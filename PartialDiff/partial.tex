\documentclass[11 pt, twoside]{article}
\usepackage{textcomp}
\usepackage[margin=1in]{geometry}
\usepackage[utf8]{inputenc}
\usepackage{color}
\usepackage{indentfirst} %Comment out for no first paragraph indent
\usepackage[parfill]{parskip}
\usepackage{setspace}
\usepackage{tikz}
\usepackage{amsmath}
\usepackage{amsfonts}
\usepackage{amssymb}
\usepackage{enumitem}
\usepackage{outlines}

\usepackage{fancyhdr}
\pagestyle{fancy}
\cfoot{\hyperlink{content}{\thepage}}
\lhead{}
\chead{}
\rfoot{}
\lfoot{}
\rhead{}
\renewcommand{\headrulewidth}{0pt}
\renewcommand{\footrulewidth}{0pt}


\usepackage{hyperref}
\hypersetup {
	colorlinks,
	citecolor=black,
	filecolor=black,
	linkcolor=black,
	urlcolor=black
}

\newcommand{\sepitem}{0pt} %Added room between items on the list, not including a list and its sublist
\newcommand{\seppar}{1pt} %Between items and lists overall

\setenumerate[1]{itemsep=\sepitem, parsep=\seppar}
\setenumerate[2]{itemsep=\sepitem, parsep=\seppar}
\setenumerate[3]{itemsep=\sepitem, parsep=\seppar}
\setenumerate[4]{itemsep=\sepitem, parsep=\seppar}

\newenvironment{outline*}
{
	\begin{outline}[enumerate]
	}
	{\end{outline}
}

\newcommand{\foot}[1]{\hyperlink{#1}{$_#1$}}

\begin{document}

\title{Partial Differential Equations}
\author{Avery Karlin}
\date{Spring 2017}
\newcommand{\textbook}{}
\newcommand{\teacher}{}

\maketitle
\newpage
\hypertarget{content}{\tableofcontents}
\vspace{11pt}
\noindent
\underline{Primary Textbook}: \textbook\\
\underline{Teacher}: \teacher
\newpage

\section{Introduction} {
Needed - Partial Derivatives
   - Ordinary Differential Equations
   - Green Theorem, Divergence, Etc
   - Complex Numbers - $\bar{z + w} = \bar{z} + \bar{w}, \bar{zw} = \bar{z} * \bar{w}$
   					 - z*bar(z) = |z|^2 = a^2 + b^2
   					 - z^(-1) = bar(z)/(|z|^2), z != 0
   					 - C is a field (closure under +, *, assocative for both, distributive, identity for both, inverses except 0 for both)
   					 - |zw| = |z||w|, s.t. product of unit vectors is a unit vector
- Conservation Laws and Flows, for some body bound by by $\partial R$, flows have a flux
	- $M_r = \int \int \int_R = \rho(\vec{v})dV, E_R(t) = \int \int \int_R  e(\vec{v}, t) dV, Q_R(t) = \int \int \int_R Q(\vec{v}, t)dV$

- For f(x(t), y(t, s)), $\frac{\partial f}{\partial t} = \frac{\partial f}{\partial x}\frac{dx}{dt} + \frac{\partial f}{\partial y}\frac{\partial y}{\partial t}$
	- $\int^b_a \frac{\partial f}{\partial x}dx = f(b, y) - f(a, y) + c(y)$ for f(x, y)
}

\section{Chapter 1 - Heat Equation}
\begin{outline*}
\1 The analysis of a physical problem requires three stages, formulation, solution, and interpretation
\1 For some one dimensional rod of constant cross-section and length L, the thermal energy density is defined by $e(x, t)$, assumed to be constant across a cross-section, such that for some cross-section, the heat energy $E = e(x, t)A\delta x$
	\2 It is assumed that heat energy change with respect to time ($\frac{\partial }{\partial t}(e(x, t)A\delta x)$ is equal to the energy flowing across boundaries combined with the energy generated inside
	\2 Heat flux is defined as the energy flowing to the right per unit time per unit surface area, $\phi (x, t)$, such that $\phi < 0$ means it is flowing to the left
	\2 Heat energy generated per unit volume per unit time is denoted as $Q(x, t)$, such that the conservation of heat energy can be written as $\frac{\partial e}{\partial t} = -\frac{\partial \phi}\partial x} + Q$ for some slice
		\3 Alternatively, it can be written not approximating for a small slice then taking the limit, such that $\frac{d}{dt} \int^b_a edx = \phi(a, t) - \phi(b, t) + \int^b_a Q dx$
		\3 This is found to also be equal to $\int^b_a \frac{\partial e}{\partial t}dx$ if a, b are constants and e is continuous \textbf{***HOW***}
		\3 It is also noted that $\phi(a, t) - \phi(b, t) = -\int^b_a \frac{\partial \phi}{\partial x}dx$ if $\phi$ is continuous differentiable, such that $\int^b_a (e_t + \phi_x - Q)dx = 0$, or $e_t = -\phi_x + Q$, equal to the differential form above assuming continuity, such that the integral form is more general
	\2 Temperature is defined as $u(x, t)$ with c(u) as the specific heat, or the heat energy per unit mass to raise the temperature one unit for some material, approximately constant over small temperature intervals
\end{outline*}

\end{document}
