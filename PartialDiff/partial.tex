\documentclass[11 pt, twoside]{article}
\usepackage{textcomp}
\usepackage[margin=1in]{geometry}
\usepackage[utf8]{inputenc}
\usepackage{color}
\usepackage{indentfirst} %Comment out for no first paragraph indent
\usepackage[parfill]{parskip}
\usepackage{setspace}
\usepackage{tikz}
\usepackage{amsmath}
\usepackage{amsfonts}
\usepackage{amssymb}
\usepackage{enumitem}
\usepackage{outlines}
\usepackage{esint}

\usepackage{fancyhdr}
\pagestyle{fancy}
\cfoot{\hyperlink{content}{\thepage}}
\lhead{}
\chead{}
\rfoot{}
\lfoot{}
\rhead{}
\renewcommand{\headrulewidth}{0pt}
\renewcommand{\footrulewidth}{0pt}


\usepackage{hyperref}
\hypersetup {
	colorlinks,
	citecolor=black,
	filecolor=black,
	linkcolor=black,
	urlcolor=black
}

\newcommand{\sepitem}{0pt} %Added room between items on the list, not including a list and its sublist
\newcommand{\seppar}{1pt} %Between items and lists overall

\setenumerate[1]{itemsep=\sepitem, parsep=\seppar}
\setenumerate[2]{itemsep=\sepitem, parsep=\seppar}
\setenumerate[3]{itemsep=\sepitem, parsep=\seppar}
\setenumerate[4]{itemsep=\sepitem, parsep=\seppar}

\newenvironment{outline*}
{
	\begin{outline}[enumerate]
	}
	{\end{outline}
}

\newcommand{\foot}[1]{\hyperlink{#1}{$_#1$}}
\newcommand\pd[2]{\frac{\partial #1}{\partial #2}}

\begin{document}

\title{Partial Differential Equations}
\author{Avery Karlin}
\date{Spring 2017}
\newcommand{\textbook}{}
\newcommand{\teacher}{}

\maketitle
\newpage
\hypertarget{content}{\tableofcontents}
\vspace{11pt}
\noindent
\underline{Primary Textbook}: \textbook\\
\underline{Teacher}: \teacher
\newpage

\section{Introduction} {
Needed - Partial Derivatives
   - Ordinary Differential Equations
   - Green Theorem, Divergence, Etc
   - Complex Numbers - $\bar{z + w} = \bar{z} + \bar{w}, \bar{zw} = \bar{z} * \bar{w}$
   					 - $z*bar(z) = |z|^2 = a^2 + b^2$
   					 - $z^(-1) = bar(z)/(|z|^2), z != 0$
   					 - C is a field (closure under +, *, assocative for both, distributive, identity for both, inverses except 0 for both)
   					 - |zw| = |z||w|, s.t. product of unit vectors is a unit vector
- Conservation Laws and Flows, for some body bound by by $\partial R$, flows have a flux
	- $M_r = \int \int \int_R = \rho(\vec{v})dV, E_R(t) = \int \int \int_R  e(\vec{v}, t) dV, Q_R(t) = \int \int \int_R Q(\vec{v}, t)dV$

- For f(x(t), y(t, s)), $\frac{\partial f}{\partial t} = \frac{\partial f}{\partial x}\frac{dx}{dt} + \frac{\partial f}{\partial y}\frac{\partial y}{\partial t}$
	- $\int^b_a \frac{\partial f}{\partial x}dx = f(b, y) - f(a, y) + c(y)$ for f(x, y)
}

\section{Chapter 1 - Heat Equation}
\begin{outline*}
\1 The analysis of a physical problem requires three stages, formulation, solution, and interpretation
\1 For some one dimensional rod of constant cross-section and length L, the thermal energy density is defined by $e(x, t)$, assumed to be constant across a cross-section, such that for some cross-section, the heat energy $E = e(x, t)A\delta x$
	\2 It is assumed that heat energy change with respect to time ($\frac{\partial }{\partial t}(e(x, t)A\delta x)$ is equal to the energy flowing across boundaries combined with the energy generated inside
	\2 Heat flux is defined as the energy flowing to the right per unit time per unit surface area, $\phi (x, t)$, such that $\phi < 0$ means it is flowing to the left
	\2 Heat energy generated per unit volume per unit time is denoted as $Q(x, t)$, such that the conservation of heat energy can be written as $\frac{\partial e}{\partial t} = -\frac{\partial \phi}{\partial x} + Q$ for some slice
		\3 Alternatively, it can be written not approximating for a small slice then taking the limit, such that $\frac{d}{dt} \int^b_a edx = \phi(a, t) - \phi(b, t) + \int^b_a Q dx$
		\3 This is found to also be equal to $\int^b_a \frac{\partial e}{\partial t}dx$ if a, b are constants and e is continuous \textbf{***HOW***}
		\3 It is also noted that $\phi(a, t) - \phi(b, t) = -\int^b_a \frac{\partial \phi}{\partial x}dx$ if $\phi$ is continuous differentiable, such that $\int^b_a (e_t + \phi_x - Q)dx = 0$, or $e_t = -\phi_x + Q$, equal to the differential form above assuming continuity, such that the integral form is more general
	\2 Temperature is defined as $u(x, t)$ with c(u) as the specific heat, or the heat energy per unit mass to raise the temperature one unit for some material, approximately constant over small temperature intervals
		\3 As a result, $e(x, t) = c(x)\rho(x)u(x, t)$, where $\rho(x)$ is the mass density of the tube, giving the relationship between thermal density and temperature, able to be substituted into the equation
	\2 This provides the relationship between temperature and flux, but does not give a conversion between, found to be $\phi = -K_0 \frac{\partial u}{\partial x}$, called Fourier's Law of Heat Conduction
		\3 This is found by the facts that heat goes from hotter to lower, does not flow if temperature is equal, higher differences cause more flow, and the flow will based on materials
		\3 $K_0$ is the ability of a material to conduct heat, called the thermal conductivity, such that for heterogeneous materials, it is a function of x, and varies with temperature, though is generally constant in some range
		\3 Thus, for constant c, $\rho, K_0$, the heat equation is found to be $\frac{\partial u}{\partial t} = k\frac{\partial^2 u}{\partial x^2}$, where $k = \frac{K_0}{c\rho}$, called thermal diffusivity
	\2 If the heat energy is originally isolated into one location, it describes the spreading of it, or the diffusion, such that it is also called the diffusion equation
		\3 Similarly, for chemical diffusion, u(x, t) is the density/concentration of the chemical, gaining Fick's Law of Diffusion, analogous to Fourier's Law
\1 For PDEs, the number of initial conditions equal to higher derivative of the spacial or temporal factor must be given, for 1D heat equation, generally the initial boundary conditions
	\2 For a prescribed fluid bath reservoir temperature at one end, the condition is such that $u(0, t) = u_B(t)$
	\2 The flux can also be prescribed, such as if the boundary is insulated $\frac{\partial u}{\partial x}(0, t) = 0$, such that flux is also 0 at that boundary
	\2 Newton's Law of Cooling is used if the rod is in contact with a mooving fluid, such that heat will continuously move to/from the air, found to be proportional to the temperature difference between the external temperature and the rod at that location
		\3 Thus, at the boundary, it is written as $-K_0(0)\frac{\partial u}{\partial x}(0, t) = -H(u(0, t) - u_B(t))$, where H is the heat transfer coefficient
		\3 The heat transfer coefficient represent the degree of insulation of the boundary, such that 0 is complete insulation, to infinity for uninsulated
\1 Steady initial conditions are those that do not depend on time, while equilibrium/steady-state solutions are solutions that do not depend on time, such that for the heat equation, $\frac{d^2u}{dx^2} = 0$
	\2 As a result, for steady boundary temperatures, $u(x) = T_1 + \frac{T_2 - T_1}{L}x$, such that for some initial stateit will eventually reach the steady state solution, while for insultated edges, the steady solution is a constant
		\3 To get a specific constant, some initial function of temperature at the initial time is given, f(x), such that $u(x) = C_2 = \frac{1}{L} \int^L_0 f(x)dx$, such that it is the average of the initial temperature distribution
\1 This equation is able to be extended to higher dimensions by the $E = \int \int \int_R c\rho u dV$ and heat flux is defined as a vector, positive for outward rather than right, using the outward normal vector $\vec{n}$
	\2 Thus, the conservation law can be written by $\frac{d}{dt} \int\int\int_R c\rho udV = - \oiint_{\partial R} \phi \cdot \vec{n}dS + \int\int\int_R QdV$
		\3 The divergence theorem states that $\int\int\int_R \nabla \cdot \vec{A} dV = \oiint_{\partial R} \vec{A} \cdot \vec{n} dS$
	\2 As a result, by the same reasoning as for 1D, $c\rho \pd{u}{t} + \nabla \cdot \phi - Q = 0$ and $\phi = -K_0\nabla u$, combined for Fourier's Law of Conduction
	\2 For Q = 0, $\pd{u}{t} = k\nabla \cdot \nabla u = k\nabla^2 u$, where $\nabla^2 u$ is the Laplacian of u
	\2 For the boundary conditions, the boundary can have a known constant temperature, or be partially insulated, such that $\nabla u \cdot \vec{n} = 0$ (directional derivative outward at the boundary is 0)
		\3 Newton's Law of Cooling can also apply, such that $-K_0 \nabla u \cdot \vec{n} = H(u - u_b)$
	\2 The steady state solution is such that $\nabla^2u = \frac{-Q}{K_0}$, called Poisson's equation, such that if Q = 0, $\nabla^2u = 0$, called Laplace's/the potential equation
\1 For cylindrical coordinates, the Laplacian is shown to be $\nabla^2 u = \frac{1}{r}\pd{}{r}(r\pd{u}{r}) + \frac{1}{r^2}\pd{^2u}{\theta^2} + \pd{^2u}{z^2}$
	\2 Situations where u is constant for $\theta$ are said to be circularly or axially symmetric
	\2 Spherical coordinates are written $(p, \theta, \phi)$, where $0 \leq \phi \leq \pi$, such that $x = psin(\phi)cos(\theta), y = psin(\phi)sin(\theta), z = pcos(\phi)$
		\3 Thus, $\nabla^2u = \frac{1}{p^2}\pd{}{p}(p^2\pd{u}{p}) + \frac{1}{p^2}{sin\phi}\pd{}{\phi}(sin\phi\pd{u}{\phi}) + \frac{1}{p^2sin^2\phi}\pd{^2u}{\theta^2}$
\end{outline*}
\section{Seperation of Variables}
\subsection{Introduction and Heat Equation}
\begin{outline*}
\1 The method of seperation of variables is used when the partial differential equation and boundary conditions are linear and homogeneous
	\2 Linear operators are those that satisfy the linearity property, $L(c_1u_1 + c_2u_2) = c_1L(u_1) + c_2L(u_2)$
		\3 The heat operator is a linear operator, $L(u) = \pd{u}{t} - k\pd{^2u}{x^2}$
		\3 Linear equations are those of the form L(u) = f, where f is a known function and L is a linear operator
	\2 Homogeneous equations are those of the form L(u) = 0, such that if L is a linear operator, it is a linear homogeneous equation, such that u = 0 is always a solution, called the trivial solution
	\2 Linear homogeneous equations have the principle of superpoosition, such that if $u_1, u_2$ are solutions, then all linear combinations of them are also solutions
		\3 Linear homogeneous properties also must be tested for the boundary conditions
\1 For the 1D homogeneous heat equation with zero temperatures at both ends, and initial condition u(x, 0) = f(x), it acts as a linear homogeneous partial differential with linear homogeneous boundary conditions
	\2 Separation of variables attempts to find solutions of $u(x, t) = \phi(x)G(t)$ (product form), ignoring the initial conditions, due to generally not satisfying in this form
		\3 As a result, by the boundary conditions, either u is the trivial solution, or $\phi(0) = 0, \phi(L) = 0$, providing new boundary condition forms
	\2 This can be converted to the form $\frac{1}{kG}\frac{dG}{dt} = \frac{1}{\phi}\frac{d^2\phi}{dx^2}$, such that for each variable to be independent, rather than a function of the other, it is set equal to the separation constant, $-\lambda$, forming 2 ODEs
		\3 Thus, $G(t) = ce^{-\lambda kt}$, such that physically, $\lambda \geq 0$, due to otherwise the temperature increasing exponentially
\1 The spacial ODE for the 1D heat equation with 0 temperature boundaries is a boundary value problem, rather than an IVP, not automatically providing a unique solution, allowing nontrivial solutions to be found
	\2 The values such that there is a nontrivial solution are eigenvalues, where the nontrivial $\phi(x)$ is called the eigenfunction of the value
	\2 It is assumed that $\lambda$ is real, where the solutions are of the form $\phi = e^{rx}, r^2 = -\lambda$
		\3 If $\lambda > 0$, $r = \pm i\sqrt{\lambda}$, such that the solutons oscillate by each of the respective components as separate solutions, such that $\phi = c_1 cos(\sqrt{\lambda} x) + c_2 sin(\sqrt{\lambda} x)$, though any linearly independent solution can be used
			\4 Thus, by the boundary conditions, $c_1 = 0$, such that $\phi(x) = c_2 sin(\frac{n\pi x}{L})$, where n is any positive integer, giving eigenvalues/eigenfunctions, denoted $\phi_n(x)$ for each respective n
		\3 If $\lambda = 0$, $\phi = c_1 + c_2x$, such that $c_1 = c_2 = 0$ by the boundary conditions, such that it is the trivial solution, such that $\lambda = 0$ is not an eigenvalue for the heat equation
		\3 If $\lambda < 0$, $\phi = c_1 e^{\sqrt{-\lambda}x} + c_2 e^{-\sqrt{-\lambda}x}$, this can be written in terms of the hyperbolic functions for simplicity
			\4 These are written as $cosh(x) = \frac{e^x + e^{-x}}{2}$ and $sinh(x) = \frac{e^x - e^{-x}}{2}$, where $cosh'(x) = sinh(x)$ and $sinh'(x) = cosh(x)$
			\4 Thus, $\phi = c'_1 cosh(\sqrt{s}x) + c'_2 sinh(\sqrt{s}x)$
			\4 The boundary conditions produce only the trivial solution for this form of the equation, such that there are no negative eigenvalues
	\2 As a result, $u(x, t) = Bsin(\frac{n\pi x}{L})e^{-k(\frac{n\pi}{L})^2t}$, such that as $t \to \infty, u(x, t) = 0$
		\3 This solution can be used to satisfy an IVP assuming the initial condition is of the correct format, $u(x, 0) = Bsin(\frac{n\pi x}{L}$ for some n
		\3 Similarly, for initial conditions which are the sum of this form, the solution can be found as the sum of the respective solutions for each initial condition
\1 Since the linear combination of solutions is a solution, the sum of each solution for n is a solution, with possibly different aplitudes for each
	\2 Thus, $u(x, t) = \sum_{n = 1}^M B_n sin(\frac{n \pi x}{L}e^{-k(\frac{n\pi}{L})^2t}$ with initial condition, $u(x, 0) = \sum_{n = 1}^M B_n sin(\frac{n \pi x}{L}$
	\2 This is useful, due to the infinite sum of sine curves of this form being a type of Fourier series, such that any function f(x) can be approximated as it as the series approaches infinity
		\3 It is noted that $\int^L_0 sin(\frac{n\pi x}{L})sin(\frac{m\pi x}{L})dx = 0$ if $m \neq n$ and $= \frac{L}{2}$ if m = n
		\3 Thus, it can be multiplied by an additional term, such that for some initial condition f(x), $f(x) = \sum^{\infty}_{n = 1} B_n sin(\frac{n\pi x}{L}$, $\int^L_0 f(x) sin(\frac{m\pi x}{L})dx = \sum_{n = 1}^{\infty} B_n \int^L_0 sin(\frac{n\pi x}{L}) sin(\frac{m\pi x}{L}) dx$, such that the series reduces to $B_m \int^L_0 sin^2(\frac{m\pi x}{L}dx$
		\3 Thus, $B_m = \frac{2}{L} \int^L_0 f(x) sin(\frac{m\pi x}{L}dx$, such that the initial condition function can be plugged in for f(x) to give the coefficients for the solution
	\2 The fact that the integral over some number of complete half-periods of the square of a sinusoidal function is half the interval is needed often in periodic calculations
\1 Orthogonal functions over some interval are those whose product intergrated over the interval is 0, such that a set of functions where each is orthogonal to each other function in the set is an orthogonal set of functions
	\2 Thus, the set of functions, f(x) = $sin(\frac{n\pi x}{L})$ is an orthogonal set, shown to fit the orthogonal condition
\1 Thus, the basic process for separation of variables is to make sure both the equation and boundary conditions are linear, homogeneous, ignore the nonzero initial condiiton, introduce the seperation constant
	\2 Then, the constants are determined as eigenvalues of a boundary value problem, the other differential equations are solved, the solutions are combined, initial condition is applied, and coefficients are found using the orthogonality of the eigenfuntions
	\2 It can then be approximated as the earlier terms often as the time increases, moving towards equilibrium solution
\end{outline*}
\subsection{Other Examples}
\begin{outline*}
\1 For heat conduction in 1D with insulated ends, such that there is no internal generation of heat or outside sources, it is found to have both a zero and positive solution, such that $u(x, t) = A_0 + \sum^{\infty}_{n = 1} A_n cos(\frac{n \pi x}{L})e^{-(\frac{n\pi}{L})^2kt}$
	\2 This is able to be rewritten $u(x, t) = \sum^{\infty}_{n = 0} A_n cos(\frac{n \pi x}{L})e^{-(\frac{n\pi}{L})^2kt}$
	\2 Thus, the initial condition is valid for this solution if $f(x) = A_0 + \sum^{\infty}_{n = 1} A_n cos(\frac{n\pi x}{L})$ for $0 \leq x \leq L$
	\2 The orthogonality relations, $\int^L_0 cos(\frac{n\pi x}{L})cos(\frac{m\pi x}{L})dx = L$ if m = n = 0, = $\frac{L}{2}$ if $n = m \neq 0$, and = 0 if $n \neq m$, such that they are an orthogonal set of functions
		\3 As a result, $A_0 = \frac{1}{L} \int^L_0 f(x)dx$ and $A_m = \frac{2}{L} \int^L_0 f(x) cos(\frac{m\pi x}{L})dx$, where $m \geq 1$
	\2 The steady-state solution as a result is the constant soution, due to the remainder decaying over time
\1 For a circular 1D wire bound at the ends of length 2L, such that both the temperature and the flux are equal at both ends, with no internal heat sources, the boundary conditions are called mixed type due to involving both boundaries and periodic, due to applying over the entire x axis, as the x values repeat
	\2 It is found to be valid for $\lambda \geq 0$, such that $u(x, t) = a_0 + \sum_{n = 1}^{\infty} a_n cos(\frac{n\pi x}{L})e^{-\frac{-n\pi}{L}^2kt} + \sum_{n = 1}^{\infty} b_n sin(\frac{n\pi x}{L})e^{-\frac{-n\pi}{L}^2kt}$
	\2 The orthogonality relations state for -L to L, $\int^L_{-L} cos(\frac{n\pi x}{L})cos(\frac{m\pi x}{L}dx = 0 if n \neq m, L if n = m \neq 0, 2L if n = m = 0$, $\int^L_{-L} sin(\frac{n\pi x}{L})sin(\frac{m\pi x}{L})dx = 0 if n \neq m, = L if n = m \neq 0$, and $\int^L_{-L} sin(\frac{n\pi x}{L})cos(\frac{m\pi x}{L})dx = 0$
		\3 As a result, $a_0 = \frac{1}{2L} \int^L_{-L} f(x)dx, a_m = \frac{1}{L} \int^L_{-L} f(x) cos(\frac{m\pi x}{L})dx$, and $b_m = \frac{1}{L} \int^L_{-L} f(x) sin(\frac{m\pi x}{L})dx$
\1 For some rectangle, such that each edge has a nonhomogeneous boundary condition, with the temperature of the rectangle being based on the Laplacian, it is solved by dividing into four problems, each with a single nonhomogeneous condition and three homogeneous conditions with maximum x of L, maximum y of H
	\2 For the boundary problem such that $u_1(0, y) = g_1(y)$, using the product solution, $u_1(0, y) = h(x)\phi(y)$, such that $\frac{1}{h}\frac{d^2h}{dx^2} = \frac{-1}{\phi}\frac{d^2\phi}{dy^2} = \lambda$
	\2 Since the x component is not a BVP due to not having two homogeneous boundary conditions, using y to determine eigenvalues of $\lambda = (\frac{n\pi}{H})^2$, with $\phi(y) = sin(\lambda y)$
		\3 The eigenvalues are then used to produce an ODE to solve for the x component, giving a hyperbolic linear combination, neither ideal for the h(L) = 0 boundary condition
			\4 Since the differential equation remains the same for a translation, called invariant on translation, the linear combination is able to be shifted L, such that it is made a function of (x - L) instead of x
\1 For some circle undere the Laplace equation with a single nonhomogeneous boundary condition on the radius, additional conditions must be determined to allow it to be solved
	\2 Additional boundaries are found by the fact that the center of the circle must be bounded, such that $|u(0, \theta)| < \infty$, and due to being a circle, there are periodicity conditions such that $u(r, -\pi) = u(r, \pi), \frac{\partial u}{\partial \theta}(r, -\pi) = \frac{\partial u}{\partial \theta}(r, \pi)$
		\3 These conditions are homogeneous, such that they act as the three homogeneous conditions for separation of variables
		\3 The periodic boundary conditions are the forms of the periodicity conditions applying purely to the $\theta$ component of the product solution
			\4 As a result, the eigenvalue problem can be solved to find that $\lambda = n^2$, for $n \geq 0$, with $\phi(\theta) = Asin(n\theta) + Bcos(n\theta)$
		\3 The radial component is then found to be valid for the equation $r^2\frac{d^2G}{dr^2} + r\frac{dG}{dr} - n^2G = 0$, such that it is a Cauchy-Euler/equidimensional equation, of the form $r^p$, such that $G(r) = Cr^n + Dr^{-n}$ for $n \neq 0$, $G(r) = C + Dln(r)$ for n = 0
			\4 Since it cannot approach infinity as $r \to 0$, D must be equal to 0, such that it remains finite as n approaches infinity
		\3 Thus, $u(r, \theta) = \sum^{\infty}_{n = 0} A_nr^ncos(n\theta) + \sum_{n=1}^{\infty} B_nr^nsin(n\theta)$, such that for boundary condition at r = a, $A_0 = \frac{1}{2\pi}\int^{\pi}_{-\pi} f(\theta)d\theta, A_na^n = \frac{1}{\pi}\int^{\pi}_{-\pi} f(\theta)cos(n\theta)d\theta, B_na^n = \frac{1}{\pi}\int^{\pi}_{-\pi} f(\theta)sin(n\theta)d\theta,$
	\2 As a result, the Laplace Equation is found to have the properties that any point inside a circle is equal to the average of the values of the border, such that the maximum and minimum must lie on the border in the steady state
		\3 It is also found that as a result of these characteristics, the Laplacian is well-posed (such that a small charge in the boundary condiitons leads to a small change in the solution), and that the solution is unique
	\2 The solvability condition for Laplace's equation is found that if the heat flux is specified, $0 = \int\int \nabla^2 u dxdy = \oint \nabla u \cdot \vec{n} ds$ by the Divergence Theorem or there is no solution, called the solvability/compatibility condition
		\3 This is due to going against the steady state assumption, due to a change in time of the thermal energy
\end{outline*}
\section{Chapter 3 - Fourier Series Theory}
\begin{outline*}
\1 The Fourier series is defined for f(x) over the interval $-L \leq x \leq L$ as $a_0 + \sum^{\infty}_{n = 1} a_ncos(\frac{n\pi x}{L}) + \sum^{\infty}_{n = 1} b_nsin(\frac{n\pi x}{L})$, with Fourier coefficients $a_0 = \frac{1}{2L} \int^L_{-L} f(x)dx, a_n = \frac{1}{L}\int^L_{-L} f(x)cos(\frac{n\pi x}{L})dx, b_n = \frac{1}{L}\int^L_{-L} f(x)sin(\frac{n\pi x}{L})dx$
	\2 Thus, the Fourier series only exists if the coefficients exist for the functions, and cannot be assumed to be precisely equal to f(x), such that it is written for Fourier series g(x) as $f(x) ~ g(x)$
\1 The Convergence Theorem for Fourier Series states that if f(x) is piecewise smooth on the interval $-L \leq x \leq L$, the series converges to the periodic extension of f(x) where the extension if continuous, and to the average of the two limits at jump discontinuities
	\2 Piecewise smooth is defined as continuous in the function and its first derivative except with possible jump discontinuities, in which both the left and right limit exist, but are unequal
	\2 Periodic extension of a function is the function drawn over a 2L period, then made periodic and repeating
	\2 As a result, Fourier series can be drawn by drawing f(x) over the period, then making the periodic extension, with an x to mark the average of two values at jump discontinuities
	\2 Thus, for f(x) without jump discontinuities in the extension intersections or the function itself, the Fourier series will precisely equal the function and act as a continuous function
		\3 Thus, for an odd extension, it requires a boundary value of 0 for it to be continuous, while the boundary of an even extension is always continuous
\1 For some odd function, f(x), it must be an infinite series of sine functions since by symmetry, if taken over some symmetrical about the y-axis, the integral of the cosine coefficients is 0
	\2 In addition, since it is anti-symmetric, $b_n = \frac{1}{L} \int^L_{-L} f(x)sin(\frac{n\pi x}{L})dx = \frac{2}{L}\int^L_0 f(x)sin(\frac{n\pi x}{L})dx$
	\2 As a result, if f(x) is given only for a positive region, it can be extended as an odd function, such that the Fourier is for the odd extension, only using it over the required region, called the Fourier sine series of f(x) over $0 \leq x \leq L$
\1 The Gibbs phenomenon is found for finite Fourier approximations, in which at a jump discontinuity, the function with overshoot in the opposite direction by approximately 9\% of the jump
\1 For some even function f(x), the sine coefficients are 0, and since it is symmetric, $a_0 = \frac{1}{2L} \int^L_{-L} f(x)dx = \frac{1}{L} \int^L_0 f(x)dx$ and $a_n = \frac{1}{L} \int^L_{-L} f(x)cos(\frac{n\pi x}{L})dx = \frac{2}{L} \int^L_0 f(x)cos(\frac{n\pi x}{L})dx$
	\2 As a result, this uses the even extension of f(x), similar to the odd extension for sines
\1 For a general function f(x), as a result, both sets of terms are needed, such that $a_0 = \frac{1}{2L}\int^L_{-L} f(x)dx, a_n = \frac{1}{L} \int^L_{-L} f(x) cos(\frac{n\pi x}{L})dx, b_n = \frac{1}{L} \int^L_{-L} f(x) sin(\frac{n\pi x}{L})dx$, unable to use the symmetry half-period forms
	\2 It is noted that the coefficients of the sine or cosine series are generally not the same as that of a general Fourier series
	\2 On the other hand, for some f(x), since f(x) = $\frac{1}{2}(f(x) + f(-x)) + \frac{1}{2}(f(x) - f(-x))$, the former term as the even part of f(x), the latter as the odd part, any function can be written as an odd and even portion
		\3 Thus, any Fourier series is equal to the sine series of the odd portion and the cosine series of the even portion
		\3 It is noted that the sine series of the odd portion/cosine of the even portion are distinct from the odd/even extensions of the half periods as a result
\1 Fourier series are not able to be term by term differentiated due to being an infinite series, unless the Fourier series of f(x) is continuous and f'(x) is piecewise smooth
	\2 As a result, a Fourier sine series must be of a function f(x) where f(0) = f(L) = 0 for the series to be able to be differentiated, while a cosine series just has the condition on f'(x) being piecewise smooth
	\2 It is also found that for some function f(x), where f'(x) is piecewise smooth, but the Fourier series is not continuous, the Fourier series of f'(x) is $\frac{1}{L}(f(L) - f(0)) + \sum_{n = 1}^{\infty} (\frac{n\pi}{L}B_n + \frac{2}{L}((-1)^nf(L) - f(0)))cos(\frac{n\pi x}{L}$
	\2 The method of Eigenfunction expansion solves for the eigenfunctions as a function of x, taking the Fourier series of it, with the coefficients as functions of t, to provide the full function solution
		\3 This is used in cases where the boundary conditions are nonhomogeneous or there are internal sources ($Q \neq 0$), but in which the Fourier series, and its partials and second partials are continuous
		\3 It is noted that in this case, the Fourier series can be differentiated with respect to t if the u is continuous and $\frac{\partial u}{\partial t}$ is piecewise smooth
\1 Fourier series are able to be integrated term by term for some piecewise smooth f(x), even with jump discontinuities, though the integrated function may not be a pure Fourier series
\1 Complex exponentials can be used in the Fourier series instead by the fact that $cos\theta = \frac{e^{i\theta} + +e^{-i\theta}}{2}$ and $sin\theta = \frac{e^{i\theta} - e^{-i\theta}}{2i}$
	\2 Using the negation of the n-index and $c_0 = a_0, c_n = \frac{a_n + ib_n}{2}$, $f(x) ~ \sum_{n = -\infty}^{\infty} c_n e^{-\frac{in\pi x}{L}}$, called the complex form of the Fourier series of f(x)
	\2 As a result, for $n \neq 0$, $c_n = \frac{1}{2L} \int^L_{-L} f(x)e^{\frac{in\pi x}{L}}dx$, such that if f(x) is real, $c_{-n} = \bar{c}_n$
		\3 This formula can also be determined by orthogonality of eigenfunctions, such that complex functions are said to be orthogonal if $\int \bar{f}gdx = 0$, such that $\int^L_{-L} \bar{e^{-\frac{im\pi x}{L}}} e^{-\frac{in\pi x}{L}}dx = 2L$ if n = m, 0 otherwise
		\3 Since $\bar{e^{-\frac{im\pi x}{L}}} = e^{\frac{im\pi x}{L}}$, this is used to solve for the coefficients
\end{outline*}
\section{Chapter 4 - Wave Equation of Vibrating Strings and Membranes}
\begin{outline*}
\1 For some tightly stretched string with constant mass density, if the slope of the vibrating portion is small, the motion of a particle can be thought to be entirely vertical
	\2 It is assumed to be perfectly flexible, such that there is no resistance, with the gravitational force and the tangential tension force at each point, using $\theta(x, t)$ as the angle from the x-axis of the string at some point
		\3 For some portion of string $\Delta x$ long, it is described by $\rho(x)\Delta x \frac{\partial^2 u}{\partial t^2} = T(x + \Delta x, t)sin(\theta(x + \Delta x, t)) - T(x, t)sin(\theta(x, t)) + \rho(x)\Delta x Q(x, t)$, where Q is the vertical body force per mass
		\3 This is differentiated and using the small angle appoximation stating that $\frac{du}{dx} = sin(\theta)$, $\rho(x)\frac{\partial^2 u}{\partial t^2} = \frac{\partial}{\partial x}(T\frac{\partial u}{\partial x}) + \rho(x)Q(x, t)$
	\2 For a perfectly elastic string, the tension is based on the angle, such that for a small angle, the tension is constant, and for a uniform string, $\rho$ is constant 
	\2 Assuming the body force is negligable, or assuming it is a sagged equilibrium, but irrelevent to the equation, $\frac{\partial^2 u}{\partial t^2} = c^2\frac{\partial^2 u}{\partial x^2}$, where $c^2 = \frac{T_0}{\rho}$, with c as the velocity
		\3 This is called the general form of the one-dimensional wave equation
\1 String waves are able to have variable support boundary conditions, due to one end attached to a dynamical system, such as a spring-mass system, designated by $u(0, t) = y(t)$
	\2 The function y(t) itself is a ordinary differential equation based on Newton's laws, such that for unstretched length l, the string stretching is $y(t) - l - y_{base}(t)$
	\2 Thus, it is described by $m\frac{d^2y}{dt^2} = -k(y(t) - y_{base}(t) - l) + T_0\frac{du}{dx}(0, t) + Q(t)$, allowing vertical tension, found by the perfectly elastic string and small-angle approximation
	\2 For no external forces on the mass, such that g = 0, it can be written in terms of the equlibirum position as $T_0\frac{\partial u}{\partial x}(0, t) = k(u(0, t) - u_E(t))$, appearing as Newton's Law of Cooling
		\3 Similarly, for x = L, -k is used, such that the signs are equal of the function and first derivative, since $T_0, k > 0$
	\2 The other end of the variable support string can be a free end, able to move up and down at the specific x position, found to be valid as $k \to 0$
\1 Strings with fixed ends are solved giving an initial condition for both displacement and velocity, set equal to $-\lambda$, such that $u(x, t) = \sum^{\infty}_{n = 1}(A_nsin(\frac{n\pi x}{L})cos(\frac{n\pi ct}{L}) + B_nsin(\frac{n\pi x}{L})sin({n \pi ct}{L}))$
	\2 Thus, if $u(x, 0) = f(x), \frac{\partial u}{\partial t}(x, 0) = g(x)$, $A_n = \frac{2}{L}\int^L_0 f(x)sin(\frac{n\pi x}{L})dx$ and $B_n\frac{n\pi c}{L} = \frac{2}{L}\int^L_0 g(x)sin(\frac{n\pi x}{L})dx$
	\2 The normal modes of vibration are each set of terms of the vertical displacement, made up of a single standing wave, such that there is a node at which it is constantly u = 0 other than the boundaries
		\3 It is found that for a first harmonic, there are zero nodes, with each subsequent harmonic having an additional node
		\3 It can be found by trig identities that it is the superposition of two travelling waves of the same form moving in opposite directions, such that it can be written as $u(x, t) = C_1f_1(x - ct) + C_2f_2(x + ct)$, valid for all boundary condition 1D wave equation solutions
	\2 The intensity of the sound is proportional to the amplitude ($\sqrt{A_n^2 + B_n^2}$), while the circular/angular frequency of each normal mode $(\omega = \frac{f}{2\pi})$ is equal to $\frac{n\pi c}{L}$
		\3 Each term frequency is called the natural frequency, such that higher frequency is higher pitch, each called the $n^{th}$ harmonic with the first as the fundemental frequency, such that each higher harmonic is a multiple
\1 The wave equation is extended to higher dimensions as $\pd{^2u}{t^2} = c^2\nabla^2u$, similarly to that of a heat equation, applying to a 2D membrane, derived as the 1D version
	\2 It also applies to fluid such that for pressure P, $c^2 = \pd{P}{\rho}$, as well as for light, such that $c^2 = \frac{c_{light}^2}{\mu\epsilon}$
	\2 Plane travelling waves of light satisfy the wave equation of the form $u = Ae^{i(k_1x + k_2y + k_3z - \omega t)}$, where $(k_1, k_2, k_3)$ is the wave vector, in the direction of propogation, with magnitude as the wavenumber and such that $ck = \omega$
		\3 For two different materials with a plane boundary between at z = 0, with light velocity $c_+, z > 0$ and $c_-, z < 0$, with an incident plane wave with wave vector $k_I$ with angle $\theta_I$ to the normal and amplitude 1, it is reflected and transmitted
		\3 It is assumed that there is a reflected wave remaining with $z > 0$, the solution above the origin is the sum of the initial and reflected waves, the latter with $k_R$
			\4 It is also assumed that there is a refracted/transmitted wave with $k_T$ and $\theta_I$ from the normal, as the solution for z < 0
			\4 **FINISH 4.6 IF NEEDED**
\end{outline*}
\end{document}



$u_n(x, 0) = B_nsin(\frac{n\pi}{L}x)$ if $f(x) = 5sin(\frac{7\pi}{L}x) + 7sin(\frac{13\pi}{L}x)$, then $u = 5sin(\frac{7\pi}{L}x)e^{-k(\frac{7\pi}{L})^2t} + 7sin(\frac{13\pi}{L}x)e^{-k(\frac{13\pi}{L})^2t}$

Orthogonality relations can be found by integrating or by complex exponential