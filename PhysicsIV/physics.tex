\documentclass[11 pt, twoside]{article}
\usepackage{textcomp}
\usepackage[margin=1in]{geometry}
\usepackage[utf8]{inputenc}
\usepackage{color}
\usepackage{indentfirst} %Comment out for no first paragraph indent
\usepackage[parfill]{parskip}
\usepackage{setspace}
\usepackage{tikz}
\usepackage{amsmath}
\usepackage{amsfonts}
\usepackage{amssymb}
\usepackage{enumitem}
\usepackage{outlines}

\usepackage{fancyhdr}
\pagestyle{fancy}
\cfoot{\hyperlink{content}{\thepage}}
\lhead{}
\chead{}
\rfoot{}
\lfoot{}
\rhead{}
\renewcommand{\headrulewidth}{0pt}
\renewcommand{\footrulewidth}{0pt}


\usepackage{hyperref}
\hypersetup {
	colorlinks,
	citecolor=black,
	filecolor=black,
	linkcolor=black,
	urlcolor=black
}

\newcommand{\sepitem}{0pt} %Added room between items on the list, not including a list and its sublist
\newcommand{\seppar}{1pt} %Between items and lists overall

\setenumerate[1]{itemsep=\sepitem, parsep=\seppar}
\setenumerate[2]{itemsep=\sepitem, parsep=\seppar}
\setenumerate[3]{itemsep=\sepitem, parsep=\seppar}
\setenumerate[4]{itemsep=\sepitem, parsep=\seppar}

\newenvironment{outline*}
{
	\begin{outline}[enumerate]
	}
	{\end{outline}
}

\newcommand{\foot}[1]{\hyperlink{#1}{$_#1$}}

\begin{document}

\title{Principles of Physics IV: Modern Physics}
\author{Avery Karlin}
\date{Spring 2017}
\newcommand{\textbook}{Modern Physics by Tipler and Llewellyn}
\newcommand{\teacher}{Dr. Paul Heiney}

\maketitle
\newpage
\hypertarget{content}{\tableofcontents}
\vspace{11pt}
\noindent
\underline{Primary Textbook}: \textbook\\
\underline{Teacher}: \teacher
\newpage

\section{Course Introduction}
\begin{outline*}
\1 Kinematic assumptions originally stated that all items have a well-defined position that can be measured to any precision, and all observers agree on the time and position for a measurement
\1 Newtonian mechanics were based on the idea of the deist clockwork universe
	\2 Conservation of momentum follows from Newton's third law, conservation of energy follows from the work energy theorm
	\2 Conservative forces have potential energy defined such that F = $-\nabla V$ (or U, depending on notation)
	\2 Newton's Laws combined with the fact that all known major forces act on the center of an object, implies conservation of angular momentum
\1 Relativistic mechanics has $p = \gamma mv$ and $E^2 = c^2p^2 + m_0^2c^4$, and defined rigid bodies as impossible, often using the center of mass
\1 Waves are travelling disturbances in a medium, averaged over space to consider bulk properties, carrying energy and momentum
	\2 Wave equations often relate curvature (second derivative w/ respect to x) to acceleration, $f_{xx} = (1/u^2) f_{tt}$ (u can be replaced with v, acting as an alternate velocity notation)
	\2 Waves are travelling, such that f(x, t) = g(x - ut)
	\2 y = exp(x - ct) is travelling, while sin(x)cos(ct) is a sum of two travelling, but not a travelling on its own, such that any g(x - ut) works
\1 Waves but not particles can diffract, refract, and superposition, while both have velocity, generally localized position, momentum, and energy
	\2 Superposition creates nodes of no displacement, antinodes of max displacement in a standing wave
	\2 Energy is proportional to the square of the amplitude
	\2 Wave properties include that $v_p = f\lambda = \frac{\omega}{\lambda}$, $\omega = 2\pi f$, and $k = 2 \pi \lambda$
\1 Maxwell's equations show that since changing magnetic fields produce electric fields and vice versa, waves can be created by oscillating or accelerating charges, travelling forever in vacuum
\1 Statistical mechanics are used to solve systems with too many obects to calculate, instead using statistics to predict average properties
	\2 For some complicated system with total conserved energy and many ways to distribute the energy, the probability of any entity having a particular energy is independent of others, such that it depends only on others, but limits the possibilities of remaining energies
		\3 Thus, $P(E_1)P(E_2) = f(E_1 + E_2)$, since the probability of the joint energy depends only on the sum, found to be true if $P(E) = Ab^{\pm cE} = Ae^{-\beta E}$, where $\beta = \frac{1}{k_BT}$, $k_b = 1.38 * 10^{-23} J/K$
		\3 n(E) is defined as the number of distinct ways an object could have energy E, such that the sum of the energies must be 1, either a sum or an integral if discrete or continuous, such that $P(E) = \frac{e^{-\beta E}}{\sum_j n(E_j) e^{-\beta E_j}}$
	\2 Thus, the average value is the sum of the value multiplied by the probability for all possible energy values, $\bar{g(E)} = \frac{\sum_j n(E_j)g(E_j)e^{-\beta E_j}}{\sum_j n(E_j)e^{-\beta E_j}}$
	\2 The Boltzmann Distribution assumes n(E) = 1 for all E, and the energy distribution is continuous, such that $P(E) = \frac{e^{-\beta E}}{k_BT}, \bar{E} = k_BT$
\1 Single photon cannot pair produce an electron and a positron, due to the center of mass frame proving lack of conservation of momentum, though a single photon can collide into a mass and pair produce
\1 Existance of atoms based on the macroscopic shape of crystals, the inability to convert elements to another type, stoichiometry chemistry rules, and periodic table able to predict properties of atoms
\1 de Broglie later stated that massive particles could have wavelength by the same formuila, $p\lambda = \lambda \sqrt{2mK} = h$, where K is kinetic energy
\1 For two spheres on a collision course, they will collide if the center-center distance is at most $r_1 + r_2$, such that effective area/cross-section, $\sigma = \pi (r_1 + r_2)^2$
	\2 Thus, the probability of an electron colliding in a box with sides of area A/$L^2$ is $P = \frac{M\sigma}{A} = nL\sigma$, where M is the mass of the box and n is the density of the box in particles/volume
	\2 For some surface, the number colliding per area, $\Delta N = -PN$ where N is the number incident on the surface per second, and the number through the box of thickness x is thus, $\frac{\Delta N}{\Delta x} = -nN\sigma$
	\2 Thus, the number remaining in the beam, $N(x) = N(0)e^{-n\sigma x} = N(0)e^{-x/l}$ where l is the mean scattering length, or the average distance before a collision
\end{outline*}

\section{Chapter 3 - Quantization of Charge, Light, and Energy}
\begin{outline*}
\1 In the 1800s, Faraday proved that a specific quantity of electricty could decompose one gram-ionic weight of monovalent ions, equal to a Faraday, or a mole of electrons, such that $Q = N_Ae$, called Faraday's Law of Electrolysis, displaying discrete electric charges
	\2 Zeeman later discovered that discrete spectral lines emitted by an atom in a magnetic field separate into three spaced lines of different frequencies, caused by the slightly different charge to mass ratios
		\3 This proved that the particles producing the light were negative, and found the charge to mass ratio of the electrons
	\2 Thomsons cathode ray experiment later measured the same ratio as Zeeman, proving the existance of the electron, with the same ratio, as being the atomic negative component
		\3 This combined with Faraday's charge allowed the mass of the electron to be determined
		\3 He also used a uniform magnetic field creating a circular path to measure the same ratio, found to be the same for all materials, showing it was universal for atoms
	\2 Millikan attempted to use a cloud of water droplets with a charge, such that Q = Ne, using the mass of the cloud and the radius of the drop to find e, found to be difficult because of the evaporation
		\3 On the other hand, he found a single drop could be balanced in the air by an electric field, eventually picking up an ion causing a movement in some direction
		\3 This resulted in the oil drop experiment, giving each charge and preserving it in midair, measuring the force on it, to confirm the electron charge
\1 Absorbed radiation increases the kinetic energy of oscillating atoms, increasing the temperature, but resulting in increased radiation emission by electrons, reducing kinetic energy, called thermal radiation
	\2 At thermal equilibrium, the rate of absorption and emission are equal, such that higher freqencies are present at higher temperatures, due to higher energy
	\2 Ideal blackbodies emit and absorb all incident radiation by $R = P/A = \omega T^4$, where Stefan's constant ($\omega$) is $5.67 * 10^8 W/m^2K^4$
		\3 Non-blackbodies emit multiplied by some emissivity constant, based on factors such as color, temperature, and composition
	\2 Spectral distribution of the radiation of a blackbody also only depends on temperature, where the maximum emitted wavelength, $\lambda_{max} T = 2.898 * 10^{-3} m*K$, called Wien's displacement law
	\2 Blackbodies are approximated by a cavity with a small hole to let radiation in, found that the power radiated out, $R = \frac{1}{4}cu$, where U is the total radiation energy density in the cavity
		\3 As a result, both are proportional to the wavelength, such that the energy density distribution can be found by the number of modes of oscillation
		\3 It is found that the number of modes of standing wave oscillation per unit volume, $n = 8\pi\lambda^{-4}$, and the Rayleigh-Jeans equation states that $u = kTn$, such that R can be calculated
		\3 As a result, while at higher wavelengths, it fits experimentally, at low wavelengths, it appears $R \to \infty$, called the ultraviolet catastrophy, such that total energy density over the spectrum from 0 to $\infty$ would be infinite as well
	\2 Planck's Law corrects for this, stating that since as $\lambda$ approaches 0, n approaches infinity by classical formulas, u must be a function of wavelength, such that it approaches 0
		\3 Classically, electrons oscillating produce waves with equal frequency, where the average energy is found by the energy distribution function, $n(E) = Ae^{-E/kT}$ based on the Maxwell-Boltzmann distribution function, such that average energy, $\bar{E} = \int^\infty_0 Ef(E)dE = kT$
			\4 In this equation, n is the fraction of oscillators with energy E
			\4 This is based on the fact that for some standing wave, in space a, $\lambda = \frac{2a}{n}$ must be true for soem integer n, such that the number of modes, $\Delta n = \int^{f_2}_{f_1} \frac{4a}{c}df$, the additional multiple of 2 to account for two polarizations
			\4 In higher dimensions, $f = \frac{c}{2a}\sqrt{n_x^2 + n_y^2 + n_z^2}$, such that $N(f)df = 2(\frac{2a}{c})^3(\frac{1}{8}4\pi f^2df)$, when the space is modeled as a sphere of radius a, only in a single octant
			\4 The higher dimensional mode equation combined with Boltzmann distribution gives the Rayleigh-Jeans formula
		\3 Planck found it agreed with expermental data if oscillator energy is a multiple of a discrete value, such that $E = nhf$, where h is Planck's constant
			\4 The sum can be taken similarly as the sum of the probabilities for some particular frequency, to get the average energy, using geometric sums to simplify, such that $\bar{E} = \frac{hf}{e^{hf/kT} - 1}$
			\4 It follows that $u = \frac{8\pi hc\lambda^{-5}}{e^{hc/\lambda kT} - 1}$, called Planck's Law, found to be a generalization of all known laws
	\2 Blackbody radiation has been used as proof of the Big Bang Theory, due to the universe predicted to act as a perfect black body in terms of energy distribution
\1 In Hertz's spark gap experiment to generate EM waves and detect them, proving Maxwell's Theories, finding that light hitting a surface produced an electron current
	\2 Lenard later proved that it was electrons, and observed the current was proportional to the intensity (P = IA), but found that there was no minimum intensity needed as would be classically expected, due to requiring enough energy to
		\3 Fluxs of photons are the photons per second per unit area, related to intensity 
		\3 Since the kinetic energy had to be great enough to avoid being pulled back to the metal surface cathode if there was a negative voltage, it required a voltage produced greater than $-V_0$, called the stopping potential
		\3 Thus, it was found that $KE_{max} = eV_0$, such that $KE_{max}$ was independent of the light intensity as well, rather than increasing the electron kinetic energy
	\2 Einstein postulated as a result that Planck's quantization was universal, such that E = hf for all light quanta, such that $eV_0 = hf - \phi$, where $\phi$ is the work function, characteristic of the metal, to remove an electron
		\3 This is equal to the maximum kinetic energy by energy conservation, though some electrons lose energy when leaving the metal further
		\3 As a result, the threshold wavelength is equal to the work function divided by h
		\3 This also explains the lack of a time lag for the production of photoelectrons, instead of the calculated time for enough energy if it is spread evenly over the surface, as assumed classically
\1 Xrays originally were discovered by Roentgen with a cathode ray tube, noticing rays from the collision of electrons and the glass tube could activate flurescent photographic film and pass through opaque materials
	\2 He later observed no material was opaque, though less rays could pass through with higher densities
	\2 He stated that their apparent lack of magnetic deflection, refraction, or interference, was due to a very short wavelength, finding they were defracted by a crystal lattice, also proving a regular crystal array
	\2 He found they were produced by eletrons when deflected then stopped by the atoms of a target
	\2 He then thought to use Bragg places, or face-centered cubic molecular structures of NaCl crystals, analyzing the scattered waves from each atom to view xray diffraction
		\3 The waves are in phase regardless of wavelength if the scattering angle is equal to the incident angle for two waves hitting atoms in a plane
		\3 This condition is called the Bragg condition, true if $2dsin(\theta) = m\lambda$, where d is the distance between atoms and $\theta$ is the angle from the surface (half the scattering angle), for constructive interference, found in soap bubbles
		\3 The amount of ionization from xrays can then be measured to get the intensity of each wavelength, after it has been corrected by the Bragg condition to get the correct angle
	\2 This produces a spectra for the xrays produced, based on the anode of the material, able to be used to determine the atomic spacing
		\3 This produces both the characteristic spectrum of sharp lines and the continuous spectrum, the former which is specific to the material
			\4 Maxwell had previously predicted the continuous spectrum was due to the electron bombardment/deacceleration in the strong electric field (called bremsstrahlung), though the cause of the characteristic was unknown
		\3 There is also a cutoff wavelength, independent of the material, based on the energy of the bombarding electrons, by $\lambda = \frac{1.24 * 10^3 nm}{V}$, called the Duane-Hunt Rule
			\4 This is explained as the opposite of the photoelectric effect, such that all the kinetic energy is converted ($\lambda = \frac{hc}{eV}$)
\1 Compton later measured the scattering of xrays by free electrons, proving further both the photon and special relativity
	\2 Classical EM would have predicted a dipole oscillation due to the light field of the particle, with the maximum at the original wavelength, based on $1 + cos^2(\theta)$ for the angle of oscillation, rather than a shifted maximum
	\2 He observed that the scattered xrays were more easily absorbed, considering that the collision allowed an electron to absorb some of the photon energy, such that the wavelength became longer
	\2 As a result, he derived the Compton equation mechanically through relativity and quantization stating $\Delta \lambda = \frac{h}{mc}(1 - cos(\theta))$, where $\frac{h}{mc}$ is called the Compton wavelength of the electron
		\3 This was observed for xrays due to the percent change in wavelength only being noticable for very short original wavelengths
		\3 The unshifted, but scattered, portion is due to electrons tightly bound to the atom, such that the entire atom recoils
\end{outline*}

\section{Chapter 4 - The Nuclear Atom}
\begin{outline*}
\1 Newton's dispersion of white light was the first spectroscope, later shown to have hundreds of numbers of dark lines inside, and bright light spectra formed by flames, forming the study of spectroscopy
	\2 Continuous spectra are emitted by incandescent solids, showing no specific lines in any spectrscope
	\2 Band spectra are closely packed groups of lines apparently continuous in low resolution produced by fire, while line spectra are those produced by unbound chemical elements, both characteristic of material
\1 In 1885, Balmer found that visible and near-UV spectrum of H could be represented by $\lambda_n = 364.6 \frac{n^2}{n^2 - 4} nm$, where n = 3, 4, ..., called the Balmer series
	\2 The general form for other elements was later found as the Rydberg-Ritz formula, such that $\frac{1}{\lambda_{mn}} = R(\frac{1}{m^2} - \frac{1}{n^2}) \forall n > m$
		\3 R is the rydberg constant, varying slightly for elements, but negligably, such that $R_H = 1.096776 * 10^7 m^{-1}$ and $R_{\infty} = 1.097373 * 10^7 m^{-1}$
		\3 Thus, for the Balmer series, m = 2, such that the maximum wavelength is as n approaches infinity
	\2 The other series from higher energy to lower are Lyman, then Balmer, Paschen, Brackett, Pfund, and then Humphreys
\1 Thomson's plum pudding model was the most popular atomic model after Balmer and Rydberg-Ritz formulas, trying to find stable configurations with normal modes of vibration of spectral lines
	\2 On the other hand, the lack of continual emission due to electrostatic radiation and inability to make a model hindered this theory
\1 Rutherford, while studying radioactivity, finding that alpha particles were doubly ionized helium, checking the spectral lines to prove it, used the particles to study atomic interiors
	\2 He sent the radiation into gold foil, measuring the scattering angles, mainly undeflected or barely detected, though a few with right angles or more, showing the positively charged sphere could not have created such high angle scattering
		\3 Thomson's model did not have enough force at any point for a large enough deflection
	\2 Rather there must be a dense center, such that Rutherford calculated angular distribution and dependence on nuclear charge, angle, and kinetic energy, validated experimentally
	\2 This can be calculated mechanically, where b is the impact parameter, or the distance from the line through the nucleus, such that $b = \frac{kq_{\alpha}Q}{m_{\alpha}v^2}cot(\frac{\theta}{2})$
		\3 Thus, it is found that for intensity $I_0$ (in particles per second), the number scattered by one nucleus through angles greater than $\theta$ is the number with impact parameters less than $b(\theta)$, equal to $\pi b^2 I_0$
		\3 $\pi b^2$ is called the cross section $\omega$ for scattering angles greater than $\theta$, such that it is multiplied by intensity and number of nuclei for the number of scatterings above that value
			\4 The number of nuclei is calculated such that $n = \frac{\rho N_A}{M}$, where M is the molar mass
			\4 As a result, the fraction of scatters above some angle $\theta$ is $\pi b^2 nt$, where t is the thickness of the surface
	\2 Rutherford later derived an equation for the number of particles scattered at some angle, $\Delta N = (\frac{I_0A_{sc}ntkZe^2}{2r^2E_k})^2\frac{1}{sin^4(\frac{\theta}{2})}$, where $A_{sc}$ is the detector area and r is the distance from the foil to the detector
		\3 In this case, Z is the atomic number and t is the thickness
		\3 This was later verified experimentally, showing the legitimacy of atomic theory
	\2 Rutherford's model does not assume a point charge, but rather just a ball, and assumes the alpha particle does not penetrate it, such that for each angle, the closest distance from the nucleus can be calculated
		\3 Thus, for the largest angle ($180^o$), the collision is almost head on, providing an upper bound for the atomic radius
		\3 For some nucleus, by conservation of energy, $r = \frac{kq_{\alpha}Q}{\frac{1}{2}m_{\alpha}v^2}$, using the point when the number of collisions at some angle changes as the radius based on the initial kinetic energy
\1 Bohr later specified the charge and mass of electrons, stating they orbited by Coulomb force, but was unstable due to accelerating toward the center, with EM predicting that light of $f = \frac{v}{2\pi r}$, proportional to $\frac{1}{r^{\frac{3}{2}}}$
	\2 For the electrons, $E = \frac{1}{2}mv^2 - \frac{kZe^2}{r} = -\frac{kZe^2}{2r_n}$, proportional to $-\frac{1}{r}$, such that it will be losing energy, with decreasing radius
		\3 As the energy is lost due to the emission of radiation, emitting a continuous spectrum as the radius changes, the atom will decay, rendering it invalid
	\2 Bohr solved this by stating electrons has certain stationary states they could orbit without emission, and that the radiation emitted was due to the change in state, $hf = E_i - E_f$, called the Bohr frequency condition
		\3 He also made the correspondence principle, stating that the limit of large orbits and energies (or as quantum numbers reach higher, rather than $\hbar$ reaches 0, as was thought during the ultraviolet catastrophe), all quantum must align with classical
	\2 This allowed him to determine that angular momentum is quantied, assuming values of $\frac{nh}{2\pi}$, determining the spectra for any atom with -e charge
		\3 By the centripetal Coulomb force, $v = (\frac{kZe^2}{mr})^{1/2}$, with $L = mvr = n\frac{h}{2\pi} = n\hbar$, where n is a quantum number, such that $r_n = \frac{n^2 a_0}{Z}$, where $a_0$ is the Bohr electron radius
		\3 Thus, $E_n = -E_0\frac{Z^2}{n^2}$, where $E_0 = \frac{mk^2e^4}{2\hbar^2} = 13.6 eV$, such that the energy is quantized, with n = 1 as the ground state for the single electron
			\4 The energy of the ground state is the ionization/binding energy to fully remove the electron from the atom, with the differences between states able to be drawn by an energy level diagram
		\3 This is able to be converted to the Rydberg-Ritz equation, finding the Rydberg constant (R) as $\frac{E_0}{hc}$
	\2 The Rydberg constant assumes a stationary nucleus due to infinite mass, such that if it has mass M, the nucleus and the electron have total kinetic energy of $E = \frac{p^2}{2\mu}$ where $\mu = \frac{m}{1 + \frac{m}{M}}$
		\3 As a result, $\mu$ is the reduced electron mass, substituted for the electron mass to account for regular nuclei, such that $R = R_{\infty}\mu$
		\3 $\mu$ is taken from classical physics, assuming central forces, acting similarly on electrons as on astronomical bodies
	\2 The correspondence principle as a result implies that as energy levels are closely enough spaced, quantization should not matter, such that it is true as $n \to \infty$
		\3 As a result, the frequency of an single electron jump emission is equal to the frequency of the revolution of the electron
		\3 While it appears that multiple jumps would be greater, this is remedied by allowing elliptical orbits, such that the energy depends on the major axis, rather than the eccentricity
	\2 Relativistic corrections implied that a highly eccentric orbit would need a larger correction as the electron moves closer, approximated for n = 1 in Hydrogen to $\frac{v}{c} = \frac{ke^2}{\hbar c} = \alpha$, called the fine-structure constant
		\3 As a result, there are many ellipses for a single energy level, and since the velocity depends on the orbit, the energy radiated is slightly different, approximately ranging $\alpha^2$
		\3 $\alpha$ is also able to be used to simplify equations acting as a universal constant
	\2 Large atoms are called Rydberg atoms, as the electron energy approaches the ionization energy, the atomic radius is able to increase, used to study the correspondence principle and electromagnetic fields
\1 Xray spectra were found by Moseley to change from element to element regularly, due to being caused by inner electron transitions, depending only on the less complex lower orbits and not affected by interatomic binding forces
	\2 The Mosley plot is based on the fact that energy levels are proportional to $Z^2$, such that characteristic X-ray lines are found by $f^{1/2} = A_n(Z - b)$, where $A_n, b$ are constants for the line
		\3 The K series has b = 1 while the L series has b = 7.4
		\3 These occur when the bombarding electron cause the ionization of an inner electron, resulting in another electron falling down into the n = 1/K shell, and so forth
	\2 Bohr's equation can provide a similar relation, using $n_f = 1$ with Z = Z - 1 to find the frequency, finding $f = cR_{\infty}(Z - b)^2(\frac{1}{n_f^2} - \frac{1}{n^2})$
		\3 Thus, for the K-series, $A^2_n = cR_{\infty}(1 - \frac{1}{n^2})$
		\3 The (Z - b) factor is due to the shielding of nuclear charge by other electrons in the lower shells, allowing corrections to the periodic table from weight to Z, solving property discrepencies
	\2 The Auger effect is an alternative to x-ray emission, ejecting an extra electron from an outer shell, such that each element has a characteristic Auger electron kinetic energy spectrum
\1 The Franck-Hertz experiment used a cathode ray tube to send electrons to atoms of some material by voltage, after which there is a small voltage in the opposite direction, measuring the current that reaches it
	\2 It was noted that the current increased as the initial voltage did, until it suddenly dropped, repeating as a pattern
	\2 This was due to a forced elastic collision with the atom until the energy was high enough to excite an electron, which was found to correspond to light emitted on the spectra
	\2 This formed the basis of electron energy loss spectroscopy, measuring the locations of inelastic collisions to gain the energy structure of materials
\end{outline*}
\section{Chapter 5 - Wavelike Properties of Particles}
\begin{outline*}
\1 The De Broglie Hypothesis stated that the cause of the quantization of electrons was an effect of the standing wave condition for the wavelength of the electron, such that $E = \frac{hc}{\lambda} = pc$, such that nonrelativistically for electrically accelerated particles, $\lambda = \frac{hc}{(2mc^2eV_0)^2}$
	\2 This had not been previously noticed due to the extremely small wavelength of larger mass objects
	\2 It was determined that low energy electrons would have a low enough wavelength to detect diffraction in a Bragg plane, noticing the maxima and minima in the scattering by the Bragg condition
		\3 As the energy is varied, the diffraction maximum changes, such that **ANGLE VOLTAGE ADD**
	\2 The wave properties of neutral atoms and molecules were foun using natural thermal eneremy as motion, only using the top plane of the Bragg's planes due to the lack of energy, measuring the sscattering
\1 The De Broglie equation for relativistic particles is found by $E^2 + (pc)^2 + E_0^2$, where $E = K + E_0$, such that $\frac{\lambda}{\lambda_c} = \frac{1}{(2(\frac{K}{E_0}) + (\frac{K}{E_0}^2))^{1/2}}$, where the Compton wavelength, $\lambda_c = \frac{h}{mc}$
	\2 This is by the relations stating that $p\lambda = h$ and $hf = E$, where E is the total energy, rather than just kinetic or mass energy
\1 Particle waves do not have a material or ether, but rather the wave is the probability of finding the particle, represented by superposition of a group of harmonic waves to form a time-space localized wave packet
	\2 Wave packets have a group velocity ($v_g = \frac{d\omega}{dk}$), and as more waves with infinitessimally similar values of k are added, it becomes more localized
		\3 Phase velocities, $v_p = f\lambda = \frac{\omega}{k}$, are the velocities of the individual harmonic waves, such that $v_g = v_p + k\frac{dv_p}{dk}$
		\3 Mediums in which the phase velocity is the same for all frequencies is nondispersive, such that it is the same as the group velocity, in which the wave shape is constant
	\2 It is found classically that for wave packets, $\Delta k \Delta x \approx 1$ where $\Delta k$ is the range of k values of the harmonic waves, $\Delta x$ is the approximate length of the wave packet
		\3 It follows that $\Delta \omega \Delta t \approx 1$ as well, called the response time-bandwidth relation
		\3 Both are approximate due to depending on how the range is defined as well as the shape of the packets, serving as a magnitude and defining the approximate wave characteristics
		\3 The derivative of the equation relating k and $\lambda$ gives $\Delta k = \frac{2\pi \Delta \lambda}{\lambda^2}$, such that $\Delta x \Delta \lambda \approx \frac{\lambda^2}{2\pi}$
		\3 This represents the minimum uncertainty/error of the measurement, though human error in the measurement would increase the actual error
	\2 The particle wavefunction is denoted $\psi(x, t)$, such that for each harmonic wave, $v_p = \frac{E}{p} = \frac{v}{2}$, where v is the velocity of the particle itself
		\3 The group velocity, on the other hand, is equal to the velocity of the particle itself, acting as a nondispersive wave packet, found nonrelativistically by $v_g = \frac{d\omega}{dk} = \frac{dE}{dp} = v$
	\2 Wave behavior of photon particles are seen in low intensity diffraction, where it is the probability of hitting some location, where $E = hf$ per photon and $E_{tot}$ is proportional to $\vec{E}^2$
		\3 The electric field, $\vec{E}$, is the wavefunction for light, cooresponding to Schrodinger's equation for electrons
		\3 For matter waves, $P(x)dx = |\psi|^2dx$, where P is the probability at that location, acting as complex wavefunctions
\1 Since the wavefunction is nonzero for multiple values, there is uncertainty to the exact position of the electron, such that by the classical uncertainty relations, $\Delta x \Delta p \approx \Delta E \Delta t \approx \hbar$, called the Uncertainty Principle
	\2 For Gaussian distribution functions (total probability of 1), $\sigma_x \sigma_k = \frac{1}{2}$, where $\sigma$ is the standard deviation, such that uncertainty is defined as the standard deviation, such that $\Delta x \Delta p \geq \frac{\hbar}{2}$
	\2 As a result, particles cannot have 0 average kinetic energy, since for some box of length L, $\Delta x \leq L$, such that $\Delta p \geq \frac{\hbar}{L}$, ignoring the Gaussian $\frac{1}{2}$ term due to generally not being Gaussian and not affecting the order of magnitude
		\3 As a result, $(\Delta p)^2 = (p - \bar{p})^2_{av} = \bar{p^2} \geq (\frac{\hbar}{L})^2$, since $\bar{p}$ is equal to 0 for a symmetric box
		\3 Thus, $\bar{E} \geq \frac{\hbar^2}{2mL^2}$, called the zero point energy of the box, since the average energy must be $\geq 0$
	\2 For an electron with momentum p and distance r from the proton, $E = \frac{p^2}{2m} - \frac{ke^2}{r}$, $\Delta x = r$ such that $(\Delta p)^2 = \bar{p^2} \geq \frac{\hbar^2}{r^2}$, providing a radius for the minimum energy and a minimum electron energy at that radius
	\2 Since a precise measurement of the energy of a system requires infinite time, such that the mean decay time, or the lifetime ($\tau$), used as the time of measurement
		\3 As a result, the measurement of the wavelength of spectral lines has a degree of uncertainty due to the uncertain in measurement, with the natural line width being $\Delta E = \frac{\hbar}{\tau}$, denoted $\Gamma_0$
		\3 By the formula that $E - E_0 = \frac{hc}{\lambda}$, the derivative is taken such that $dE = -hc\frac{d\lambda}{\lambda^2}$, the range of $\lambda$ can be found for $\Delta E$
\1 It is seen that wave-particle duality is seen in matter and light, such that emission and absorption are particle-based, propogation is wave-based
	\2 This is also able to be thought of as interactions and observations of matter and light are particle-based, while predictions of observations are described by waves
		\3 Interactions cause the changing of the wavefunction, changing the propogation after
	\2 For wavelengths smaller than any objects, particle theory can describe propogation as well as wave theory, giving the same results, due to wavelike behavior being too small to be observed, while for time averages of energy/momentum exchange, wave theory works as well
\end{outline*}
\section{Chapter 6 - The Schrodinger Equation}
\begin{outline*}
\1 The Schrodinger Equation is a fundamental law, unable to be derived, and is non-relativistic, replaced by Dirac's Relativistic Wave Equation later
	\2 This wave equation is similar to that of light, $\frac{\partial^2\vec{E}}{\partial x^2} = \frac{1}{c^2}\frac{\partial^2\vec{E}}{\partial t^2}$, with solution $\vec{E} = \vec{E}_0 cos(kx - \omega t)$ to give the equation, $\omega = kc$, which is equivelent to $E = pc$
		\3 Similarly for electrons, since $E = \frac{p^2}{2m} + V$ is the energy of the electron where V is the potential energy of the electron, $\hbar \omega = \frac{\hbar^2 k^2}{2m} + V$, such that k and $\omega$ are not linearly related
		\3 This implies that for a harmonic electron wavefunction, the first time derivative is related to the second spacial derivative, and will involve potential energy
		\3 The wave equation must be consistent with the De Broglie relations, classical conservation of energy, squared as the probability of finding the particle at that location, and if potential is constant, energy and momentum is constant
			\4 As a result, an ideal form for V = 0 would be able to be reduced to $E = \frac{p^2}{2m}$, such that cos/sin fail, but a complex exponential works
		\3 The wave equation must also be linear with respect to the wavefunction, due to allowing constructive and destructive interference, such that each term is linear with respect to $\phi$ or some derivative of $\phi$
	\2 The equation in 1D states that $\frac{-\hbar^2}{2m} \frac{\partial^2\psi(x, t)}{\partial x^2} + V(x, t)\psi(x, t) = i\hbar \frac{\partial \psi(x, t)}{\partial t}$
		\3 For a free particle, such that $V(x, t) = V_0$ is constant, it is found that individual harmonics are not solutions, but a complex exponential is, such that $\psi = Ae^{i(kx - \omega t)}$ provides the previous equation
	\2 By the probabilistic interpretation of the wavefunction, $P(x, t)dx = |\psi(x, t)|^2dx = \psi^*(x, t)\psi(x, t)dx$, where $\psi^*(x, t)$ is the conjugate, such that P(x, t) is called the probability density, and $\psi(x, t)$ is called the probability density amplitude/probability amplitude
		\3 Thus, the normalization condition states that $\int^{\infty}_{-\infty} \psi^* \psi dx = 1$
	\2 Schrodinger then showed that energy quantization can be explained in terms of standing waves, called the stationary states, or eigenstates, of the particles, in which the potential energy is independent of time
		\3 For these situations, the solution is seperable, such that $\psi(x, t) = \gamma(x)\phi(t)$, producing ordinary derivatives instead of partial, with each side as a function of a single variable
		\3 Since they are single variable, they both must be equal to some separation constant, such that $\phi(t) = e^{\frac{-iCt}{\hbar}} = cos(2\pi \frac{Ct}{h}) - isin(2\pi \frac{Ct}{h})$, such that $f = \frac{C}{h}$, or C = E
			\4 With E in the spacial portion of the equation, it gives the time-independent Schrodinger equation, with the time dependent side replaced by $E\psi(x)$ whose probability density is found to be equal for this situation to the multivariable density
	\2 Wavefunctions must fit with the type of potential energy function, which is allowed to be discontinuous, solved seperately in each region, but require smooth joining at the point of discontinuity
		\3 The wavefunction itself must be continuous as well as the first derivative, such that the function is smooth, except possibly at the boundary (since if there is infinite potential energy, the wavefunction must be 0)
		\3 Both also must be finite and single-valued to have measurable quantities, and must follow the normalizataion condition
\1 The infinite square well is a frictionless wire with increasing potential from points near the end, until the maximum at the border of the box, able to be made arbitrarily steep and large potentials, such that $V(x) = 0 if 0 < x < L, V(x) = \infty$ otherwise
	\2 This problem is related to the classical vibrating string problem and is a good approximation to the motion of free electrons in a metal
	\2 As a result, the boundary conditions state that it must be 0 at both ends of the box, allowing quantization similar to the standing wave condition
		\3 For a standing wavelength of the particle, $E = n^2\frac{\pi^2\hbar^2}{2mL^2}$, able to be derived from the time-independent Schrodinger equation by providing $\psi(L) = Asin(kx)$, such that $k_n = \frac{n\pi}{L}$, providing quantized energy values/energy eigenvalues
			\4 n is the quantum number of the system, allowing the specific mode of the system to be determined
			\4 This also displays the minimum/zero-point energy, since n = 0 is only valid if the particle is not in the box
		\3 The normalization condition can then be used to get that $A_n = (\frac{2}{L})^{1/2}$, such that the eigenfunctions are $\psi_n(x) = \sqrt{\frac{2}{L}}sin(\frac{n\pi x}{L})$
		\3 This provides the probability of finding the particle at each location, depending on the total energy of the particle system
	\2 Classically, since there is no force except an infinitely large force at the edges, any speed and energy is possible, but by the uncertainty principle the velocity and position can't be found simultaneously, the minimum energy is below the energy uncertainty
		\3 In addition, as quantum number increases, the peaks approach infinite, similar to classical prediction
	\2 The complete wavefunction is found as $\psi_n(x, t) = \psi_n(x)e^{-i\omega t} = \psi_n(x)e^{-i(E_n/\hbar)t}$, such that by the identity, $sin(k_n x) = \frac{(e^{ik_nx} - e^{-ik_nx})}{2i}$, multiplied by the time portion to show that the standing wave is equal to two equal and opposite travelling waves
\1 For a finite square well, such that the bounds of the region have potential $V_0$, where the total energy is assumed to be less than $V_0$, such that within the well, the equation is the same as the infinite well, $\psi''(x) = -k^2\psi(x), k^2 = \frac{2mE}{\hbar^2}$, such that it is a sine/cosine within the well
	\2 Outside the well, it requires the complete time-independent equation, but is not required to be 0, rather only needing to follow the normalization condition
	\2 Since the second derivative/curvature and the derivative have the same sign, the function always curves away from the x-axis, rather than towards like a sine/cosine
		\3 This will appear in the form, $\psi(x) = ce^{\pm\alpha x} = ce^{\pm\sqrt{\frac{2m}{\hbar^2}(V_0 - E)}x}$
		\3 As a result, for most energy values of the wavefunction, it goes towards $\pm \infty$ as x approaches $\pm \infty$, such that it is invalid, such that it must move towards 0 as $x \to \pm \infty$, restricting energy states
		\3 It is found that there is a finite number of possible energies, decreasing in quantity as the voltage decreases, to 1 possible state at very low $V_0$
	\2 Since it is possible for the particle to be outside the well, kinetic energy appears to be negative, but if the probability is assumed to be negligable at $\Delta x = \alpha^{-1}$, the momentum uncertainty gives a minimum energy equal to the negative kinetic energy to cancel it
	\2 This applies to all functions with $E > V$ in the box, even if V varies such that it is not simple sinusoidal, with $E < V$ outside the box, restricting energy states
\1 The expectation value of f(x), $<f(x)> = \int^{\infty}_{-\infty} \psi^*f(x)\psi dx$, is the average value of f(x) expected to be obtained by the measurement of particles with the wavefunction (either the time-independent or time-dependent)
	\2 Operators are the set of functions replacing f(x), acting on the wavefunction to provide the weighted average, not always functions of x such as when the uncertainty principle prevents it, like momentum
		\3 The momentum x-component operator is $\frac{\hbar}{i}\frac{\partial}{\partial x}$, squared for the $p^2$ operator
		\3 The Hamiltonian/total energy operator in classical mechanics is $H = \frac{p^2}{2m} + V$, such that the Hamiltonian operator is found by replacing p by the p operator
			\4 As a result, the time independent Schrodinger equation can be replaced by $H_{op}\psi = E\psi$
		\3 The time dependent Hamiltonian operator is $i\hbar \frac{\partial}{\partial t}$, kinetic energy is $-\frac{\hbar^2}{2m}\frac{\partial^2}{\partial x^2}$, and the z component of angular momentum is $-i\hbar \frac{\partial}{\partial \phi}$
\1 The simple harmonic oscillator, such that $V(x) = \frac{1}{2}kx^2 = \frac{1}{2}m\omega^2x^2$, where k is the force constant, acting as the finite well, has a classical probability of $P(x)dx = c\frac{dx}{v}$, with any energy value possible
	\2 Due to the symmetic potential function, the probability must be symmetric, such that the wave function must be either symmetric or anti-symmetric
		\3 Parity operations are those flipping the x-axis, such that $x \to -x$, true for the Hamiltonian operator such that $H_op\psi(-x) = E\psi(-x)$
		\3 If there are multiple separate solutions for one energy state, it is called degenerate, doubly degenerate for two, though since it is found for the Schrodinger equation to be at most by a factor of -1 for odd parity ($\psi(-x) = -\psi(x)$)
	\2 It is found that $E_n = (n + \frac{1}{2})\hbar\omega$, such that the minimum energy is $E_0 = \frac{1}{2}\hbar\omega$, due to the minimum uncertainty principle energy
		\3 Thus, it is found that $\psi_n(x) = C_n e^{-m\omega x^2/2\hbar}H_n(x)$, where $H_n(x)$ is the Hermite polynomial of order n, symmetric for even n, antisymmetric for odd n, such that it decays outside the classically allowed region, oscillating within the region
		\3 The approximate length uncertainty can be thought of as the width under which the exponential term is greater than 1
	\2 It is also found that $\int^{\infty}_{-\infty} \psi^*_nx\psi_mdx = 0$ unless $n = m \pm 1$, called the selection rule, such that radiation can only change the energy of the oscillator by one at a time
\1 For an unbound state problem, such that E is greater than V(x) as $x \to \pm \infty$, the second derivative and wavefunction must have opposite signs to avoid moving towards infinity, curving to the x axis, such that any value of E is possible
	\2 While it is not normalizable over the entire domain, it is bounded within a specific range, such that $\int^b_a |\psi(x)|^2dx = \int^b_a \rho dx = \int^b_a dN = N$, where N is the number of particles in the interval
	\2 For some step potential, such that for $x < 0, V(x) = 0$ and for $x > 0, V(x) = V_0$, such that by the Schrodinger equation, for $x < 0, \frac{d^2\psi(x)}{dx^2} = -\frac{\sqrt{2mE}}{\hbar}$, and $x > 0, \frac{d^2\psi(x)}{dx^2} = -\frac{\sqrt{2m(E - V_0)}}{\hbar}$
		\3 For solutions of beams of particles moving to the right multiplied by the time portion, each regions solution is the sum of the travelling waves in each direction
		\3 There is assumed to be no leftward moving beam from the $x > 0$ side, such that the coefficient of that term is 0, with the functions for each side required to be continuous at x = 0
		\3 The coefficients of reflection R and transmission T are the relative rates by which particles are refleceted and transmitted, with $R = (\frac{k_1 - k_2}{k_1 + k_2})^2$, $T = \frac{4k_1^2}{(k_1 + k_2)^2}$, such that $T + R = 1$
	\2 As a result of Schrodinger's wave nature, unlike the classical idea that none would be reflected due to the change in potential, a portion of the particles are, depending on the change in wavenumbers, but not in the sign of the change
	\2 For $V_0 > E$, the equation is a real negative exponential in the potential region, such that the particle is able to somewhat permeate, appearing as if it has negative velocity by the uncertainty principles
	\2 For a barrier in which $V_0 > E$, as a result, the particle has the possibility of reaching all the way through the barrier, such that it is possible for it to tunnel through the higher potential then proceed through
		\3 It is found that the percent transmitted, $T = 16\frac{E}{V_0}(1 - \frac{E}{V_0})e^{-2\frac{\sqrt{2m(V_0 - E)}}{\hbar}L} \approx e^{-2\frac{\sqrt{2m(V_0 - E)}}{\hbar}L}$, where L is the length of the tunneling region
		\3 As a result, the probability of a standard barrier is the integral of the probability of transmission for each x value 
		\3 This is found to be the cause of the likelihood of spontaneous alpha decay, such that the force increases linearly as the particle gets further out of the nucleus, decreasing by $\frac{1}{r^2}$ outside
		\3 Tunnelling microscopes view metals as an infinite well, moving the servo close to the material, sending a voltage through the air to measure the light emitted from the tunnelling based on the height, to measure the surface layout and height
			\4 This is done with the current proportional to the transmission probability, keeping the tip-surface distance constant for each scan
	\2 Flux of particles, $J = \rho v = \frac{p}{m} \rho = \frac{\hbar k}{m} \rho$, such that for initial density of a single particle, $J_I = v, J_{R} = vR$, and $J_T = vT$
\end{outline*}
\section{Chapter 7 - Atomic Physics}
\begin{outline*}
\1 The time-independent Schrodinger Equation is expanded to 3 dimensions, such that $\frac{-\hbar^2}{2m}(\frac{\partial^2 \psi}{\partial x^2} + \frac{\partial^2 \psi}{\partial y^2} + \frac{\partial^2 \psi}{\partial z^2}) + V\psi = E\psi$
	\2 Since the standing wave bounds in one dimension are independent of the other dimensions, it is broken down to a product function, $\psi(x, y, z) = \psi_x(x)\psi_y(y)\psi_z(z)$
		\3 Thus, it is found that $E = \frac{\hbar^2}{2m}(k_1^2 + k_2^2 + k_3^2 = \frac{\hbar^2\pi^2}{2m}(\frac{n_1^2}{L_1^2} + \frac{n_2^2}{L_2^2} + \frac{n_3^2}{L_3^2})$, such that it has 3 quantum numbers, each from a different boundary
	\2 Thus, for a symmetrical cube, three energy state combinations are found for each energy state, such that it is triple-degenerate, split apart otherwise
	\2 For spherical coordinates, such that it simulates a Hydrogen atom, using a reduced mass, $\mu$, to compensate for a stationary center, the equation is found to be $\frac{-\hbar^2}{2\mu r^2}(\frac{\partial}{\partial r}(r^2 \frac{\partial \psi}{\partial r}) + \frac{1}{sin(\theta)}\frac{\partial}{\partial \theta}(sin(\theta)\frac{\partial \psi}{\partial \theta}) + \frac{1}{sin^2(\theta)}\frac{\partial^2 \psi}{\partial \phi^2}) + V(r)\psi = E\psi$
		\3 Within this equation, $\theta$ is along the z-axis, restricted to $\pi$, while $\phi$ is along the x-axis, restricted to $2\pi$
\1 For a hydrogen atom, the equation can be separated ($\psi = R(r)f(\theta)g(\phi)$) into the radial equation and the angular equation, $\frac{1}{R}\frac{d}{dr}(r^2\frac{dR}{dr}) + \frac{2\mu r^2}{\hbar^2}(E - V(r)) = -(\frac{1}{fsin(\theta)}\frac{d}{d\theta}(sin(\theta)\frac{df}{d\theta}) + \frac{1}{gsin^2(\theta)}\frac{d^2g}{d\phi^2}$ = l(l + 1), due to being separable
	\2 The angular equation can then be separated further, with the constant as $-m^2$, such that $g_m(\phi) = e^{im\phi}$, periodic, such that m must be integral, called the azimuthal function
	\2 Further, $f_lm(\theta) = \frac{(sin(\theta))^{|m|}}{2^ll!}(\frac{d}{dcos(\theta)})^{l + |m|}(cos^2(\theta) - 1)^l$, such that to be finite-valued/periodic, $|m| \leq l$
		\3 f is called the associated Legendre functions, while for m = 0, they are callec the Legendre polynomials, where the product of f and g is called the spherical harmonics
	\2 The angular momentum, $L = r x p$, can be found by separating momentum into radial and tangential, such that $L = rp_t$, where $p_r = \mu(\frac{dr}{dt}), p_t = \mu r(\frac{dA}{dt})$
		\3 Thus, kinetic energy can be shown to be $\frac{p^2}{2\mu} = \frac{p_r^2}{2\mu} + \frac{L^2}{2\mu r^2}$, such that $V_{eff} = \frac{L^2}{2\mu r^2} + V(r)$, such that $E = \frac{p_r^2}{2\mu} + V_{eff}(r)$, acting as an energy conservation formula, able to be used to form the Schrodinger equation
		\3 As a result, it is derived that $(p_r^2)_{op} = -\hbar^2 \frac{1}{r^2}\frac{\partial}{\partial r}(r^2 \frac{\partial}{\partial r})$, and $(L^2)_{op} = -\hbar^2(\frac{1}{sin(\theta)}\frac{\partial}{\partial \theta}(sin(\theta)\frac{\partial}{\partial \theta}) + \frac{1}{sin^2\theta}\frac{\partial^2}{\partial \phi^2})$
			\4 This is noted to be similar to the angular side of the separated Schrodinger equation, such that when operating on fg, it is equal to $\hbar^2l(l + 1)fg$, such that $L^2$ is quantized, where l is the angular momentum/orbital quantum number
			\4 By the same reasoning, it is found that $L_z = m\hbar$, where m is integral equal to or less than $\pm l$, such that L is space quantized, pointing in specific spacial directions, with m as the magnetic quantum number
		\3 The other dimensional operators can be derived similarly, but cannot be found definitively due to relying on both angles, which the uncertainty principle forbids
			\4 In addition, the uncertainty principle also forbids knowing more than one component of the angular momentum simultaneously
	\2 For the potential energy function, $V(r) = -\frac{Zke^2}{r}$, the radial equation is solved to find that it is well-defined only if the energy levels fit the Bohr energy levels, with n as the principle quantum number
		\3 The radial function as a result is found to be $R_{nl}(r) = A_{nl}e^{-\frac{Zr}{a_0n}}r^lL_{nl}(\frac{Zr}{a_0})$, where $L_{nl}(\frac{r}{a_0})$ are the Laguerre polynomials, and $a_0$ is the Bohr radius, $\frac{\hbar^2}{ke^2\mu}$
		\3 The energy is found as well to be $E = -\frac{1}{2}m_ec^2Z^2\alpha^2\frac{1}{n^2}$, equal to the Bohr energy levels, with $\alpha$ as the fine structure constants
		\3 Since the radial equation itself is $\frac{-\hbar^2}{2m_e}(\frac{d^2}{dr^2} + \frac{2}{r}\frac{d}{dr} - \frac{l(l+1)}{r^2})R(r) - \frac{Ze^2}{4\pi \epsilon_0 r}R(r) = ER(r)$, such that the $\frac{l(l + 1)}{r^2}$ doesn't allow the particle to be at r = 0 if $l \neq 0$ (angular momentum is nonzero)
			\4 This is called the centrifugal barrier, preventing the atom from decaying into the origin
	\2 As a result, the Schrodinger model predicted the total orbital angular momentum of the energy levels differently from the Bohr model, but agreed with Bohr on the quantization of the z-component of angular momentum
		\3 Bohr also did not predict degeneracy of energy states, the uncertainty principle, and did not predict the 3D wavefunction
	\2 The energy depends only on n due to the inverse square force causing it to only depend on the major axis, not eccentricity, and since higher l provides lower eccentricity, and since m provides the z-component, which has no preferred direction
		\3 The l states for each n are described by the letters S (l = 0), P, D, F, G, continuing alphabetically, while the n states are called K, L, M, ... shells
		\3 For some absorbed photon, the selection rules state that m can change by 0, 1, or -1, while l must change by either 1 or -1 to conserve angular momentum, due to photons having intrinsic angular momentum of $\pm \hbar$
		\3 As a result, there is a series of degenerate states ($0 \leq l \leq n -1, -l \leq m \leq l$) for each possible energy level
	\2 For the ground state, l and m are 0 and $L_{10} = 1$, such that $\psi_{100} = C_100e^{\frac{-Zr}{a_0}}$, such that since $dV = r^2sin\theta d\phi d\theta dr$, $C_100 = \frac{1}{\sqrt{\pi}}(\frac{Z}{a_0})^{3/2}$
		\3 Thus, the electron is most likely to be found at the origin, but the radius at which it is most likely to be found is the Bohr electron radius $a_0$, though it can be found at any radius
		\3 Electrons tend to be thought of as a result as a charged cloud of charge density, equal to the probability multiplied by the charge
		\3 It is also noted that the angular momentum of the ground state is 0, unlike the Bohr model which assumed the minimum angular momentum was $\hbar$
		\3 The set of n and l quantum numbers is called the electron configuration
	\2 For the excited state n = 2, $\psi_{200} = C_{200}(2 - \frac{Zr}{a_0})e^{\frac{-Zr}{2a_0}}, \psi_{210} = C_{210}\frac{Zr}{a_0}e^{\frac{-Zr}{2a_0}}cos(\theta)$, and $\psi_{20\pm 1} = C_{21 \pm 1}\frac{Zr}{a_0}e^{\frac{-Zr}{2a_0}}sin(\theta)e^{\pm i\phi}$
		\3 The maximum for l = 1 is the Bohr orbit, $r = 4a_0$, while for l = 0, it is slightly higher, but still close
		\3 It is found that for some n, the wavefunction is greatest near the origin when l is small, such that it gets more likely to be near the originn
		\3 It is also noted that for l = 0, the orbital density is spherically symmetric, while otherwise, it depends on the angle
\1 Electron spin/intrinsic angular momentum is used to explain the fine structure of the spectral lines (such that there are several infinitessimally seperate spectral lines), created analogous to Schrodinger's angular momentum ($|\vec{S}| = \sqrt{s(s + 1)}\hbar$)
	\2 On the other hand, Pauli suggested that there could only be two magnetic spin numbers, $\pm \frac{1}{2}$, such that s must be restricted to $\frac{1}{2}$
	\2 Magnetic moment is defined by $\vec{\mu} = \vec{g_LI}A = \frac{g_Levr}{2} = \frac{-g_Le\vec{L}}{2m_e} = g_L\frac{q}{2M}\vec{L} = g_L\sqrt{l(l + 1)}\mu_B$, where $\mu_B$ is the Bohr magneton, equal to $\frac{e\hbar}{2m_e} = 9.27 * 10^{-24} J/T$
		\3 In addition, $\mu_z = -g_Lm\mu_B$, with $g_L$ as the gyromagnetic ratio/g-factor, taking into account the complex situation, 1 for a single revolving electron
		\3 This shows that the quantiztation of angular momentum creates the quantization of magnetic moments
		\3 The Stern-Gerlach Experiment measured the quantization of the z-component, placing am atomic beam in a magnetic field, checking the collision pattern ($F_z = \frac{\partial B_z}{\partial z}\mu_z$)
	\2 The potential energy for a megnetic moment is found to be $U = -\vec{\mu} \cdot \vec{B}$, such that for B in the z-axis, $U = -\mu_zB$, causing the slight energy differences for the possible $\mu_z$ values creating the fine structure
	\2 It is found that the g-factor for an electron is approximately 2, rather than the 1 expected for an electron revolving normally, showing the distinction $(\mu_z = -\mu_B(m_l + 2m_s))$
	\2 Thus, the full form of the wave-equation is $\psi = C_1\psi_{100\frac{1}{2}} + C_2\psi_{100\frac{-1}{2}}$, with $C_1^2 = C_2^2 = \frac{1}{2}$ for an atom not in a magnetic field
		\3 As a result, $\psi_{100\frac{1}{2}} = \psi_{100}\chi_{\frac{1}{2}}$, with the quantized spin and spin component as eigenvalues of $\chi$, with the general Hamiltonian as $H = \frac{p_1^2}{2m_1} + \frac{p_2^2}{2m_2} + ... + V(\vec{r_1}, \vec{r_2}, ...) + f(\vec{r_1}, \vec{r_2}, ..., \chi_{1}, \chi{2}, ...)$
		\3 For noninteracting, time-independent particles, f = 0 and V is the sum of the individual potentials, $V_n(\vec{r_n}, \chi_n)$, such that the total energy is the sum of the individual energiees
\1 While classically, the sum of the revolving and intrinsic angular momentum can be any value based on the vector angles, the quantization limits it, such that $|\vec{J}| = |\vec{L} + \vec{S}| = \sqrt{j(j + 1)}\hbar$
	\2 j is allowed to be either the sum of l and s (parallel) or the positive difference of them (antiparallel), or integral moves between the two, with the z component given by $J_z = m_j\hbar = (m_l + m_s)\hbar$, where $-j \leq m_j \leq j$ as a result
		\3 Spectroscopic notation is denoted by $n^{2s + 1}L_j$, where L is the angular momentum state of the electon (S, P, D, F, G, ...)
		\3 As a result, the fine structure splitting can also be explained by different values of j causing different potential energy, such that antiparallel J has lower energy than parallel
		\3 In addition, within an external magnetic field, the energy is slightly different for each $m_j$, splitting the lines further, called the Zeeman effect
	\2 Similarly, the sum of the total angular momentum of two atoms has a value of j with a maximum of the sum, the minimum of the positive difference
	\2 Since the fine-structure depends only on n and j, rather than on l, different l states have the same energy, and since there is only a single l value for n = 1, S, $2^2S_{1/2}$ is metastable, due to the selection rule requiring a change in the l value
		\3 On the other hand, it was discovered that $2^2S_{1/2}$ had slightly more energy than $2^2P_{1/2}$, allowing the Lamb shift transition with photon emission
\1 The Schrodinger Equation for multiple particles, assumed to be electrons, written time-independently as $-\frac{\hbar^2}{2m}\frac{\partial^2\psi(x_1, x_2)}{\partial x_1^2} -\frac{\hbar^2}{2m}\frac{\partial^2\psi(x_1, x_2)}{\partial x_2^2} + V\psi(x_1, x_2) = E\psi(x_1, x_2)$
	\2 Potential must take into account both the independent potential of each and the interacting potential, both assumed to be 0 for the infinite square well
	\2 The solution is written as a product solution, such that for states m and n respectively, $\psi_{mn}(x_1, x_2) = \psi_m(x_1)\psi_n(x_2)$
		\3 The probability of finding particle 1 at $dx_1$, particle 2 at $dx_2$ is $|\psi(x_1, x_2)|^2dx_1dx_2 = (|\psi_m(x_1)|^2dx_1)(|\psi_n(x_2)|^2dx_2)$
		\3 For identical particles as a result, the interchanging of the particles would result in the same overall wave probability ($|\psi(x_1, x_2)|^2 = |\psi(x_2, x_1)|^2$), considered a symmetric wavefunction if they are equal if reversed, antisymmetric if negative
	\2 Since identical particles are assumed to be indistinguishable within quantum mechanics, the wavefunctions must be the linear combination of both with equal probability, such that $\psi = C(\psi_m(x_1)\psi_n(x_2) \pm \psi_m(x_2)\psi_n(x_1))$, depending if symmetric or not
	\2 It is found that electrons can only have antisymmetric wavefunctions, such that if all wavenumbers are the same, then the wavefunction at any point is 0, forbidden by Pauli's exclusion principle
\1 For a Helium atom, potential energy is the sum of the potential with the nucleus and $V_{int} = \frac{ke^2}{|\vec{r_2} + \vec{r_1}|}$, preventing the separation of equations for each electron
	\2 If ignoring the interaction term, the energy of each of the electrons is individual, with total energy as the sum of the individual electron energy states, the wavefunction as the product of the standard electron wavefunctions
		\3 First-order perturbation theory is then done to correct for the interactions, calculating the expectation value of the interaction potential, adding it to the ground state energy to get the real energy
	\2 The first ionization potential is the difference between the ground state energy with two electrons and a single election, rather than the electrostatic potential of a single electron
\1 For a lithium ion, the energy of the third electron depends on the l state, due to the force not being an inverse square force, but rather depending on the negative K-shell charge, spread over the volume at the radius $\frac{a_0}{Z}$
	\2 The electron acts as an electron in either the 2S or 2P state of a hydrogen atom, due to the electrons in the K shell shielding the nucleus, such that $Z_{eff} = 1$
		\3 On the other hand, in the 2S state, it is more likely to be near the center, below the shielding, such that the energy state is lower, true generally, such that for the electron in 2S, $Z_{eff} = 1.3$
	\2 For beryllium, since the effective force within the Z-shell is greater, the energy is lower than that of the third electron in lithium
	\2 The electron of boron has higher energy than beryllium, due to the lack of penetration of the 2S subshell, with lower energy for subsequent electrons due to higher nucleus charge, though with neon still having higher energy than helium 
	\2 Thus, for 3S electrons, the energy is drastically higher due to the shielding of the L shell, proceeding lower as subsequent electrons are added into 3S then 3P
	\2 The shielding by the M shell is strong enough as well, that there is less energy for 4S than for 3D, such that the former is filled first, before filling 3D in the transition elements
\1 Visible light excited transitions typically are only able to take place in the outer electron shells, the excitation energy being approimately that of the electron in hydrogen, due to shielding
	\2 The different states have spin-orbit energy difference, such that there is slightly more energy in higher angular momentum combinations
	\2 By the selection rule ($\Delta l = \pm 1, \Delta j = -1, 0, 1$), transitions to and from the S state have double lines due to the energy difference from the doublet P state (two possible initial l values), while other transitions have three possible lines (called a compound double)
		\3 Two of the lines are far closer than the other, due to the distance between the upper and lower P levels being further apart
\end{outline*}
\section{Chapter 8 - Statistical Mechanics}
\begin{outline*}
\1 Classical statistical physics contains a large number of identical, but distinguishable particles, able to be tracked, with the energies of each given by the Boltzmann distribution function, $f_B(E) = Ae^{-E/kt}$
	\2 The function without A is called the Boltzmann factor, while k is the Boltzmann constant, $1.381 * 10^{-21} J/K$, with A as the normalization constant
	\2 $n(E) = g(E)f_B(E)$ is the number of particles, where g is the statistical weight/degeneracy of the energy state/density of states, with E as a continuous variable
	\2 Maxwell's Distribution of Molecular Speeds assumes the velocity in each direction is distinct, with $F(v_x, v_y, v_z) = f(v_x)f(v_y)f(v_z)$, where $f(v_x) = Ce^{-\frac{mv_x^2}{2kT}}$, such that $n(v)dv = 4\pi v^2 N f(v)dv$, such that $4\pi v^2$ is the surface area of the sphere of velocities, acting as the degeneracy
		\3 It is found that the normalization constant for the three dimensional equation is $(\frac{m}{2\pi k T})^{3/2}$
		\3 As a result, $v_{max}(T) = (\frac{2kT}{m})^{1/2}$ and $<v> = \frac{1}{N}\int^{\infty}_{0} v fnv) dv$
		\3 Similarly, by the relationship between velocity and kinetic energy, $n(E)dE = \frac{2\pi N}{(\pi k T)^{3/2}}E^{1/2}e^{-E/kT}$, such that $<E> = \frac{3}{2}kT$
	\2 The Equipartition Theorem states that in equilibrium, each degree of freedom contributes $\frac{1}{2}kT$ to the average energy per molecule, where a degree of freedom is a coordinate/velocity component that is squared in the expression of total energy
		\3 As a result, $C_V = \frac{1}{2}N_Ak = \frac{1}{2}R$ for each degree of freedom
		\3 Thus, rotational kinetic by axis, translational by axis, and both kinetic and potential vibrational by axis are degrees of freedom
			\4 Thus, for a monotomic atom, assuming no vibration, it should be 3R, but it rather found rise as the temperature rises, starting with purely translation, gaining rotation only in the two non-atomic axes, and increasing vibrationally
			\4 Most diatomic molecules then decay before it is able to reach $\frac{7}{2}R$, but the theorem still doesn't explain the increase with temperature
		\3 Solid heat capacity of approximately 3R was explained by three axes of vibrational potential and three axes of vibrational kinetic, true at high temperatures, but dropping as temperature drops
			\4 This is not different for metals, rather than the expected three translational kinetic energy degrees of freedom combined with the six vibrational degrees
\1 Quantum statistical mechanics attempted to correct classical by the fact that particles were indistinguishable and identical, providing the Bose-Einstein and Fermi-Dirac distribution functions for bosons and fermions respectively
	\2 Bosons, or particles with an integral spin, do not obey the exclusion principle, distributed by $f(E) = \frac{1}{e^{\alpha}e^{E/kt} - 1}$, where $e^{\alpha}$ is the normalization constant
		\3 Fermions are those with a half integer spin that obey the exclusion principle, given by $f(E) = \frac{1}{e^{\alpha}e^{E/kt} + 1}$
	\2 These are similar to the Boltzmann distribution, differing only by the $\pm 1$ in the denominator, approaching the Boltzmann distribution as the energy increases or as $\alpha >> \frac{E}{kT}$
		\3 Since bosons must have symmetric wavefunctions and fermions, asymmetric, $\psi = \frac{1}{\sqrt{2}}(\psi_n(1)\psi_m(2) \pm \psi_n(2)\psi_m(1))$, the function of both of the same state is $\psi = \frac{\sqrt{2}}\psi_n(1)\psi_n(2)$
		\3 By the Boltzmann distribution, for distinguishable particles, the function doesn't have the $\sqrt{2}$ factor, due to not being the linear combination
		\3 Thus, it is found that the presence of a boson in a quantum state improves the probability others would by in the state from the classical probability, while fermions prevent other fermions from being found in the same state
	\2 The Boltzmann distribution is usable when the particles are distinguishable, such that the distance between them is far greater than the De Broglie Wavelength
		\3 Since, $\lambda = \frac{h}{2mE_k} = \frac{h}{3mkT}$, since $<E_k> = \frac{3kT}{2}$, and it is found that $<d> = (\frac{V}{N})^{1/3}$, such that this is valid when $\frac{N}{V}\frac{h^3}{(3mkT)^{3/2}} << 1$
		\3 Within an infinite well, $E_n = E_0(n_x^2 + n_y^2 + n_z^2)$, such that it forms an eighth of a sphere with radius $(\frac{E}{E_0})^{1/2}$, where each quantum number must be an integer, such that $N = \frac{1}{8}(\frac{4\pi R^3}{3})$, or $\frac{N}{2}$ for two possible spin-states
			\4 The infinite well radius is called the Fermi radius, acting as the maximum energy state an electron exists in within the well
			\4 Thus, $g(E) = \frac{dN}{dE} = \frac{2\pi(2m)^{3/2}V}{h^3}E^{1/2}$, called the density of states, where for an electron/fermion, it is multipled by 2 for each of the spins, otherwise not, after which it can be integrated such that $N = \int^{\infty}_0 g(E)f(E)dE$, assuming a high enough number of states that it can be integrated
			\4 As a result, for a fermion, $e^{-\alpha} = \frac{Nh^3}{2(2\p im_ekT)^{3/2}V}$, twice that for a boson
	\2 For liquid helium, there is approximately one particle per quantum state at such low temperatures, such that Boltzmann did not apply, instead treating it as an ideal gas under Bose-Einstein distribution, applying below 2.17 K
		\3 It was found to increase in density until that point, called the lambda point, apparently as a phase transition, called Helium I above, Helium II below (made of Helium I and superfluid Helium, which had viscosity 0)
			\4 As a result, the density of Helium II was the sum of Helium I and superfluid helium, remaining constant due to the superfluid increasing as the temperature drops
			\4 Superfluid helium was considered to be ground state helium atoms, only exerting van der Waals forces, combined with the low density to act as an ideal gas
		\3 The distribution must have $\alpha \geq 0$, such that the number of particles is non-negative for all temperatures, integrated over all positive real numbers, where the ground state is assumed to be E = 0
			\4 As temperature drops past the critical temperature, such that $\alpha = 0$, it is impossible to normalize
			\4 Since g(E) = 0 for E = 0, it ignores ground state particles, irrelevent for fermions due to a maximum of 2 in that state, but causing the lack of normalization
			\4 As a result, it can be normalized if $N = N_0 + \int_0^{\infty} n(E)dE$, where $N_0 = \frac{g_0}{e^{\alpha}e^{E_0/kT} - 1} = \frac{1}{e^{\alpha} - 1}$
			\4 Thus, $\frac{N_0}{N} \approx 1 - (\frac{T}{T_c})^{3/2}$, such that as the temperature approaches 0, the condensate approaches the entire material
		\3 Atoms, especially in their ground states, tend to be bosons, although made up of fermions, though in small spaces, the interparticle interaction creates fermion behavior, making the production of condensates difficult
			\4 On the other hand, when produced, it begins to exhibit macroscopic quantum wavefunction behavior in terms of appearance and properties, forming coherent matter acting like lasers
\1 **ADD 8.4**
\1 The free electron theory was used to explain conduction, but fails due to not having the expected $\frac{3kT}{2}$ added average translational kinetic energy from collision above the degree of freedom 3kT
	\2 This is explained by the indistinguishability of the electrons, such that they are defined under Fermi-Dirac distribution rather than expected Boltzmann 
	\2 By the constant Fermi energy, $f_{FD}(E) = \frac{1}{e^{\alpha + \frac{E}{kT}} + 1} = \frac{1}{e^{\frac{E - E_F}{kT}} + 1}$, such that $\alpha = \frac{-E_F}{kT}$
		\3 As a result, for any temperature, $E = E_F$, $f_{FD} = \frac{1}{2}$, while for 0 K, if $E < E_F$, f(E) = 1, $E > E_F$, f(E) = 0
		\3 Thus, the Fermi energy is the maximum energy state filled at 0 K, while at higher temperatures, electrons in energy states within approximately kT of the Fermi are somewhat dispersed in the kT above
			\4 As the temperature gets vastly higher, even those further from kT below the Fermi energy move above, such that $f(0) < 1$
	\2 **FINISH 8.5**
\end{outline*}
\section{Chapter 9 - Molecular Structure and Spectra}
\begin{outline*}
\1 Molecular bonds are generally ionic, covalent, dipole-dipole, or metallic, created stabally to lower the total energy of the atom from the total energy of the individual parts seperate, caused by electrostatic forces and Pauli's exclusion principle
\1 The ionic bond is generally the strongest, with the seperate energy assumed to be 0 eV, while the ionization energy of $K^+$ is 4.34 eV, while Cl has electron affinity (energy released by acquisition of one electron) is 3.62 eV, such that the energy needed to create it (net ionization energy) is 0.72 eV
	\2 The release of energy is due to the extra electron penetrating the outer shell, such that it is attached to a net positive charge
	\2 Since the electrostatic potential energy is $\frac{-ke^2}{r}$, such that at some low distance, the potential is high enough to create the ion
		\3 While the atoms would classically keep moving together, exclusion principle repulsion occurs due to the overlapping of wavefunctions of electrons with the same energy states as they move close enough together
		\3 Since the electrons overlap, they require moving into a higher energy state to avoid overlapping, such that the total energy increases, even as the electrostatic potential decreases
			\4 The total energy of the ssytem can be written as $U(r) = \frac{-ke^2}{r} + E_{totalIon} + E_{exclusion} = \frac{-ke^2}{r} + E_{ion} - E_{affinity} + \frac{A}{r^n}$, where A and n are molecular constants
			\4 This ignores the small, but positive zero-point energy, and the small, but negative van der Waals attraction (due to induced dipole moments)
	\2 As a result, the minimum energy is the dissociation energy, with the radius at that point as the equilibrium seperation, for ground states of molecules
		\3 For excited electrons of molecules, the total potential energy of the curve is higher and more spread over the radii
\1 Covalent bonds are used when the net ionization energy is too high to allow a negative energy at any distance, based on the symmetry of two infinite square wells in the same state, either symmetric or antisymmetric
	\2 As the wells get close enough together, symmetric wavefunctions have a higher likelihood of being found in between the wells (the bonding orbital), while the anti-symmetric approaches the first excited state in structure and energy
		\3 For an antisymmetric wavefunction, the electrons are between the atoms enough to allow electrostatic force to hold the molecules together, such that there is no minimum energy (antibonding orbital)
	\2 The Hamiltonian of the system is $H_{op} = \frac{p_{op}^2}{2m} + ke^2(-\frac{1}{r_1} - \frac{1}{r_2} + \frac{1}{r_0})$, where $r_0$ is the distance between atoms, and $r_1/r_2$ is the distance between electrons
		\3 Due to the lack of multiple electrons, there is no exclusion-principle repulsion
		\3 Thus, for the symmetric wavefunction, $E_1 = -13.6 Z^2 = -54.4 eV$ for $H^2_+$, approaching -13.6 eV as the distance approaches infinity, while the anti-symmetric approaches -13.6 eV as the first excited at both 0 and infinity
		\3 It is found as a result that the binding energy is 2.7 eV
	\2 For an $H_2$ system, the symmetric total wavefunction is the one with a spacially symmetric portion, antisymmetric spin, both approaching 2 * -13.6 as the distance goes to infinity
		\3 In this case, due to two molecular states, the energy goes to infinity as r goes to 0, but the symmetric form has a negative minimum
		\3 Since the molecular orbitals are subject to the Pauli exclusion principle, the electrons are considered bonded by the angular momentum state, called a saturated bond if filled, due to further electrons requiring higher energy states
		\3 It is found in this case that there is a lower equilibrium separation, with higher binding energy of 4.5 eV, due to increased charge binding
	\2 The amount of covalent and ionic bonding is determined by the electric dipole moment, $p_{ion} = er_0$, where the percentage of ionic bonding is $\frac{p_{meas}}{p_{ion}}$
\1 Dipole-dipole/molecular bonding is due to electrostatic forces, pulling molecules and non-bonding atoms together, causing solid and liquid forms, with the electric field of a dipole as $\vec{E} = k(\frac{\vec{p}}{r^3} - \frac{3(\vec{p} \cdot \vec{r})}{r^5}\vec{r})$
	\2 The dipole, $\vec{p} = qa$, where a is the length between the charges, such that as $r >> a$, $E_d = \frac{k\vec{p}}{r^3}$
	\2 As a result, another dipole will have potential energy, $U = -\vec{p_{other}} \cdot \vec{E}$, orienting itself along the electric field lines as a result
		\3 Thus, there is attraction between permanent electric dipoles/polar molecules, called a hydrogen bond when involving hydrogen, acting as sharing a proton, used to hold ice and DNA structure, and produce friction and surface tension
		\3 In addition, molecules with higher dipoles cam be assumed to have a higher boiling and melting point as a result
	\2 Nonpolar molecules are able to be polarized by a field of a dipole, such that $\vec{p_{induced}} = \alpha \vec{E_d}$, where $\alpha$ is the polarizability
		\3 As a result, the resulting potential energy, $U = -\vec{p_{induced}} \cdot \vec{E_d}$, producing an attractive force
		\3 This is able to exist between two nonpolar molecules, due to the average square dipole moment being nonzero due to constant electron movement, called the van der Walls/London dispersion force
\end{outline*}

\section{Notes}
\begin{outline*}
\1 Absorption is a subset of the emission spectra, due to multiple possible paths of emission for higher levels of hydrogen atoms
\1 $\frac{1}{\lambda} = \frac{m_e \alpha^2 Z^2}{2} (\frac{1}{n^2} - \frac{1}{m^2})$, $\alpha = \frac{e^2}{4\pi \epsilon_0 \hbar c} \approx \frac{1}{137}$, called the fine structure constant
\1 Fractional change of n vs m = $\frac{n - m}{n}$
\1 Hydrogen atom is $f = cZ^2R(\frac{1}{n^2} - \frac{1}{m^2}$, such that Moseley's Law applies this to xrays for analyzing wavelength
\1 Regular as n goes to infinity, L'Hopital used 
\1 The area of a sphere is $4\pi r^2$
\1 $\lambda = \frac{h}{\sqrt{2mK}}$
\1 $L = mvr = I\omega = mr^2 \omega$ (for single particle)
\1 \textbf{**Pay attention to units**}
\1 Things maybe to know: Light frequencies of different types, eV vs J conversion, angular momentum mechanics
\1 Powder scattering is the scattering of light by a collection of crystallites with random orientations, forming circles of peaks in a target formation
\1 Electrons observed appear to behave with probabilities due to the light preventing the interference pattern, only going through one slit, changing the outcome, or low intensity makes not enough scatter and low momentum/high wavelength don't measure properly
	\2 Complementary principle, only one type of property, wave or particle, can be measured, but not both can be
		\3 Since momentum is proportional inversely to wavelength, it is a wave property, such that it or position can be measured, but not both, leading to Heisenberg
\1 Copenhagen interpretation states that initial conditions can only determine probability, not definite future conditions
\1 Oscillating functions are often represented in complex exponential notation, keeping only the real portion for the actual equation
\1 Useful integrals include $\int^{\infty}_{-\infty} e^{-(x/a)^2}dx = a\sqrt{\pi}$, $\int^{\infty}_{0} xe^{-(x/a)^2}dx = \frac{a^2}{2}$, $\int^{\infty}_{-\infty} x^2 e^{-(x/a)^2}dx = \frac{a^3\sqrt{\pi}}{2}$, $\int^{\infty}_{-\infty} e^{ikx-(x/a)^2}dx = a\sqrt{\pi}e^{-(a^2k^2)/4}$
\1 $R_{\infty} = 1.0973 * 10^7 m^-1$ (Rydberg Constant)
\1 The operators for the wavefunction are found by the complex exponential wavefunction, using the expression which would produce the correct terms
	\2 Eigenfunctions are functions which when acted on by an operator, produce a constant times the function ($Q\psi = \lambda \psi$), such that the expectation value of the operator is always the eigenvalue, able to use it simplify expectation value calculations
	\2 As a result, the standard deviation ($<A^2> - <A>^2$) of 0, providing a definite value for some value
	\2 For $e^{ipx}$, the momentum operator forms an eigenfunction, due to the momentum being fixed in the equation or some other reason?
\1 For two functions, they are orthogongal if the product of the conjugate of either by the other is 0 over the entire domain, combined with the fact that $<1> = 1$ to simplify expressions
	\2 Orthogonality exercise 2 is true due to why??
\1 Standard deviation, $\sigma = \sqrt{<x^2> - <x>^2}$
\1 Uncertainty principle for standard deviation is divided by 1/2, for half width is 1, otherwise it can be either multiplier
\1 **Solid angle/differential cross section??**

\1 Quantum mechanics can be written in terms of matrix operations, related to operators, noted to similarly not necessarily be commutable
	\2 Similarly, for a wavefunction as a linear combination of eigenfunctions, the vector element cooresponds to the vector of the amplitudes of each eigenfunction
	\2 Thus, the operator can be denoted as some matrix, (often as the diagonal matrix of the eigenvalues), such that eigenfunctions are like eigenvectors (when multiplied by the operator matrix, it provides the vector multiplied by a constant)
	\2 Expectation values are calculated by the adjoint vector, or the conjugate transpose, denoted as $<f(x)> = v^*f(x)v$
	\2 Spin has matrix values as $\chi_{Z+} = (1, 0)^T, \chi_{Z-} = (0, 1)^T$, with $S_z = \frac{\hbar}{2}((1 0)^T (0 -1)^T), S_x = \frac{\hbar}{2}((0 1)^T (1 0)^T), S_y = \frac{\hbar}{2}((0 i)^T (-i 0)^T)$
		\3 As a result, for some fixed value of x, either positive or negative eigenvalues, it has equal probability of the positive or negative z value
		\3 $\chi_{X+} = \frac{1}{\sqrt{2}}(1, 1)^T, \chi_{X-} = \frac{1}{\sqrt{2}}(1, -1)^T$, such that the sum of them provides $\chi_{Z+}$, difference for $\chi_{Z-}$

\1 It is found that an eigenfunction cannot be simultaneously found for multiple $\vec{L} (\vec{r} x \vec{p} = mvR = mR^2\omega = I\omega = mR^2\frac{d\theta}{dt})$ components, but can be for $L^2, L_z$, similar to being unable to be found for p and x
	\2 As a result, it is noted that the maximum angular momentum along a single axis is less than the total angular momentum, since otherwise, the angular momentum along all axes would be known
	\2 $L^2_{op} = -\hbar^2(\frac{\partial^2}{\partial \theta^2} + cot(\theta)\frac{\partial}{\partial \theta} + \frac{1}{sin^2(\theta)}\frac{\partial^2}{\partial \phi^2}$, $L_z = -i\hbar(x\frac{\partial}{\partial y} - y\frac{\partial}{\partial x}) = -i\hbar\frac{\partial}{\partial \phi}$, such that the wavefunction is already an eigenfunction
	\2 It is noted that angular momentum can be put in terms of moment of inertia, I, as the sum of the values for each particle, such that for a sphere, $I = \frac{2}{5}MR^2$, with rotational kinetic energy as $K = \frac{1}{2}I\omega^2 = \frac{L^2}{2I}$
	\2 The force analog of angular momentum, $\vec{\tau} = \vec{r} x \vec{F} = \frac{d\vec{L}}{dt} = I\alpha$, such that it is conserved for an isolated system

\1 Hydrogen atoms involve only radius based electrostatic potential, $V = \frac{-Ze^2}{4\pi \epsilon_0 r} = \frac{-kZe^2}{r}$
	\2 Spherical coordinates are $x = rsin(\theta)cos(\phi), y = rsin(\theta)sin(\phi), z = rcos(\theta), \theta = tan^{-1}(\frac{\sqrt{x^2 + y^2}}{z}), \phi = tan^{-1}(\frac{y}{x})$, assuming $\theta$ is from the z-axis, $\phi$ is from the x-axis
		\3 In addition, $dV = r^2sin\theta d\phi d\theta dr$ for integrating in spherical coordinates, using radians rather than degrees
		\3 $\nabla^2 = \frac{\partial^2}{\partial r^2} + \frac{2}{r}\frac{\partial}{\partial r} + \frac{1}{r^2}(\frac{\partial^2}{\partial \theta^2} + cot(\theta)\frac{\partial}{\partial \theta} + \frac{1}{sin^2(\theta)}\frac{\partial^2}{\partial \phi^2})$
	\2 As a result, the surface area integral is the integral with a set radius, such that there is no radius bound
\1 The hydrogen atom product solutions for angles and radial must be independently normalized, due to needing to be at some radius and angle
	\2 Energy less than 0 implies a bound state, such that potential energy is greater than kinetic energy, which are the proper solutions
	\2 The degeneracy of the state is equal to $n^2$, equal to the number of states with the same energy in a given potential

\1 The beat frequency of the superposition of two waves is the absolute value of the difference of the two frequencies
	\2 As a result, for some wave $f_0$, modulated by frequencies from 0 to f, where $f_0 >> f$, the result can be decomposed into frequencies from $f_0 - f$ to $f_0 + f$
	\2 For the range considered the half-width at maximum, it is $\hbar$, while for the standard deviation ($\sqrt{<x^2> - <x>^2}$), it is $\frac{\hbar}{2}$
		\3 For some distribution function, f(x), $<g(x)> = \int g(x) f(x)dx$
\1 Zero point energy provides the stability of the atom by preventing it into collapsing into itself, by the uncertainty principle
\1 Reduced mass is derived by Newton's Third Law, within the reference frame of $X = x_1 - x_2$, with the only force as that acting on both
\1 Larmor precession, $\omega = \frac{g\mu_B}{\hbar}\vec{B}$, is the respective of $\vec{\mu}$ for $\vec{B}$ rather than $\vec{L}$

\1 The general wavefunction of the infinite well is the linear combination of solutions to the equation with some coefficient to provide the combination, such that it is not simply a single eigenfunction, though the energy measured be an eigenvalue
	\2 $A_n = \int u_n^*(x)\psi(x)dx$, where $u_n(x)$ is the nth eigenfunction of the equation
	\2 The probability of measuring any particular energy is $|A_n|^2$, such that $<H> = E_{avg} = \int \psi^* H \psi dx = \sum_n |A_n|^2E_n$

\1 The gyromagnetic ratio of a proton is 5.56, though the spin of the proton has almost no difference in energy due to the larger mass
	\2 Nuclei can only have angular momentum l = 0, such that only the spin of the nucleons play a role

\1 Symmetries are related to conservation laws, such that in constant potential (momentum conserved), translation of the system creates no change in the outcome, while rotational relates to angular momentum, and time-invariant potential means energy conservation
	\2 As a result, symmetry of the Hamiltonian creates symmetry in the probability density of the same form, such that the wavefunction is restricted
	\2 Thus, since the Hamiltonian is invariant under the exchange of two identical particles, the wavefunction is either symmetric or antisymmetric under particle exchange
		\3 This is found to be the former for bosons with integer spin, the latter for fermions with half-integer spins, requiring linear combinations to produce
		\3 Thus, to produce the antisymmetric wavefunction, it is the linear combination of opposite spins, negatively combined, such that the spin of each is equally likely to be measured, though the opposite will be measured after for the other
	\2 Thus, there is a triplet of S = 1, spin-symmetric, space-antisymmetric wavefunction forms (each with a distinct joint m-value), and a singlet of S = 0, spin-antisymmetric, space-symmetric wavefunction forms
		\3 Thus, for a spacially-antisymmetric state, the probability of both particles being at the same location is 0, unlike in a spacially-symmetric, such that the motions are related

\1 Selection rule states that m must go +1, -1, or 0 in a transition, while l must go +1 or -1

\1 The reflection probability $\frac{B^2}{A^2}$, while the transmission probability is $\frac{k_2 C^2}{k_1 A^2}$, due to being the relative rates

\1 Since the spin of the electron in fermions keeps electrons away from eachother, it is called the exchange force of the electron

\1 For some infinite square well, by Pauli's exclusion principle, there is able to be two electrons per state, based on the spin of the electron
	\2 It is found as a result that the total ground energy for $E_{total} = \frac{N^3E_1}{12}$, where there are N particles, such that $n_{max} = \frac{N}{2}$
	\2 Fermi energy is the energy of the highest filled energy state, such that $E_F = \frac{\hbar^2 \pi^2 n^2}{8m}$, where n is the number of particles per unit length

1/24 - Exp Basis Not Done
3/01 - Hydrogen Not Done
3/16 - Many Not Done
3/21 - Many Not Done
3/21 - StatMech Not Done
3/23 - StatMech Not Done

\end{outline*}
\end{document}