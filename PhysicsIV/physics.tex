\documentclass[11 pt, twoside]{article}
\usepackage{textcomp}
\usepackage[margin=1in]{geometry}
\usepackage[utf8]{inputenc}
\usepackage{color}
\usepackage{indentfirst} %Comment out for no first paragraph indent
\usepackage[parfill]{parskip}
\usepackage{setspace}
\usepackage{tikz}
\usepackage{amsmath}
\usepackage{amsfonts}
\usepackage{amssymb}
\usepackage{enumitem}
\usepackage{outlines}

\usepackage{fancyhdr}
\pagestyle{fancy}
\cfoot{\hyperlink{content}{\thepage}}
\lhead{}
\chead{}
\rfoot{}
\lfoot{}
\rhead{}
\renewcommand{\headrulewidth}{0pt}
\renewcommand{\footrulewidth}{0pt}


\usepackage{hyperref}
\hypersetup {
	colorlinks,
	citecolor=black,
	filecolor=black,
	linkcolor=black,
	urlcolor=black
}

\newcommand{\sepitem}{0pt} %Added room between items on the list, not including a list and its sublist
\newcommand{\seppar}{1pt} %Between items and lists overall

\setenumerate[1]{itemsep=\sepitem, parsep=\seppar}
\setenumerate[2]{itemsep=\sepitem, parsep=\seppar}
\setenumerate[3]{itemsep=\sepitem, parsep=\seppar}
\setenumerate[4]{itemsep=\sepitem, parsep=\seppar}

\newenvironment{outline*}
{
	\begin{outline}[enumerate]
	}
	{\end{outline}
}

\newcommand{\foot}[1]{\hyperlink{#1}{$_#1$}}

\begin{document}

\title{Principles of Physics IV: Modern Physics}
\author{Avery Karlin}
\date{Spring 2017}
\newcommand{\textbook}{Modern Physics by Tipler and Llewellyn}
\newcommand{\teacher}{Dr. Paul Heiney}

\maketitle
\newpage
\hypertarget{content}{\tableofcontents}
\vspace{11pt}
\noindent
\underline{Primary Textbook}: \textbook\\
\underline{Teacher}: \teacher
\newpage

\section{Course Introduction}
\begin{outline*}
\1 Kinematic assumptions originally stated that all items have a well-defined position that can be measured to any precision, and all observers agree on the time and position for a measurement
\1 Newtonian mechanics were based on the idea of the deist clockwork universe
	\2 Conservation of momentum follows from Newton's third law, conservation of energy follows from the work energy theorm
	\2 Conservative forces have potential energy defined such that F = $-\nabla V$ (or U, depending on notation)
	\2 Newton's Laws combined with the fact that all known major forces act on the center of an object, implies conservation of angular momentum
\1 Relativistic mechanics has $p = \gamma mv$ and $E^2 = c^2p^2 + m_0^2c^4$, and defined rigid bodies as impossible, often using the center of mass
\1 Waves are travelling disturbances in a medium, averaged over space to consider bulk properties, carrying energy and momentum
	\2 Wave equations often relate curvature (second derivative w/ respect to x) to acceleration, $f_{xx} = (1/u^2) f_{tt}$ (u can be replaced with v, acting as an alternate velocity notation)
	\2 Waves are travelling, such that f(x, t) = g(x - ut)
	\2 y = exp(x - ct) is travelling, while sin(x)cos(ct) is a sum of two travelling, but not a travelling on its own, such that any g(x - ut) works
\1 Waves but not particles can diffract, refract, and superposition, while both have velocity, generally localized position, momentum, and energy
	\2 Superposition creates nodes of no displacement, antinodes of max displacement in a standing wave
	\2 Energy is proportional to the square of the amplitude
	\2 Wave properties include that $v_p = f\lambda = \frac{\omega}{\lambda}$, $\omega = 2\pi f$, and $k = 2 \pi \lambda$
\1 Maxwell's equations show that since changing magnetic fields produce electric fields and vice versa, waves can be created by oscillating or accelerating charges, travelling forever in vacuum
\1 Statistical mechanics are used to solve systems with too many obects to calculate, instead using statistics to predict average properties
	\2 For some complicated system with total conserved energy and many ways to distribute the energy, the probability of any entity having a particular energy is independent of others, such that it depends only on others, but limits the possibilities of remaining energies
		\3 Thus, $P(E_1)P(E_2) = f(E_1 + E_2)$, since the probability of the joint energy depends only on the sum, found to be true if $P(E) = Ab^{\pm cE} = Ae^{-\beta E}$, where $\beta = \frac{1}{k_BT}$, $k_b = 1.38 * 10^{-23} J/K$
		\3 n(E) is defined as the number of distinct ways an object could have energy E, such that the sum of the energies must be 1, either a sum or an integral if discrete or continuous, such that $P(E) = \frac{e^{-\beta E}}{\sum_j n(E_j) e^{-\beta E_j}}$
	\2 Thus, the average value is the sum of the value multiplied by the probability for all possible energy values, $\bar{g(E)} = \frac{\sum_j n(E_j)g(E_j)e^{-\beta E_j}}{\sum_j n(E_j)e^{-\beta E_j}}$
	\2 The Boltzmann Distribution assumes n(E) = 1 for all E, and the energy distribution is continuous, such that $P(E) = \frac{e^{-\beta E}}{k_BT}, \bar{E} = k_BT$
	\2 Thus, for some kinetic energy with $\vec{v} = v_x + v_y + v_z$, normalized such that the probability is equal to 1 for all possible velocities with $P(v_x, v_y, v_z) = Ae^{-\beta K}$, it is found that $<K> = \frac{3k_B T}{2}$
\1 Single photon cannot pair produce an electron and a positron, due to the center of mass frame proving lack of conservation of momentum, though a single photon can collide into a mass and pair produce
\1 Existance of atoms based on the macroscopic shape of crystals, the inability to convert elements to another type, stoichiometry chemistry rules, and periodic table able to predict properties of atoms
\1 de Broglie later stated that massive particles could have wavelength by the same formuila, $p\lambda = \lambda \sqrt{2mK} = h$, where K is kinetic energy
\1 For two spheres on a collision course, they will collide if the center-center distance is at most $r_1 + r_2$, such that effective area/cross-section, $\sigma = \pi (r_1 + r_2)^2$
	\2 Thus, the probability of an electron colliding in a box with sides of area A/$L^2$ is $P = \frac{M\sigma}{A} = nL\sigma$, where M is the mass of the box and n is the density of the box in particles/volume
	\2 For some surface, the number colliding per area, $\Delta N = -PN$ where N is the number incident on the surface per second, and the number through the box of thickness x is thus, $\frac{\Delta N}{\Delta x} = -nN\sigma$
	\2 Thus, the number remaining in the beam, $N(x) = N(0)e^{-n\sigma x} = N(0)e^{-x/l}$ where l is the mean scattering length, or the average distance before a collision
\end{outline*}

\section{Chapter 3 - Quantization of Charge, Light, and Energy}
\begin{outline*}
\1 In the 1800s, Faraday proved that a specific quantity of electricty could decompose one gram-ionic weight of monovalent ions, equal to a Faraday, or a mole of electrons, such that $Q = N_Ae$, called Faraday's Law of Electrolysis, displaying discrete electric charges
	\2 Zeeman later discovered that discrete spectral lines emitted by an atom in a magnetic field separate into three spaced lines of different frequencies, caused by the slightly different charge to mass ratios
		\3 This proved that the particles producing the light were negative, and found the charge to mass ratio of the electrons
	\2 Thomsons cathode ray experiment later measured the same ratio as Zeeman, proving the existance of the electron, with the same ratio, as being the atomic negative component
		\3 This combined with Faraday's charge allowed the mass of the electron to be determined
		\3 He also used a uniform magnetic field creating a circular path to measure the same ratio, found to be the same for all materials, showing it was universal for atoms
	\2 Millikan attempted to use a cloud of water droplets with a charge, such that Q = Ne, using the mass of the cloud and the radius of the drop to find e, found to be difficult because of the evaporation
		\3 On the other hand, he found a single drop could be balanced in the air by an electric field, eventually picking up an ion causing a movement in some direction
		\3 This resulted in the oil drop experiment, giving each charge and preserving it in midair, measuring the force on it, to confirm the electron charge
\1 Absorbed radiation increases the kinetic energy of oscillating atoms, increasing the temperature, but resulting in increased radiation emission by electrons, reducing kinetic energy, called thermal radiation
	\2 At thermal equilibrium, the rate of absorption and emission are equal, such that higher freqencies are present at higher temperatures, due to higher energy
	\2 Ideal blackbodies emit and absorb all incident radiation by $R = P/A = \omega T^4$, where Stefan's constant ($\omega$) is $5.67 * 10^8 W/m^2K^4$
		\3 Non-blackbodies emit multiplied by some emissivity constant, based on factors such as color, temperature, and composition
	\2 Spectral distribution of the radiation of a blackbody also only depends on temperature, where the maximum emitted wavelength, $\lambda_{max} T = 2.898 * 10^{-3} m*K$, called Wien's displacement law
	\2 Blackbodies are approximated by a cavity with a small hole to let radiation in, found that the power radiated out, $R = \frac{1}{4}cu$, where U is the total radiation energy density in the cavity
		\3 As a result, both are proportional to the wavelength, such that the energy density distribution can be found by the number of modes of oscillation
		\3 It is found that the number of modes of standing wave oscillation per unit volume, $n = 8\pi\lambda^{-4}$, and the Rayleigh-Jeans equation states that $u = kTn$, such that R can be calculated
		\3 As a result, while at higher wavelengths, it fits experimentally, at low wavelengths, it appears $R \to \infty$, called the ultraviolet catastrophy, such that total energy density over the spectrum from 0 to $\infty$ would be infinite as well
	\2 Planck's Law corrects for this, stating that since as $\lambda$ approaches 0, n approaches infinity by classical formulas, u must be a function of wavelength, such that it approaches 0
		\3 Classically, electrons oscillating produce waves with equal frequency, where the average energy is found by the energy distribution function, $n(E) = Ae^{-E/kT}$ based on the Maxwell-Boltzmann distribution function, such that average energy, $\bar{E} = \int^\infty_0 Ef(E)dE = kT$
			\4 In this equation, n is the fraction of oscillators with energy E
			\4 This is based on the fact that for some standing wave, in space a, $\lambda = \frac{2a}{n}$ must be true for soem integer n, such that the number of modes, $\Delta n = \int^{f_2}_{f_1} \frac{4a}{c}df$, the additional multiple of 2 to account for two polarizations
			\4 In higher dimensions, $f = \frac{c}{2a}\sqrt{n_x^2 + n_y^2 + n_z^2}$, such that $N(f)df = 2(\frac{2a}{c})^3(\frac{1}{8}4\pi f^2df)$, when the space is modeled as a sphere of radius a, only in a single octant
			\4 The higher dimensional mode equation combined with Boltzmann distribution gives the Rayleigh-Jeans formula
		\3 Planck found it agreed with expermental data if oscillator energy is a multiple of a discrete value, such that $E = nhf$, where h is Planck's constant
			\4 The sum can be taken similarly as the sum of the probabilities for some particular frequency, to get the average energy, using geometric sums to simplify, such that $\bar{E} = \frac{hf}{e^{hf/kT} - 1}$
			\4 It follows that $u = \frac{8\pi hc\lambda^{-5}}{e^{hc/\lambda kT} - 1}$, called Planck's Law, found to be a generalization of all known laws
	\2 Blackbody radiation has been used as proof of the Big Bang Theory, due to the universe predicted to act as a perfect black body in terms of energy distribution
\1 In Hertz's spark gap experiment to generate EM waves and detect them, proving Maxwell's Theories, finding that light hitting a surface produced an electron current
	\2 Lenard later proved that it was electrons, and observed the current was proportional to the intensity (P = IA), but found that there was no minimum intensity needed as would be classically expected, due to requiring enough energy to
		\3 Fluxs of photons are the photons per second per unit area, related to intensity 
		\3 Since the kinetic energy had to be great enough to avoid being pulled back to the metal surface cathode if there was a negative voltage, it required a voltage produced greater than $-V_0$, called the stopping potential
		\3 Thus, it was found that $KE_{max} = eV_0$, such that $KE_{max}$ was independent of the light intensity as well, rather than increasing the electron kinetic energy
	\2 Einstein postulated as a result that Planck's quantization was universal, such that E = hf for all light quanta, such that $eV_0 = hf - \phi$, where $\phi$ is the work function, characteristic of the metal, to remove an electron
		\3 This is equal to the maximum kinetic energy by energy conservation, though some electrons lose energy when leaving the metal further
		\3 As a result, the threshold wavelength is equal to the work function divided by h
		\3 This also explains the lack of a time lag for the production of photoelectrons, instead of the calculated time for enough energy if it is spread evenly over the surface, as assumed classically
\1 Xrays originally were discovered by Roentgen with a cathode ray tube, noticing rays from the collision of electrons and the glass tube could activate flurescent photographic film and pass through opaque materials
	\2 He later observed no material was opaque, though less rays could pass through with higher densities
	\2 He stated that their apparent lack of magnetic deflection, refraction, or interference, was due to a very short wavelength, finding they were defracted by a crystal lattice, also proving a regular crystal array
	\2 He found they were produced by eletrons when deflected then stopped by the atoms of a target
	\2 He then thought to use Bragg places, or face-centered cubic molecular structures of NaCl crystals, analyzing the scattered waves from each atom to view xray diffraction
		\3 The waves are in phase regardless of wavelength if the scattering angle is equal to the incident angle for two waves hitting atoms in a plane
		\3 This condition is called the Bragg condition, true if $2dsin(\theta) = m\lambda$, where d is the distance between atoms and $\theta$ is the angle from the surface (half the scattering angle), for constructive interference, found in soap bubbles
		\3 The amount of ionization from xrays can then be measured to get the intensity of each wavelength, after it has been corrected by the Bragg condition to get the correct angle
	\2 This produces a spectra for the xrays produced, based on the anode of the material, able to be used to determine the atomic spacing
		\3 This produces both the characteristic spectrum of sharp lines and the continuous spectrum, the former which is specific to the material
			\4 Maxwell had previously predicted the continuous spectrum was due to the electron bombardment/deacceleration in the strong electric field (called bremsstrahlung), though the cause of the characteristic was unknown
		\3 There is also a cutoff wavelength, independent of the material, based on the energy of the bombarding electrons, by $\lambda = \frac{1.24 * 10^3 nm}{V}$, called the Duane-Hunt Rule
			\4 This is explained as the opposite of the photoelectric effect, such that all the kinetic energy is converted ($\lambda = \frac{hc}{eV}$)
\1 Compton later measured the scattering of xrays by free electrons, proving further both the photon and special relativity
	\2 Classical EM would have predicted a dipole oscillation due to the light field of the particle, with the maximum at the original wavelength, based on $1 + cos^2(\theta)$ for the angle of oscillation, rather than a shifted maximum
	\2 He observed that the scattered xrays were more easily absorbed, considering that the collision allowed an electron to absorb some of the photon energy, such that the wavelength became longer
	\2 As a result, he derived the Compton equation mechanically through relativity and quantization stating $\Delta \lambda = \frac{h}{mc}(1 - cos(\theta))$, where $\frac{h}{mc}$ is called the Compton wavelength of the electron
		\3 This was observed for xrays due to the percent change in wavelength only being noticable for very short original wavelengths
		\3 The unshifted, but scattered, portion is due to electrons tightly bound to the atom, such that the entire atom recoils
\end{outline*}

\section{Chapter 4 - The Nuclear Atom}
\begin{outline*}
\1 Newton's dispersion of white light was the first spectroscope, later shown to have hundreds of numbers of dark lines inside, and bright light spectra formed by flames, forming the study of spectroscopy
	\2 Continuous spectra are emitted by incandescent solids, showing no specific lines in any spectrscope
	\2 Band spectra are closely packed groups of lines apparently continuous in low resolution produced by fire, while line spectra are those produced by unbound chemical elements, both characteristic of material
\1 In 1885, Balmer found that visible and near-UV spectrum of H could be represented by $\lambda_n = 364.6 \frac{n^2}{n^2 - 4} nm$, where n = 3, 4, ..., called the Balmer series
	\2 The general form for other elements was later found as the Rydberg-Ritz formula, such that $\frac{1}{\lambda_{mn}} = R(\frac{1}{m^2} - \frac{1}{n^2}) \forall n > m$
		\3 R is the rydberg constant, varying slightly for elements, but negligably, such that $R_H = 1.096776 * 10^7 m^{-1}$ and $R_{\infty} = 1.097373 * 10^7 m^{-1}$
		\3 Thus, for the Balmer series, m = 2, such that the maximum wavelength is as n approaches infinity
	\2 The other series from higher energy to lower are Lyman, then Balmer, Paschen, Brackett, Pfund, and then Humphreys
\1 Thomson's plum pudding model was the most popular atomic model after Balmer and Rydberg-Ritz formulas, trying to find stable configurations with normal modes of vibration of spectral lines
	\2 On the other hand, the lack of continual emission due to electrostatic radiation and inability to make a model hindered this theory
\1 Rutherford, while studying radioactivity, finding that alpha particles were doubly ionized helium, checking the spectral lines to prove it, used the particles to study atomic interiors
	\2 He sent the radiation into gold foil, measuring the scattering angles, mainly undeflected or barely detected, though a few with right angles or more, showing the positively charged sphere could not have created such high angle scattering
		\3 Thomson's model did not have enough force at any point for a large enough deflection
	\2 Rather there must be a dense center, such that Rutherford calculated angular distribution and dependence on nuclear charge, angle, and kinetic energy, validated experimentally
	\2 This can be calculated mechanically, where b is the impact parameter, or the distance from the line through the nucleus, such that $b = \frac{kq_{\alpha}Q}{m_{\alpha}v^2}cot(\frac{\theta}{2})$
		\3 Thus, it is found that for intensity $I_0$ (in particles per second), the number scattered by one nucleus through angles greater than $\theta$ is the number with impact parameters less than $b(\theta)$, equal to $\pi b^2 I_0$
		\3 $\pi b^2$ is called the cross section $\omega$ for scattering angles greater than $\theta$, such that it is multiplied by intensity and number of nuclei for the number of scatterings above that value
			\4 The number of nuclei is calculated such that $n = \frac{\rho N_A}{M}$, where M is the molar mass
			\4 As a result, the fraction of scatters above some angle $\theta$ is $\pi b^2 nt$, where t is the thickness of the surface
	\2 Rutherford later derived an equation for the number of particles scattered at some angle, $\Delta N = (\frac{I_0A_{sc}ntkZe^2}{2r^2E_k})^2\frac{1}{sin^4(\frac{\theta}{2})}$, where $A_{sc}$ is the detector area and r is the distance from the foil to the detector
		\3 In this case, Z is the atomic number and t is the thickness
		\3 This was later verified experimentally, showing the legitimacy of atomic theory
	\2 Rutherford's model does not assume a point charge, but rather just a ball, and assumes the alpha particle does not penetrate it, such that for each angle, the closest distance from the nucleus can be calculated
		\3 Thus, for the largest angle ($180^o$), the collision is almost head on, providing an upper bound for the atomic radius
		\3 For some nucleus, by conservation of energy, $r = \frac{kq_{\alpha}Q}{\frac{1}{2}m_{\alpha}v^2}$, using the point when the number of collisions at some angle changes as the radius based on the initial kinetic energy
\1 Bohr later specified the charge and mass of electrons, stating they orbited by Coulomb force, but was unstable due to accelerating toward the center, with EM predicting that light of $f = \frac{v}{2\pi r}$, proportional to $\frac{1}{r^{\frac{3}{2}}}$
	\2 For the electrons, $E = \frac{1}{2}mv^2 - \frac{kZe^2}{r} = -\frac{kZe^2}{2r_n}$, proportional to $-\frac{1}{r}$, such that it will be losing energy, with decreasing radius
		\3 As the energy is lost due to the emission of radiation, emitting a continuous spectrum as the radius changes, the atom will decay, rendering it invalid
	\2 Bohr solved this by stating electrons has certain stationary states they could orbit without emission, and that the radiation emitted was due to the change in state, $hf = E_i - E_f$, called the Bohr frequency condition
		\3 He also made the correspondence principle, stating that the limit of large orbits and energies (or as quantum numbers reach higher, rather than $\hbar$ reaches 0, as was thought during the ultraviolet catastrophe), all quantum must align with classical
	\2 This allowed him to determine that angular momentum is quantied, assuming values of $\frac{nh}{2\pi}$, determining the spectra for any atom with -e charge
		\3 By the centripetal Coulomb force, $v = (\frac{kZe^2}{mr})^{1/2}$, with $L = mvr = n\frac{h}{2\pi} = n\hbar$, where n is a quantum number, such that $r_n = \frac{n^2 a_0}{Z}$, where $a_0$ is the Bohr electron radius
		\3 Thus, $E_n = -E_0\frac{Z^2}{n^2}$, where $E_0 = \frac{mk^2e^4}{2\hbar^2} = 13.6 eV$, such that the energy is quantized, with n = 1 as the ground state for the single electron
			\4 The energy of the ground state is the ionization/binding energy to fully remove the electron from the atom, with the differences between states able to be drawn by an energy level diagram
		\3 This is able to be converted to the Rydberg-Ritz equation, finding the Rydberg constant (R) as $\frac{E_0}{hc}$
	\2 The Rydberg constant assumes a stationary nucleus due to infinite mass, such that if it has mass M, the nucleus and the electron have total kinetic energy of $E = \frac{p^2}{2\mu}$ where $\mu = \frac{m}{1 + \frac{m}{M}}$
		\3 As a result, $\mu$ is the reduced electron mass, substituted for the electron mass to account for regular nuclei, such that $R = R_{\infty}\mu$
		\3 $\mu$ is taken from classical physics, assuming central forces, acting similarly on electrons as on astronomical bodies
	\2 The correspondence principle as a result implies that as energy levels are closely enough spaced, quantization should not matter, such that it is true as $n \to \infty$
		\3 As a result, the frequency of an single electron jump emission is equal to the frequency of the revolution of the electron
		\3 While it appears that multiple jumps would be greater, this is remedied by allowing elliptical orbits, such that the energy depends on the major axis, rather than the eccentricity
	\2 Relativistic corrections implied that a highly eccentric orbit would need a larger correction as the electron moves closer, approximated for n = 1 in Hydrogen to $\frac{v}{c} = \frac{ke^2}{\hbar c} = \alpha$, called the fine-structure constant
		\3 As a result, there are many ellipses for a single energy level, and since the velocity depends on the orbit, the energy radiated is slightly different, approximately ranging $\alpha^2$
		\3 $\alpha$ is also able to be used to simplify equations acting as a universal constant
	\2 Large atoms are called Rydberg atoms, as the electron energy approaches the ionization energy, the atomic radius is able to increase, used to study the correspondence principle and electromagnetic fields
\1 Xray spectra were found by Moseley to change from element to element regularly, due to being caused by inner electron transitions, depending only on the less complex lower orbits and not affected by interatomic binding forces
	\2 The Mosley plot is based on the fact that energy levels are proportional to $Z^2$, such that characteristic X-ray lines are found by $f^{1/2} = A_n(Z - b)$, where $A_n, b$ are constants for the line
		\3 The K series has b = 1 while the L series has b = 7.4
		\3 These occur when the bombarding electron cause the ionization of an inner electron, resulting in another electron falling down into the n = 1/K shell, and so forth
	\2 Bohr's equation can provide a similar relation, using $n_f = 1$ with Z = Z - 1 to find the frequency, finding $f = cR_{\infty}(Z - b)^2(\frac{1}{n_f^2} - \frac{1}{n^2})$
		\3 Thus, for the K-series, $A^2_n = cR_{\infty}(1 - \frac{1}{n^2})$
		\3 The (Z - b) factor is due to the shielding of nuclear charge by other electrons in the lower shells, allowing corrections to the periodic table from weight to Z, solving property discrepencies
	\2 The Auger effect is an alternative to x-ray emission, ejecting an extra electron from an outer shell, such that each element has a characteristic Auger electron kinetic energy spectrum
\1 The Franck-Hertz experiment used a cathode ray tube to send electrons to atoms of some material by voltage, after which there is a small voltage in the opposite direction, measuring the current that reaches it
	\2 It was noted that the current increased as the initial voltage did, until it suddenly dropped, repeating as a pattern
	\2 This was due to a forced elastic collision with the atom until the energy was high enough to excite an electron, which was found to correspond to light emitted on the spectra
	\2 This formed the basis of electron energy loss spectroscopy, measuring the locations of inelastic collisions to gain the energy structure of materials
\end{outline*}
\section{Chapter 5 - Wavelike Properties of Particles}
\begin{outline*}
\1 The De Broglie Hypothesis stated that the cause of the quantization of electrons was an effect of the standing wave condition for the wavelength of the electron, such that $E = \frac{hc}{\lambda} = pc$, such that nonrelativistically for electrically accelerated particles, $\lambda = \frac{hc}{(2mc^2eV_0)^2}$
	\2 This had not been previously noticed due to the extremely small wavelength of larger mass objects
	\2 It was determined that low energy electrons would have a low enough wavelength to detect diffraction in a Bragg plane, noticing the maxima and minima in the scattering by the Bragg condition
		\3 As the energy is varied, the diffraction maximum changes, such that **ANGLE VOLTAGE ADD**
	\2 The wave properties of neutral atoms and molecules were foun using natural thermal eneremy as motion, only using the top plane of the Bragg's planes due to the lack of energy, measuring the sscattering
\1 The De Broglie equation for relativistic particles is found by $E^2 + (pc)^2 + E_0^2$, where $E = K + E_0$, such that $\frac{\lambda}{\lambda_c} = \frac{1}{(2(\frac{K}{E_0}) + (\frac{K}{E_0}^2))^{1/2}}$, where the Compton wavelength, $\lambda_c = \frac{h}{mc}$
	\2 This is by the relations stating that $p\lambda = h$ and $hf = E$, where E is the total energy, rather than just kinetic or mass energy
\1 Particle waves do not have a material or ether, but rather the wave is the probability of finding the particle, represented by superposition of a group of harmonic waves to form a time-space localized wave packet
	\2 Wave packets have a group velocity ($v_g = \frac{d\omega}{dk}$), and as more waves with infinitessimally similar values of k are added, it becomes more localized
		\3 Phase velocities, $v_p = f\lambda = \frac{\omega}{k}$, are the velocities of the individual harmonic waves, such that $v_g = v_p + k\frac{dv_p}{dk}$
		\3 Mediums in which the phase velocity is the same for all frequencies is nondispersive, such that it is the same as the group velocity, in which the wave shape is constant
	\2 It is found classically that for wave packets, $\Delta k \Delta x \approx 1$ where $\Delta k$ is the range of k values of the harmonic waves, $\Delta x$ is the approximate length of the wave packet
		\3 It follows that $\Delta \omega \Delta t \approx 1$ as well, called the response time-bandwidth relation
		\3 Both are approximate due to depending on how the range is defined as well as the shape of the packets, serving as a magnitude and defining the approximate wave characteristics
		\3 The derivative of the equation relating k and $\lambda$ gives $\Delta k = \frac{2\pi \Delta \lambda}{\lambda^2}$, such that $\Delta x \Delta \lambda \approx \frac{\lambda^2}{2\pi}$
		\3 This represents the minimum uncertainty/error of the measurement, though human error in the measurement would increase the actual error
	\2 The particle wavefunction is denoted $\psi(x, t)$, such that for each harmonic wave, $v_p = \frac{E}{p} = \frac{v}{2}$, where v is the velocity of the particle itself
		\3 The group velocity, on the other hand, is equal to the velocity of the particle itself, acting as a nondispersive wave packet, found nonrelativistically by $v_g = \frac{d\omega}{dk} = \frac{dE}{dp} = v$
	\2 Wave behavior of photon particles are seen in low intensity diffraction, where it is the probability of hitting some location, where $E = hf$ per photon and $E_{tot}$ is proportional to $\vec{E}^2$
		\3 The electric field, $\vec{E}$, is the wavefunction for light, cooresponding to Schrodinger's equation for electrons
		\3 For matter waves, $P(x)dx = |\psi|^2dx$, where P is the probability at that location, acting as complex wavefunctions
\1 Since the wavefunction is nonzero for multiple values, there is uncertainty to the exact position of the electron, such that by the classical uncertainty relations, $\Delta x \Delta p \approx \Delta E \Delta t \approx \hbar$, called the Uncertainty Principle
	\2 For Gaussian distribution functions (total probability of 1), $\sigma_x \sigma_k = \frac{1}{2}$, where $\sigma$ is the standard deviation, such that uncertainty is defined as the standard deviation, such that $\Delta x \Delta p \geq \frac{\hbar}{2}$
	\2 As a result, particles cannot have 0 average kinetic energy, since for some box of length L, $\Delta x \leq L$, such that $\Delta p \geq \frac{\hbar}{L}$, ignoring the Gaussian $\frac{1}{2}$ term due to generally not being Gaussian and not affecting the order of magnitude
		\3 As a result, $(\Delta p)^2 = (p - \bar{p})^2_{av} = \bar{p^2} \geq (\frac{\hbar}{L})^2$, since $\bar{p}$ is equal to 0 for a symmetric box
		\3 Thus, $\bar{E} \geq \frac{\hbar^2}{2mL^2}$, called the zero point energy of the box, since the average energy must be $\geq 0$
	\2 For an electron with momentum p and distance r from the proton, $E = \frac{p^2}{2m} - \frac{ke^2}{r}$, $\Delta x = r$ such that $(\Delta p)^2 = \bar{p^2} \geq \frac{\hbar^2}{r^2}$, providing a radius for the minimum energy and a minimum electron energy at that radius
	\2 Since a precise measurement of the energy of a system requires infinite time, such that the mean decay time, or the lifetime ($\tau$), used as the time of measurement
		\3 As a result, the measurement of the wavelength of spectral lines has a degree of uncertainty due to the uncertain in measurement, with the natural line width being $\Delta E = \frac{\hbar}{\tau}$, denoted $\Gamma_0$
		\3 By the formula that $E - E_0 = \frac{hc}{\lambda}$, the derivative is taken such that $dE = -hc\frac{d\lambda}{\lambda^2}$, the range of $\lambda$ can be found for $\Delta E$
\1 It is seen that wave-particle duality is seen in matter and light, such that emission and absorption are particle-based, propogation is wave-based
	\2 This is also able to be thought of as interactions and observations of matter and light are particle-based, while predictions of observations are described by waves
		\3 Interactions cause the changing of the wavefunction, changing the propogation after
	\2 For wavelengths smaller than any objects, particle theory can describe propogation as well as wave theory, giving the same results, due to wavelike behavior being too small to be observed, while for time averages of energy/momentum exchange, wave theory works as well
\end{outline*}
\section{Chapter 6 - The Schrodinger Equation}
\begin{outline*}
\1 The Schrodinger Equation is a fundamental law, unable to be derived, and is non-relativistic, replaced by Dirac's Relativistic Wave Equation later
	\2 This wave equation is similar to that of light, $\frac{\partial^2\vec{E}}{\partial x^2} = \frac{1}{c^2}\frac{\partial^2\vec{E}}{\partial t^2}$, with solution $\vec{E} = \vec{E}_0 cos(kx - \omega t)$ to give the equation, $\omega = kc$, which is equivelent to $E = pc$
		\3 Similarly for electrons, since $E = \frac{p^2}{2m} + V$ is the energy of the electron where V is the potential energy of the electron, $\hbar \omega = \frac{\hbar^2 k^2}{2m} + V$, such that k and $\omega$ are not linearly related
		\3 This implies that for a harmonic electron wavefunction, the first time derivative is related to the second spacial derivative, and will involve potential energy
		\3 The wave equation must be consistent with the De Broglie relations, classical conservation of energy, squared as the probability of finding the particle at that location, and if potential is constant, energy and momentum is constant
			\4 As a result, an ideal form for V = 0 would be able to be reduced to $E = \frac{p^2}{2m}$, such that cos/sin fail, but a complex exponential works
		\3 The wave equation must also be linear with respect to the wavefunction, due to allowing constructive and destructive interference, such that each term is linear with respect to $\phi$ or some derivative of $\phi$
	\2 The equation in 1D states that $\frac{-\hbar^2}{2m} \frac{\partial^2\psi(x, t)}{\partial x^2} + V(x, t)\psi(x, t) = i\hbar \frac{\partial \psi(x, t)}{\partial t}$
		\3 For a free particle, such that $V(x, t) = V_0$ is constant, it is found that individual harmonics are not solutions, but a complex exponential is, such that $\psi = Ae^{i(kx - \omega t)}$ provides the previous equation
	\2 By the probabilistic interpretation of the wavefunction, $P(x, t)dx = |\psi(x, t)|^2dx = \psi^*(x, t)\psi(x, t)dx$, where $\psi^*(x, t)$ is the conjugate, such that P(x, t) is called the probability density, and $\psi(x, t)$ is called the probability density amplitude/probability amplitude
		\3 Thus, the normalization condition states that $\int^{\infty}_{-\infty} \psi^* \psi dx = 1$
	\2 Schrodinger then showed that energy quantization can be explained in terms of standing waves, called the stationary states, or eigenstates, of the particles, in which the potential energy is independent of time
		\3 For these situations, the solution is seperable, such that $\psi(x, t) = \gamma(x)\phi(t)$, producing ordinary derivatives instead of partial, with each side as a function of a single variable
		\3 Since they are single variable, they both must be equal to some separation constant, such that $\phi(t) = e^{\frac{-iCt}{\hbar}} = cos(2\pi \frac{Ct}{h}) - isin(2\pi \frac{Ct}{h})$, such that $f = \frac{C}{h}$, or C = E
			\4 With E in the spacial portion of the equation, it gives the time-independent Schrodinger equation, with the time dependent side replaced by $E\psi(x)$ whose probability density is found to be equal for this situation to the multivariable density
	\2 Wavefunctions must fit with the type of potential energy function, which is allowed to be discontinuous, solved seperately in each region, but require smooth joining at the point of discontinuity
		\3 The wavefunction itself must be continuous as well as the first derivative, such that the function is smooth, except possibly at the boundary (since if there is infinite potential energy, the wavefunction must be 0)
		\3 Both also must be finite and single-valued to have measurable quantities, and must follow the normalizataion condition
\1 The infinite square well is a frictionless wire with increasing potential from points near the end, until the maximum at the border of the box, able to be made arbitrarily steep and large potentials, such that $V(x) = 0 if 0 < x < L, V(x) = \infty$ otherwise
	\2 This problem is related to the classical vibrating string problem and is a good approximation to the motion of free electrons in a metal
	\2 As a result, the boundary conditions state that it must be 0 at both ends of the box, allowing quantization similar to the standing wave condition
		\3 For a standing wavelength of the particle, $E = n^2\frac{\pi^2\hbar^2}{2mL^2}$, able to be derived from the time-independent Schrodinger equation by providing $\psi(L) = Asin(kx)$, such that $k_n = \frac{n\pi}{L}$, providing quantized energy values/energy eigenvalues
			\4 n is the quantum number of the system, allowing the specific mode of the system to be determined
			\4 This also displays the minimum/zero-point energy, since n = 0 is only valid if the particle is not in the box
		\3 The normalization condition can then be used to get that $A_n = (\frac{2}{L})^{1/2}$, such that the eigenfunctions are $\psi_n(x) = \sqrt{\frac{2}{L}}sin(\frac{n\pi x}{L})$
		\3 This provides the probability of finding the particle at each location, depending on the total energy of the particle system
	\2 Classically, since there is no force except an infinitely large force at the edges, any speed and energy is possible, but by the uncertainty principle the velocity and position can't be found simultaneously, the minimum energy is below the energy uncertainty
		\3 In addition, as quantum number increases, the peaks approach infinite, similar to classical prediction
	\2 The complete wavefunction is found as $\psi_n(x, t) = \psi_n(x)e^{-i\omega t} = \psi_n(x)e^{-i(E_n/\hbar)t}$, such that by the identity, $sin(k_n x) = \frac{(e^{ik_nx} - e^{-ik_nx})}{2i}$, multiplied by the time portion to show that the standing wave is equal to two equal and opposite travelling waves
\1 For a finite square well, such that the bounds of the region have potential $V_0$, where the total energy is assumed to be less than $V_0$, such that within the well, the equation is the same as the infinite well, $\psi''(x) = -k^2\psi(x), k^2 = \frac{2mE}{\hbar^2}$, such that it is a sine/cosine within the well
	\2 Outside the well, it requires the complete time-independent equation, but is not required to be 0, rather only needing to follow the normalization condition
	\2 Since the second derivative/curvature and the derivative have the same sign, the function always curves away from the x-axis, rather than towards like a sine/cosine
		\3 This will appear in the form, $\psi(x) = ce^{\pm\alpha x} = ce^{\pm\sqrt{\frac{2m}{\hbar^2}(V_0 - E)}x}$
		\3 As a result, for most energy values of the wavefunction, it goes towards $\pm \infty$ as x approaches $\pm \infty$, such that it is invalid, such that it must move towards 0 as $x \to \pm \infty$, restricting energy states
		\3 It is found that there is a finite number of possible energies, decreasing in quantity as the voltage decreases, to 1 possible state at very low $V_0$
	\2 Since it is possible for the particle to be outside the well, kinetic energy appears to be negative, but if the probability is assumed to be negligable at $\Delta x = \alpha^{-1}$, the momentum uncertainty gives a minimum energy equal to the negative kinetic energy to cancel it
	\2 This applies to all functions with $E > V$ in the box, even if V varies such that it is not simple sinusoidal, with $E < V$ outside the box, restricting energy states
\1 The expectation value of f(x), $<f(x)> = \int^{\infty}_{-\infty} \psi^*f(x)\psi dx$, is the average value of f(x) expected to be obtained by the measurement of particles with the wavefunction (either the time-independent or time-dependent)
	\2 Operators are the set of functions replacing f(x), acting on the wavefunction to provide the weighted average, not always functions of x such as when the uncertainty principle prevents it, like momentum
		\3 The momentum x-component operator is $\frac{\hbar}{i}\frac{\partial}{\partial x}$, squared for the $p^2$ operator
		\3 The Hamiltonian/total energy operator in classical mechanics is $H = \frac{p^2}{2m} + V$, such that the Hamiltonian operator is found by replacing p by the p operator
			\4 As a result, the time independent Schrodinger equation can be replaced by $H_{op}\psi = E\psi$
		\3 The time dependent Hamiltonian operator is $i\hbar \frac{\partial}{\partial t}$, kinetic energy is $-\frac{\hbar^2}{2m}\frac{\partial^2}{\partial x^2}$, and the z component of angular momentum is $-i\hbar \frac{\partial}{\partial \phi}$
\1 The simple harmonic oscillator, such that $V(x) = \frac{1}{2}kx^2 = \frac{1}{2}m\omega^2x^2$, where k is the force constant, acting as the finite well, has a classical probability of $P(x)dx = c\frac{dx}{v}$, with any energy value possible
	\2 Due to the symmetic potential function, the probability must be symmetric, such that the wave function must be either symmetric or anti-symmetric
		\3 Parity operations are those flipping the x-axis, such that $x \to -x$, true for the Hamiltonian operator such that $H_op\psi(-x) = E\psi(-x)$
		\3 If there are multiple separate solutions for one energy state, it is called degenerate, doubly degenerate for two, though since it is found for the Schrodinger equation to be at most by a factor of -1 for odd parity ($\psi(-x) = -\psi(x)$)
	\2 It is found that $E_n = (n + \frac{1}{2})\hbar\omega$, such that the minimum energy is $E_0 = \frac{1}{2}\hbar\omega$, due to the minimum uncertainty principle energy
		\3 Thus, it is found that $\psi_n(x) = C_n e^{-m\omega x^2/2\hbar}H_n(x)$, where $H_n(x)$ is the Hermite polynomial of order n, symmetric for even n, antisymmetric for odd n, such that it decays outside the classically allowed region, oscillating within the region
		\3 The approximate length uncertainty can be thought of as the width under which the exponential term is greater than 1
	\2 It is also found that $\int^{\infty}_{-\infty} \psi^*_nx\psi_mdx = 0$ unless $n = m \pm 1$, called the selection rule, such that radiation can only change the energy of the oscillator by one at a time
\1 For an unbound state problem, such that E is greater than V(x) as $x \to \pm \infty$, the second derivative and wavefunction must have opposite signs to avoid moving towards infinity, curving to the x axis, such that any value of E is possible
	\2 While it is not normalizable over the entire domain, it is bounded within a specific range, such that $\int^b_a |\psi(x)|^2dx = \int^b_a \rho dx = \int^b_a dN = N$, where N is the number of particles in the interval
	\2 For some step potential, such that for $x < 0, V(x) = 0$ and for $x > 0, V(x) = V_0$, such that by the Schrodinger equation, for $x < 0, \frac{d^2\psi(x)}{dx^2} = -\frac{\sqrt{2mE}}{\hbar}$, and $x > 0, \frac{d^2\psi(x)}{dx^2} = -\frac{\sqrt{2m(E - V_0)}}{\hbar}$
		\3 For solutions of beams of particles moving to the right multiplied by the time portion, each regions solution is the sum of the travelling waves in each direction
		\3 There is assumed to be no leftward moving beam from the $x > 0$ side, such that the coefficient of that term is 0, with the functions for each side required to be continuous at x = 0
		\3 The coefficients of reflection R and transmission T are the relative rates by which particles are refleceted and transmitted, with $R = (\frac{k_1 - k_2}{k_1 + k_2})^2$, $T = \frac{4k_1^2}{(k_1 + k_2)^2}$, such that $T + R = 1$
	\2 As a result of Schrodinger's wave nature, unlike the classical idea that none would be reflected due to the change in potential, a portion of the particles are, depending on the change in wavenumbers, but not in the sign of the change
	\2 For $V_0 > E$, the equation is a real negative exponential in the potential region, such that the particle is able to somewhat permeate, appearing as if it has negative velocity by the uncertainty principles
	\2 For a barrier in which $V_0 > E$, as a result, the particle has the possibility of reaching all the way through the barrier, such that it is possible for it to tunnel through the higher potential then proceed through
		\3 It is found that the percent transmitted, $T = 16\frac{E}{V_0}(1 - \frac{E}{V_0})e^{-2\frac{\sqrt{2m(V_0 - E)}}{\hbar}L} \approx e^{-2\frac{\sqrt{2m(V_0 - E)}}{\hbar}L}$, where L is the length of the tunneling region
		\3 As a result, the probability of a standard barrier is the integral of the probability of transmission for each x value 
		\3 This is found to be the cause of the likelihood of spontaneous alpha decay, such that the force increases linearly as the particle gets further out of the nucleus, decreasing by $\frac{1}{r^2}$ outside
		\3 Tunnelling microscopes view metals as an infinite well, moving the servo close to the material, sending a voltage through the air to measure the light emitted from the tunnelling based on the height, to measure the surface layout and height
			\4 This is done with the current proportional to the transmission probability, keeping the tip-surface distance constant for each scan
	\2 Flux of particles, $J = \rho v = \frac{p}{m} \rho = \frac{\hbar k}{m} \rho$, such that for initial density of a single particle, $J_I = v, J_{R} = vR$, and $J_T = vT$
\end{outline*}
\section{Chapter 7 - Atomic Physics}
\begin{outline*}
\1 The time-independent Schrodinger Equation is expanded to 3 dimensions, such that $\frac{-\hbar^2}{2m}(\frac{\partial^2 \psi}{\partial x^2} + \frac{\partial^2 \psi}{\partial y^2} + \frac{\partial^2 \psi}{\partial z^2}) + V\psi = E\psi$
	\2 Since the standing wave bounds in one dimension are independent of the other dimensions, it is broken down to a product function, $\psi(x, y, z) = \psi_x(x)\psi_y(y)\psi_z(z)$
		\3 Thus, it is found that $E = \frac{\hbar^2}{2m}(k_1^2 + k_2^2 + k_3^2 = \frac{\hbar^2\pi^2}{2m}(\frac{n_1^2}{L_1^2} + \frac{n_2^2}{L_2^2} + \frac{n_3^2}{L_3^2})$, such that it has 3 quantum numbers, each from a different boundary
	\2 Thus, for a symmetrical cube, three energy state combinations are found for each energy state, such that it is triple-degenerate, split apart otherwise
	\2 For spherical coordinates, such that it simulates a Hydrogen atom, using a reduced mass, $\mu$, to compensate for a stationary center, the equation is found to be $\frac{-\hbar^2}{2\mu r^2}(\frac{\partial}{\partial r}(r^2 \frac{\partial \psi}{\partial r}) + \frac{1}{sin(\theta)}\frac{\partial}{\partial \theta}(sin(\theta)\frac{\partial \psi}{\partial \theta}) + \frac{1}{sin^2(\theta)}\frac{\partial^2 \psi}{\partial \phi^2}) + V(r)\psi = E\psi$
		\3 Within this equation, $\theta$ is along the z-axis, restricted to $\pi$, while $\phi$ is along the x-axis, restricted to $2\pi$
\1 For a hydrogen atom, the equation can be separated ($\psi = R(r)f(\theta)g(\phi)$) into the radial equation and the angular equation, $\frac{1}{R}\frac{d}{dr}(r^2\frac{dR}{dr}) + \frac{2\mu r^2}{\hbar^2}(E - V(r)) = -(\frac{1}{fsin(\theta)}\frac{d}{d\theta}(sin(\theta)\frac{df}{d\theta}) + \frac{1}{gsin^2(\theta)}\frac{d^2g}{d\phi^2}$ = l(l + 1), due to being separable
	\2 The angular equation can then be separated further, with the constant as $-m^2$, such that $g_m(\phi) = e^{im\phi}$, periodic, such that m must be integral, called the azimuthal function
	\2 Further, $f_lm(\theta) = \frac{(sin(\theta))^{|m|}}{2^ll!}(\frac{d}{dcos(\theta)})^{l + |m|}(cos^2(\theta) - 1)^l$, such that to be finite-valued/periodic, $|m| \leq l$
		\3 f is called the associated Legendre functions, while for m = 0, they are callec the Legendre polynomials, where the product of f and g is called the spherical harmonics
	\2 The angular momentum, $L = r x p$, can be found by separating momentum into radial and tangential, such that $L = rp_t$, where $p_r = \mu(\frac{dr}{dt}), p_t = \mu r(\frac{dA}{dt})$
		\3 Thus, kinetic energy can be shown to be $\frac{p^2}{2\mu} = \frac{p_r^2}{2\mu} + \frac{L^2}{2\mu r^2}$, such that $V_{eff} = \frac{L^2}{2\mu r^2} + V(r)$, such that $E = \frac{p_r^2}{2\mu} + V_{eff}(r)$, acting as an energy conservation formula, able to be used to form the Schrodinger equation
		\3 As a result, it is derived that $(p_r^2)_{op} = -\hbar^2 \frac{1}{r^2}\frac{\partial}{\partial r}(r^2 \frac{\partial}{\partial r})$, and $(L^2)_{op} = -\hbar^2(\frac{1}{sin(\theta)}\frac{\partial}{\partial \theta}(sin(\theta)\frac{\partial}{\partial \theta}) + \frac{1}{sin^2\theta}\frac{\partial^2}{\partial \phi^2})$
			\4 This is noted to be similar to the angular side of the separated Schrodinger equation, such that when operating on fg, it is equal to $\hbar^2l(l + 1)fg$, such that $L^2$ is quantized, where l is the angular momentum/orbital quantum number
			\4 By the same reasoning, it is found that $L_z = m\hbar$, where m is integral equal to or less than $\pm l$, such that L is space quantized, pointing in specific spacial directions, with m as the magnetic quantum number
		\3 The other dimensional operators can be derived similarly, but cannot be found definitively due to relying on both angles, which the uncertainty principle forbids
			\4 In addition, the uncertainty principle also forbids knowing more than one component of the angular momentum simultaneously
	\2 For the potential energy function, $V(r) = -\frac{Zke^2}{r}$, the radial equation is solved to find that it is well-defined only if the energy levels fit the Bohr energy levels, with n as the principle quantum number
		\3 The radial function as a result is found to be $R_{nl}(r) = A_{nl}e^{-\frac{Zr}{a_0n}}r^lL_{nl}(\frac{Zr}{a_0})$, where $L_{nl}(\frac{r}{a_0})$ are the Laguerre polynomials, and $a_0$ is the Bohr radius, $\frac{\hbar^2}{ke^2\mu}$
		\3 The energy is found as well to be $E = -\frac{1}{2}m_ec^2Z^2\alpha^2\frac{1}{n^2}$, equal to the Bohr energy levels, with $\alpha$ as the fine structure constants
		\3 Since the radial equation itself is $\frac{-\hbar^2}{2m_e}(\frac{d^2}{dr^2} + \frac{2}{r}\frac{d}{dr} - \frac{l(l+1)}{r^2})R(r) - \frac{Ze^2}{4\pi \epsilon_0 r}R(r) = ER(r)$, such that the $\frac{l(l + 1)}{r^2}$ doesn't allow the particle to be at r = 0 if $l \neq 0$ (angular momentum is nonzero)
			\4 This is called the centrifugal barrier, preventing the atom from decaying into the origin
	\2 As a result, the Schrodinger model predicted the total orbital angular momentum of the energy levels differently from the Bohr model, but agreed with Bohr on the quantization of the z-component of angular momentum
		\3 Bohr also did not predict degeneracy of energy states, the uncertainty principle, and did not predict the 3D wavefunction
	\2 The energy depends only on n due to the inverse square force causing it to only depend on the major axis, not eccentricity, and since higher l provides lower eccentricity, and since m provides the z-component, which has no preferred direction
		\3 The l states for each n are described by the letters S (l = 0), P, D, F, G, continuing alphabetically, while the n states are called K, L, M, ... shells
		\3 For some absorbed photon, the selection rules state that m can change by 0, 1, or -1, while l must change by either 1 or -1 to conserve angular momentum, due to photons having intrinsic angular momentum of $\pm \hbar$
		\3 As a result, there is a series of degenerate states ($0 \leq l \leq n -1, -l \leq m \leq l$) for each possible energy level
	\2 For the ground state, l and m are 0 and $L_{10} = 1$, such that $\psi_{100} = C_100e^{\frac{-Zr}{a_0}}$, such that since $dV = r^2sin\theta d\phi d\theta dr$, $C_100 = \frac{1}{\sqrt{\pi}}(\frac{Z}{a_0})^{3/2}$
		\3 Thus, the electron is most likely to be found at the origin, but the radius at which it is most likely to be found is the Bohr electron radius $a_0$, though it can be found at any radius
		\3 Electrons tend to be thought of as a result as a charged cloud of charge density, equal to the probability multiplied by the charge
		\3 It is also noted that the angular momentum of the ground state is 0, unlike the Bohr model which assumed the minimum angular momentum was $\hbar$
		\3 The set of n and l quantum numbers is called the electron configuration
	\2 For the excited state n = 2, $\psi_{200} = C_{200}(2 - \frac{Zr}{a_0})e^{\frac{-Zr}{2a_0}}, \psi_{210} = C_{210}\frac{Zr}{a_0}e^{\frac{-Zr}{2a_0}}cos(\theta)$, and $\psi_{20\pm 1} = C_{21 \pm 1}\frac{Zr}{a_0}e^{\frac{-Zr}{2a_0}}sin(\theta)e^{\pm i\phi}$
		\3 The maximum for l = 1 is the Bohr orbit, $r = 4a_0$, while for l = 0, it is slightly higher, but still close
		\3 It is found that for some n, the wavefunction is greatest near the origin when l is small, such that it gets more likely to be near the originn
		\3 It is also noted that for l = 0, the orbital density is spherically symmetric, while otherwise, it depends on the angle
\1 Electron spin/intrinsic angular momentum is used to explain the fine structure of the spectral lines (such that there are several infinitessimally seperate spectral lines), created analogous to Schrodinger's angular momentum ($|\vec{S}| = \sqrt{s(s + 1)}\hbar$)
	\2 On the other hand, Pauli suggested that there could only be two magnetic spin numbers, $\pm \frac{1}{2}$, such that s must be restricted to $\frac{1}{2}$
	\2 Magnetic moment is defined by $\vec{\mu} = \vec{g_LI}A = \frac{g_Levr}{2} = \frac{-g_Le\vec{L}}{2m_e} = g_L\frac{q}{2M}\vec{L} = g_L\sqrt{l(l + 1)}\mu_B$, where $\mu_B$ is the Bohr magneton, equal to $\frac{e\hbar}{2m_e} = 9.27 * 10^{-24} J/T$
		\3 In addition, $\mu_z = -g_Lm\mu_B$, with $g_L$ as the gyromagnetic ratio/g-factor, taking into account the complex situation, 1 for a single revolving electron
		\3 This shows that the quantiztation of angular momentum creates the quantization of magnetic moments
		\3 The Stern-Gerlach Experiment measured the quantization of the z-component, placing am atomic beam in a magnetic field, checking the collision pattern ($F_z = \frac{\partial B_z}{\partial z}\mu_z$)
	\2 The potential energy for a megnetic moment is found to be $U = -\vec{\mu} \cdot \vec{B}$, such that for B in the z-axis, $U = -\mu_zB$, causing the slight energy differences for the possible $\mu_z$ values creating the fine structure
	\2 It is found that the g-factor for an electron is approximately 2, rather than the 1 expected for an electron revolving normally, showing the distinction $(\mu_z = -\mu_B(m_l + 2m_s))$
	\2 Thus, the full form of the wave-equation is $\psi = C_1\psi_{100\frac{1}{2}} + C_2\psi_{100\frac{-1}{2}}$, with $C_1^2 = C_2^2 = \frac{1}{2}$ for an atom not in a magnetic field
		\3 As a result, $\psi_{100\frac{1}{2}} = \psi_{100}\chi_{\frac{1}{2}}$, with the quantized spin and spin component as eigenvalues of $\chi$, with the general Hamiltonian as $H = \frac{p_1^2}{2m_1} + \frac{p_2^2}{2m_2} + ... + V(\vec{r_1}, \vec{r_2}, ...) + f(\vec{r_1}, \vec{r_2}, ..., \chi_{1}, \chi{2}, ...)$
		\3 For noninteracting, time-independent particles, f = 0 and V is the sum of the individual potentials, $V_n(\vec{r_n}, \chi_n)$, such that the total energy is the sum of the individual energiees
\1 While classically, the sum of the revolving and intrinsic angular momentum can be any value based on the vector angles, the quantization limits it, such that $|\vec{J}| = |\vec{L} + \vec{S}| = \sqrt{j(j + 1)}\hbar$
	\2 j is allowed to be either the sum of l and s (parallel) or the positive difference of them (antiparallel), or integral moves between the two, with the z component given by $J_z = m_j\hbar = (m_l + m_s)\hbar$, where $-j \leq m_j \leq j$ as a result
		\3 Spectroscopic notation is denoted by $n^{2s + 1}L_j$, where L is the angular momentum state of the electon (S, P, D, F, G, ...)
			\4 This is extended to atoms with L as the total angular momentum, and with S as the total spin of the orbital
		\3 As a result, the fine structure splitting can also be explained by different values of j causing different potential energy, such that antiparallel J has lower energy than parallel
		\3 In addition, within an external magnetic field, the energy is slightly different for each $m_j$, splitting the lines further, called the Zeeman effect
	\2 Similarly, the sum of the total angular momentum of two atoms has a value of j with a maximum of the sum, the minimum of the positive difference
	\2 Since relativistically, the fine-structure depends only on n and j, rather than on l, different l states have the same energy, and since there is only a single l value for n = 1, S, $2^2S_{1/2}$ is metastable, due to the selection rule requiring a change in the l value
		\3 On the other hand, it was discovered that $2^2S_{1/2}$ had slightly more energy than $2^2P_{1/2}$ due to electron interactions with their own electric fields (virtual electron-positron pairs), allowing the Lamb shift transition with photon emission
\1 The Schrodinger Equation for multiple particles, assumed to be electrons, written time-independently as $-\frac{\hbar^2}{2m}\frac{\partial^2\psi(x_1, x_2)}{\partial x_1^2} -\frac{\hbar^2}{2m}\frac{\partial^2\psi(x_1, x_2)}{\partial x_2^2} + V\psi(x_1, x_2) = E\psi(x_1, x_2)$
	\2 Potential must take into account both the independent potential of each and the interacting potential, both assumed to be 0 for the infinite square well
	\2 The solution is written as a product solution, such that for states m and n respectively, $\psi_{mn}(x_1, x_2) = \psi_m(x_1)\psi_n(x_2)$
		\3 The probability of finding particle 1 at $dx_1$, particle 2 at $dx_2$ is $|\psi(x_1, x_2)|^2dx_1dx_2 = (|\psi_m(x_1)|^2dx_1)(|\psi_n(x_2)|^2dx_2)$
		\3 For identical particles as a result, the interchanging of the particles would result in the same overall wave probability ($|\psi(x_1, x_2)|^2 = |\psi(x_2, x_1)|^2$), considered a symmetric wavefunction if they are equal if reversed, antisymmetric if negative
	\2 Since identical particles are assumed to be indistinguishable within quantum mechanics, the wavefunctions must be the linear combination of both with equal probability, such that $\psi = C(\psi_m(x_1)\psi_n(x_2) \pm \psi_m(x_2)\psi_n(x_1))$, depending if symmetric or not
	\2 It is found that electrons can only have antisymmetric wavefunctions, such that if all wavenumbers are the same, then the wavefunction at any point is 0, forbidden by Pauli's exclusion principle
\1 Non-hydrogen atoms can have their energy approximated by the Fermi energy model, such that the Fermi sphere is the atom, with the radius and particle density providing eachother and the Fermi energy
	\2 Thus, $E_F = \frac{\hbar^2}{2m}(3\pi^2n_f)^{2/3}$ for an atom, where $n_f$ is the fermi particle density
\1 For a Helium atom, potential energy is the sum of the potential with the nucleus and $V_{int} = \frac{ke^2}{|\vec{r_2} + \vec{r_1}|}$, preventing the separation of equations for each electron
	\2 If ignoring the interaction term, the energy of each of the electrons is individual, with total energy as the sum of the individual electron energy states, the wavefunction as the product of the standard electron wavefunctions
		\3 This is because the charge of a sphere is concentrated at the center, such that the effective charge is just the sum of the nucleus and other electrons
			\4 On the other hand, inner shell electrons have virtually no effect from the outer electron shel interactions
		\3 First-order perturbation theory is then done to correct for the interactions, calculating the expectation value of the interaction potential, adding it to the ground state energy to get the real energy
			\4 Thus, $V_{int} = \frac{e^2}{4\pi\epsilon_0}\intd^3r \frac{|\phi(\vec{r_2})|^2}{|\vec{r_1} - \vec{r_2}|}$
			\4 This model further assumes that the Hamiltonian of the respective electrons in a shell are equal, after which a spin modification is done, and the interaction energy is added
			\4 This creates the correct energy of the electrons, but ignores the coupled motions of the electrons, by Pauli and Coulumb
	\2 The first ionization potential is the difference between the ground state energy with two electrons and a single election, rather than the electrostatic potential of a single electron
\1 For a lithium ion, the energy of the third electron depends on the l state, due to the force not being an inverse square force, but rather depending on the negative K-shell charge, spread over the volume at the radius $\frac{a_0}{Z}$
	\2 The electron acts as an electron in either the 2S or 2P state of a hydrogen atom, due to the electrons in the K shell shielding the nucleus, such that $Z_{eff} = 1$ and a far higher radius
		\3 On the other hand, in the 2S state, it is more likely to be near the center, below the shielding, such that the energy state is lower, true generally, such that for the electron in 2S, $Z_{eff} = 1.3$
	\2 For beryllium, since the effective force within the Z-shell is greater, the energy is lower than that of the third electron in lithium
	\2 The electron of boron has higher energy than beryllium, due to the lack of penetration of the 2S subshell, with lower energy for subsequent electrons due to higher nucleus charge, though with neon still having higher energy than helium 
	\2 Thus, for 3S electrons, the energy is drastically higher due to the shielding of the L shell, proceeding lower as subsequent electrons are added into 3S then 3P
	\2 The shielding by the M shell is strong enough as well, that there is less energy for 4S than for 3D, such that the former is filled first, before filling 3D in the transition elements
	\2 The higher l values of electrons in multi-electron atoms causes the further splitting of energy levels by l values, due to the effective charge being lower for higher l values (less tightly bound/higher energy)
\1 Visible light excited transitions typically are only able to take place in the outer electron shells, the excitation energy being approimately that of the electron in hydrogen, due to shielding
	\2 The different states have spin-orbit energy difference, such that there is slightly more energy in higher angular momentum combinations
	\2 By the selection rule ($\Delta l = \pm 1, \Delta j = -1, 0, 1$), transitions to and from the S state have double lines due to the energy difference from the doublet P state (two possible initial l values), while other transitions have three possible lines (called a compound double)
		\3 Two of the lines are far closer than thae other, due to the distance between the upper and lower P levels being further apart
\end{outline*}
\section{Chapter 8 - Statistical Mechanics}
\begin{outline*}
\1 Classical statistical physics contains a large number of identical, but distinguishable particles, able to be tracked, with the energies of each given by the Boltzmann distribution function, $f_B(E) = Ae^{-E/kt}$
	\2 The function without A is called the Boltzmann factor, while k is the Boltzmann constant, $1.381 * 10^{-21} J/K$, with A as the normalization constant
	\2 $n(E) = g(E)f_B(E)$ is the probability of a particular state/number of particles, where g is the statistical weight/degeneracy of the energy state/density of states, with E as a continuous variable
	\2 Maxwell's Distribution of Molecular Speeds assumes the velocity in each direction is distinct, with $F(v_x, v_y, v_z) = f(v_x)f(v_y)f(v_z)$, where $f(v_x) = Ce^{-\frac{mv_x^2}{2kT}}$, such that $n(v)dv = 4\pi v^2 N f(v)dv$, such that $4\pi v^2$ is the surface area of the sphere of velocities, acting as the degeneracy
		\3 It is found that the normalization constant for the three dimensional equation is $(\frac{m}{2\pi k T})^{3/2}$
		\3 As a result, $v_{maxProb}(T) = \frac{2kT}{m})^{1/2}$, $<v> = \frac{1}{N}\int^{\infty}_{0} v n(v) dv = \sqrt\frac{{8k_BT}{\pi m}}$, and $v_{rms} = \sqrt{\frac{1}{N}\int^{\infty}_0 v^2n(v)dv} = \sqrt{\frac{3k_bT}{m}}$
		\3 Similarly, by the relationship between velocity and kinetic energy, $n(E)dE = \frac{2\pi N}{(\pi k T)^{3/2}}E^{1/2}e^{-E/kT}$, such that $<E> = \frac{3}{2}kT$, and $c_V = (\frac{\partial u}{\partial T})_V = \frac{3}{2}N_Ak_B = \frac{3}{2}R$
	\2 The Equipartition Theorem states that in equilibrium, each degree of freedom contributes $\frac{1}{2}kT$ to the average energy per molecule, where a degree of freedom is a coordinate/velocity component that is squared in the expression of total energy
		\3 As a result, $C_V = \frac{1}{2}N_Ak_B = \frac{1}{2}R$ for each degree of freedom
		\3 Thus, rotational kinetic by axis, translational by axis, and both kinetic and potential vibrational by axis are degrees of freedom
			\4 Thus, for a monotomic atom, assuming no vibration, it should be 3R, called the Dulong-Petit value, but it rather found rise as the temperature rises, starting with purely translation, gaining rotation only in the two non-atomic axes, and increasing vibrationally
			\4 Most diatomic molecules then decay before it is able to reach $\frac{7}{2}R$, but the theorem still doesn't explain the increase with temperature
		\3 Solid heat capacity of approximately 3R was explained by three axes of vibrational potential and three axes of vibrational kinetic, true at high temperatures, but dropping as temperature drops
			\4 This is not different for metals, rather than the expected three translational kinetic energy degrees of freedom combined with the six vibrational degrees
\1 Quantum statistical mechanics attempted to correct classical by the fact that particles were indistinguishable and identical, providing the Bose-Einstein and Fermi-Dirac distribution functions for bosons and fermions respectively
	\2 Bosons, or particles with an integral spin, do not obey the exclusion principle, distributed by $f(E) = \frac{1}{e^{\alpha}e^{E/kt} - 1}$, where $e^{\alpha}$ is the normalization constant
		\3 Fermions are those with a half integer spin that obey the exclusion principle, given by $f(E) = \frac{1}{e^{\alpha}e^{E/kt} + 1}$
	\2 These are similar to the Boltzmann distribution, differing only by the $\pm 1$ in the denominator, approaching the Boltzmann distribution as the energy increases or as $\alpha >> \frac{E}{kT}$
		\3 Since bosons must have symmetric wavefunctions and fermions, asymmetric, $\psi = \frac{1}{\sqrt{2}}(\psi_n(1)\psi_m(2) \pm \psi_n(2)\psi_m(1))$, the function of both of the same state is $\psi = \frac{\sqrt{2}}\psi_n(1)\psi_n(2)$
		\3 By the Boltzmann distribution, for distinguishable particles, the function doesn't have the $\sqrt{2}$ factor, due to not being the linear combination
		\3 Thus, it is found that the presence of a boson in a quantum state improves the probability others would by in the state from the classical probability, while fermions prevent other fermions from being found in the same state
	\2 The Boltzmann distribution is usable when the particles are distinguishable, such that the distance between them is far greater than the De Broglie Wavelength
		\3 Since, $\lambda = \frac{h}{2mE_k} = \frac{h}{3mkT}$, since $<E_k> = \frac{3kT}{2}$, and it is found that $<d> = (\frac{V}{N})^{1/3}$, such that this is valid when $\frac{N}{V}\frac{h^3}{(3mkT)^{3/2}} << 1$
		\3 Within an infinite well, $E_n = E_0(n_x^2 + n_y^2 + n_z^2)$, such that it forms an eighth of a sphere with radius $(\frac{E}{E_0})^{1/2}$, where each quantum number must be an integer, such that $N = \frac{1}{8}(\frac{4\pi R^3}{3})$, or $\frac{N}{2}$ for two possible spin-states
			\4 The infinite well radius is called the Fermi radius, acting as the maximum energy state an electron exists in within the well
			\4 Thus, this can be related to the particle density within the Fermi sphere, such that $E_F = \frac{\hbar^2}{2m}(3\pi^2n_f)^{2/3}$
			\4 Thus, $g(E) = \frac{dN}{dE} = \frac{2\pi(2m)^{3/2}V}{h^3}E^{1/2}$, called the density of states, where for an electron/fermion, it is multipled by 2 for each of the spins, otherwise not, after which it can be integrated such that $N = \int^{\infty}_0 g(E)f(E)dE$, assuming a high enough number of states that it can be integrated
			\4 As a result, for a fermion, $e^{-\alpha} = \frac{Nh^3}{2(2\p im_ekT)^{3/2}V}$, twice that for a boson
	\2 For liquid helium, there is approximately one particle per quantum state at such low temperatures, such that Boltzmann did not apply, instead treating it as an ideal gas under Bose-Einstein distribution, applying below 2.17 K
		\3 It was found to increase in density until that point, called the lambda point, apparently as a phase transition, called Helium I above, Helium II below (made of Helium I and superfluid Helium, which had viscosity 0)
			\4 As a result, the density of Helium II was the sum of Helium I and superfluid helium, remaining constant due to the superfluid increasing as the temperature drops
			\4 Superfluid helium was considered to be ground state helium atoms, only exerting van der Waals forces, combined with the low density to act as an ideal gas
		\3 The distribution must have $\alpha \geq 0$, such that the number of particles is non-negative for all temperatures, integrated over all positive real numbers, where the ground state is assumed to be E = 0
			\4 As temperature drops past the critical temperature, such that $\alpha = 0$, it is impossible to normalize
			\4 Since g(E) = 0 for E = 0, it ignores ground state particles, irrelevent for fermions due to a maximum of 2 in that state, but causing the lack of normalization
			\4 As a result, it can be normalized if $N = N_0 + \int_0^{\infty} n(E)dE$, where $N_0 = \frac{g_0}{e^{\alpha}e^{E_0/kT} - 1} = \frac{1}{e^{\alpha} - 1}$
			\4 Thus, $\frac{N_0}{N} \approx 1 - (\frac{T}{T_c})^{3/2}$, such that as the temperature approaches 0, the condensate approaches the entire material
		\3 Atoms, especially in their ground states, tend to be bosons, although made up of fermions, though in small spaces, the interparticle interaction creates fermion behavior, making the production of condensates difficult
			\4 On the other hand, when produced, it begins to exhibit macroscopic quantum wavefunction behavior in terms of appearance and properties, forming coherent matter acting like lasers
\1 Planck's Law can be derived as a photon gas of bosons, and since photons are created and destroyed constantly, there cannot be a normalization integral to solve for $\alpha$
	\2 Thus, $\alpha = 0$ and by the density of states function, $g(E)dE = \frac{8\piVE^2dE}{c^3h^3}$, such that since $u(E)dE = \frac{Eg(E)f(E)dE}{V}$, $u(f)df = \frac{8\pif^3hdf}{c^3(e^{hf/kT} - 1)}$, proving Bose-Einstein distribution
		\3 As a result, this can be derived both by Boltzmann distribution for distinguishable electromagnetic waves and Bose-Einstein for indistinguishable particles, showing wave-particle duality
	\2 Einstein found that the low heat capcity of solids at low temperatures/failure of the equipartition theorem was due to quantization of energy states, extending it to matter
		\3 He assumed that all molecules can oscillate with the same frequency f in each direction, such that each molecule acted as 3 1D oscillators, each with energy values along $E_n = nhf$
		\3 By the Boltzmann distribution, $<E> = \int^{\infty}_0 En_B(E)dE = \frac{hf}{e^{hf/kT} - 1}$, such that at high temperatures, $<E> = kT$, such that total energy of N particles is $U = 3N<E>$
			\4 Thus, it is calculated by $C_V = \frac{dU}{dT} = 3R(\frac{\hbar\omega}{k_bT})^2\frac{e^{\hbar\omega/k_BT}{(e^{\hbar\omega/k_BT} - 1)^2}} = 0$ as $T \to 0$, $= 3R$ as $T \to \infty$
		\3 The critical temperature at which classical theory works approximately is $T_E = \frac{hf}{k}$ by Einstein theory, at which point changes in E create approximately no change in the Boltzmann distribution
			\4 As a result, beyond the Einstein temperature, classical theory is valid due to energy acting approximately continuous
			\4 For hard solids, the binding forces are stronger, such that the oscillation frequency and Einstein temperatures are higher
	\2 The Einstein theory gave slightly low results of the heat capacity, fied by Debye theory, who assumed that the molecules could vibrate at a range of frequencies up to the maximum Debye frequency, $f_D$
		\3 This defined a new temperature at which classical theory worked, as the Debye temperature, $T_D = \frac{hf_D}{k}$
		\3 He also argued that the number of possible frequencies was limited at maximum to the number of degrees of freedom
	\2 As a result, for quantized vibrational energy as $E = nhf$, with f as the frequency of vibration, the critical temperature of vibration is defined as the Debye/Einstein temperature, $E_v = \frac{hf}{k}$
		\3 The rotational energy distribution can be defined by quantized rotational energy, such that the energy distribution is proportional to $\frac{e^{-l(l + 1)h^2/8\pi^2 IkT}}$, with $T_R = \frac{E_R}{k} = \frac{h^2}{8\pi^2 Ik}$
			\4 As a result, it is found for monotomic gases, the moment of inertia is low enough that the rotational temperature is too high to be relevant
\1 The free electron theory was used to explain conduction, but fails due to not having the expected $\frac{3kT}{2}$ added average translational kinetic energy from collision above the degree of freedom 3kT
	\2 This is explained by the indistinguishability of the electrons, such that they are defined under Fermi-Dirac distribution rather than expected Boltzmann 
	\2 By the constant Fermi energy, $f_{FD}(E) = \frac{1}{e^{\alpha + \frac{E}{kT}} + 1} = \frac{1}{e^{\frac{E - E_F}{kT}} + 1}$, such that $\alpha = \frac{-E_F}{kT}$
		\3 For a 1D infinite well of electrons, g(E) = 2, such that $n(E) = 2f_{FD}(E)$
		\3 As a result, for any temperature, $E = E_F$, $f_{FD} = \frac{1}{2}$, while for 0 K, if $E < E_F$, f(E) = 1, $E > E_F$, f(E) = 0
		\3 Thus, the Fermi energy is the maximum energy state filled at 0 K, while at higher temperatures, electrons in energy states within approximately kT of the Fermi are somewhat dispersed in the kT above
			\4 As the temperature gets vastly higher, even those further from kT below the Fermi energy move above, such that $f(0) < 1$
			\4 On the other hand, only the higher energy electrons tend to move to higher states, due to lower ones being unable to gain enough energy to surpass the Fermi energy
		\3 By the density of states for electrons, $n_{FD}(E) = \frac{\pi}{2}(\frac{8m}{h^2})^{3/2}\frac{VE^{1/2}}{e^{(E - E_F)/kT} + 1}$
	\2 The quantum degenerate fermion gas state is that of the electrons filling up all states up to the Fermi energy, occuring gradually at lower temperatures
\end{outline*}
\section{Chapter 9 - Molecular Structure and Spectra}
\begin{outline*}
\1 Molecular bonds are generally ionic, covalent, dipole-dipole, or metallic, created stabally to lower the total energy of the atom from the total energy of the individual parts seperate, caused by electrostatic forces and Pauli's exclusion principle
	\2 The additional atomic energy is made up of electron kinetic energy, nuclear kinetic energy
	\2 The Born-Oppenheimer Approximation is based on the far higher mass and lower speed of the neucleus, finding a seperation or nucleus arrangement, solving the Schrdoinger equation for the electrons
		\3 The ground state energy as a function of nuclear seperation is then analyzed for the minimum, calculating excited, rotational, and vibrational states based on the found seperation distance
\1 The ionic bond is generally the strongest, with the seperate energy assumed to be 0 eV, while the ionization energy of $K^+$ is 4.34 eV, while Cl has electron affinity (energy released by acquisition of one electron) is 3.62 eV, such that the energy needed to create it (net ionization energy) is 0.72 eV
	\2 The release of energy is due to the extra electron penetrating the outer shell, such that it is attached to a net positive charge
	\2 Since the electrostatic potential energy is $\frac{-ke^2}{r}$, such that at some low distance, the potential is high enough to create the ion
		\3 While the atoms would classically keep moving together, exclusion principle repulsion occurs due to the overlapping of wavefunctions of electrons with the same energy states as they move close enough together
		\3 Since the electrons overlap, they require moving into a higher energy state to avoid overlapping, such that the total energy increases, even as the electrostatic potential decreases
			\4 The total energy of the ssytem can be written as $U(r) = \frac{-ke^2}{r} + E_{totalIon} + E_{exclusion} = \frac{-ke^2}{r} + E_{ion} - E_{affinity} + \frac{A}{r^n}$, where A and n are molecular constants
			\4 This ignores the small, but positive zero-point energy, and the small, but negative van der Waals attraction (due to induced dipole moments)
	\2 As a result, the minimum energy is the dissociation energy, with the radius at that point as the equilibrium seperation, for ground states of molecules
		\3 For excited electrons of molecules, the total potential energy of the curve is higher and more spread over the radii
\1 Covalent bonds are used when the net ionization energy is too high to allow a negative energy at any distance, based on the symmetry of two infinite square wells in the same state, either symmetric or antisymmetric
	\2 As the wells get close enough together, symmetric wavefunctions have a higher likelihood of being found in between the wells (the bonding orbital), while the anti-symmetric approaches the first excited state in structure and energy
		\3 For an antisymmetric wavefunction, the electrons are between the atoms enough to allow electrostatic force to hold the molecules together, such that there is no minimum energy (antibonding orbital)
	\2 The Hamiltonian of the system is $H_{op} = \frac{p_{op}^2}{2m} + ke^2(-\frac{1}{r_1} - \frac{1}{r_2} + \frac{1}{r_0})$, where $r_0$ is the distance between atoms, and $r_1/r_2$ is the distance between electrons
		\3 Due to the lack of multiple electrons, there is no exclusion-principle repulsion
		\3 Thus, for the symmetric wavefunction, $E_1 = -13.6 Z^2 = -54.4 eV$ for $H^2_+$, approaching -13.6 eV as the distance approaches infinity, while the anti-symmetric approaches -13.6 eV as the first excited at both 0 and infinity
		\3 It is found as a result that the binding energy is 2.7 eV
	\2 For an $H_2$ system, the symmetric total wavefunction is the one with a spacially symmetric portion, antisymmetric spin, both approaching 2 * -13.6 as the distance goes to infinity
		\3 In this case, due to two molecular states, the energy goes to infinity as r goes to 0, but the symmetric form has a negative minimum
		\3 Since the molecular orbitals are subject to the Pauli exclusion principle, the electrons are considered bonded by the angular momentum state, called a saturated bond if filled, due to further electrons requiring higher energy states
		\3 It is found in this case that there is a lower equilibrium separation, with higher binding energy of 4.5 eV, due to increased charge binding
	\2 The amount of covalent and ionic bonding is determined by the electric dipole moment, $p_{ion} = er_0$, where the percentage of ionic bonding is $\frac{p_{meas}}{p_{ion}}$
\1 Dipole-dipole/molecular bonding is due to electrostatic forces, pulling molecules and non-bonding atoms together, causing solid and liquid forms, with the electric field of a dipole as $\vec{E} = k(\frac{\vec{p}}{r^3} - \frac{3(\vec{p} \cdot \vec{r})}{r^5}\vec{r})$
	\2 The dipole, $\vec{p} = qa$, where a is the length between the charges, such that as $r >> a$, $E_d = \frac{k\vec{p}}{r^3}$
	\2 As a result, another dipole will have potential energy, $U = -\vec{p_{other}} \cdot \vec{E}$, orienting itself along the electric field lines as a result
		\3 Thus, there is attraction between permanent electric dipoles/polar molecules, called a hydrogen bond when involving hydrogen, acting as sharing a proton, used to hold ice and DNA structure, and produce friction and surface tension
		\3 In addition, molecules with higher dipoles cam be assumed to have a higher boiling and melting point as a result
	\2 Nonpolar molecules are able to be polarized by a field of a dipole, such that $\vec{p_{induced}} = \alpha \vec{E_d}$, where $\alpha$ is the polarizability
		\3 As a result, the resulting potential energy, $U = -\vec{p_{induced}} \cdot \vec{E_d}$, producing an attractive force
		\3 This is able to exist between two nonpolar molecules, due to the average square dipole moment being nonzero due to constant electron movement, called the van der Walls/London dispersion force
\1 Similarly to an atom, molecules can absorb and emit radiation, seperated into three types of energy levels, electronic, vibrational (1/10x), and rotational (1/1000x)
	\2 Rotational kinetic energy, $E = \frac{1}{2}I\omega^2 = \frac{L^2}{2I}$ has quantized angular momentum, such that $E = l(l + 1)\frac{\hbar^2}{2I}$, where $E_{0r} = \frac{\hbar^2}{2I}$, or the characteristic rotatinal energy
		\3 As a result, transitions in the angular momentum produce the rotatioanl spectrum, though it is only able to be seen in dipoles, by the selection rule, $\Delta l = \pm 1$
		\3 For diatomic molecules, since $I = m_1r_1^2 + m_2r_2^2 = \mu (r_1 + r_2)^2 = \mu r_0^2$, with $r_{1/2}$ as the distance from the centers of mass
			\4 The mass is simplified by the unified mass unit, u, equal to $1.6605 * 10^{-27} kg = \frac{1 g}{N_A}$
		\3 Due to the low energy, kinetic energy of collisions is often enough to excite molecules to a higher rotational level, unlike electronic and vibrational
	\2 Vibrational energy levels are measured by the quantum simple harmonic oscillator, such that $E_v = (v + \frac{1}{2})hf$, where v is the vibrational quantum number with selection rule $\Delta v = \pm 1$
		\3 In terms of the force constant, k, it is found that $f = \frac{1}{2\pi}\sqrt{\frac{k}{\mu}}$
		\3 As the v gets higher, the levels become more spaced out, due to the approximation failing to account for increasingly rising potential
		\3 In addition, the selection rule fails for vibrational during an electronic state transition, with the fundamental frequency of vibration different from distinct electronic energy states
	\2 Spectroscopy is generally on the vibration-rotation spectrum in the ground electronic state, such that for $\Delta v = 0, \pm1$, $\Delta l = \pm 1$
		\3 This selection rule only applies to those with a permanent dipole moment, such that those without are unable to have transitions that emit radiation without an electronic transition
		\3 As a result, the change in energy for some absorption is $hf \pm n2E_{0r}$, where n is the final l value for increasing l, or the initial for falling l
		\3 The different isotopes result in slightly different moments of inertia, forming a double peak structure
		\3 The frequencies of the different peaks are due to the probability of the different vibrational states, resulting from the thermal energy, such that $n(E_l) = g(E_l)e^{-E_l/kT}$, where $E_l = \frac{1}{2}hf + l(l+1)E_{0r}$ for the ground vibrational state
			\4 The degeneracy of a rotational state is 2l + 1, such that the density is not greatest at the l = 0 state, instead peaking at $l_{max} = \frac{1}{2}(\sqrt{\frac{4kT}{h^2/mr^2}} - 1)$
		\3 It is noted that the moment of inertia is not constant, but rather as the rotational level increases, the moment of inertia does as well, such that the energy between the levels get closer
			\4 There is also a peak missing for the l = 0 value where it would hypothetically drop
			\4 On the other hand, for polyatomic atoms, it is possible for a vibrational energy change without a rotational, called the Q branch
		\3 For changes in the electronic spectrum, there is no selection rule for v, though $\Delta l = \pm 1$ still
\1 Scattering is either elastic/Rayleigh (without a frequency change) or inelastic, the former caused by electrons absorbing the energy, then oscillating at the same frequency in phase with the original wave
	\2 While there is a Compton shift, for a large wavelength, it is insignificant and cannot be measured
	\2 Inelastic scattering is called Raman scattering, either greater or less depending on if it excited or deexcites the atom from a vibrational/rotational state
		\3 It is found that each Raman line is seperate by a specific interval, $\Delta f$, characteristic of the scattering molecule, providing the rotational levels of the molecule
		\3 The selection rule for the rotational quantum number is $\Delta l = 0, \pm 2$, the former of which is Rayleigh
			\4 This is as a result of rising two, then falling one, or falling two then rising one, or vice versa
	\2 Absorption/dark-line spectra provides further information about the energy levels, generally only showing the Lyman spectra, due to atoms generally beginning at their ground/low-level states
		\3 The probability of a light wave of the correct frequency being absorbed by the atom per unit time is $B_{n_in_f}u(f)$, where u is the energy density per unit frequency and B is Einstein's coefficient of absorption
		\3 Later emission moving back to a lower energy state is calleed fluorescence, where if it moves to the first excited state, it is called resonance absorption, with the meission called the resonance radiation
			\4 The emissions and the incident have no cooresponding direction or phase, with each particular drop having a specific probability based on the Einstein coefficient of spontaneous emission, independent of energy density
		\3 Emission can also be stimulated by oscillating EM field of the incident, dependent on energy density, with a seperate coefficient of stimulated emission ($B_{n_in_f}u(f)$)
			\4 The electric field is made to have the same frequency, such that the stimulated radiation is in the same direction and phase, called coherent
	\2 By the Boltzmann distribution, $\frac{N_2}{N_1} = e^{-(E_2 - E_1)/kT}$, where $N_1/2$ is the number of atoms in that particular energy level at temperature T
		\3 Since the number of absorptions is equal to emissions at equilibrium, $N_1B_{12}u(f) = N_2(A_{21} + B_{21}u(f))$
		\3 Combining Planck's Law for energy density, $u(f) = \frac{8\pi hf^3}{c^3}(\frac{1}{e^{hf/kT} - 1})$, it is found that $B_{12} = B_{21}, A_{21}c^3 = B_{21}8\pi hf^3$, such that spontaneous emission is more likely for higher energies
			\4 Further, for $hf >> kT$ as a result, stimulated emission can mainly be ignored due to the collision energy being not enough
\1 **ADD Section 9.6**
\end{outline*}
\section{Chapter 10 - Solid-State Physics}
\begin{outline*}
\1 The three main phases of matter, solids, liquids, and gases are distinct in terms of the molecular interaction, such that in liquids, short range bonds are continuously formed and broen by kinetic energy
	\2 In solids, the molecules arrange themselves in an array to maximize the number of bonds, though if cooled quickly, is amorphous, appearing as a solid version of the liquid structure
		\3 As a result, proper long-range crystal order has a definite melting point, while amorphous solids just soften over a range
		\3 Some solids only exist in either amorphous or crystalline forms, the most common form being polycrystalline (many unit crystrals randomly oriented)
	\2 Single crystals have symmetry and regularity, made up of repeating unit cell, the cell depending on the type of bonding, relative atomic size of the molecular atoms, pressure, temperature, and metallic bonding
		\3 Simple-cubics are a set of interconnected cuibic structures, while body-centered cubics are a subset with the same atom at each vertex and center
		\3 Hexagonal close-packed crystal structures are stacks of hexagonally laid-out atoms
	\2 Ionic solids are face-centered cubic, with an ion at the center of each face, not paired into ionic molecules, with the attractive potential of each ion as $U_{att} = -\alpha \frac{ke^2}{r}$, where r is the seperation distance
		\3 $\alpha$ is the Madelung constant, based on the geometry of the crystal, equal to the number of ions at each distance ($\alpha = \frac{n_1r}{r_1} + ...$), converging only for a cube
		\3 Since the ions repel each other at too close of a distance due to exclusion-principle repulsion by $U_{rep} = \frac{A}{r^n}$, with equilibrium seperation where $F = -\frac{dU}{dr} = 0$
			\4 It is found that $A = \frac{\alpha ke^2 r_0^{n - 1}}{n}$, with $r_0$ as the equilbrium distance, such that $U(r_0) = -\alpha\frac{ke^2}{r_0}(1 - \frac{1}{n})$, giving the dissociation/lattice energy
		\3 The cohesive energy is the potential energy per atomic pair, such that it is the dissociation energy, combined with the energy to ionzie the atoms, equal to the energy needed to remove the ions from the crystal/melting point
			\4 As a result, while the cohesive energy per atom may be high, per molecule, it may be low causing the crystal to fall apart
	\2 Covalent solids are made of series of covalent bonds, such that the dissociation energy is simply the energy of the individual bonds
		\3 The molar volume of a solid is equal to $nN_Ar_0^3$, where n is the number of different ions per molecule
		\3 The angles and number of covalent bonds per atom determine the possible layouts of the molecule
	\2 Metallic solids are those made of a metallic bond, in which the electrons are generally free to move, allowing high thermal and electric conductivity, shared between all atoms
		\3 For closely held atoms, the wavefunction of the electrons are summed, creating a repeating wavefunction, with the peaks narrowed, such that the positional uncertainty is smaller
		\3 Thus, metallic bonds rely on the increse in kinetic energy from the uncertainty principle being overruled by the decrease in potential from the electrons being closer to the atom
			\4 As a result, it is most likely when the valence electron number is small (making a smaller increse in kinetic energy), and with a larger narrowing in the peak for larger potential difference
\1 The Classical Drude Model of Electrical Conduction predicts thermal and electrical conduction and Ohm's law, though it fails to predict the change in conductivity with temperature and the lack of an increase in heat capacity by $\frac{3}{2}R$
	\2 Without an external electric field, $<v> = \sqrt{\frac{8kT}{\pi m_e}}$, is the average speed, though the direction makes the average velocity 0
		\3 In an applied electric field, this is with an additional average/drift velocity of $I = neAv_d$, where A is the cross-sectional area and n is the density of atoms
		\3 This can be rewritten in terms of the current density by $j = \frac{I}{A}$
		\3 Ohm's Law states that $I = \frac{V}{R}$, where $R = \frac{\rho L}{A}$ with $\rho$ is the resistivity, equal to $\frac{1}{\sigma}$, where $\sigma$ is conductivity
			\4 This can also be written as $\vec{j}\rho = \vec{E} = \frac{\vec{F}}{q}$, since $j = \frac{I}{A}$, where A is the cross sectional area, called the current density
			\4 $\rho = \frac{n\sigma m <v>}{n_e e^2}$, such that as the mean velocity increases, collisions happen more often, but the drift velocity is lowered, such that resistivity increases
			\4 Thus, if each atom releases one electron and electrons are assumed as point charges, $n_e = n$ and $\sigma = \pi r^2$, where r is the ionic radius
	\2 The mean free path, $\lambda = <v>\tau$, where $\tau$ is the relaxation time, or the average time between collisions, such that the mean free path is the average distance travelled without a collision
		\3 The mean free path is also equal to the mean scattering length, calculated by the scattering cross section
		\3 After each collision, the drift velocity is 0, such that the average drift velocity, $v_d = \frac{F_e\tau}{m} = \frac{eE\lambda}{<v>m} = \frac{eE}{n\sigma <v> m}$
\1 The electron gas would be expected to have a kinetic energy of $\frac{3}{2}kT$, but by the zero-point energy
	\2 Since within a 1D square well, $E_F = \frac{h^2N^2}{32mL^2}$, with average energy as the sum of the energies divided by the number of particles, N, found that $<E> = \frac{1}{3}E_F$, greater than room temperature energy
		\3 By $n(E)dE = g(E)f_{FD}(E)dE = 2\frac{dN}{dE}f_{FD}(E)dE$, where since $E = n^2E_1$, $g(E) = E_1^{-1/2}E^{-1/2}$, since $f_{FD}(E) = 1$ at 0 K for $E < E_F$, 0 for $E > E_F$, $n(E) = E_1^{-1/2}E^{-1/2}$
		\3 This can be viewed as $E_k = \frac{\hbar^2k^2}{2m}$, such that the current causes the electrons to move in the k direction of the current, causing conduction
	\2 Within 3D, $\frac{N}{V} = \frac{1}{V}\int^{\infty}_0 n_{FD}(E)dE = \frac{\pi}{2}(\frac{8m}{h^2})^{3/2}\int^{\infty}_0 \frac{E^{1/2}dE}{e^{(E - E_f)/kT} + 1}$, simplified at T = 0 K, such that $E_F = \frac{h^2}{2m}(\frac{3N}{8\pi V})^{2/3}$, where m is the mass of the electrons
		\3 As a result, as there are more atoms within a region, it requires more states filled, such that the Fermi energy is higher
		\3 Further, it is found that $n(E) = \frac{\pi}{2}(\frac{8m}{h^2})^{3/2}VE^{1/2} = \frac{3N}{2}E_F^{-3/2}E^{1/2}$, with average energy $<E> = \frac{1}{N}\int^{E_F}_0 En(E)dE = \frac{3}{5}E_F$ at 0 K
			\4 At higher temperatures, since the kinetic energy is approximately kT, only those within kT of the Fermi temperature can gain energy, hardly different for thousands of degrees
			\4 Thus, at non-zero temperatures, there is no cutoff Fermi energy, instead defined by $f_{FD}(E) = \frac{1}{2}$ for higher temperatures
			\4 As a result, unlike the classical where all particles gain energy as temperature increases, only a small portion gain energy, such that $C_V = \frac{N_Ad<E>}{dT}$ is less than the classical
	\2 Fermi temperature, $T_F$ is defined by $E_F = kT_F$, such that it is the temperature beyond which the Fermi-Dirac distribution approaches the Boltzmann
\1 The Quantum Theory of Conduction uses the Fermi distribution rather than Boltzmann for the electron gas of valence electrons, and accounts for the effects of the wave properties on electron scattering by the ions
	\2 Since the electric field acceelerates all electrons, they all take part in conduction, with the Fermi speed, $u_F = \sqrt{\frac{2E_F}{m_e}}$ as the approximate maximum speed in 1D at 0 K, distributed like the energy distribution
		\3 Thus, under an electric field, the distribution of electrons is shifted by the drift velocity with $\rho = \frac{m_eu_F}{ne^2\lambda}$, such that it is independent of temperature and uses Fermi speed rather than $<v>$
	\2 It is made proportional to temperature and of the correct magnitude by adjusting the calculation of $\lambda$, to account for the wave properties of electrons, such that the scattering of an electron by a collision is not a particle collision
		\3 Instead of Bragg scattering, which would require a smaller wavelength, electrons are only able to be scattered by crystal lattice imperfections
		\3 Imperfections are either due to impurities or thermal vibrations, with the scattering cross-section unrelated to the ion radius, but rather the array imperfections
			\4 Assuming point charge ions, the scattering cross section, $\sigma = \pi \bar{r^2}$, uses r as the average vibrational displacement
			\4 By the equiparition theorem, $\frac{1}{2}K\bar{r^2} = kT$, assuming only vibrational energy is allowed, such that $\lambda = \frac{1}{n\pi\bar{r^2}}$, such that it is proportional to temperature
		\3 By the Einstein model of the solid, accurate for non-low temperatures, it is assumed all atoms oscillate at the same frequency, such that $T_E = \frac{\hbar\omega}{k}$
	\2 The resisitivity due to the thermal vibration can be added to that of the lattice impurities, the latter of which is independent of the temperature, but generally irrelenant at non-low temperatures
		\3 This uses the density of the impurities, instead of the density of the atoms for the resistivity, usually $10^{-5}$ smaller
	\2 It is found that the energy of the excited electrons by nonzero temperatures can be modeled such that for electrons, $U = \frac{3}{5}NE_F + \frac{\pi^2}{4} N \frac{kT}{E_F}kT$
		\3 As a result, the molar heat capacity, $C_V = \frac{\pi^2}{2}R\frac{T}{T_F}$ for electrons, such that at most temperatures, the electron gas related heat capacity is far less than the expected $\frac{3}{2}R$
\1 The electron Fermi gas and lattice do not explain the varying conductivity of different solids, solved by finding the effect of the lattice on energy levels, able to be done either by the periodic potential or by the joining of levels of individual atoms
	\2 The Kronig-Penney model defines a periodic potential energy function by the positive ion lattice, such that energy travelling-waves, $\psi(x) = u_k(x)e^{ikx}$ exist for each possible potential
		\3 By the periodicity, $u_k(x) = u_k(x + nL)$, with the wavenumber $k = \frac{2\pi}{\lambda}$, called the Bloch function
		\3 This is taken to the solution to the Schrodinger equation with periodic potential function U(x), such that $\psi(x) = Ae^{\alpha x} + Be^{-\alpha x}, \alpha = \frac{\sqrt{2m(U - E)}}{\hbar^2}$
		\3 The requirement for continuity of the wavefunction and its first derivative creates energy bands in which the energy ($E = \frac{\hbar^2k^2}{2m}$) can exist, seperated by energy gaps
			\4 The energy gaps are found to exist at $ka = \pm n\pi$ for infinite potential and atomic potential length of 0, where a is the lattice spacing, acting as the points of Bragg reflection, producing a standing wave
			\4 Standing waves are able to have of the form, $\psi_1 = sin(\frac{\pi x}{a})$ or $\psi_2 = cos(\frac{\pi x}{a})$, the latter which is maximum at each a, the former between each
			\4 As a result, the second wavefunction has higher energy, such that the difference in energy produces the energy gap
			\4 The ranges of k values are called Brillouin zones, the first as $|k| < \frac{\pi}{a}$, the second as $\frac{\pi}{a} < |k| < \frac{2\pi}{a}$
	\2 Conductors are solids with an unfilled valence band, or an overlapping conductor band (as the band directly above the valencec vand), such as metals
		\3 Semimetals are those with only a small amount of overlap, limiting the number of states available
		\3 These work to conduct by allowing the electric field to excite the electrons to a higher energy state
	\2 Similarly, insulators are those with a filled valance band and an energy gap to the conduction band larger than 2 eV, at which thermal temperatures are unlikely to have an effect, such as ionic crystal solids which have a filled shell as a result, requiring exciting the individual ions
		\3 The dielectric breakdown is the point where the electric field is strong enough to overcome the separation gap
	\2 Intrinsic semiconductors are those with a smaller gap, such that the probability of thermal energy providing enough energy for a transition increases exponentially
		\3 For covalent solids, half of the states per atom in each shell are spacially-symmetric, half antisymmetric
		\3 Thus, as the radius decreases, the spacially-symmetric energy decreases, the antisymmetric increases, and the radius determines the gap beetween the states, such that it is able to be excited to antisymmetric states
	\2 Further, as atoms move to a higher state, other atoms are able to move to the states they were previously in, causing a move of the empty states in the direction of the field
	\2 The effective mass of a semiconductor, $\frac{1}{m^*} = \frac{1}{\hbar^2}\frac{d^2E}{dk^2}$, such that they are equal for a free-electron, less than otherwise
\end{outline*}
\section{Chapter 11 - Nuclear Physics}
\begin{outline*}
\1 It was found by Moseley that the atomic mass was approimately half the atomic number, while the total charge to the proton charge times the atomic number
	\2 The discrepency of the atomic number and mass number was solved by the electrons in the nucleus to balance out surplus charge, supported by beta decay, but found the minimum kinetic energy in the nucleus would be far higher than the beta energy
		\3 In addition, this was contradicted by there not being a strong enough electrostatic force found in the nucleus to counteract the kinetic energy
		\3 Otherwise, if the electron energy was positive, beta decay would occur more, and there would be no alpha decay barrier
		\3 Finally, it was found the magnetic moments were at the far lower proton moment, rather than the electron magnitude
	\2 Since protons and neutrons are fermions, bound by the exclusion principle, such that there is a splitting of atomic spectral lines by hyperfine structure, each with different angular momentum quantum numbers, prohibiting electrons contained
	\2 Rutherford postulated the neutral neutron particle, discovered by Chadwick, as the second nucleon along with protons
\1 N is used to note the number of neutrons, Z for the number of protons/atomic number, A for the number of nucleons/atomic mass, with isotopes are atoms with the same Z, isotones with same N, and isobars with the same A
	\2 Isotopes have similar checmical properties, except Hydrogen, while isobars have different chemical properties, but may have certain similar nuclear properties
\1 The nuclear radius can be measured by either Rutherford scattering or mirror nuclides, which are those with the Z and N values switched, such that nuclear forces are equal, while electrostatic are not
	\2 Since the electrostatic energy of a ball of charge, $U = \frac{3q^2}{20R\pi\epsilon_0}$, beta decay energy can also be measured (from Z to Z - 1) to find the radius
		\3 It is also found by electron bombardment, measuring the scattering, found that the first minimum is described by $sin(\theta) = \frac{0.16 \lambda}{R}$, where $\lambda$ is the electron wavelength
		\3 The cross section of a beam of neutrons through an atomic sample can be used, with $\sigma = 2\pi(R + \frac{\lambda}{2\pi})^2$, measuring the radius purely by the nuclear force instead of charge radius
	\2 It is found that $R = R_0 A^{1/3}$ where $R_0 = 1.2 fm$ for Z > 10, accounting for quantum mechanical correction of charge distribution with surface thickness of 2.4 fm
		\3 Since the volume is proportional to A as a result and the mass is proportional to A, the densities of the nuclei are approximately the same, found to be $0.138 nucleon/fm^3$
\1 The nuclear shape is generally spherical, with a few ellipsoidal exceptions in the rare earth elements, causing small variations in energy levels
	\2 The variation in energy levels coincide with a nonzero electric quadrupole by nonsymmetric distribution, such that $<Q> = Z\int \psi^* (3(z^2)_{av} - (x^2 + y^2 + z^2)_{av})\psi dV$, allowing determining of the shape
\1 It is found that nuclei are most stable on the line of stability with N = Z, due to the exclusion principle allowing the lowest energy when they are equal, and thus the most likely
	\2 Nucleons also generally pair with identical ones, such that there tends to be even numbers of each type of nucleon
	\2 It is found that there are approximately 2.6 isotopes per element, with Z = 20, 28, 50, and 82 as magic numbers with higher isotopes, and similarly for N with higher isotones
		\3 This is due to the closed shell structure, similar to that of electron shells, hypothetically including 2, 8, and several around 126 (the island of stability)
		\3 It is found that for the same numbers, adding that numbered neutron to a nucleus has a binding energy less than would be expected based on the mass energy difference as well
		\3 The neutron shells cannot be calculated like electrons by shielded charge due to being noncentralized, and are more difficult due to lack of knowledge of the nuclear forces
			\4 It is known that the strong force is saturated, short range, charge independent, and spin dependent, with each nucleon orbiting freely, with virtually no overlap in position
			\4 By the exclusion principle, only collisions causing an exchange of states is allowed, causing no effect, such that energy is only allowed to be transferred for high energy nucleons, where they can be excited, causing a Fermi energy
			\4 There is a strong spin-orbit interaction, called j-j coupling, such that if the angular momentum are parallel, the energy is higher, opposite that of electrons 
	\2 The mass is found by a mass spectrometer, measuring the charge to mass ratio of a beam within a magnetic field
\1 The binding energy of the atom, $B = M_Nc^2 + Zm_ec^2 - M_Ac^2$, where $M_N$ is the nucleus mass and $M_A$ is the atomic mass
	\2 Similarly, the far higher binding energy of the nucleus, $B = Zm_pc^2 + Nm_nc^2 - M_Ac^2$
	\2 It is found that the binding energy per nucleon raises drastically until Z = 16, at which point it remains constant, such that the nuclear force is saturated
		\3 This implies a very short radius of the nuclear force, such that nucleons only interact with their near neighbors
\1 The nuclear angular momenta is defined similarly and combined with the electronic angular momentum, producing a hyperfine atomic energy structure, with selection rule $\Delta F = \pm 1, 0$, with the nuclear magneton using the proton mass, such that it is vastly smaller energy difference than electrons
	\2 Only if the nuclear spin is less than the electron spin can the total nuclear spin be calculated, found that if Z and N are even, the spin is 0 in ground state, generally coupling, though not as simple in other cases
		\3 This is due to the lesser spin providing the number of fine structure deviations, such that there are 2I + 1 possible states
	\2 It is found that the common value of the proton moment is $2.79 \mu_N$, while the value of the neutron moment is $-1.91 \mu_N$, though it depends specifically on the detailed motion
\1 Radioactivity is found in all unstable nuclides, decaying into others by emitting particle and electromagnetic radiation, found to decrease exponentially with time
	\2 Since the nucleus is shielded by the electrons from other atoms, there is no depedence on temperature or pressure 
	\2 By $-dN = \lambda N dt$, $N(t) = N_0e^{-\lambda t}$, where $\lambda$ is the decay constant, with the decay rate as $R = \frac{-dN}{dt} = \lambda N_0e^{-\lambda t} = R_0e^{-\lambda t}$
		\3 The mean lifetime of a particle is calculated by the fact that the number with lifetimes from t to t + dt is $\lambda N dt$, such that $\tau = \int^{\infty}_0 t f(t)dt = \int^{\infty}_0 \frac{\lambda N t dt}{N_0} = \frac{1}{\lambda}$
		\3 The half life, $t_{1/2}$ is the time by which the number of radioactive nuclei and the decay rate has decreased by half, equal to 0.693$\tau$
	\2 Decay is measured in Becquerel (Bq), equal to 1 decay per second, formerly in Curie (Ci), equal to $3.7 * 10^10 Bq$
	\2 All radioactive decay must follow conservation of mass-energy, charge, momentum, angular momentum, nucleon number, and lepton number, and must produce atoms within the proton and neutron driplines, at which a single nucleon is spontaneously emitted
		\3 In addition, to be naturally found, the half-life must be comparable to that of the Earth or must be continuously produced by another reactions
	\2 Alpha decay is the spontaneous emission of a helium nucleus, generally found in heavy nuclei, with a long half-life due to the Couloumb barrier requiring tunneling
		\3 The transmission coefficient is based on the kinetic energy of the alpha particle determining the distance required to be traversed, due to the $\frac{1}{r}$ proportional potential
		\3 It is found that $\lambda = \frac{Tv}{2R}$, where R is the radius of the barrier and T is the transmission coefficient, with v as the particle velocity
			\4 It is also found that similarly $log(t_{1/2}) = 1.61(ZE_{\alpha}^{-1/2} - Z^{2/3}) - 28.9$, where Z is the resultant atomic charge, called the Geiger-Nuttall Rule
		\3 There are 4 possible families, the A = 4N, 4N + 1, 4N + 2, and 4N + 3, with the second not naturally found due to short decay times
	\2 Beta decay modes create a change in Z and N, with A remaining constant, unlike alpha which changes all three, made up of beta minus, plus, and electron capture
		\3 $\beta -$ decay involves the loss of an electron combined with the conversion of a neutron into a proton, such that the decay energy, $Q = c^2(M_i - M_f - m_e + m_e - \mu_e) = c^2(M_i - M_f - \mu_e)$, where $M_f$ is the mass of the neutral version of the final atom, rather than the charged product
			\4 $\mu_e \approx 0$, where $\mu_e$ is an electron neutrino, produced, allowing a range of possible beta momenta
		\3 Similarly, $\beta +$ decay has a proton decay into a neutron and positron, such that $Q = c^2(M_i - M_f - 2m_e)$, due to the positron and the lost electron to create a neutral atom, requiring more energy than - decay
			\4 The only positron emitter naturally found is $^{40}K$, able to decay by both other modes of beta decay
		\3 Within both types of decay, electrons are produced by the deexcitation of the nuclear bound states, found to have a longer decay time than strong nuclear force, and not suceptible to EM or strong force, such that there must be an additional short-range nuclear weak force
			\4 As a result, while double beta minus decay is possible theoretically, due to the relatively long ($10^{-21} s$) lifetime, it is unlikely
		\3 Electron capture/internal conversion has a proton capture an electron to convert to a neutron, releasing a neutrino, such that $Q = c^2(M_i - M_f)$, allowing decay which decreases the nuclear charge without requiring energy to allow two electrons
			\4 This only occurs for electrons within the K (or rarely the L) shell
			\4 As a result, this is used when the energy is not high enough to bring to a more stable state by beta plus decay
	\2 Gamma decay is the release of a photon due to nucleon deexcitation, generally with a $10^{-11} s$ halflife, such that there is a large natural line width, though it is potentially far longer, called isomers/metastable states
		\3 Large spin changes in the nucleus are forbidden, such that transitions that would require it are metastable states
		\3 The photon energy is modeled by the change in states, though it requires a portion of the energy to be given to the nucleus for conservation of momentum
			\4 Since the gamma ray energy is small compared to the mass energy, it is negligable, such that the approximation is valid, and $E_{recoil} = \frac{(hf)^2}{2Mc^2}$, where M is the nuclear mass
		\3 It is also possible for an electron to be emitted instead, called internal conversion, emitting a K or L shell electron, allowing measurement of the nuclear energy states
	\2 Neutron emission is also a possible mode of decay, changing the isotope, purely emitting a neutron with no other particles
\1 Nuclear/hadronic strong force has been studied by scattering and spectroscopy experiments, found to have equal strength between any two nucleons, acts very short-range, depends on velcoities and spins of composites, and is saturated
	\2 By approximating the width of the nuclei as an infinite square well, an estimate can be made for the strength of the nuclear force with the ground state as the potential of the strong force (approximately 50 MeV)
	\2 Protons with nuclear sized wavelengths are diffracted, providing the potential curves, found to be equal, though the proton-proton curve is shifted upward by electrostatic force
		\3 It is saturated due to the position within the nuclei not having an effect, due to only being acted on by its nearest neighbors
		\3 The saturation allows the hard core with constant density of the nuclear force, due to electrostatic repulsion acting on non-neighbors to keep them spaced at a constant distance
	\2 It was previously postulated that all particles are constantly emitting and absorbing photons, called virtual photons, to produce the EM force
		\3 Photons can be emitted without the particle changing its energy or momentum, as long as the time of photon existing is less than the uncertainty for the photon energy $\Delta t = \frac{\hbar}{\Delta E}$
			\4 Thus, longer distance forces have lower energy virtual photons
			\4 This was due to a change in a field propogating at the speed of light, creating time-dependent waves in the electric field, or waves
		\3 Yukawa stated that mesons were transmitted between nucleons for nuclear force, creating a meson field, causing a short-range by the large energy uncertainty, due to the non-massless meson
			\4 As a result, the range $R = c\Delta t = \frac{c\hbar}{\Delta E} = \frac{\hbar}{mc} = \frac{\lambda_c}{2\pi}$, where $\lambda_c$ is the meson Compton wavelength
			\4 The charge independence was accounted for by allowing the meson to have +e, -e, or 0 charge, such that the charged nucleon exchange would switch the particle's charges
			\4 He also postulated that if a collision provided enough energy to allow a real particle to be emitted, it could be observed, found as pi mesons/pions, though other mesons exist
		\3 It has since been found that mesons and nucleons are made up of smaller quarks, exchanging gluons instead
			\4 Attempts to seperate quarks produce enough energy to overcome the attractive force, which increases with distance like a spring,  that eventually a quark-antiquark pair is produced to prevent isolation
\1 Particle collisions with a nucleus can either be elastic, inelastic, and absorption combined with an emission of the same or different particle
	\2 Elastic scattering generally refers to the reflection of the particle at the edge of the nucleus, called Rutherford scattering
		\3 Direct interaction refers to the particle causing the removal of a nucleon by the energy
		\3 Compound nuclei are nuclei excited states formed by the collision, possibly including the combination of the particle with the atom, decaying as a statistical process
	\2 The energy change of a reaction is called the Q value, exothermic if positive, endothermic if negative, such that additional energy must be added
		\3 The threshold energy is Q in the center-of-mass frame, such that the total momentum is 0, while in other frames, it must be greater to allow conservation
		\3 For non-relativistic frames, $E_{CM} = K_1 + K_2$, after which when shifted by $v_2$ into the lab frame, $E_{lab} = \frac{m_1 + m_2}{m_2}E_{CM}$
	\2 The cross-section of a collision depends on the energy and type of the particle within the collision, with $\sigma = \frac{R}{I}$, where I is the intensity of particles (number per unit area and unit time) and R is the reactions per unit time per nucleus
		\3 Collisions are written by the notation, $A_i (p_i, p_f) A_f$, where the atom is starred to indicate a compound or excited nucleus, and the particle has an apostrophe to indicate an inelastic collision
		\3 Each possible outcome has a partial cross section, such that the total cross-section is the sum of each partial
		\3 Cross sections are measured by Barn with area dimensions, equal to $10^{-28} m^2$
	\2 It was found by Neils Bohr that most low-energy reactions were found to be two-parts, the formation of the compound nucleus and statistical decay, the latter of which is independent of the former
		\3 This is proven by the time of the decay taking longer ($10^{-16} s$) than the time it would take the particle to cross the nucleus ($10^{-21} s$)
		\3 For some complex nucleus, the possible reactions to produce it are the entrance channels, and the possible decays are exit channels
		\3 As a result, the cross-section can be broken up into the cross-section of the complex nucleus multiplied by the probability of a specific decay
	\2 Excited states of nuclei can be measured by the cross-section as a function of energy, with the lifetime of the states measured by the peak width, called the resonance
		\3 It can also be measured by the energy of inelastically scattered particles by the energy loss
		\3 The cross-section measuring finds higher energy levels, while inelastic scattering measures the lower levels
	\2 Neutron reactions are studied to avoid electromagnetic effects, finding that each scattering, some energy is lost to preserve momentum conservation, until the energy is kT, at which point losing and gaining energy are equally likely
		\3 Neutrons with energy of kT are called thermal neutrons
		\3 The cross-sections of thermal neutrons are found to decrease proportional to $\frac{1}{v}$, due to the lower time to leave the substance, with resonance peaks at the energy levels
\1 Nuclear fusion and fission produce large amounts of energy, due to the high mass per nucleon of the earliest atoms, as well as the higher values of heavy atoms, moving towards atoms with lower mass per nucleon, releasing the additional
	\2 Fission was discovered in 1938 by Hahn and Strassmann for Uranium-235, finding that when bombarded by thermal neutrons, it would have fission 85\%, gamma radiation 15\%
		\3 This was due to the neutrons carrying enough energy to reach the critical energy, escaping the Coulumb barrier, similar to alpha particles
		\3 On the other hand, it was found the thermal neutron bombardment of Uranium-238 was not enough to reach the U-239 critical energy
			\4 Atoms that are able to fission with standard energy thermal neutrons are called fissile, such as U-235
		\3 Heavy nuclei, such as those above Z = 90 were found to spontaneously fission, due to the surface tension of the nuclear force increasing proportionally to $R^2$, the electrostatic force by $R^3$
			\4 Thus, as the size of the nuclei gets higher, it gets more unstable, and has a higher chance of fissioning, though with a lower likelihood than other decay due to the Coulumb barrier
		\3 Nuclei generally fission into two smaller masses and 1-3 neutrons, each given very high speeds by electrostatic repulsion as the mode of energy release
			\4 This neutron release was used to create a chain reaction and produce nuclear weapons and energy
		\3 Fission also sometimes produces an alpha or gamma decay as well simultaneously, though this can be considered a seperate decay
	\2 Fusion is the joining of light nuclei into a medium weight, often releasing a neutron in the process, releasing vastly more energy than fission
		\3 The Coulomb repulsion causes the kinetic energy to need to be at least 1 MeV, and the probability of scattering is so much higher, it is inefficient
		\3 Thermal kinetic energy of the particles as a result, must be used, such that approximately $<KE> = \frac{3}{2}kT = 10 keV$ is used, combined with quantum tunnelling through the barrier
			\4 At the temperatures necessary, the gas is a plasma of ions and negative electrons, making it difficult to contain, present in stars by the gravitational field
			\4 It is found that the energy produced is proportional to $n^2\tau$, where $\tau$ is the confinement time and n is the density, while the energy needed is proportional to the density
			\4 Thus, the Lawson Criteria states that $n\tau > 10^{20} s * particles/m^3$
		\3 This is achieved either by magnetic confinement, placing inside a solenoid/tokamak around the plasma, moving it around to produce its own magnetic field, containing it, or inertial confinement
			\4 Inertial confinement has a portion of frozen atoms bombared by lasers, compressing it to a high enough temperature, producing $10^6 J$ in $10^{-10} s$, remaining in place for that time purely by inertia
	\2 The Sun's energy at any given time would radiate away far faster than the lifespan thus far, such that the energy supply is extended by fusion
		\3 When the sun contracted due to gravity, the temperature raised enough to produce fusion in the proton-proton cycle, proposed by Bethe
			\4 This is made up of 2H-1 to a positron, H-2, an electron-neutrino, and 0.42 MeV, followed by H-1 and H-2 to He-3, gamma rays, and 5.49 MeV
			\4 The cycle is ended by 2He-3 to He-4, 2H-1, gamma rays, and 12.86 MeV
			\4 Excess electrons and positrons are removed by annihilation into gamma radiation
		\3 Due to the low probability of the first reaction in the cycle, called the bottleneck of the cycle, only allowed for the highest energy atoms, the energy is produced slowly over millions of years
		\3 It has been found that the rate of solar neutrinos is half the expected value theoretically calculated, called the solar-neutrino problem
			\4 The neutrinos can also be used to measure the quantity of each type of reaction by the energy spectrum, allowing study of the cycle
		\3 The rate of reaction can also be found by the expected energy release, combined by the radiated power of the sun
		\3 When the hydrogen is used up, it gains enough weight for the gas and Fermi pressure to be outweighed, causing it to collapse into a supernova explosion, releasing high weight atoms
\1 Neutron activation analysis is performed for chemical analysis by exposing the sample to a high, slow neutron flux, fusing with a neutron and releasing a gamma ray, producing a radioactive isotope
	\2 Since $R(t) = \lambda N(t) = R_0 (1 - e^{-\lambda t})$, where $R_0 = N_0\sigma I$, such that $R_0$ is the production rate of the radioactive isotope, $\sigma$ is the cross section, and I is the intensity
		\3 Thus, $N_0$ is the number of original atoms, such that as time increases, the rate approaches $R_0$, or a constant rate of production .....
\1 Nuclear magnetic resonance is used to measure the distance between the up and down spin states in an external magnetic field, $\Delta E = 2(\mu_z)_pB$
	\2 These can be irradiated with photons of the same energy, called the RF (radio frequency) radiation and absorbed to change spins temporarily in resonance
	\2 The magnetic field of the molecule is the sum of the external/planetary and internal/structural magnetic field
	\2 This allows medical scans to test for the location in the body of materials, based on a localized magnetic field of varying strengths, and produces almost no heat or biological damage
\1 Computer-assisted Tomography uses short-life gamma-emitting radioisotopes to image specific organs, providing depth by rotating an x-ray beam and detector ring around the patient
	\2 This allows the reconstruction of transverse slices of the body, called a tomograph, and when combined with a CT, called single-photon emission computer tomography
	\2 This was improved by removing the gamma ray aspects, replacing with positrons, due to these annihilating shortly into two photons in opposite directions in opposite directions, called a PET scan, using bismuth germanate (BGO)
		\3 Positron emitters have a short-enough half life, such that it can only be done when near a cyclotron to produce it
\1 Radioactive dating is performed
\1 Accelerator mass spectrometry
\1 Particle-induced xray emission
\end{outline*} 
\section{Chatper 12 - Particle Physics}
\begin{outline*}
\1 In 1932, Anderson discovered the positron, with the same spin and mass as an electon, but positive charge, such that the intrinsic magnetic moment is parallel to the spin, rather than antiparallel
	\2 This was predicted by the relativistic wave equation, stating that since $E^2 = (pc)^2 + (mc^2)^2$, E could be positive or negative
		\3 Dirac's wave equation required negative-energy states to  be filled, postulating that electrons filled the, only exerting net force by holes in the negative energy, acting as positive chages
		\3 The positron was interpreted as the positive energy state, confirming this idea, but rendered unnecessary by QED
	\2 Pauli found that the Schrodinger equation could be broken up into 4 coupled equations, predicting spin and explaining negative energy solutions as antiparticles
	\2 QED interpreted the negative energy solutions as the positive energy solutions of the positron, with no need for them to be filled with virtual electrons
		\3 It also required that all fermions, by the Dirac equation, have a similar antiparticle, able to be pair-produced if given enough energy in an interactions, denoted with a bar (though positrons are denoted $e^+$)
			\4 Particles and anti-particles are also able to annihilate in an interaction, producing several photons to conserve momentum
		\3 Bosons also must have an anti-particle, described by the Klein-Gordon relativistic wave equation, with the same mass, but opposite charge, baryon number, and strangeness
	\2 The problem of matter-antimatter asymmetry remains a predominant issue in cosmology and particle physics
	\2 Feynman diagrams are used to describe particle interactions, as spacetime diagrams of ct vs x, with a backwards time arrow showing antiparticles, forward showing particles
		\3 All electromagnetic interactions are drawn as the combination of an electron and a positron colliding from the left, moving right as a photon, able to be rotated or combined to form complex interactions
			\4 These interactions are able to be overlapping, as long as the arrows are in the same direction
		\3 Particles that start and end within the diagram are virtual particles, such as force mediator particles or virtually created leptons/hadrons particles
	\2 The standard model, which predicts the strong and electroweak force interactions, combined with indirect and direct observation by conservation laws and detectors, are used to determine particles
		\3 Leptons are made up of three generations/families/flavors, each with a charged lepton and a lepton neutrino, electron, muon, and tau
			\4 Each has an antiparticle as well, though it is possible that neutrinos are Majorana neutrinos, such that each is its own antiparticle
			\4 Each lepton is given a weak isospin, $T_z$, $-\frac{1}{2}$ for the main particles, $\frac{1}{2}$ for neutrinos
		\3 Quarks are also in three generations, each with two flavors and antiparticles, binding together to form the majority of particles, bound into hadrons by gluon force carriers of the strong force, not found isolated, each with half-spin
			\4 Three-quark hadrons are called baryons, while quark-antiquark hadrons are called mesons (with B = 0), the latter of which is always unstable, decaying to other hadrons or leptons
			\4 Each quark also has a color charge, either red, blue, or green, such that each type has three possible properties, with each antiparticle having an anticolor
			\4 Up-type quarks are those with $\frac{2e}{3}$ charge (up, charm, top from generation 1-3) and $\frac{1}{2}$ weak isospin, while down-type quarks are those with $\frac{-e}{3}$ charge (down, strange, and bottom) and $\frac{-1}{2}$ weak isospin
\1 The four basic forces, from strongest to weakest, are strong nuclear, electromagnetic, weak nuclear, and then gravitational, unifying the middle two first by Glashow, Salam, and Weinberg in 1979 into electroweak theory
	\2 The strength of an interaction is the magnitude of the dimensionless coupling constant multiplying the space dependent portion of the potential energy
		\3 Thus, for electric force, $U'(r) = \frac{U(r)}{\hbar c} = \frac{-e^2}{r\pi\epsilon_0 \hbar c}\frac{1}{r} = -\alpha\frac{1}{r} = \frac{-1}{37}\frac{1}{r}$, creating U' to make it dimensionless
	\2 The Standard Model, which attempts to unify the strong and electroweak forces, assumes a boson mediator of each force interaction
	\2 Hadrons interact by the strong interaction, with a range of 1 fm and characteristic interaction time of $10^{-23} s$, with a coupling constant of approximately 1, due to the color charge, mediated by gluons
		\3 Each gluon has a color and anticolor charge, with 8 gluons rather than the 9 expected, such that the emission can change the color of the quark
			\4 In addition, as a result, gluons are able to couple to other gluons, such that they produce colorless particles, which are the required form
		\3 The interaction designates the time the particles must stay within range to have a probability of exchanging gluons and the strong interaction decay time
		\3 All baryons have been found to eventually decay to a proton, but there are many which are stable to the strong force
		\3 As a result of being composites, hadrons have exited states, decaying by strong interaction creating large energy width, found in resonances of hadron scattering, such that they are called resonance particles
		\3 The cross-section of the strong interaction is approximated by the range of the interaction as the radius
	\2 Electromagnetic force have a characteristic interaction time of $10^{-18} s$, mediated by a photon, with an infinite range, designated in a particle by an electric charge/magnetic moment, mediated by a photon with spin 1
		\3 Particles can also participate in the electromagnetic force by emitting charged virtual particles, causing them to have a temporary charge
	\2 Weak nuclear force acts short-range ($10^{-3} fm$) between all particles, with characteristic interaction time of $10^{-13} s$, acting on the weak/flavor charge, with coupling constant of $10^{-5}$
		\3 The weak force is carried by the $W^{\pm}$ of mass $80 GeV/c^2$ for the charged weak force, and $Z^0$ of mass $91 GeV/c^2$ for the neutral weak force, each with spin 1, as the second and third heaviest particles respectively
		\3 Charged weak force turns one quark flavor into another, but does not change the flavor of a lepton
	\2 Gravitational force has infinite range, with a hypothetical graviton mediator with spin 2, produced by the mass property with coupling constant $10^{-38}$
	\2 The range of the force cooresponds to the mass of the particles exchanged, such that the average distance of an exchange is $d = c\tau = \frac{\hbar}{mc}$ by the uncertainty principle
		\3 Thus, the probability of an interaction is denoted $P = e^{-r/d}$, where r is the distance between the interacting particles
		\3 The general form of the force law for nuclear forces is thus thought to be $V = \pm \frac{g^2e^{-\mu cr/\hbar}}{r}$, where g is the strength of the interaction
\1 The totalitarian principle of particle physics states that anything that can happen does happen, such that reactions that don't happen are forbidden by conservation laws
	\2 Each conservation law is due to a symmetry that controls the universe, such that momentum is from spacial translations, angular momentum from rotations, energy from temporal translations, and charge from gauge/scale transformations
		\3 This is called Noether's theorem, not yet validated for quantum conservation laws that have been found
	\2 Conservation of baryon number gives +1 to baryons, antibaryons have -1, and must be conserved, forcing the stability of the proton
	\2 Conservation of lepton number is done similarly, but is valid separately for each family of leptons
		\3 This was proven by the fact that neutrinos cannot have mass due to always being left-handed, anti-neutrinos righthanded, such that spin is antiparallel to momentum, unable to interact with the Higgs boson
		\3 Thus, by conservation of energy and lepton number, muon and tau neutrinos are assumed to be stable, assumed that all neutrinos were unable to have interactions once produced
		\3 It has since been found the neutrinos have mass and oscillate between flavors, such that there is not a conservation of lepton by family
			\4 It also states that since they are not always found left-handed, since they can have their frame of reference changed due to moving at less than c by a relativistic transformation
			\4 Lack of this conservation would make no distinction between neutrinos and antineutrinos, creating a further problem
	\2 Internal quantum number conservation other than charge and baryon is not conserved in weak interaction, denoted by upness, downness, stangeness, charmness, bottomness, and topness (U, D, S, C, B', T)
		\3 Up, charm, and top quarks have 1 respectively, the antiparticles -1, while down, strange, and bottom have -1 respectively, the antiparticles 1
		\3 These were determined by the unusually long lifetime of heavy baryons, due to no lower mass particles with the same quarks
		\3 In weak interactions, these internal quantum numbers are limited by the selection rule of $\pm 1, 0$
	\2 Hadrons have clusters of charge multiplets with similar mass, such that the only different property is the electric charge, such that the mass difference is due to the splitting of mass states of the same particle by charge
		\3 The isospin, I, is a 3D charge space vector, with the z-component denoted $I_3$, found to be quantized integrally, such that there are 2I + 1 $I_3$ states
			\4 Thus, it is added together similarly to spin or angular momentum vectors
		\3 It is defined such that $q = eQ = e(I_3 + \frac{B + S}{2})$, where Q is the charge quantum number
		\3 Isospin is conserved by the strong interaction, such that $I_{op}$ and $H_{op}$ commute in the strong interaction
	\2 Hypercharge is used to relate the quantum numbers to simplify conservation, such that $Y = 2(Q - I_3)$, equal to twice the average charge of a multiplet
		\3 By the previous rules for strangeness and baryon number, it is conserved only in strong interactions, with selection rule $\Delta Y = \pm 1, 0$
	\2 The parity quantum number, P, is defined for particles such that if the wavefunction changes sign after the parity operator, it is odd/-1, otherwise even/1, such that it is the only multiplicative property, rather than additive
		\3 It is related to orbital angular momentum, such that $P = (-1)^l$, conserved in strong and electromagnetic interactions
		\3 Thus, it is not conserved in a weak interaction, such that there are more particles produced of a particular parity, such that the right-handed and left-handed coordinate systems can be differentated against
			\4 This is seen by the amount of parallel vs antiparallel spin-angular momenta combinations changing by the parity
	\2 It is found that for any relativistic quantum theory in which speeds are maximum at the speed of light, time reversal, charge conjugation (particle to antiparticle), and parity do not change the wavefunction (TCP invariance)
		\3 Thus, since parity is changed in a weak interaction, $CP = -1$ in a weak interaction
		\3 This allows the matter-antimatter symmetry and three generations of quarks within the Standard Model
	\2 This is coupled with the strongest force principle, stating that the rate/probability of the reaction is determined by the strongest force consistant with conserved quantities
\1 **STANDARD MODEL**
\1 Grand unification theories are those which attempt to join the grvitational force with the others, account for the issues with the standard model, and the equal charges of the electron and proton
	\2 These require the coupling constants of all four interactions to approach the same value at higher energies than can be currently produced, approximately at the fine-structure constant, $\alpha$
	\2 Supersymmetry assigns each elementary particle a superpartner, with all properties the same except the spin, such that the spin of the fermions are all 0, the spin of the bosons as $\frac{1}{2}$
		\3 Fermions are given the same name with an ``s'' at the start, while bosons are given the same name with the ``ino'' suffix
		\3 The masses would also be different, at least that of the W bosons
		\3 String and superstring theories build off of supersymmetry, stating that particles are quantized 10+ dimensional strings, shorter than the uncertainty to allow quantized gravity
			\4 Superstring theories result in gauge theories to explain the exchange bosons, but have no experimental basis, and do not yet explain the color charge, isospin, or possible meson sea
	\2 GUTs state that leptons and quarks are states of a single leptoquark particle, occuring symmetrically in a mutiplet, allowing conversion of one into the other, stating that the proton lifetime is $10^32$ years, preventing measuring
		\3 This is possibly used to explain matter-antimatter asymmetry, and would result in the lack of conservation of lepton and baryon number at high energies
	\2 Neutrinos, originally believed to have mass, were proven to have mass experimentally, having been hypothesized to explain the low amount of solar neutrinos, called the solar neutrino problem
		\3 The mass is thought to be far lower for electron neutrinos than muon and respectively than tau, with mass below 1 eV
		\3 This could possibly help to explain the missing mass/energy problem in cosmology as well
		\3 It is also thought that this results in neutrino oscillation, called the MSW effect, where the three flavors each have the superposition of three flavor states
			\4 Thus, when scattered by solar matter, they change phase differently, and when moving, move at different rates, producing an oscillation of flavor
	\2 Magnetic monopoles are also hypothesized, originally by Dirac, arguing it causes a quantized magnetic charge, $q_m = n\frac{\hbar c}{2e}$, where n is a positive integer
		\3 These are predicted to only exist at extremely high masses, such that the energies are beyond what can be produced, with a very low flux due to not destroying the galactic magnetic field
\end{outline*}
\section{Notes}
\begin{outline*}
\1 Absorption is a subset of the emission spectra, due to multiple possible paths of emission for higher levels of hydrogen atoms
\1 $\frac{1}{\lambda} = \frac{m_e \alpha^2 Z^2}{2} (\frac{1}{n^2} - \frac{1}{m^2})$, $\alpha = \frac{e^2}{4\pi \epsilon_0 \hbar c} \approx \frac{1}{137}$, called the fine structure constant
\1 Fractional change of n vs m = $\frac{n - m}{n}$
\1 Hydrogen atom is $f = cZ^2R(\frac{1}{n^2} - \frac{1}{m^2}$, such that Moseley's Law applies this to xrays for analyzing wavelength
\1 Regular as n goes to infinity, L'Hopital used 
\1 The area of a sphere is $4\pi r^2$
\1 $\lambda = \frac{h}{\sqrt{2mK}}$
\1 $L = mvr = I\omega = mr^2 \omega$ (for single particle)
\1 \textbf{**Pay attention to units**}
\1 Things maybe to know: Light frequencies of different types, eV vs J conversion, angular momentum mechanics
\1 Powder scattering is the scattering of light by a collection of crystallites with random orientations, forming circles of peaks in a target formation
\1 Electrons observed appear to behave with probabilities due to the light preventing the interference pattern, only going through one slit, changing the outcome, or low intensity makes not enough scatter and low momentum/high wavelength don't measure properly
	\2 Complementary principle, only one type of property, wave or particle, can be measured, but not both can be
		\3 Since momentum is proportional inversely to wavelength, it is a wave property, such that it or position can be measured, but not both, leading to Heisenberg
\1 Copenhagen interpretation states that initial conditions can only determine probability, not definite future conditions
\1 Oscillating functions are often represented in complex exponential notation, keeping only the real portion for the actual equation
\1 Useful integrals include $\int^{\infty}_{-\infty} e^{-(x/a)^2}dx = a\sqrt{\pi}$, $\int^{\infty}_{0} xe^{-(x/a)^2}dx = \frac{a^2}{2}$, $\int^{\infty}_{-\infty} x^2 e^{-(x/a)^2}dx = \frac{a^3\sqrt{\pi}}{2}$, $\int^{\infty}_{-\infty} e^{ikx-(x/a)^2}dx = a\sqrt{\pi}e^{-(a^2k^2)/4}$
\1 $R_{\infty} = 1.0973 * 10^7 m^-1$ (Rydberg Constant)
\1 The operators for the wavefunction are found by the complex exponential wavefunction, using the expression which would produce the correct terms
	\2 Eigenfunctions are functions which when acted on by an operator, produce a constant times the function ($Q\psi = \lambda \psi$), such that the expectation value of the operator is always the eigenvalue, able to use it simplify expectation value calculations
	\2 As a result, the standard deviation ($<A^2> - <A>^2$) of 0, providing a definite value for some value
	\2 For $e^{ipx}$, the momentum operator forms an eigenfunction, due to the momentum being fixed in the equation or some other reason?
\1 For two functions, they are orthogongal if the product of the conjugate of either by the other is 0 over the entire domain, combined with the fact that $<1> = 1$ to simplify expressions
	\2 Orthogonality exercise 2 is true due to why??
\1 Standard deviation, $\sigma = \sqrt{<x^2> - <x>^2}$
\1 Uncertainty principle for standard deviation is divided by 1/2, for half width is 1, otherwise it can be either multiplier
\1 **Solid angle/differential cross section??**

\1 Quantum mechanics can be written in terms of matrix operations, related to operators, noted to similarly not necessarily be commutable
	\2 Similarly, for a wavefunction as a linear combination of eigenfunctions, the vector element cooresponds to the vector of the amplitudes of each eigenfunction
	\2 Thus, the operator can be denoted as some matrix, (often as the diagonal matrix of the eigenvalues), such that eigenfunctions are like eigenvectors (when multiplied by the operator matrix, it provides the vector multiplied by a constant)
	\2 Expectation values are calculated by the adjoint vector, or the conjugate transpose, denoted as $<f(x)> = v^*f(x)v$
	\2 Spin has matrix values as $\chi_{Z+} = (1, 0)^T, \chi_{Z-} = (0, 1)^T$, with $S_z = \frac{\hbar}{2}((1 0)^T (0 -1)^T), S_x = \frac{\hbar}{2}((0 1)^T (1 0)^T), S_y = \frac{\hbar}{2}((0 i)^T (-i 0)^T)$
		\3 As a result, for some fixed value of x, either positive or negative eigenvalues, it has equal probability of the positive or negative z value
		\3 $\chi_{X+} = \frac{1}{\sqrt{2}}(1, 1)^T, \chi_{X-} = \frac{1}{\sqrt{2}}(1, -1)^T$, such that the sum of them provides $\chi_{Z+}$, difference for $\chi_{Z-}$

\1 It is found that an eigenfunction cannot be simultaneously found for multiple $\vec{L} (\vec{r} x \vec{p} = mvR = mR^2\omega = I\omega = mR^2\frac{d\theta}{dt})$ components, but can be for $L^2, L_z$, similar to being unable to be found for p and x
	\2 As a result, it is noted that the maximum angular momentum along a single axis is less than the total angular momentum, since otherwise, the angular momentum along all axes would be known
	\2 $L^2_{op} = -\hbar^2(\frac{\partial^2}{\partial \theta^2} + cot(\theta)\frac{\partial}{\partial \theta} + \frac{1}{sin^2(\theta)}\frac{\partial^2}{\partial \phi^2}$, $L_z = -i\hbar(x\frac{\partial}{\partial y} - y\frac{\partial}{\partial x}) = -i\hbar\frac{\partial}{\partial \phi}$, such that the wavefunction is already an eigenfunction
	\2 It is noted that angular momentum can be put in terms of moment of inertia, I, as the sum of the values for each particle, such that for a sphere, $I = \frac{2}{5}MR^2$, with rotational kinetic energy as $K = \frac{1}{2}I\omega^2 = \frac{L^2}{2I}$
	\2 The force analog of angular momentum, $\vec{\tau} = \vec{r} x \vec{F} = \frac{d\vec{L}}{dt} = I\alpha$, such that it is conserved for an isolated system

\1 Hydrogen atoms involve only radius based electrostatic potential, $V = \frac{-Ze^2}{4\pi \epsilon_0 r} = \frac{-kZe^2}{r}$
	\2 Spherical coordinates are $x = rsin(\theta)cos(\phi), y = rsin(\theta)sin(\phi), z = rcos(\theta), \theta = tan^{-1}(\frac{\sqrt{x^2 + y^2}}{z}), \phi = tan^{-1}(\frac{y}{x})$, assuming $\theta$ is from the z-axis, $\phi$ is from the x-axis
		\3 In addition, $dV = r^2sin\theta d\phi d\theta dr$ for integrating in spherical coordinates, using radians rather than degrees
		\3 $\nabla^2 = \frac{\partial^2}{\partial r^2} + \frac{2}{r}\frac{\partial}{\partial r} + \frac{1}{r^2}(\frac{\partial^2}{\partial \theta^2} + cot(\theta)\frac{\partial}{\partial \theta} + \frac{1}{sin^2(\theta)}\frac{\partial^2}{\partial \phi^2})$
	\2 As a result, the surface area integral is the integral with a set radius, such that there is no radius bound
\1 The hydrogen atom product solutions for angles and radial must be independently normalized, due to needing to be at some radius and angle
	\2 Energy less than 0 implies a bound state, such that potential energy is greater than kinetic energy, which are the proper solutions
	\2 The degeneracy of the state is equal to $n^2$, equal to the number of states with the same energy in a given potential

\1 The beat frequency of the superposition of two waves is the absolute value of the difference of the two frequencies
	\2 As a result, for some wave $f_0$, modulated by frequencies from 0 to f, where $f_0 >> f$, the result can be decomposed into frequencies from $f_0 - f$ to $f_0 + f$
	\2 For the range considered the half-width at maximum, it is $\hbar$, while for the standard deviation ($\sqrt{<x^2> - <x>^2}$), it is $\frac{\hbar}{2}$
		\3 For some distribution function, f(x), $<g(x)> = \int g(x) f(x)dx$
\1 Zero point energy provides the stability of the atom by preventing it into collapsing into itself, by the uncertainty principle
\1 Reduced mass is derived by Newton's Third Law, within the reference frame of $X = x_1 - x_2$, with the only force as that acting on both
\1 Larmor precession, $\omega = \frac{g\mu_B}{\hbar}\vec{B}$, is the respective of $\vec{\mu}$ for $\vec{B}$ rather than $\vec{L}$

\1 The general wavefunction of the infinite well is the linear combination of solutions to the equation with some coefficient to provide the combination, such that it is not simply a single eigenfunction, though the energy measured be an eigenvalue
	\2 $A_n = \int u_n^*(x)\psi(x)dx$, where $u_n(x)$ is the nth eigenfunction of the equation
	\2 The probability of measuring any particular energy is $|A_n|^2$, such that $<H> = E_{avg} = \int \psi^* H \psi dx = \sum_n |A_n|^2E_n$

\1 The gyromagnetic ratio of a proton is 5.56, though the spin of the proton has almost no difference in energy due to the larger mass
	\2 Nuclei can only have angular momentum l = 0, such that only the spin of the nucleons play a role

\1 Symmetries are related to conservation laws, such that in constant potential (momentum conserved), translation of the system creates no change in the outcome, while rotational relates to angular momentum, and time-invariant potential means energy conservation
	\2 As a result, symmetry of the Hamiltonian creates symmetry in the probability density of the same form, such that the wavefunction is restricted
	\2 Thus, since the Hamiltonian is invariant under the exchange of two identical particles, the wavefunction is either symmetric or antisymmetric under particle exchange
		\3 This is found to be the former for bosons with integer spin, the latter for fermions with half-integer spins, requiring linear combinations to produce
		\3 Thus, to produce the antisymmetric wavefunction, it is the linear combination of opposite spins, negatively combined, such that the spin of each is equally likely to be measured, though the opposite will be measured after for the other
	\2 Thus, there is a triplet of S = 1, spin-symmetric, space-antisymmetric wavefunction forms (each with a distinct joint m-value), and a singlet of S = 0, spin-antisymmetric, space-symmetric wavefunction forms
		\3 Thus, for a spacially-antisymmetric state, the probability of both particles being at the same location is 0, unlike in a spacially-symmetric, such that the motions are related

\1 Selection rule states that m must go +1, -1, or 0 in a transition, while l must go +1 or -1

\1 The reflection probability $\frac{B^2}{A^2}$, while the transmission probability is $\frac{k_2 C^2}{k_1 A^2}$, due to being the relative rates

\1 Since the spin of the electron in fermions keeps electrons away from eachother, it is called the exchange force of the electron

\1 For some infinite square well, by Pauli's exclusion principle, there is able to be two electrons per state, based on the spin of the electron
	\2 It is found as a result that the total ground energy for $E_{total} = \frac{N^3E_1}{12}$, where there are N particles, such that $n_{max} = \frac{N}{2}$
	\2 Fermi energy is the energy of the highest filled energy state, such that $E_F = \frac{\hbar^2 \pi^2 n^2}{8m}$, where n is the number of particles per unit length

\1 Hund's Rules determine the fillin up of energy levels, in which the n levels are filled increasing, then l in increasing order
	\2 Spin is maximized first, with total spin ($m_s$) denoted as S (the First Rule), followed by total angular momentum maximized ($m_l$) denoted as L (the Second Rule)
	\2 The Third Rule states that for shells less than half filled, J ($m_j$) is minimized, while for greater than, it is maximized
	\2 The specific orbital state of an atom is denoted $^{2S + 1}L_J$, with L written by S, P, D, F, etc notation
	\2 Conservation of angular momentum, momentum, and energy are followed by selection rules, such that for energy changed, $\Delta n$ can be any value, $\Delta L = \pm 1, \Delta S = 0, \pm 1, \Delta m_l = 0, \pm 1, \Delta J = 0, \pm 1$

\1 Electronic configuration notation is written as $nL^{num}$, where num is the number of electrons in the orbital, L is the letter form of the l quantum number

\1 Neutrons and protons themselves as fermions act as particles within a spherical box, forming shells with quantized energy levels, similar to electrons

\1 Constants include the Bohr radius, planck's constant, the electron mass, the proton/neutron mass, and the electron charge

\1 The idea of quantization and rotational kinetic energy can be combined with classical thermodynamics to give the $C_V$, assuming purely allowed rotational motion, and the probability of each rotational level classically
	\2 It is found that the classical equidistnce theorem is only valid for higher temperatures if quantization of rotational energy is taken into account
	\2 In this case, the degeneracy of states is 2l + 1, due to the possible $m_l$ values
	\2 Similarly, it can be combined with vibrational energy to give the probability of each vibrational level classically

\1 Protons and neutrons are fermions, rather than bosons, such that the hydrogen wavefunction must be antisymmetric, either singlet (S = 0, antisymmetric spins), or triplet (S = 1, symmetric spin)
	\2 It is found that the spacial component of the wavefuntion is multiplied by $(-1)^l$, such that for even l, it is singlet, called para-hydrogen, otherwise triplet/ortho-hydrogen
	\2 Since all states are equally likely for some energy, the odds of a triplet state are three times as great as the singlet state, though at low temperatures, this requires a catalyst to produce equilibrium distribution

\1 1 amu = 1 u = $1.66 * 10^{-27} kg$, $m_p = 1.007276 u, m_n = 1.008665 u, m_e = 5.49 * 10^{-4} u, m_{\mu} = 0$
	\2 The radius of a proton or a neutron is 1 fm, with the latter having a lifetime of 900 s when free, before decaying into a proton, electron, and electron antineutrino
	\2 Conservation of lepton number has particles as 1, antiparticles as -1, with a seperate conservation for each family

\1 Nuclei stability is determined by free neutrons decaying into a proton, but neutrons and protons tending to pair up and further pair into alpha particles to minimize energy levels, though as Z increases, N increases slightly faster
	\2 Binding energy is measured by the mass deficit, such that the decrease in mass is the potential energy holding it together (negative potential)
	\2 Iron-56 is the more stable element, due to the highest binding-energy per nucleon, such that all greater decay towards, and all below would, though Coulumb barrier can prevent

\1 The lifetime of a nucleus is probabilistic, such that $\frac{\Delta N}{\Delta t} = -\lambda N$, where $\lambda$ is the decay constant
	\2 As a result, $N = N_0e^{-\lambda t}$, from which the half-life can be derived

\1 The force constant for a vibration is $k = \frac{d^2 V}{dx^2}|_{x = x_0}$, where $x_0$ is the minimum, or the point at which the first derivataive is 0
	\2 This is due to the Fourier series, $V(x) = \sum \frac{1}{n!}(x - x_0)^n \frac{d^nV}{dx^n}|_{x = x_0}$ being able to remove the first derivative term, approximating k as the quadratic term

\1 The semiempirical mass formula is written as $E = (Zm_p + Nm_n)c^2 - B$, where B is the binding energy, such that $B = a_{vol}A - a_AA^{2/3} - 0.72 MeV * \frac{Z(Z - 1)}{A^{1/3}} - a_S\frac{(N - Z)^2}{A} + \delta$
	\2 $\delta$ is the alpha pairing term, such that it is equal to $\frac{33 MeV}{A^{3/4}}$ for even-even nuclei, 0 for even-odd, and the negative for odd-odd
	\2 The first term is the neighbor binding energy term, with $a_{vol} = 14 MeV$, while the second is the surface tension lower binding energy, due to fewer neighbors, with $a_A = 13 MeV$
	\2 The third term is the Coulomb repulsion of the protons, while the fourth is the binding energy due to equal numbers of protons and neutrons with $a_S = 19 MeV$

\1 Uranium-238 has natural decay taking 4.5 billion years for the first step, with the path designated by $U-238 \to Th-234 \to Pa-234 \to ... \to Pb-206$
	\2 Within reactors, it is sped up by collisions with a neutron, such that $U-238 \to U-239 \to Np-239 \to Pu-239$, which is able to be fissioned
	\2 U-235 is also seperated/enriched from mixed Uranium to allow fission to take place, releasing an average of 2.5 additional neutrons
		\3 This must produce an average of at least 1 reaction if there is a critical mass of uranium to allow a sustaining reaction
	\2 U-235 releases approximately 200 MeV per atom within a chain reaction
	\2 Fusion reactors would most likely use Deuterium magnetically-confined plasma, within the International Thermonuclear Experimental Reactor
	\2 Fusion bombs funtion by fusing Li-6 + n to He-4 + H-3, after which H-3 + H-2 to He-4 + n, the neutrons and heat for compression supplied by a previous fission
		\3 Styrofoam is used to absorb gamma radiation from the fission, but allow the neutrons to get through
		\3 More neutrons are released than from a hydrogen bomb, causing radiation poisoning over a larger area

\1 The density of states can be approximated as 2 if it is assumed that the energy levels are continuous rather than discrete, assumed for large energies, also true specifically at the energy levels
	\2 $n(E)dE = \frac{V}{2\pi^2}(\frac{2m}{\hbar^2})^{3/2}\frac{\sqrt{E}dE}{e^{(E - E_F)/kT} + 1}$, integrated to give the total number of particles (under this assumption or not??)

\1 It is found classically that for N connected oscillators, there are N modes of vibration, such that for N in each direction in 3D, there are 3N modes, each with two degrees of freedom
	\2 It is also found that if there are N free particles in 3D, there are 3N degrees of freedom for them total

\1 By the equipartition theorem, average translational kinetic energy within 3D is approximated as $\frac{3}{2}kT$, while average vibrational potential and vibrational kinetic are $\frac{kT}{2}$ each
	\2 Tunnelling allows fusion to take place, even if there is not enough thermal energy in the Sun to produce it naturally, though there are also extremes with high enough energy

\1 Electron gas energy can be approximated quantum dynamically as a constant with the scale of increase term as $N\frac{\pi^2k_BT^2}{4T_F}$

\1 Phonons are the quantized versions of the modes of atoms in a lattice, due to not being able to vibrate independently, acting as bosons
	\2 These are quantized, such that $E_K = (n + \frac{1}{2})\hbar\omega_K, \omega_K = \sqrt{\frac{4k}{m}}sin(\frac{\pi a}{\lambda})$, where a is the lattice spacing, k is the spring constant
	\2 $<E_K> = \frac{\hbar\omega_K}{e^{\beta \hbar \omega_K} - 1}$, equal to $U = \sum_{\lambda, p} <E_K> = \sum_p \int <E> D_p(\omega)d\omega$, where p are the polarizations and branches of the dispersion and $D_p(\omega)$ is the number of modes with a specific polarization at that wavelength
	\2 This is too difficult to calculate analytically, such that there are different models to simplify the calculation

\1 Photons do not have electric charge, but W, Z bosons have weak charge, and quarks have color charge
\1 Color charge was invented to allow anti-symmetric wavefunctions for the $\Delta^{++}$ particle, due to the spin being identical and parallel for each (j = 3/2, symmetric), flavor identical, and due to being a small particle, l must be 0, such that it is spacially symmetric
	\2 Color was added to allow each to have a different particle, making it anti-symmetric, such that $\psi = \psi_{spin}\psi_{space}\psi_{flavor}\psi_{color}$
		\3 Thus, $\psi_{color} = \frac{1}{\sqrt{6}}(RGB + BRG + GBR - RBG - GRB - BGR)$ for hadrons
		\3 Similarly, mesons are symmetric, as a boson with integer spin, such that $\psi_{color} = \frac{1}{\sqrt{3}}(R\bar{R} + G\bar{G} + B\bar{B})$
	\2 All unconfined, non-virtual objects must be color neutral, made up of red, blue, and green in equal numbers

\1 Particle accelerators are used to produce high enough energies for production of heavy particles, such as Roentgen's xray tube
	\2 Cyclotrons have an electric field along an axis, starting the particle in the center, keeping the period equal, accelerating it each time it crosses the axis by flipping the field, keeping it in a constant magnetic field
		\3 Thus, it is limited by the radius of the cyclotron, with $qvB = \frac{mv^2}{R}$
	\2 Cyclotrons are limited by relativistic effects creating non-constant periods, since $E = K + mc^2 = \gamma mc^2 = \sqrt{c^2p^2 + m^2c^4}$, where $\gamma = \frac{1}{\sqrt{1 - \frac{v^2}{c^2}}}$
		\3 Synchotrons correct for this, changing the electric field to both account for the electric field, and keep it in a circle with the same radius, allowing higher energies
		\3 This has the downside of energy loss due to the circular motion, not present in linear accelerators, but do not have the length issue

1/24 - Exp Basis Not Done
3/01 - Hydrogen Not Done
3/29 - StatMech Not Done
3/30 - Molecules Not Done
4/05 - Conductivity Not Done (Start with p8)
4/06 - Conductivity Not Done
4/20 - Particle Not Done

\end{outline*}
\end{document}