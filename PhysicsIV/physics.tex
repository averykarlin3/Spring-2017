\documentclass[11 pt, twoside]{article}
\usepackage{textcomp}
\usepackage[margin=1in]{geometry}
\usepackage[utf8]{inputenc}
\usepackage{color}
\usepackage{indentfirst} %Comment out for no first paragraph indent
\usepackage[parfill]{parskip}
\usepackage{setspace}
\usepackage{tikz}
\usepackage{amsmath}
\usepackage{amsfonts}
\usepackage{amssymb}
\usepackage{enumitem}
\usepackage{outlines}

\usepackage{fancyhdr}
\pagestyle{fancy}
\cfoot{\hyperlink{content}{\thepage}}
\lhead{}
\chead{}
\rfoot{}
\lfoot{}
\rhead{}
\renewcommand{\headrulewidth}{0pt}
\renewcommand{\footrulewidth}{0pt}


\usepackage{hyperref}
\hypersetup {
	colorlinks,
	citecolor=black,
	filecolor=black,
	linkcolor=black,
	urlcolor=black
}

\newcommand{\sepitem}{0pt} %Added room between items on the list, not including a list and its sublist
\newcommand{\seppar}{1pt} %Between items and lists overall

\setenumerate[1]{itemsep=\sepitem, parsep=\seppar}
\setenumerate[2]{itemsep=\sepitem, parsep=\seppar}
\setenumerate[3]{itemsep=\sepitem, parsep=\seppar}
\setenumerate[4]{itemsep=\sepitem, parsep=\seppar}

\newenvironment{outline*}
{
	\begin{outline}[enumerate]
	}
	{\end{outline}
}

\newcommand{\foot}[1]{\hyperlink{#1}{$_#1$}}

\begin{document}

\title{Principles of Physics IV: Modern Physics}
\author{Avery Karlin}
\date{Spring 2017}
\newcommand{\textbook}{Modern Physics by Tipler and Llewellyn}
\newcommand{\teacher}{Dr. Paul Heiney}

\maketitle
\newpage
\hypertarget{content}{\tableofcontents}
\vspace{11pt}
\noindent
\underline{Primary Textbook}: \textbook\\
\underline{Teacher}: \teacher
\newpage

\section{Course Introduction}
\begin{outline*}
\1 Kinematic assumptions originally stated that all items have a well-defined position that can be measured to any precision, and all observers agree on the time and position for a measurement
\1 Newtonian mechanics were based on the idea of the deist clockwork universe
	\2 Conservation of momentum follows from Newton's third law, conservation of energy follows from the work energy theorm
	\2 Conservative forces have potential energy defined such that F = $-\nabla V$ (or U, depending on notation)
	\2 Newton's Laws combined with the fact that all known major forces act on the center of an object, implies conservation of angular momentum
\1 Relativistic mechanics has $p = \gamma mv$ and $E^2 = c^2p^2 + m_0^2c^4$, and defined rigid bodies as impossible, often using the center of mass
\1 Waves are travelling disturbances in a medium, averaged over space to consider bulk properties, carrying energy and momentum
	\2 Wave equations often relate curvature (second derivative w/ respect to x) to acceleration, $f_{xx} = (1/u^2) f_{tt}$ (u can be replaced with v, acting as an alternate velocity notation)
	\2 Waves are travelling, such that f(x, t) = g(x - ut)
	\2 y = exp(x - ct) is travelling, while sin(x)cos(ct) is a sum of two travelling, but not a travelling on its own, such that any g(x - ut) works
\1 Waves but not particles can diffract, refract, and superposition, while both have velocity, generally localized position, momentum, and energy
	\2 Superposition creates nodes of no displacement, antinodes of max displacement in a standing wave
	\2 Energy is proportional to the square of the amplitude
\1 Maxwell's equations show that since changing magnetic fields produce electric fields and vice versa, waves can be created by oscillating or accelerating charges, travelling forever in vacuum
\1 Statistical mechanics are used to solve systems with too many obects to calculate, instead using statistics to predict average properties
	\2 For some complicated system with total conserved energy and many ways to distribute the energy, the probability of any entity having a particular energy is independent of others, such that it depends only on others, but limits the possibilities of remaining energies
		\3 Thus, $P(E_1)P(E_2) = f(E_1 + E_2)$, since the probability of the joint energy depends only on the sum, found to be true if $P(E) = Ab^{\pm cE} = Ae^{-\beta E}$, where $\beta = \frac{1}{k_BT}$, $k_b = 1.38 * 10^{-23} J/K$
		\3 n(E) is defined as the number of distinct ways an object could have energy E, such that the sum of the energies must be 1, either a sum or an integral if discrete or continuous, such that $P(E) = \frac{e^{-\beta E}}{\sum_j n(E_j) e^{-\beta E_j}}$
	\2 Thus, the average value is the sum of the value multiplied by the probability for all possible energy values, $\bar{g(E)} = \frac{\sum_j n(E_j)g(E_j)e^{-\beta E_j}}{\sum_j n(E_j)e^{-\beta E_j}}$
	\2 The Boltzmann Distribution assumes n(E) = 1 for all E, and the energy distribution is continuous, such that $P(E) = \frac{e^{-\beta E}}{k_BT}, \bar{E} = k_BT$
\1 Single photon cannot pair produce an electron and a positron, due to the center of mass frame proving lack of conservation of momentum, though a single photon can collide into a mass and pair produce
\1 Existance of atoms based on the macroscopic shape of crystals, the inability to convert elements to another type, stoichiometry chemistry rules, and periodic table able to predict properties of atoms
\1 de Broglie later stated that massive particles could have wavelength by the same formuila, $p\lambda = \lambda \sqrt{2mK} = h$, where K is kinetic energy
\end{outline*}

\section{Chapter 3 - Quantization of Charge, Light, and Energy}
\begin{outline*}
\1 In the 1800s, Faraday proved that a specific quantity of electricty could decompose one gram-ionic weight of monovalent ions, equal to a Faraday, or a mole of electrons, such that $Q = N_Ae$, called Faraday's Law of Electrolysis, displaying discrete electric charges
	\2 Zeeman later discovered that discrete spectral lines emitted by an atom in a magnetic field separate into three spaced lines of different frequencies, caused by the slightly different charge to mass ratios
		\3 This proved that the particles producing the light were negative, and found the charge to mass ratio of the electrons
	\2 Thomsons cathode ray experiment later measured the same ratio as Zeeman, proving the existance of the electron, with the same ratio, as being the atomic negative component
		\3 This combined with Faraday's charge allowed the mass of the electron to be determined
		\3 He also used a uniform magnetic field creating a circular path to measure the same ratio, found to be the same for all materials, showing it was universal for atoms
	\2 Millikan attempted to use a cloud of water droplets with a charge, such that Q = Ne, using the mass of the cloud and the radius of the drop to find e, found to be difficult because of the evaporation
		\3 On the other hand, he found a single drop could be balanced in the air by an electric field, eventually picking up an ion causing a movement in some direction
		\3 This resulted in the oil drop experiment, giving each charge and preserving it in midair, measuring the force on it, to confirm the electron charge
\1 Absorbed radiation increases the kinetic energy of oscillating atoms, increasing the temperature, but resulting in increased radiation emission by electrons, reducing kinetic energy, called thermal radiation
	\2 At thermal equilibrium, the rate of absorption and emission are equal, such that higher freqencies are present at higher temperatures, due to higher energy
	\2 Ideal blackbodies emit and absorb all incident radiation by $R = P/A = \omega T^4$, where Stefan's constant ($\omega$) is $5.67 * 10^8 W/m^2K^4$
		\3 Non-blackbodies emit multiplied by some emissivity constant, based on factors such as color, temperature, and composition
	\2 Spectral distribution of the radiation of a blackbody also only depends on temperature, where the maximum emitted wavelength, $\lambda_{max} T = 2.898 * 10^{-3} m*K$, called Wien's displacement law
	\2 Blackbodies are approximated by a cavity with a small hole to let radiation in, found that the power radiated out, $R = \frac{1}{4}cu$, where U is the total radiation energy density in the cavity
		\3 As a result, both are proportional to the wavelength, such that the energy density distribution can be found by the number of modes of oscillation
		\3 It is found that the number of modes of standing wave oscillation per unit volume, $n = 8\pi\lambda^{-4}$, and the Rayleigh-Jeans equation states that $u = kTn$, such that R can be calculated
		\3 As a result, while at higher wavelengths, it fits experimentally, at low wavelengths, it appears $R \to \infty$, called the ultraviolet catastrophy, such that total energy density over the spectrum from 0 to $\infty$ would be infinite as well
	\2 Planck's Law corrects for this, stating that since as $\lambda$ approaches 0, n approaches infinity by classical formulas, u must be a function of wavelength, such that it approaches 0
		\3 Classically, electrons oscillating produce waves with equal frequency, where the average energy is found by the energy distribution function, $n(E) = Ae^{-E/kT}$ based on the Maxwell-Boltzmann distribution function, such that average energy, $\bar{E} = \int^\infty_0 Ef(E)dE = kT$
			\4 In this equation, n is the fraction of oscillators with energy E
			\4 This is based on the fact that for some standing wave, in space a, $\lambda = \frac{2a}{n}$ must be true for soem integer n, such that the number of modes, $\Delta n = \int^{f_2}_{f_1} \frac{4a}{c}df$, the additional multiple of 2 to account for two polarizations
			\4 In higher dimensions, $f = \frac{c}{2a}\sqrt{n_x^2 + n_y^2 + n_z^2}$, such that $N(f)df = 2(\frac{2a}{c})^3(\frac{1}{8}4\pi f^2df)$, when the space is modeled as a sphere of radius a, only in a single octant
			\4 The higher dimensional mode equation combined with Boltzmann distribution gives the Rayleigh-Jeans formula
		\3 Planck found it agreed with expermental data if oscillator energy is a multiple of a discrete value, such that $E = nhf$, where h is Planck's constant
			\4 The sum can be taken similarly  such that $\bar{E} = \frac{hf}{e^{hf/kT} - 1}$
			\4 It follows that $u = \frac{8\pi hc\lambda^{-5}}{e^{hc/\lambda kT} - 1}$, called Planck's Law, found to be a generalization of all known laws
	\2 Blackbody radiation has been used as proof of the Big Bang Theory, due to the universe predicted to act as a perfect black body in terms of energy distribution
\1 In Hertz's spark gap experiment to generate EM waves and detect them, proving Maxwell's Theories, finding that light hitting a surface produced an electron current
	\2 Lenard later proved that it was electrons, and observed the current was proportional to the intensity (P = IA), but found that there was no minimum intensity needed as would be classically expected, due to requiring enough energy to
		\3 Fluxs of photons are the photons per second per unit area, related to intensity 
		\3 Since the kinetic energy had to be great enough to avoid being pulled back to the metal surface cathode if there was a negative voltage, it required a voltage produced greater than $-V_0$, called the stopping potential
		\3 Thus, it was found that $KE_{max} = eV_0$, such that $KE_{max}$ was independent of the light intensity as well, rather than increasing the electron kinetic energy
	\2 Einstein postulated as a result that Planck's quantization was universal, such that E = hf for all light quanta, such that $eV_0 = hf - \phi$, where $\phi$ is the work function, characteristic of the metal, to remove an electron
		\3 This is equal to the maximum kinetic energy by energy conservation, though some electrons lose energy when leaving the metal further
		\3 As a result, the threshold wavelength is equal to the work function divided by h
		\3 This also explains the lack of a time lag for the production of photoelectrons, instead of the calculated time for enough energy if it is spread evenly over the surface, as assumed classically
\1 Xrays originally were discovered by Roentgen with a cathode ray tube, noticing rays from the collision of electrons and the glass tube could activate flurescent photographic film and pass through opaque materials
	\2 He later observed no material was opaque, though less rays could pass through with higher densities
	\2 He stated that their apparent lack of magnetic deflection, refraction, or interference, was due to a very short wavelength, finding they were defracted by a crystal lattice, also proving a regular crystal array
	\2 He found they were produced by eletrons when deflected then stopped by the atoms of a target
	\2 He then thought to use Bragg places, or face-centered cubic molecular structures of NaCl crystals, analyzing the scattered waves from each atom to view xray diffraction
		\3 The waves are in phase regardless of wavelength if the scattering angle is equal to the incident angle for two waves hitting atoms in a plane
		\3 This condition is called the Bragg condition, true if $2dsin(\theta) = m\lambda$, where d is the distance between atoms and $\theta$ is the angle from the surface, for constructive interference, found in soap bubbles
		\3 The amount of ionization from xrays can then be measured to get the intensity of each wavelength, after it has been corrected by the Bragg condition to get the correct angle
	\2 This produces a spectra for the xrays produced, based on the anode of the material, able to be used to determine the atomic spacing
		\3 This produces both the characteristic spectrum of sharp lines and the continuous spectrum, the former which is specific to the material
			\4 Maxwell had previously predicted the continuous spectrum was due to the electron bombardment/deacceleration in the strong electric field, though the cause of the characteristic was unknown
		\3 There is also a cutoff wavelength, independent of the material, based on the energy of the bombarding electrons, by $\lambda = \frac{1.24 * 10^3 nm}{V}$, called the Duane-Hunt Rule
			\4 This is explained as the opposite of the photoelectric effect, such that all the kinetic energy is converted ($\lambda = \frac{hc}{eV}$)
\1 Compton later measured the scattering of xrays by free electrons, proving further both the photon and special relativity
	\2 Classical EM would have predicted a dipole oscillation due to the light field of the particle, with the maximum at the original wavelength, based on $1 + cos^2(\theta)$ for the angle of oscillation, rather than a shifted maximum
	\2 He observed that the scattered xrays were more easily absorbed, considering that the collision allowed an electron to absorb some of the photon energy, such that the wavelength became longer
	\2 As a result, he derived the Compton equation mechanically through relativity and quantization stating $\Delta \lambda = \frac{h}{mc}(1 - cos(\theta))$, where $\frac{h}{mc}$ is called the Compton wavelength of the electron
		\3 This was observed for xrays due to the percent change in wavelength only being noticable for very short original wavelengths
		\3 The unshifted, but scattered, portion is due to electrons tightly bound to the atom, such that the entire atom recoils
\end{outline*}

\section{Chapter 4 - The Nuclear Atom}
\begin{outline*}
\1 Newton's dispersion of white light was the first spectroscope, later shown to have hundreds of numbers of dark lines inside, and bright light spectra formed by flames, forming the study of spectroscopy
	\2 Continuous spectra are emitted by incandescent solids, showing no specific lines in any spectrscope
	\2 Band spectra are closely packed groups of lines apparently continuous in low resolution produced by fire, while line spectra are those produced by unbound chemical elements, both characteristic of material
\1 In 1885, Balmer found that visible and near-UV spectrum of H could be represented by $\lambda_n = 364.6 \frac{n^2}{n^2 - 4} nm$, where n = 3, 4, ..., called the Balmer series
	\2 The general form for other elements was later found as the Rydberg-Ritz formula, such that $\frac{1}{\lambda_{mn}} = R(\frac{1}{m^2} - \frac{1}{n^2}) \forall n > m$
		\3 R is the rydberg constant, varying slightly for elements, but negligably, such that $R_H = 1.096776 * 10^7 m^{-1}$ and $R_{\infty} = 1.097373 * 10^7 m^{-1}$
		\3 Thus, for the Balmer series, m = 2, such that the maximum wavelength is as n approaches infinity
\1 Thomson's plum pudding model was the most popular atomic model after Balmer and Rydberg-Ritz formulas, trying to find stable configurations with normal modes of vibration of spectral lines
	\2 On the other hand, the lack of continual emission due to electrostatic radiation and inability to make a model hindered this theory
\1 Rutherford, while studying radioactivity, finding that alpha particles were doubly ionized helium, checking the spectral lines to prove it, used the particles to study atomic interiors
	\2 He sent the radiation into gold foil, measuring the scattering angles, mainly undeflected or barely detected, though a few with right angles or more, showing the positively charged sphere could not have created such high angle scattering
		\3 Thomson's model did not have enough force at any point for a large enough deflection
	\2 Rather there must be a dense center, such that Rutherford calculated angular distribution and dependence on nuclear charge, angle, and kinetic energy, validated experimentally
	\2 This can be calculated mechanically, where b is the impact parameter, or the distance from the line through the nucleus, such that $b = \frac{kq_{\alpha}Q}{m_{\alpha}v^2}cot(\frac{\theta}{2})$
		\3 Thus, it is found that for intensity $I_0$ (in particles per second), the number scattered by one nucleus through angles greater than $\theta$ is the number with impact parameters less than $b(\theta)$, equal to $\pi b^2 I_0$
		\3 $\pi b^2$ is called the cross section $\omega$ for scattering angles greater than $\theta$, such that it is multiplied by intensity and number of nuclei for the number of scatterings above that value
			\4 The number of nuclei is calculated such that $n = \frac{\rho N_A}{M}$, where M is the molar mass
			\4 As a result, the fraction of scatters above some angle $\theta$ is $\pi b^2 nt$, where t is the thickness of the surface
	\2 Rutherford later derived an equation for the number of particles scattered at some angle, $\Delta N = (\frac{I_0A_{sc}ntkZe^2}{2r^2E_k})^2\frac{1}{sin^4(\frac{\theta}{2})}$, where $A_{sc}$ is the detector area and r is the distance from the foil to the detector
		\3 In this case, Z is the atomic number and t is the thickness
		\3 This was later verified experimentally, showing the legitimacy of atomic theory
	\2 Rutherford's model does not assume a point charge, but rather just a ball, and assumes the alpha particle does not penetrate it, such that for each angle, the closest distance from the nucleus can be calculated
		\3 Thus, for the largest angle ($180^o$), the collision is almost head on, providing an upper bound for the atomic radius
		\3 For some nucleus, by conservation of energy, $r = \frac{kq_{\alpha}Q}{\frac{1}{2}m_{\alpha}v^2}$, using the point when the number of collisions at some angle changes as the radius based on the initial kinetic energy
\1 Bohr later specified the charge and mass of electrons, stating they orbited by Coulomb force, but was unstable due to accelerating toward the center, with EM predicting that light of $f = \frac{v}{2\pi r}$, proportional to $\frac{1}{r^{\frac{3}{2}}}$
	\2 For the electrons, $E = \frac{1}{2}mv^2 - \frac{kZe^2}{r}$, proportional to $-\frac{1}{r}$, such that it will be losing energy, with decreasing radius
		\3 As the energy is lost due to the emission of radiation, emitting a continuous spectrum as the radius changes, the atom will decay, rendering it invalid
	\2 Bohr solved this by stating electrons has certain stationary states they could orbit without emission, and that the radiation emitted was due to the change in state, $hf = E_i - E_f$, called the Bohr frequency condition
		\3 He also made the correspondence principle, stating that the limit of large orbits and energies, all quantum must align with classical
\end{outline*}
\end{document}