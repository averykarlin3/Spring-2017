\documentclass[11 pt, twoside]{article}
\usepackage{textcomp}
\usepackage[margin=1in]{geometry}
\usepackage[utf8]{inputenc}
\usepackage{color}
\usepackage{indentfirst} %Comment out for no first paragraph indent
\usepackage[parfill]{parskip}
\usepackage{setspace}
\usepackage{tikz}
\usepackage{amsmath}
\usepackage{amsfonts}
\usepackage{amssymb}
\usepackage{enumitem}
\usepackage{outlines}
\usepackage{physymb}

\usepackage{fancyhdr}
\pagestyle{fancy}
\cfoot{\hyperlink{content}{\thepage}}
\lhead{}
\chead{}
\rfoot{}
\lfoot{}
\rhead{}
\renewcommand{\headrulewidth}{0pt}
\renewcommand{\footrulewidth}{0pt}


\usepackage{hyperref}
\hypersetup {
	colorlinks,
	citecolor=black,
	filecolor=black,
	linkcolor=black,
	urlcolor=black
}

\newcommand{\sepitem}{0pt} %Added room between items on the list, not including a list and its sublist
\newcommand{\seppar}{1pt} %Between items and lists overall

\setenumerate[1]{itemsep=\sepitem, parsep=\seppar}
\setenumerate[2]{itemsep=\sepitem, parsep=\seppar}
\setenumerate[3]{itemsep=\sepitem, parsep=\seppar}
\setenumerate[4]{itemsep=\sepitem, parsep=\seppar}

\newenvironment{outline*}
{
	\begin{outline}[enumerate]
	}
	{\end{outline}
}

\newcommand{\foot}[1]{\hyperlink{#1}{$_#1$}}

\begin{document}

\title{Electromagnetics I}
\author{Avery Karlin}
\date{Fall 2017}
\newcommand{\textbook}{Introduction to Electrodynamics by Griffiths}
\newcommand{\teacher}{Dr. Gene Mele}

\maketitle
\newpage
\hypertarget{content}{\tableofcontents}
\vspace{11pt}
\noindent
\underline{Primary Textbook}: \textbook\\
\underline{Teacher}: \teacher
\newpage

\section{Course Introduction}
\begin{outline*}
\1 Matter in charged state responds to extra forces, based on the charge of both, size and shape of both, proximity to other charge, and size, shape, and proximity of uncharged shapes as well
	\2 $\vec{F}_{1\to2} = \frac{q_1q_2}{4\pi\epsilon_0}(\frac{1}{d^2})(\frac{\vec{d}}{d})$, where $\frac{\vec{d}}{d} = \hat{e_d}$
	\2 In reality, there is a retarded time component, $t' = t - \frac{\vec{r}_2(t) - \vec{r}_1(t')}{c}$ and a velocity $\vec{u} = c\hat{e}_{r_2(t) - r_1(t')} - v_1(t') = c\hat{e}_{r'}(t') - \vec{v}_1(t')$, where r' is the difference between the r vectors
		\3 See Formula From Notes Page for Modified Coulumb's Law
		\3 By extension, if the velocity and acceleration of the source and the velocity of the target are 0, this simplifies to the conventional form of Coulumb's Law
		\3 If the source velocity only is non-zero, it becomes proportional to the relativistic $\gamma$, rewritten by $c(t - t') = |\vec{r}_2(t) - \vec{r}_1(t')|$, such that $F = \frac{kq_1q_2\hat{e}_{d}}{d^2}(\frac{1 - \frac{v_1^2}{c^2}}{(\hat{e}_{r_1} \cdot \hat{e}_{r_2})^{3/2}}$
		\3 Otherwise, this can be rewritten as $F_{1\to2} = q_2\vec{E} + q_2\vec{v}_2 \times \vec{B}$, providing the field formulation of the law
\1 $\hat{e}_r(t, t')$ is the vector pointing from r2 at time t to r1 at time t'
\1 **8/31 Notes**
\1 Complex force law is able to be rewritten $\vec{F} = q\vec{E} + q\vec{v} \times \vec{B}$, due to fields being easier by geometric constraints
	\2 This is done by using the flux through the object, equal to the surface integral of the vector field over the surface ($\Phi = \oint_S \vec{V} \cdot d\vec{a}$)
		\3 As the surface size approaches 0, the flux approaches the divergence times the volume of the element, showing the Divergence Theorem
		\3 Similar derivation can be done for the Stokes Theorem
	\2 This also allows it to be broken up into transverse and longitudinal components, by the Helmholtz Theorem
\end{outline*}
\section{Chapter 1 - Vector Analysis}
\begin{outline*}
\1 Vector addition is commutative and associative, while multiplication by a scalar is distributive over vector addition
\1 The dot product is defined by $\vec{A} \cdot \vec{B} = ABcos(\theta)$, where $\theta$ is the tail-to-tail angle between the vectors, returning a scalar, and is both commutative and distributive over vector addition
\1 The cross product is defined by $\vec{A} \times \vec{B} = ABsin(\theta)\hat{n}$, where the direction is defined by the right hand rule, curling from the first to the second
	\2 As a result, it is distributive over vector addition, but not commutative or associative, but rather the inverse when commuted
	\2 Geometrically, it evaluates to the area of the parallelogram made up of two vectors joined at the tails
		\3 Thus, a triple cross product is the volume of the parallelepiped, able to be rewritten by the formula $A \times (B \times C) = B(A \cdot C) - C(A \cdot B)$
\1 Vector components are used in terms of normalized basis vectors, generally using mutually perpendicular unit vectors, such that the dot product with themselves is 1, with another is 0
	\2 The cross product in components is able to be written as the determinant of a matrix with the unit vectors as the first row, the components of the first vector as the second, and so forth
	\2 It is notable that $A \cdot (B \times C) = (A \times B) \cdot C$, equal to the determinant matrix with each row as the components of A, B, and C respectively
\1 For some position vector, the unit vector is equal to $\frac{\vec{r}}{|\vec{r}|}$, while the infinitesimal displacement vector is $d\vec{l} = dx\hat{x} + dy\hat{y} + dz\hat{z}$
	\2 The source point is denoted by the $\vec{r}'$ vector, while the field point is denoted by $\vec{r}$, while the separation vector is $\vec{\pmb{\scriptr}} = \vec{r} - \vec{r}'$
\1 **1.1.5 Skipped - Possibly Not Needed
\1 The gradient, $\nabla T$ is the vector generalizzed derivative, pointing in the direction of the maximum increase of T, with the magnitude as the slope in that direction
	\2 The directional derivative ($\frac{dT}{d\vec{l}}$) is thus a unit vector in the direction intended, with the dot product against the gradient ($\nabla T \cdot \hat{e}_{\vec{l}}$)
	\2 As a result, when the gradient is 0, it is either a maximum, minimum, or saddle point, generally called a stationary point
	\2 The del operator acts as a vector operator, $\vec{\nabla} = \hat{x}\frac{\partial}{\partial x} + ...$, such that when acting on a scalar function, it is the gradient
		\3 The divergence is equal to $\vec{\nabla} \cdot \vec{v}$, acting as a measure of how much the vector function spreads out from the point (positive is an outward source, negative inward sink)
		\3 The curl is equal to the cross product with a vector function, acting as a measure of the vector function swirling around the point, the direction given by the right hand rule
	\2 The gradient of a function sum or a constant multiplied by a function is equal to the sum of the gradients of the functions or the constant multiplied by the gradient
		\3 The constant can also be moved outside the product for the dot and cross products
	\2 Due to the higher number of types of products and quotients, there are more rules for del operators on functions
		\3 Thus, $\nabla(fg) = f\nabla g + g\nabla f$
			\4 $\nabla(\vec{A} \cdot \vec{B}) = \vec{A} \times (\nabla \times \vec{B}) + \vec{B} \times (\nabla \times \vec{A}) + (\vec{A} \cdot \nabla)\vec{B} + (B \cdot \nabla)\vec{A}$
			\4 $\nabla \cdot (f\vec{A}) = f(\nabla \cdot \vec{A}) + \vec{A} \cdot (\nabla f)$
			\4 $\nabla \times (f\vec{A}) = f(\nabla \times \vec{A}) - \vec{A} \times (\nabla f)$
			\4 $\nabla \cdot (\vec{A} \times \vec{B}) = \vec{B} \cdot (\nabla \times \vec{A}) - \vec{A} \cdot (\nabla \times \vec{B})$
			\4 $\nabla \times (\vec{A} \times \vec{B}) = (\vec{B} \cdot \nabla)\vec{A} - (\vec{A} \cdot \nabla)\vec{B} + \vec{A}(\nabla \cdot \vec{B}) - \vec{B}(\nabla \cdot \vec{A})$
		\3 Further, $\nabla (\frac{f}{g}) = \frac{g\nabla f - f\nabla g}{g^2}$
			\4 $\nabla \cdot (\frac{\vec{A}}{g}) = \frac{g(\nabla \cdot \vec{A}) - \vec{A} \cdot (\nabla g)}{g^2}$
			\4 $\nabla \times (\frac{\vec{A}}{g}) = \frac{g(\nabla \times \vec{A}) + \vec{A} \times (\nabla g)}{g^2}$
	\2 The different forms are able to be combined for the divergence or curl of gradient, the gradient of divergence, or the divergence or curl of curl
		\3 The divergence of gradient, or the Laplacian, is equal to the sum of the second partial derivatives, written $\nabla^2 T$
		\3 The curl of gradient and the divergence of curl are 0, the former based on $T_{xy} = T_{yx}$, able to be thought of intuitively, though not rigorously, as the operator crossed with itself
			\4 The latter can be non-rigorously proved by the identity $A \cdot (B \times C) = (A \times B) \cdot C$
		\34 The gradient of divergence is different from the Laplacian, without much physical significance
		\3 The curl of curl is found to simplify to a formula of the others, such that $\nabla \times (\nabla \times \vec{v}) = \nabla(\nabla \cdot \vec{v}) - \nabla^2 \vec{v}$, defining a vector Laplacian
			\4 In Cartesian coordinates, a vector Laplacian reduces to the vector of the Laplacian of each component
\1 For some scalar function, the Fundmental Theorem for gradients states that $\int^{\vec{b}}_{\vec{a}} (\nabla T) \cdot d\vec{l} = T(\vec{b} - T(\vec{a})$, such that the line integral of a gradient is path independent and a closed loop integrates to 0
	\2 Similarly, the Divergence/Green's/Gauss's Theorem states that $\int(\nabla \cdot \vec{v}) dV = \oint_{S} \vec{v} \cdot d\vec{A}$
	\2 The Stoke's Theorem states that $\int_S (\nabla \times \vec{v}) \cdot d\vec{A} = \oint_P \vec{v} \cdot d\vec{l}$
		\3 The righthand rule determines the direction of the vectors, using the outward normal for a closed surface
		\3 The theorem doesn't depend on the surface within the boundary line, due to infinite surfaces within a boundary
		\3 For some closed surface, since the boundary line chosen is able to be infinitessimal, the integral of the curl is 0
\1 Spherical coordinates have r, $\theta$ (angle down from the z-axis), and $\phi$ (angle around the x-axis), the former called the polar angle, the latter the azimuthal angle
	\2 These provide the cartesian coordinates by $x = rsin(\theta)cos(\phi), y = rsin(\theta)sin(\phi), z = rcos(\theta)$
	\2 This is able to be written in terms of orthogonal normal vectors, $\hat{r} = sin(\theta)cos(\phi)\hat{x} + sin(\theta)sin(\phi)\hat{y} + cos(\theta)\hat{z}, \hat{\theta} = cos(\theta)cos(\phi)\hat{x} + cos(\theta)sin(\phi)\hat{y} - sin(\theta)\hat{z}, \hat{\phi} = -sin(\phi)\hat{x} + cos(\phi)\hat{y}$
		\3 As a result, these normal vectors change based on the point in question, with $\hat{r}$ pointing outward, relying on the angles
		\3 Thus, the vector operations cannot be used on them normally and differentiation requires them to act as function as well ($\hat{r}(\theta, \phi)...$)
		\3 In addition, an increment in $\theta, \phi$ is not a unit of length, such that rather, $dl_{\theta} = rd\theta, dl_{\phi} = rsin(\theta)d\phi$, unlike r, which is incremented normally
			\4 Similarly, $dV = dl_{\theta}dl_rdl_{\phi} = r^2sin\theta drd\theta d\phi$
			\4 Integrating over a surface replaces an increment with the unit vector, such that for constant r, $dS = r^2sin(\theta)d\theta d\phi \hat{r}$
	\2 Using the chain rule, expressions for gradient, divergence, curl, laplacian can be determined, such that $\nabla T = \frac{\partial T}{\partial r}\hat{r} + \frac{1}{r}\frac{\partial T}{\partial \theta}\hat{\theta} + \frac{1}{rsin(\theta}\frac{\partial T}{\partial \phi}\hat{\phi}$
		\3 $\nabla \cdot \vec{v} = \frac{1}{r^2}\frac{\partial(r^2v_r)}{\partial r} + \frac{1}{rsin(\theta)}\frac{\partial(sin(\theta)v_{\theta})}{\partial \theta} + \frac{1}{rsin(\theta}\frac{\partial v_{\phi}}{\partial \phi}$
		\3 $\nabla \times \vec{v} = \frac{1}{rsin(\theta)}(\frac{\partial(sin(\theta)v_{\phi)}}{\partial \theta} - \frac{\partial v_{\theta}}{\partial \phi})\hat{r} + \frac{1}{r}(\frac{1}{sin(\theta}\frac{\partial v_r}{\partial \phi} - \frac{\partial(rv_{\phi)}{\partial r})})\hat{\theta} + \frac{1}{r}(\frac{\partial(rv_{\theta})}{\partial r} - \frac{\partial v_r}{\partial \theta})\hat{\phi}$
		\3 $\nabla^2 T = \frac{1}{r^2}\frac{\partial}{\partial r}(r^2\frac{\partial T}{\partial r}) + \frac{1}{rsin(\theta)}\frac{\partial}{\partial \theta}(sin(\theta)\frac{\partial T}{\partial \theta}) + \frac{1}{r^2sin^2(\theta}\frac{\partial^2 T}{\partial \phi^2}$
\1 Cylindrical coordinates have $s/r, \phi$, but use the cartesian z rather than $\theta$, such that $x = scos(\phi), y = ssin(\phi), z = z$
	\2 As a result, $\hat{s} = cos(\phi)\hat{x} + sin(\phi)\hat{y}, \hat{\phi} = -sin(\phi)\hat{x} + cos(\phi)\hat{y}$, and $dl_{\phi} = sd\phi$, such that $dV = sdsd\phi dz$
	\2 Similarly, by the chain rule, $\nabla T = \frac{\partial T}{\partial s}\hat{s} + \frac{1}{s}\frac{\partial T}{\partial \phi}\hat{\phi} + \frac{\partial T}{\partial z}\hat{z}$
		\3 $\nabla \cdot \vec{v} = \frac{1}{s}\frac{\partial(sv_s)}{\partial s} + \frac{1}{s}\frac{\partial v_{\phi}}{\partial \phi} + \frac{\partial v_z}{\partial z}$
		\3 $\nabla \times \vec{v} = (\frac{1}{s}\frac{\partial v_{z}}{\partial \phi} - \frac{\partial v_{\phi}}{\partial z})\hat{s} + (\frac{\partial v_s}{\partial z} - \frac{\partial v_z}{\partial s})\hat{\phi} + \frac{1}{s}(\frac{\partial(sv_{\phi})}{\partial s} - \frac{\partial v_s}{\partial \phi})\hat{z}$
		\3 $\nabla^2 T = \frac{1}{s}\frac{\partial}{\partial s}(s\frac{\partial T}{\partial s}) + \frac{1}{s^2}\frac{\partial^2 T}{\partial \phi^2} + \frac{\partial^2 T}{\partial z^2}$
\1 For the vector function, $\vec{v} = \frac{1}{r^2}\hat{r}$, the divergence is found to be 0, the surface integral is $4\pi$ around the origin, and the volume integral of the divergence is 0	
	\2 While, this appears to contradict the divergence theorem, explained by the divergence not being 0 at the origin only, called the Dirac delta function
	\2 The Dirac Delta Function is defined as infinity at x = 0, 0 elsewhere, with an integral over the whole domain of one, technically not a function, called a distribution or generalized function
		\3 This is used such that $\int^{\infty}_{-\infty} f(x)\delta(x)dx = f(0)$, able to be shifted with $\delta(x - a)$ to get f(a)
		\3 While Delta functions are not actual functions, they are able to be used in functions involving integral signs
			\4 For two functions with delta functions, $D_1, D_2$, they are equal if $\int^{\infty}_{-\infty} f(x)D_1(x)dx = \int^{\infty}_{-\infty} f(x)D_2(x)dx$
		\3 The Delta function is formally defined by the equation $\nabla \cdot (\frac{\hat{r}}{\vec{r} - \vec{r'}} = 4\pi\delta^3(\vec{r} - \vec{r'})$, where it is being differentiated purely in terms of r, with r' constant
			\4 Since $\nabla \frac{1}{\vec{r} - \vec{r'}} = \frac{-\hat{r}}{\vec{r} - \vec{r'}}$, such that the Delta function can be put as a Laplacian
			\4 Further, for the seperation vector, $\int 4\pi\delta^3(\vec{\pmb{\scriptr}})f(\vec{r}')d\vec{r}' = f(\vec{r}$, such that the alternate order is equally valid
	\2 The 3D Delta function, $\delta^3(\vec{r}) = \delta(x)\delta(y)\delta(z)$, such that it is zero purely at the origin, integrating over the volume domain to 1, able to be used such that $\int_{\mathbb{R}^3} f(\vec{r})\delta^3(\vec{r} - \vec{a})dV = f(\vec{a})$
		\3 Thus, it can be stated that $\nabla \cdot (\frac{\hat{r}}{r^2}) = 4\pi\delta^3(\vec{r})$, where r is either the position or displacement vector, and $\hat{r}$ is either the unit displacement or radial vector
\1 The Helmholtz Theorem states that a field is uniquely defined by boundary conditions, divergence, and curl
	\2 Generally in electrodynamics, the boundary condition has the field go to zero as it approaches infinity, such that it is only unique in that case
	\2 As a corollary, curl-less/irrotational fields, such that the curl is $\vec{0}$, are able to be written as the negative gradient of a scalar potential function $\vec{F} = -\nabla V$
		\3 By extension, $\int^b_a \vec{F} \cdot d\vec{l}$ is path independent and for a closed loop, equal to 0, for an irrotational field
		\3 Each of these conditions imply the other conditions for a vector field
	\2 As another corollary, divergence-less/solenoidal fields imply and are implied by $\vec{F} = \nabla \times \vec{A}$ (the curl of some vector function), $\int \vec{F} \cdot d\vec{a}$ being independent of surface for any boundary, or for a closed surface, $\oint \vec{F} \cdot d\vec{a} = 0$
		\3 $\vec{A}$ is the vector potential, non-unique due to any gradient being able to be added since the curl of gradient is 0
	\2 As a result, any vector field can be defined by $\vec{F} = -\nabla U + \nabla \times \vec{W}$
		\3 Further, $V(\vec{r}) = \frac{1}{4\pi} \int \frac{D(\vec{r'})}{\pmb{\scriptr}}d\tau', \vec{W}(\vec{r}) = \frac{1}{4\pi} \int \frac{\vec{C}(\vec{r}')}{\pmb{\scriptr}}d\tau'$, where $\vec{\nabla} \cdot \vec{F} = D, \vec{\nabla} \times \vec{F} = \vec{C}$
			\4 It is assumed that $\nabla \cdot \vec{C} = 0$ since the divergence of a curl is 0
	\2 This has the example for an electrostatic field, such that $\nabla \cdot \vec{E} = \frac{\rho}{\epsilon_0}, \nabla \times \vec{E} = \vec{0}, \vec{E}(\vec{r}) = -\nabla V$
		\3 Similarly for a magnetostatic field, $\nabla \cdot \vec{B} = 0, \nabla \times \vec{B} = \mu_0 \vec{J}, \vec{B}(\vec{r}) = \nabla \times \vec{A}$
% \1 Coordinate systems are assumed to have orthogonal unit vectors, as functions of position (varying based on location) except in cartesian systems
% 	\2 Thus, the displacement by $(du, dv, dw)$ can be written as $d\vec{l} = fdu\hat{u} + gdv\hat{v} + hdw\hat{w}$, where f, g, h are multipliers (f = 1, g = r, h = r sin$\theta$ in spherical, f = g = h = 1 in cartesian)
% 	\2 The gradient can be written as $dt = \frac{\partial t}{\partial u}du = \nabla t \cdot d\vec{l} = \nabla t fdu$, such that $\nabla t = \frac{1}{f}\frac{\partial t}{\partial u}$, providing the fundamental theorem of gradients
% 	\2 **Remainder of Appendix A**
\end{outline*}
\section{Chapter 2 - Electrostatics}
\begin{outline*}
\1 The problem of the electric force exerted on a test charge by a set of source charges, given as functions of time, is solved by the principle of superposition
	\2 This found to depend on both velocities, the source acceleration, and the delayed time of field movement (speed of light)
	\2 Electrostatics are the subset of electromagnetic problems in which source charges are stationary
\1 Coulumb's Law is the main law of electrostatics, stating that $\vec{F} = \frac{qQ\hat{e}_{\pmb{\scriptr}}}{4\pi\epsilon_0\pmb{\scriptr}^2}$, where q is the source, Q is the target, $\vec{\pmb{\scriptr}}$ is the separation vector, equal to $\vec{r} - \vec{r}'$, where $\vec{r}$ is the vector to the test, $\vec{r}'$ to the source
	\2 As a result, the electric field, $\vec{E}(\vec{r})$ is defined such that $\vec{F} = Q\vec{E}$, defined at the field point of the target charge
	\2 For some continuous charge distribution, $\vec{E}(\vec{r}) = \frac{1}{4\pi\epsilon_0}\int\frac{1}{\pmb{\scriptr}^2}\hat{e}_{\pmb{\scriptr}}dq$, where $dq = \lambda(\vec{r}') dl' = \sigma(\vec{r}') da' = \rho(\vec{r}') d\tau'$
		\3 It is notable that $\hat{e}_{\pmb{\scriptr}}$ is not a constant, but rather depends on the location in the charge distribution
\1 Gauss's Law is used to simplify calculation of the electric field, based on the field line representation, with connected arrows showing the field, going to infinity,, where the magnitude is based on the density of the lines
	\2 This is notably flawed, since it is a 2D density representation of 3D fields
		\3 Electric field lines are not allowed to cross, moving from positive outward, either to infinity or ending at negative
	\2 The number of field lines through a surface is called the flux for some sampling rate of field lines (rather than the actual infinite number), defined mathematically by $\Phi_E = \int_S \vec{E} \cdot d\vec{a}$, where $\vec{a}$ is perpendicular to the field
	\2 Gauss's Law is derived for a point charge and extended to general by superposition, viewing all charge bodies as the sum of point charges, stating that $\oint \vec{E} \cdot d\vec{a} = \frac{Q_{enc}}{\epsilon_0}$
		\3 The differential form of the law, based on the fact that $Q_{enc} = \int_V \rho d\tau$, where $\rho$ is the charge density, states $\vec{\nabla} \cdot \vec{E} = \frac{\rho}{\epsilon_0}$
		\3 This is able to be derived both from the divergence of the field and the integral of the field through the surface to get either form
	\2 Gauss's Law best applies to symmetrical objects, either cylindrical, spherical, or planar, extended to approximately infinity
		\3 For a uniform sphere or spherical shell, the electric field outside is the same as if it was contained in the center as a point charge
		\3 For a plane, it is found that $\vec{E} = \frac{\omega \hat{n}}{2\epsilon_0}$, such that it is constant, independent of distance
		\3 In addition, for objects without symmetry, by the principle of superposition, it can be broken up into symmetrical parts
	\2 The curl of $\vec{E}$ can be calculated for a point charge, and extended by superposition to the general case, by the Stokes Theorem, such that it is found that $\oint \vec{E} \cdot d\vec{l} = 0$, or $\vec{\nabla} \times \vec{E} = \vec{0}$
\1 Since the curl of $\vec{E}$ is 0, such that by the Helmholtz Theorem, it is equal to the gardient of a scalar, and the loop integral around a closed loop is 0 and the the line integral of $\vec{E}$ is path independent by Stokes
	\2 As a result, since the derivatives of each component of the electric field with respect to each variable are interconnected, the function is able to be contained within a scalar function
	\2 By it being path independent, an electric potential can be defined, such that $V(\vec{r}) = -\int^{\vec{r'}}_0 \vec{E} \cdot d\vec{l'}$ or $\vec{E} = -\nabla V$
		\3 Electric potential is distinct from potential energy
		\3 Equipotential surfaces are those in which the potential is constant over the surface
			\4 It is notable that for regions of equal potential, there is no field between them, but potential is not required to be 0
		\3 Since potential is relevant in terms of potential difference between two points, the origin/reference point is unimportant, due to just changing a constant addition to the potential throughout
			\4 The point of 0 potential is generally assumed to be the point infinitely distant from all electric charges, though this causes issues for infinite charge distributions, in which case other references must be used
	\2 Electric potential obeys the principle of superposition, similar to force and electric field, measured in Volts or J/C
	\2 By Gauss's Law, it is found that $\nabla^2 V = \frac{-\rho}{\epsilon_0}$, known as Poisson's equation, where it is the Laplacian equation in regions of no charge
		\3 For some localized charge distribution, $V(\vec{r}) = \frac{1}{4\pi\epsilon_0} \int \frac{\rho(\vec{r}')}{\pmb{\scriptr}}d\tau'$, assuming a reference point at infinity, avoiding the vectors needed for direct field calculations
\1 By Gauss's Law, it is seen that there is a discontinutity in the field around a surface, such that $E_{above-perp} - E_{below-perp} = \frac{\sigma}{\epsilon_0}$, due to being undefined at the charge
	\2 This assumes that upward is the vector direction to calculate the difference in the vector fields
	\2 Further, by the fact that the curl is 0 of the electric field, the integral of the field around a closed loop is 0, such that the parallel electric field on both sides is equal
		\3 The sides have no contribution to the loop due to being of infinitessimal length, such that it is purely by the parallel portions of the loop
	\2 Thus, it is generalized to $\vec{E}_{abv} - \vec{E}_{blw} = \frac{\sigma \hat{n}}{\epsilon_0}$, where $\hat{n}$ is the upward surface normal
	\2 Since the potential must be continuous across any boundary, as the path length goes towards 0, the potential becomes constant across the boundary
		\3 On the other hand, since the derivative is the electric field, the normal derivative ($\frac{\partial V}{\partial n} = \nabla V \cdot \hat{n}$) has the same discontinuity as the electric field
\1 By classical mechanics, $W = \int^b_a \vec{F} \cdot d\vec{l} = -Q\int^b_a \vec{E} \cdot d\vec{l} = Q(V(b) - V(a))$, independent of path, such that electrostatic force is conservative
	\2 Thus, potential difference is equal to the work per unit charge, or the potential energy per unit charge
	\2 Since for each charge added, the number of potentials relative to each charge goes up, it can be written as a sum, such that $W = \frac{1}{4\pi\epsilon_0} \sum_{i = 1}^n \sum_{j > i}^n \frac{q_iq_j}{\pmb{\scriptr_{ij}}}$ is the total work to move all charges
		\3 This can also be written such that each pair is counted twice, but then divided by half, $W = \frac{1}{8\pi\epsilon_0} \sum_{i = 1}^n \sum_{j \neq i}^n \frac{q_iq_j}{\pmb{\scriptr_{ij}}}$
			\4 It can thus be written as a result in terms of $V(\vec{r})$, as the potential after all charges are in place, $W = \frac{1}{2}\sum_{i = 1}^n q_iV(\vec{r}_i)$
		\3 By extension, for a continuous charge distribution, $W = \frac{1}{2}\int \rho V d\tau = \frac{\epsilon_0}{2}\int (\nabla \cdot \vec{E})Vd\tau = \frac{\epsilon_0}{2}(\int_V E^2d\tau + \oint_S V\vec{E} \cdot d\vec{a}$
			\4 Since the region being integrated over can be any size that encompasses all of the charge, since the electric field increases as the volume integral rises, the surface integral decreases
			\4 As a result, as the domain rises infinitely, $W = \frac{\epsilon_0}{2}\int_{\mathbb{R}^3} E^2 d\tau$
	\2 While the work integral implies that the energy of a stationary charge distribution can only be positive, while the $\rho$ integral doesn't force that
		\3 This is possible due to the work integral of the electric field giving the total energy of the charge configuration, while $\rho$ not including the energy from the creation of the point charges
		\3 With point charges, since the potential created by a point charge is non-infintessimal, unlike a charge distribution, the energy from it would create a problem, and must be ignored
	\2 Since electrostatic energy is quadratic, it does not follow the principle of superposition, adding extra terms
	\2 The energy is neither considered to be stored in the charge or field in electrostatics, though it could be though to be in the charge by energy density $\frac{1}{2}\rho V$ or the field by $\frac{\epsilon_0}{2}E^2$
\end{outline*}
\section{Extra}
\begin{outline*}
\1 The Helmholtz Theorem states that for two fields with the same divergence and curl, the difference of the fields has a divergence and curl of 0
	\2 For a function with curl 0, it is equal to the gradient of a function, called a gauge function within gauge transformations
\1 Gauge transformations add an additional component to potentials to create more simplistic equations, adjusting the divergence of the vector potential, generally to 0
	\2 These added terms must create the same fields, while simplifying the equations, putting each of the added terms in terms of a gauge function, $\lambda(\vec{r}, t)$, generally equal to the function whose gradient makes the curlless field
	\2 The gauge potentials able to be used are specified by the gauge condition on the potentials themselves, not on the gauge transformation
\1 Helmholtz Theorem uniqueness depends on the fact that the divergence and curl converge to 0 faster than $\frac{1}{r^2}$ and the field itself converges to 0
	\2 As a corollary though, in other cases, if the field converges to 0 faster than $\frac{1}{r}$, it can be written as the gradient of a scalar potential plus the curl of a vector potential, though possibly not uniquely
\end{outline*}
\end{document}