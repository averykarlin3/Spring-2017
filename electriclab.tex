\documentclass[11 pt, twoside]{article}
\usepackage{textcomp}
\usepackage[margin=1in]{geometry}
\usepackage[utf8]{inputenc}
\usepackage{color}
\usepackage{indentfirst} %Comment out for no first paragraph indent
\usepackage[parfill]{parskip}
\usepackage{setspace}
\usepackage{tikz}
\usepackage{amsmath}
\usepackage{amsfonts}
\usepackage{amssymb}
\usepackage{enumitem}
\usepackage{outlines}
\usepackage{physymb}

\usepackage{fancyhdr}
\pagestyle{fancy}
\cfoot{\hyperlink{content}{\thepage}}
\lhead{}
\chead{}
\rfoot{}
\lfoot{}
\rhead{}
\renewcommand{\headrulewidth}{0pt}
\renewcommand{\footrulewidth}{0pt}


\usepackage{hyperref}
\hypersetup {
	colorlinks,
	citecolor=black,
	filecolor=black,
	linkcolor=black,
	urlcolor=black
}

\newcommand{\sepitem}{0pt} %Added room between items on the list, not including a list and its sublist
\newcommand{\seppar}{1pt} %Between items and lists overall

\setenumerate[1]{itemsep=\sepitem, parsep=\seppar}
\setenumerate[2]{itemsep=\sepitem, parsep=\seppar}
\setenumerate[3]{itemsep=\sepitem, parsep=\seppar}
\setenumerate[4]{itemsep=\sepitem, parsep=\seppar}

\newenvironment{outline*}
{
	\begin{outline}[enumerate]
	}
	{\end{outline}
}

\newcommand{\foot}[1]{\hyperlink{#1}{$_#1$}}

\begin{document}

\title{Electronics Lab}
\author{Avery Karlin}
\date{Fall 2017}
\newcommand{\textbook}{}
\newcommand{\teacher}{}

\maketitle
\newpage
\hypertarget{content}{\tableofcontents}
\vspace{11pt}
\noindent
\underline{Primary Textbook}: \textbook\\
\underline{Teacher}: \teacher
\newpage

\section{Homework 1}
\begin{outline*}
\1 Kirchoff's Current Law states that the current entering a node is equal to the current leaving a node, while Kirchoff's Voltage Law states that the voltage in a closed loop sums to 0
	\2 The passive sign convention is that the voltage across a source in the direction of current is positive, across a drain is negative
	\2 The standard method assigns currents to each branch, applying the laws to each loop and node, while the mesh loop method assigns a current to each loop
\1 Power of a circuit element is given by $P = VI$, where the power dissipated must be below the power rating, or the component will fail
\1 Ohm's Law states the V = IR, while R is based on the physical properties of the device specifically $R = \rho \frac{L}{A}$
	\2 Rheostat (two-terminal variable resistors) have a diagonal line drawn through the resistor, while the potentiometer (three-terminal) has the arrow drawn to it perpendicularly
	\2 Resistors in series are able to be summed up, while in parallel, the reciprocal of the sum of the reciprocals
\1 Thevnin's Theorem states that any two terminal network of sources and resistors can be replaced by a voltage source and resistor in series
	\2 When the terminals are not connected to anything in an open circuit, no current flows, such that the voltage of the resistor is 0, and the terminal voltage is the Thevnin voltage
	\2 Thevnin resistance is found by making an open circuit between the terminals and using that with the Thevnin voltage
		\3 It can also be found by replacing current sources with an open circuit, voltage sources with a short circuit, and then using resistance rules to calculate the total resistance
	\2 Norton's Theorem is a similar theorem which states any of the same type of network can be replaced by a current source in parallel with a resistor
		\3 The current source gives the same current as the short circuit current, and the Norton resistance is the same as the Thevnin resistance, with the open circuit terminal voltage as the same as the Thevnin voltage
\1 Resistors in parallel are able to make a current divider, such that $I_1 = (\frac{R_2}{R_1 + R_2})I$, while resistors in series can make a voltage divider, such that $V_2 = (\frac{R_2}{R_1 + R_2})V$
	\2 Voltage dividers can also be used as current limiters, to lower the voltage enough to lower the current to safe levels
	\2 Voltage and current dividers are also used in amnmeters and voltmeters respectively to allow a meter for much smaller values to scale to larger values, based on pre-defined resistor settings
\1 Measuring instruments become part of the circuit when connected, such that they have a defined input resistance which the meter provides, which must be large for voltmeters, tiny for amnmeters, to prevent changing the values
\1 **ADD ADDITIONAL READING**
\end{outline*}
\section{Homework 2}
\begin{outline*}
\1 AC quantities are currents and voltages which vary in time, though resistors still act normally
\1 Capacitors are formed by a pair of conductors, commonly parallel places in which capacitance is given by $C = \frac{\epsilon A}{d}$, where d is the distance between places, A is the plate area, and $\epsilon$ is the dielectric constant of the material between
	\2 Capacitance is measured in Farads (F, Coulumbs per Volt)
	\2 Capacitors have a voltage rating as the maximum voltage able to be applied before electrical breakdown of the dielectric material
	\2 Capacitors are used for electrical storage devices, such that Q = CV, where Q is the charge stored on each plate for an applied voltage V
		\3 By extension, $I = C\frac{dV}{dt}$, such that the AC voltage is proportional to the current flowing through the capacitor by it
	\2 Capacitance in series is equal to the reciprocal of the sum of the reciprocals, while in parallel is the sum
\1 Inductors are the result of Ampere's Law stating that currents make magnetic fields and Faraday's Law stating that time-varying magnetic fields make voltages, such that AC currents make voltages
	\2 As a result, $V = L\frac{dI}{dt}$, where L is the self-inductance/inductance, negligable for general circuit loops, made non-negligable by coils of wires
	\2 For a solenoid coil, $L = \frac{\mu N^2 \pi R^2}{L}$, where L is the length of the coil, N is the number of turns, $\mu$ is the magnetic permeability of the material in the coil, and R is the coil radius, measured in Henries (H, V*s/A)
	\2 Inductors in series are equal to the sum, while in parallel are the reciprocal of the sum of the reciprocals
\1 RC circuits are a resistor and a capacitor in series, with a switch that can connect them alone or to a battery $V_0$
	\2 By KVL, $V = IR + \frac{Q}{C}$, where V is either 0 or $V_0$, able to be differentiated with respect to t, and solved to get $I = I_0e^{-\frac{t}{RC}}$
		\3 As a result, the voltage across the capacitor, $V_C = V - I_0Re^{\frac{-t}{RC}} = V_1e^{\frac{-t}{RC}} + V_2$ for either state, written generally
	\2 The initial condition for charging is specified such that $V_C = 0, V = V_0$, such that the constants are determined to be $V_C = V_0(1 - e^{\frac{-t}{RC}})$
	\2 The initial conditions for discharging are given such that $V_C = V_0, V = 0$, such that $V_C = V_0e^{-\frac{t}{RC}}$
	\2 RC circuits responding to a square AC wave act similarly but have the voltage change from $V_0$ to $-V_0$, such that the discharge approaches $-V_0$ instead
		\3 For $RC << \frac{T}{2}$, it is able to mostly finish charging and discharging, while it only curves slightly when they are equal, and remains linear, not charging enough to curve when greater
		\3 While the current equation remains the same, V is different, such that it is solved again for each case
	\2 For AC, the voltage is a function of time, rather than constant, called time domain analysis, able to be approximated as a sine wave/frequency domain analysis
		\3 The voltage and current are approximated as $V = V_p sin(\omega t), I = I_P sin(\omega t + \phi)$, where it is found that $I_P = \frac{\omega CV_P}{\sqrt{1 + (\omega RC)^2}}, \phi = tan^{-1}(\frac{1}{\omega RC})$
		\3 Thus, the voltage on the resistor is $V_r = \frac{\omega RCV_psin(\omega t + \phi)}{\sqrt{1 + (\omega RC)^2}}$, such that it is a positive phase shifter by $\phi$
			\4 In addition $\frac{V_r}{V} = \frac{\omega RC}{\sqrt{1 + (\omega RC)^2}}$, such that it is a high-pass filter, removing low frequency voltages, leaving high
			\4 The breakpoint/half-power frequency is defined as when $\omega RC = 1$
			\4 On the other hand, for low frequencies (small values of $\omega RC$), it is found that $V_{out} \approx RC\frac{dV_{in}}{dt}$, such that it is called a differentiator
		\3 Similarly, the voltage on the capacitor can be found by the formula $Q = \int Idt$ and $cos(A) = -sin(A - \frac{\pi}{2})$, such that $V_c = \frac{\omega V_psin(\omega t + \phi - \frac{\pi}{2})}{\sqrt{1 + (\omega RC)^2}}$, acting as a negative phase shifter
			\4 In addition $\frac{V_c}{V} = \frac{1}{\sqrt{1 + (\omega RC)^2}}$, such that it is a low-pass filter
			\4 It is found that for high $\omega RC$, $V_{out} \approx \frac{1}{RC}\int V_{in}dt$, such that it is an integrator circuit
\end{outline*}
\section{Homework 3}
\begin{outline*}
\1 Based on the fact that $z = a + bi = |z|(cos\theta + isin\theta) = |z|e^{i\theta}$, where $\theta = tan^{-1} (\frac{b}{a})$, complex analysis can be used to avoid requiring trigonometric identities and simplify calculations
	\2 It is assumed that $V_{in} = V_pcos(\omega t), I = I_p cos(\omega t + \phi)$, such that it is the real component of the complex values
	\2 Complex resistance is able to be generalized to capacitors and inductors by impedance, where the real part is resistive impedance/resistance and the imaginary part is reactive impedance/reactance
		\3 $\hat{Z}_r = R, \hat{Z}_c = \frac{1}{i\omega C}, \hat{Z}_i = i\omega L$, able to be summed in series, and combined in parallel as the reciprocal of the reciprocal sum
		\3 Ohm's Law can be generalized to impedence by $\hat{V} = \hat{I}\hat{Z}$
	\2 Thus, total impedence can be used to find the complex current and voltage, taking the real component to get the actual value
	\2 This can be used to solve an LR circuit, to find that the resistor as the output is a negative phase shifter and low pass filter, while the inductor is a positive phase shifter and high pass filter
		\3 LRC circuits can be similarly solved, to find that it approaches 0 current for both high and low frequencies, such that it is a resonant system, at its peak at the resonant frequency ($\omega_0 = \frac{1}{\sqrt{LV}}$)	
			\4 Resonant systems are also called a band pass filter
			\4 At the resonant frequency, it is notable that the input and output voltages (measured over the resistor) are the same
	\2 For impedence, Thevnin's theorem holds true, able to generalize it to Thevenin impedance and voltage for circuits with capacitors and inductors
\1 For an LRC switching problem, the differential equation can be put in terms of Q, as a non-homogeneous differential equation, $L\frac{d^Q}{dt^2} + R\frac{dQ}{dt} + \frac{Q}{C} = V$, with the particular solution as the constant solution, $Q = CV$
	\2 The homogeneous equation can then be solved by plugging in the complex exponential, $\hat{Q} = Q_pe^{i\omega t}$
		\3 This is solved to get the solution, $\omega = i\frac{\gamma}{2} \pm \sqrt{\omega_0^2 - \frac{\gamma^2}{4}}$, where $\omega_0^2 = \frac{1}{LC}, \omega = \frac{R}{L}$e^
	\2 The underdamped case assumes $\omega_0^2 > \frac{\gamma^2}{4}$, such that $\omega = i\frac{\gamma}{2} \pm \omega_1$, $\omega_1 = \sqrt{\omega_0^2 - \frac{\gamma^2}{4}}$
		\3 Thus, $Q = (Q_1 + Q_2)e^{-\frac{\gamma t}{2}}cos(\omega_1 t) + CV$, able to be used to solve for the voltage by initial conditions
		\3 Since this oscillates, while the amplitude decays with respect to time, it is called ringing, approaching a constant over time (charged at $V_0$), found to be the case for both charging and discharging
		\3 This is found that for the voltage of the capacitor, $V_c = V_0(1 - e^{\frac{-\gamma}{2}t}cos(\omega_1 t)$ for charging, $V_c = V_0e^{-\frac{\gamma}{2}t}cos(\omega_1t)$ for discharging
	\2 The overdamped case assumes $\omega_0^2 < \frac{\gamma^2}{4}$, such that $\omega = j(\frac{\gamma}{2} \pm \beta), \beta = \sqrt{\frac{\gamma^2}{4} - \omega_0^2}$
		\3 This can be calculated for charging and discharging, similar to the RC response to an AC voltage
		\3 This is found to provide $V_c = \frac{V_0}{2}(2 - e^{-(\frac{\gamma}{2} + \beta)t} - e^{-(\frac{\gamma}{2} - \beta)t}$ for charging, $V_c = \frac{V_0}{2}(e^{-(\frac{\gamma}{2} + \beta)t} - e^{-(\frac{\gamma}{2} - \beta)t}$ for discharging
	\2 The critically damped case has $\omega_0^2 = \frac{\gamma^2}{4}$, such that $V_c = (A + Bt)e^{-\frac{\gamma}{2}t} + V$
	\2 Time constants are defined such that $e^{\frac{-t}{\tau}}$ is a multiplicative factor, causing exponential decay/growth
\1 The energy stored within an inductor, $U = \frac{1}{2}LI^2$, such that they store energy through current, unable to be instantaneously
	\2 Inductors are used in LC circuits as reversed high pass and low pass filters, but require much larger devices, and have parallel capacitance and series resistance within, such that they are not used commonly
	\2 In addition, since inductors can be made by loops of circuits, pseudo-inductors are often made by the archetectual construction of the circuit itself
\1 High-pass circuits can be thought of as having a high impedence at higher frequencies, due to open circuits having the most reactive voltage, open circuits with the least
\1 **Add Post Page 2 Supplementary Reading**
\end{outline*}
\section{Homework 4}
\begin{outline*}
\1 PN junctions are made of silicon primarily in modern day, acting as a semiconductor, with the conductivity vastly increased by doping by elements added in small amounts to the silicon, joining the quadruple covalent bond lattice, with an extra charge
	\2 Donor elements, such as phosphorus, arsenic, or antimony are placed in the Silicon crystal, with an extra electron in the outer orbital, easily mobile, reducing the resistivity
	\2 Acceptor atoms, such as boron or aluminum, have one electron less than silicon, with an excess positive charge, increasing the resistivity
	\2 Donor silicon is called n type while acceptor is called p type, combined to form a potential difference with the p-type as the positive anode, n as the negative cathode
		\3 As a result, current is only able to flow from p to n, drawn by an arrow with a line perpendicular to the point, showing the direction of current
\1 Diodes are non-linear, approximated most easily as a voltage activated switch, such that it conducts as an short circuit, otherwise open
	\2 For a junction diode, it is modeled such that it has a forward bias, such that voltage has to be greater than some positive value to conduct, with a voltage drop of that value for any current
		\3 Zener diodes have a forward bias voltage to conduct and a greater magnitude reverse bias voltage after which it is able to conduct in reverse
		\3 The voltage drop is often modelled as 0 for simplicity
	\2 Diode circuits are solved by assuming it is either open or short, replacing it if short with a voltage drop of the forward bias, calculating the remainder to see if the assumptions are valid
	\2 In reality, diodes are defined by the equation $I = I_S(e^{\frac{eV}{nkT}} - 1)$, where $I_s$ is the saturation current around $10^{-15} A$, k is the Boltzmann constant, $1.38 * 10^{-23} W*s/K$, and T is the temperature
		\3 n ranges from 1 to 2, depending on the diode structure, approximately 1 for surface mounted PCB ones, 2 for hand-used diodes
		\3 $\frac{kT}{e}$ is the thermal voltage, $V_T$, approximated for room temperature as 25 mV, with the -1 term able to be ignored due to the size of $I_D$
	\2 
\end{outline*}
\end{document}