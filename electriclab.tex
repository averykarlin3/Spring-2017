\documentclass[11 pt, twoside]{article}
\usepackage{textcomp}
\usepackage[margin=1in]{geometry}
\usepackage[utf8]{inputenc}
\usepackage{color}
\usepackage{indentfirst} %Comment out for no first paragraph indent
\usepackage[parfill]{parskip}
\usepackage{setspace}
\usepackage{tikz}
\usepackage{amsmath}
\usepackage{amsfonts}
\usepackage{amssymb}
\usepackage{enumitem}
\usepackage{outlines}
\usepackage{physymb}

\usepackage{fancyhdr}
\pagestyle{fancy}
\cfoot{\hyperlink{content}{\thepage}}
\lhead{}
\chead{}
\rfoot{}
\lfoot{}
\rhead{}
\renewcommand{\headrulewidth}{0pt}
\renewcommand{\footrulewidth}{0pt}


\usepackage{hyperref}
\hypersetup {
	colorlinks,
	citecolor=black,
	filecolor=black,
	linkcolor=black,
	urlcolor=black
}

\newcommand{\sepitem}{0pt} %Added room between items on the list, not including a list and its sublist
\newcommand{\seppar}{1pt} %Between items and lists overall

\setenumerate[1]{itemsep=\sepitem, parsep=\seppar}
\setenumerate[2]{itemsep=\sepitem, parsep=\seppar}
\setenumerate[3]{itemsep=\sepitem, parsep=\seppar}
\setenumerate[4]{itemsep=\sepitem, parsep=\seppar}

\newenvironment{outline*}
{
	\begin{outline}[enumerate]
	}
	{\end{outline}
}

\newcommand{\foot}[1]{\hyperlink{#1}{$_#1$}}

\begin{document}

\title{Electronics Lab}
\author{Avery Karlin}
\date{Fall 2017}
\newcommand{\textbook}{}
\newcommand{\teacher}{}

\maketitle
\newpage
\hypertarget{content}{\tableofcontents}
\vspace{11pt}
\noindent
\underline{Primary Textbook}: \textbook\\
\underline{Teacher}: \teacher
\newpage

\section{Homework 1}
\begin{outline*}
\1 Kirchoff's Current Law states that the current entering a node is equal to the current leaving a node, while Kirchoff's Voltage Law states that the voltage in a closed loop sums to 0
	\2 The passive sign convention is that the voltage across a source in the direction of current is positive, across a drain is negative
	\2 The standard method assigns currents to each branch, applying the laws to each loop and node, while the mesh loop method assigns a current to each loop
\1 Power of a circuit element is given by $P = VI$, where the power dissipated must be below the power rating, or the component will fail
\1 Ohm's Law states the V = IR, while R is based on the physical properties of the device specifically $R = \rho \frac{L}{A}$
	\2 Rheostat (two-terminal variable resistors) have a diagonal line drawn through the resistor, while the potentiometer (three-terminal) has the arrow drawn to it perpendicularly
	\2 Resistors in series are able to be summed up, while in parallel, the reciprocal of the sum of the reciprocals
\1 Thevnin's Theorem states that any two terminal network of sources and resistors can be replaced by a voltage source and resistor in series
	\2 When the terminals are not connected to anything in an open circuit, no current flows, such that the voltage of the resistor is 0, and the terminal voltage is the Thevnin voltage
	\2 Thevnin resistance is found by making an open circuit between the terminals and using that with the Thevnin voltage
		\3 It can also be found by replacing current sources with an open circuit, voltage sources with a short circuit, and then using resistance rules to calculate the total resistance
	\2 Norton's Theorem is a similar theorem which states any of the same type of network can be replaced by a current source in parallel with a resistor
		\3 The current source gives the same current as the short circuit current, and the Norton resistance is the same as the Thevnin resistance, with the open circuit terminal voltage as the same as the Thevnin voltage
\1 Resistors in parallel are able to make a current divider, such that $I_1 = (\frac{R_2}{R_1 + R_2})I$, while resistors in series can make a voltage divider, such that $V_2 = (\frac{R_2}{R_1 + R_2})V$
	\2 Voltage dividers can also be used as current limiters, to lower the voltage enough to lower the current to safe levels
	\2 Voltage and current dividers are also used in amnmeters and voltmeters respectively to allow a meter for much smaller values to scale to larger values, based on pre-defined resistor settings
\1 Measuring instruments become part of the circuit when connected, such that they have a defined input resistance which the meter provides, which must be large for voltmeters, tiny for amnmeters, to prevent changing the values
\1 **ADD ADDITIONAL READING**
\end{outline*}
\section{Homework 2}
\begin{outline*}
\1 AC quantities are currents and voltages which vary in time, though resistors still act normally
\1 Capacitors are formed by a pair of conductors, commonly parallel places in which capacitance is given by $C = \frac{\epsilon A}{d}$, where d is the distance between places, A is the plate area, and $\epsilon$ is the dielectric constant of the material between
	\2 Capacitance is measured in Farads (F, Coulumbs per Volt)
	\2 Capacitors have a voltage rating as the maximum voltage able to be applied before electrical breakdown of the dielectric material
	\2 Capacitors are used for electrical storage devices, such that Q = CV, where Q is the charge stored on each plate for an applied voltage V
		\3 By extension, $I = C\frac{dV}{dt}$, such that the AC voltage is proportional to the current flowing through the capacitor by it
	\2 Capacitance in series is equal to the reciprocal of the sum of the reciprocals, while in parallel is the sum
\1 Inductors are the result of Ampere's Law stating that currents make magnetic fields and Faraday's Law stating that time-varying magnetic fields make voltages, such that AC currents make voltages
	\2 As a result, $V = L\frac{dI}{dt}$, where L is the self-inductance/inductance, negligable for general circuit loops, made non-negligable by coils of wires
	\2 For a solenoid coil, $L = \frac{\mu N^2 \pi R^2}{L}$, where L is the length of the coil, N is the number of turns, $\mu$ is the magnetic permeability of the material in the coil, and R is the coil radius, measured in Henries (H, V*s/A)
	\2 Inductors in series are equal to the sum, while in parallel are the reciprocal of the sum of the reciprocals
\1 RC circuits are a resistor and a capacitor in series, with a switch that can connect them alone or to a battery $V_0$
	\2 By KVL, $V = IR + \frac{Q}{C}$, where V is either 0 or $V_0$, able to be differentiated with respect to t, and solved to get $I = I_0e^{-\frac{t}{RC}}$
		\3 As a result, the voltage across the capacitor, $V_C = V - I_0Re^{\frac{-t}{RC}} = V_1e^{\frac{-t}{RC}} + V_2$ for either state, written generally
	\2 The initial condition for charging is specified such that $V_C = 0, V = V_0$, such that the constants are determined to be $V_C = V_0(1 - e^{\frac{-t}{RC}})$
	\2 The initial conditions for discharging are given such that $V_C = V_0, V = 0$, such that $V_C = V_0e^{-\frac{t}{RC}}$
	\2 RC circuits responding to a square AC wave act similarly but have the voltage change from $V_0$ to $-V_0$, such that the discharge approaches $-V_0$ instead
		\3 For $RC << \frac{T}{2}$, it is able to mostly finish charging and discharging, while it only curves slightly when they are equal, and remains linear, not charging enough to curve when greater
		\3 While the current equation remains the same, V is different, such that it is solved again for each case
	\2 For AC, the voltage is a function of time, rather than constant, called time domain analysis, able to be approximated as a sine wave/frequency domain analysis
		\3 The voltage and current are approximated as $V = V_p sin(\omega t), I = I_P sin(\omega t + \phi)$, where it is found that $I_P = \frac{\omega CV_P}{\sqrt{1 + (\omega RC)^2}}, \phi = tan^{-1}(\frac{1}{\omega RC})$
		\3 Thus, the voltage on the resistor is $V_r = \frac{\omega RCV_psin(\omega t + \phi)}{\sqrt{1 + (\omega RC)^2}}$, such that it is a positive phase shifter by $\phi$
			\4 In addition $\frac{V_r}{V} = \frac{\omega RC}{\sqrt{1 + (\omega RC)^2}}$, such that it is a high-pass filter, removing low frequency voltages, leaving high
			\4 The breakpoint/half-power frequency is defined as when $\omega RC = 1$
			\4 On the other hand, for low frequencies (small values of $\omega RC$), it is found that $V_{out} \approx RC\frac{dV_{in}}{dt}$, such that it is called a differentiator
		\3 Similarly, the voltage on the capacitor can be found by the formula $Q = \int Idt$ and $cos(A) = -sin(A - \frac{\pi}{2})$, such that $V_c = \frac{\omega V_psin(\omega t + \phi - \frac{\pi}{2})}{\sqrt{1 + (\omega RC)^2}}$, acting as a negative phase shifter
			\4 In addition $\frac{V_c}{V} = \frac{1}{\sqrt{1 + (\omega RC)^2}}$, such that it is a low-pass filter
			\4 It is found that for high $\omega RC$, $V_{out} \approx \frac{1}{RC}\int V_{in}dt$, such that it is an integrator circuit
\end{outline*}
\section{Homework 3}
\begin{outline*}
\1 Based on the fact that $z = a + bi = |z|(cos\theta + isin\theta) = |z|e^{i\theta}$, where $\theta = tan^{-1} (\frac{b}{a})$, complex analysis can be used to avoid requiring trigonometric identities and simplify calculations
	\2 It is assumed that $V_{in} = V_pcos(\omega t), I = I_p cos(\omega t + \phi)$, such that it is the real component of the complex values
	\2 Complex resistance is able to be generalized to capacitors and inductors by impedance, where the real part is resistive impedance/resistance and the imaginary part is reactive impedance/reactance
		\3 $\hat{Z}_r = R, \hat{Z}_c = \frac{1}{i\omega C}, \hat{Z}_i = i\omega L$, able to be summed in series, and combined in parallel as the reciprocal of the reciprocal sum
		\3 Ohm's Law can be generalized to impedence by $\hat{V} = \hat{I}\hat{Z}$
	\2 Thus, total impedence can be used to find the complex current and voltage, taking the real component to get the actual value
	\2 This can be used to solve an LR circuit, to find that the resistor as the output is a negative phase shifter and low pass filter, while the inductor is a positive phase shifter and high pass filter
		\3 LRC circuits can be similarly solved, to find that it approaches 0 current for both high and low frequencies, such that it is a resonant system, at its peak at the resonant frequency ($\omega_0 = \frac{1}{\sqrt{LV}}$)	
			\4 Resonant systems are also called a band pass filter
			\4 At the resonant frequency, it is notable that the input and output voltages (measured over the resistor) are the same
	\2 For impedence, Thevnin's theorem holds true, able to generalize it to Thevenin impedance and voltage for circuits with capacitors and inductors
\1 For an LRC switching problem, the differential equation can be put in terms of Q, as a non-homogeneous differential equation, $L\frac{d^Q}{dt^2} + R\frac{dQ}{dt} + \frac{Q}{C} = V$, with the particular solution as the constant solution, $Q = CV$
	\2 The homogeneous equation can then be solved by plugging in the complex exponential, $\hat{Q} = Q_pe^{i\omega t}$
		\3 This is solved to get the solution, $\omega = i\frac{\gamma}{2} \pm \sqrt{\omega_0^2 - \frac{\gamma^2}{4}}$, where $\omega_0^2 = \frac{1}{LC}, \omega = \frac{R}{L}$e^
	\2 The underdamped case assumes $\omega_0^2 > \frac{\gamma^2}{4}$, such that $\omega = i\frac{\gamma}{2} \pm \omega_1$, $\omega_1 = \sqrt{\omega_0^2 - \frac{\gamma^2}{4}}$
		\3 Thus, $Q = (Q_1 + Q_2)e^{-\frac{\gamma t}{2}}cos(\omega_1 t) + CV$, able to be used to solve for the voltage by initial conditions
		\3 Since this oscillates, while the amplitude decays with respect to time, it is called ringing, approaching a constant over time (charged at $V_0$), found to be the case for both charging and discharging
		\3 This is found that for the voltage of the capacitor, $V_c = V_0(1 - e^{\frac{-\gamma}{2}t}cos(\omega_1 t)$ for charging, $V_c = V_0e^{-\frac{\gamma}{2}t}cos(\omega_1t)$ for discharging
	\2 The overdamped case assumes $\omega_0^2 < \frac{\gamma^2}{4}$, such that $\omega = j(\frac{\gamma}{2} \pm \beta), \beta = \sqrt{\frac{\gamma^2}{4} - \omega_0^2}$
		\3 This can be calculated for charging and discharging, similar to the RC response to an AC voltage
		\3 This is found to provide $V_c = \frac{V_0}{2}(2 - e^{-(\frac{\gamma}{2} + \beta)t} - e^{-(\frac{\gamma}{2} - \beta)t}$ for charging, $V_c = \frac{V_0}{2}(e^{-(\frac{\gamma}{2} + \beta)t} - e^{-(\frac{\gamma}{2} - \beta)t}$ for discharging
	\2 The critically damped case has $\omega_0^2 = \frac{\gamma^2}{4}$, such that $V_c = (A + Bt)e^{-\frac{\gamma}{2}t} + V$
	\2 Time constants are defined such that $e^{\frac{-t}{\tau}}$ is a multiplicative factor, causing exponential decay/growth
\1 The energy stored within an inductor, $U = \frac{1}{2}LI^2$, such that they store energy through current, unable to be instantaneously
	\2 Inductors are used in LC circuits as reversed high pass and low pass filters, but require much larger devices, and have parallel capacitance and series resistance within, such that they are not used commonly
	\2 In addition, since inductors can be made by loops of circuits, pseudo-inductors are often made by the archetectual construction of the circuit itself
\1 High-pass circuits can be thought of as having a high impedence at higher frequencies, due to open circuits having the most reactive voltage, open circuits with the least
\1 **Add Post Page 2 Supplementary Reading**
\end{outline*}
\section{Homework 4}
\begin{outline*}
\1 PN junctions are made of silicon primarily in modern day, acting as a semiconductor, with the conductivity vastly increased by doping by elements added in small amounts to the silicon, joining the quadruple covalent bond lattice, with an extra charge
	\2 Donor elements, such as phosphorus, arsenic, or antimony are placed in the Silicon crystal, with an extra electron in the outer orbital, easily mobile, reducing the resistivity
	\2 Acceptor atoms, such as boron or aluminum, have one electron less than silicon, with an excess positive charge, increasing the resistivity
	\2 Donor silicon is called n type while acceptor is called p type, combined to form a potential difference with the p-type as the positive anode, n as the negative cathode
		\3 As a result, current is only able to flow from p to n, drawn by an arrow with a line perpendicular to the point, showing the direction of current
\1 Diodes are non-linear, approximated most easily as a voltage activated switch, such that it conducts as an short circuit, otherwise open
	\2 For a junction diode, it is modeled such that it has a forward bias, such that voltage has to be greater than some positive value to conduct, with a voltage drop of that value for any current
		\3 Zener diodes have a forward bias voltage to conduct and a greater magnitude reverse bias voltage after which it is able to conduct in reverse
		\3 The voltage drop is often modelled as 0 for simplicity
	\2 Diode circuits are solved by assuming it is either open or short, replacing it if short with a voltage drop of the forward bias, calculating the remainder to see if the assumptions are valid
		\3 The forward bias of 0 V versus the conventional real 0.7 V forward bias is determined by the particular circuit, generally just changing the values slightly
		\3 If the current is found to be in the wrong direction, the assumption is wrong, and that diode is assumed to be wrong
	\2 In reality, diodes are defined by the equation $I = I_S(e^{\frac{eV}{nkT}} - 1)$, where $I_s$ is the saturation current around $10^{-15} A$, k is the Boltzmann constant, $1.38 * 10^{-23} W*s/K$, and T is the temperature
		\3 n ranges from 1 to 2, depending on the diode structure, approximately 1 for surface mounted PCB ones, 2 for hand-used diodes
		\3 $\frac{kT}{e}$ is the thermal voltage, $V_T$, approximated for room temperature as 25 mV, with the -1 term able to be ignored due to the size of $I_D$
	\2 Junction diodes can be modelled equivelently by a resistor and a DC voltage source in series, applying Kirchoff's Voltage Law, to get the relationship between diode voltage and current as $I = \frac{-V}{R} + \frac{V_{DC}}{R}$
		\3 This function is called the load line, such that it must satisfy the diode I-V relationship and the load line, since Kirchoff applies to non-linear as well
			\4 The intersection of the two functions is called the quiescant/operating point of the diode
		\3 An AC source added in series after the DC voltage adds an AC component to the current and voltage of the diode
			\4 Notation has a lowercase letter with an uppercase subscript refer to the total, uppercase for both for the DC component, and lowercase for both as the AC component
			\4 Unlike for a linear circuit, $i_D = I_D + i_d = f(V_D + v_d) \neq f(V_D) + f(v_d)$, $I_D \neq f(V_D)$, and $i_d \neq f(v_d)$
			\4 In addition, for an AC circuit, the AC quantities depend on the DC voltage as well as the total quantities
		\3 By the diode relationship and the small signal approximation, $I_D + i_d = I_De^{\frac{v_d}{nV_T}} = I_D(1 + \frac{v_d}{nV_T})$, or $i_d = \frac{I_D}{nV_T}v_d$, such that it can be modelled as a resistor
			\4 The superposition principle can then still be used, dividing into DC and AC, with the AC changed for a linear model and determined by the DC operating point
			\4 The resistance, called the dynamic resistance, $\rho = \frac{nV_T}{I_D}$, where $\frac{1}{\rho} = \frac{di_D}{dv_D}|_{v_D = V_D}$
			\4 This signifies that as long as the AC signal keeps the diode near the DC operating point, it responds to the AC linearly
\end{outline*}
\section{Homework 5}
\begin{outline*}
\1 Transformers are devices which transform the amplitude of AC voltage, commonly stepping down a high voltage transmission wire into a safe household level, called a step-down transformer, while the opposite is a step-up
	\2 Transformers are commonly made of a ring core, with $n_1$ primary windings around one side of the core, and $n_2$ secondary windings, where the current in the primary core and voltage are $I_1, V_1$ and $I_2, V_2$ for the secondary
		\3 The core is made of a ferrous material, keeping the magnetic field inside, flowing to the secondary part of the core
		\3 The current flowing through the primary coil causes a time-varying magnetic field inside the material, causing a current in the secondary coil
	\2 By Faraday's Law, $V_2 = \frac{n_2}{n_1}V_1$, and by conservation of energr, $I_1V_1 = I_2V_2$, such that both the voltage and the current depend on the turns ratio $\frac{n_1}{n_2}$
	\2 Transformers are denoted by a circuit with half of a coil and a line on the outside, next to the same within another circuit
	\2 Transformers are also used for impedence matching, based on the fact that the maximum power goes to the load if the load and internal impedence are equal
		\3 The transformer can be placed between the source and the load, such that it is found that $V_1 = I_1((\frac{n_1}{n_2})^2R_L)$, providing a modified effective resistance
\1 The diode piecewise function model is found graphically, looking at the intersection of the load line and diode exponential, using either a forward bias of 0.6 or 0.7 V, able to be any voltage without forward bias, but no current
\1 Diodes are commonly used as voltage droppers, putting a large number in series to drop by that value
	\2 It can also be a limited, putting a resistor in series with two diodes in parallel in opposite directions, such that whichever way the current is flowing, one of the diodes defines the voltage as the forward bias, used to prevent spikes
		\3 A variable diode clipper can be made by adding a battery in series to each of the diodes to add some set amount of voltage to the output
	\2 Diode clamps are used to shift the AC signal by a constant voltage, made up of a capacitor and a diode in series, taking the output voltage of the diode
		\3 For voltages lower than $-0.6 V$, the capacitor is charged to a voltage $V_{PP} - 0.6$, where $V_{PP}$ is peak to peak voltage, unable to discharge due to the diode only going one way
		\3 As a result, the output voltage is shifted upward by the voltage of the capacitor if the diode is pointed towards the capacitor, negative if away
	\2 Diodes can be used to protect inductors, placed in parallel, such that reverse bias is present when the inductor is on
		\3 When the circuit is turned off, the inductor produces a voltage by $V = L\frac{dI}{dt}$ in the opposite direction, causing an arc flash normally, activating the diode to short the circuit
	\2 Diodes are also used for logic circuits, connecting it to a switch that activates the circuit if any of the switches is connected to ground	
		\3 For the other setting, it is connected to a voltage equal to the separate voltage supply in series with a resistor
		\3 There is also a lightbulb connected to ground in parallel to the diodes, such that current only flows through the diodes if it is connected to ground, such that the lightbulb voltage is 0.6 V if any select no, too low to activate the lightbulb
\1 Diodes are used also for rectification, making an AC signal unidirectional, used to make a DC signal or an AM radio reciever
	\2 Half-wave rectifiers are a single diode and load in series, such that current only flows for positive voltage, creating a non-zero average voltage
	\2 Center-tapped full-wave rectifiers are made up of a center tap in parallel with a diode coming off of both, with a center tap of joined wire from the transformers in the middle, attached to a ground
		\3 The non-center taps are in parallel, joined in series to the load, which is in turn connected to ground
		\3 The center tap transformer allows identical secondary coils to be shared by two separate wire sets, providing a phase difference between the output of 180 degrees
		\3 Half-wave rectifier diodes then provide only the positive voltages, adding it them together, though with only half the original voltage amplitude, dividing the secondary windings
	\2 Full-wave bridge rectifiers are a transformer, leading to a diamond of diodes, all pointing away from the transformer, with the right-most and left-most corners leading to the load resistor
		\3 This loses twice the voltage of the forward voltage of the diode, but has the full voltage provided otherwise being given to the resistor, unidirectional without break
\1 Power supply filters are used to convert rectified AC voltage to a DC voltage, based on the idea that the recified circuit has a DC average component with an AC component
	\2 As a result, a low-pass filter is used, placing a capacitor in parallel to the output of the rectifier, such that it charges as it is positive
		\3 Since it is only able to discharge into the load resistor, it does so otherwise, in approximately $R_LC$ length of time (the time constant)
		\3 If the time constant is far larger than the period of the signal, the capacitor doesn't charge and discharge too drastically, making it more like DC, rising quickly to the peak voltage then slowly falling slightly
	\2 The quality of a power fsilter is denoted by the ripple factor, r, $\frac{V_{rms,AC}}{V_{DC}}$, found that for a capacitor filter with $R_LC >> T$, to be equal to $\frac{1}{2\sqrt{3}fR_LC}$, where f is the rectified signal frequency
		\3 It is also measured by the regulation, such that the supplied and output current doesn't impact output voltage
	\2 It is also measured that for a capacitor filter, $V_{DC} = V_{max} - \frac{V_{max}T}{2R_LC} = V_{max} - \frac{I_{DC}}{2fC}$, where it is assumed that the peak voltage is approximately the DC current multiplied by the resistive load
	\2 Alternative forms of regulation are an LC regulator, with the load parallel to the capacitor, and an RC $\pi$ regulator, with a capacitor, resistor, and capacitor in a loop, with the load parallel to the second capacitor
		\3 It is found that the ripple for both depend on a factor of $f^{-2}$, which is preferable for high frequency, and L-section (LC) is an ideal regulator
\1 **ADD END OF DIODE SECTION**
\end{outline*}
\section{Homework 6}
\begin{outline*}
\1 Signal amplification is needed due to signal weakness (sub-Volt energy) as a result of transducing, or changing the energy format, required to be linear so as to not change the signal information (distorting it)
	\2 Amplifiers function by $V_o = AV_i$, where A is the amplifier gain, called a linear amplifier in that case, otherwise causing non-linear distortion
	\2 Voltage amplifiers, often called preamplifiers, operate on small signals, while power amplifiers are used for large current gain with small voltage gain
		\3 Unlike transformers which can increase voltage, but cannot increase power, amplifiers are able to increase both voltage and power
		\3 As a result, amplifiers require an additional DC power source, both for power dissipated and outputed, requiring a positive and negative voltage terminal, denoted by an arrow going up and down from the 
			\4 The power efficiency, $\mu = \frac{P_{out}}{P_{dc}} * 100$
	\2 Voltage amplifiers have a limit to the maximum and minimum output voltages, above and below which, they will just go to the limit
\1 Voltage amplifiers generally have an input and output terminal, as well as a common ground terminal, denoted by a triangle, with the point as the output, the ground coming from a slanted side
\1 Voltage and current gain are able to be expressed in decibles as $20log|A|$, while power is represented in decibles as $20log|A|$
	\2 Absolute values are used since negative amplitudes refer to a phase difference of 180 degrees, but the gain is still the same value
\1 Since amplifiers are li3ear, the sine-wave frequency of an AC output is the same as the AC input, though other waveforms do change shape
	\2 The phase and amplitude are able to change as it moves through a linear circuit
	\2 The amplifier gain/frequency response is formally done as a vector with magnitude of the gain ratio (magnitude/amplitude response) and angle (phase response) of the phase shift
		\3 It is found that the amplitude response is constant within 3 dB over the amplifier bandwith, above and below which, there is a lower gain, generally only using the amplifier for the bandwidth frequencies
\1 Operational amplifiers take in a positive (non-inverting) and negative (inverting) input, and the output is the difference multiplied by the gain, such that if the inputs are the same, the output is 0, called common-mode rejection
	\2 Ideal op amps have infinite input resistance, 0 output resistance, and infinite gain, such that no current is drawn, and the load has no effect on the output
		\3 As a result, there is often feedback, such that the output is connected to one of the inputs, called negative or positive feedback, used for oscillators
		\3 Op amps are solved by assuming the input draws no current and adjusting the output such that the positive and negative inputs are equal
	\2 Inverting amplifiers have a resistor ($R_1$) from a voltage source to the negative input, the positive connected to ground, with a branch from the negative input to a resistor ($R_2$) to the output
		\3 As a result, $v_o = \frac{-R_2}{R_1}v_{in}$, while input resistance is $R_1$, such that it is higher than 0, shifting the signal and amplifying it
		\3 If it is calculated instead such that $v_{out} = A(v_+ - v_-)$ (assuming non-infinite gain), $\frac{v_{out}}{v_{in}} = \frac{-R_2/R_1}{1 + (1 + R_2/R_1)/A}$, such that as A approaches infinity, it approaches the previous ratio
		\3 By this configuration the gain and input resistance are inversely related, causing an issue, solved by adding an extra resister after $R_2$ branching downward to ground and another before the current reaches $v_{out}$
			\4 This has the additional resistors increase the gain magnitude, while allowing a high input resistance
	\2 Non-inverting amplifiers have the resistor connected to ground and the negative input, while $v_{in}$ is connected to the positive input, creating an effectively infinite input resistance, with $v_{out} = (1 + \frac{R_2}{R_1})v_{in}$
	\2 Voltage buffers have both resistors in the non-inverting amplifier replaced by a wire, such that input resistance is infinite, output is 0
	\2 **Add Remaining of Extra Reading**
\end{outline*}
\section{Homework 7}
\begin{outline*}
\1 In reality, small DC currents flow into op-amps, called the bias current ($I_B^{\pm}$), generally less than 500 nA, only causing an issue in certain cases
	\2 For a non-inverting amplifier with a capacitor at the non-inverting input (AC coupled non-inverting), the capacitor blocks the DC bias current, preventing the op-amp from working
	\2 For a non-inverting amplifier with the non-inverting input as ground instead of voltage, the bias current must flow through the ground for the + input, and from the output for the - input, such that the output isn't 0 V as expected ($V_{out} = I_B^-R_f$)
		\3 In addition, since the voltage outputted is higher for a higher resistor going from the output to - input, high feedback resistors increase the effect
	\2 For a non-inverting amplifier with the non-inverting input as ground, with an input resistor before it
		\3 It is found for this circuit that $V_{out} = R_f(I_B^- - I_B^+)$, where $R_f$ is the feedback resistor, such that it depends on the input offset current (the maximum mechanical bias current difference), lowering the effect of it
			\4 This equation is only valid if the added resistor, $R = \frac{R_1R_f}{R_1 + R_f}$, where $R_1$ is the inverting input resistor
		\3 Thus, a resistor is added to the non-inverting side in various circuits to prevent an effect from the bias current
\1 There is also a small offset between the voltage of the inputs, called the input offset voltage, acting like a small voltage source in series with the inputs, creating a non-zero output for two grounded inputs
	\2 Op-amps are thus made with nulling inputs, onto which a potentiometer is connected, adjusting the output to 0, called nulling/external balancing
\1 The slew rate limit is the maximum rate of change of the output voltage from a change in the input voltage
	\2 As a result for an AC signal, $\omega A$ maximum rate of change, such that the highest frequency that will prevent distortion is $f_{max} = \frac{SR}{2\pi A}$
	\2 The maximum frequency when the amplitude is equal to the saturation voltage (maximum output voltage based on the inputted voltage supply) of the op-amp is the full-power bandwidth
	\2 On the other hand, op-amps with a high slew rate have a tendency to oscillate spontaneously due to being unstable
\1 The frequency response of the op-amp is independent of the amplitude, such that the open loop (non-feedback) gain drops exponentially, while the closed loop is constant for longer, for the duration of the closed loop bandwidth
	\2 The constant of the closed loop gain and closed loop bandwidth is the gain bandwidth product/open-loop bandwidth
		\3 The open-loop bandwidth is also the frequency at which the open-loop gain is 1 
\end{outline*}
\end{document}