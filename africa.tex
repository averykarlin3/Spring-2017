\documentclass[11 pt, twoside]{article}
\usepackage{textcomp}
\usepackage[margin=1in]{geometry}
\usepackage[utf8]{inputenc}
\usepackage{color}
\usepackage{indentfirst} %Comment out for no first paragraph indent
\usepackage[parfill]{parskip}
\usepackage{setspace}
\usepackage{tikz}
\usepackage{amsmath}
\usepackage{amsfonts}
\usepackage{amssymb}
\usepackage{enumitem}
\usepackage{outlines}
\usepackage{physymb}

\usepackage{fancyhdr}
\pagestyle{fancy}
\cfoot{\hyperlink{content}{\thepage}}
\lhead{}
\chead{}
\rfoot{}
\lfoot{}
\rhead{}
\renewcommand{\headrulewidth}{0pt}
\renewcommand{\footrulewidth}{0pt}


\usepackage{hyperref}
\hypersetup {
	colorlinks,
	citecolor=black,
	filecolor=black,
	linkcolor=black,
	urlcolor=black
}

\newcommand{\sepitem}{0pt} %Added room between items on the list, not including a list and its sublist
\newcommand{\seppar}{1pt} %Between items and lists overall

\setenumerate[1]{itemsep=\sepitem, parsep=\seppar}
\setenumerate[2]{itemsep=\sepitem, parsep=\seppar}
\setenumerate[3]{itemsep=\sepitem, parsep=\seppar}
\setenumerate[4]{itemsep=\sepitem, parsep=\seppar}

\newenvironment{outline*}
{
	\begin{outline}[enumerate]
	}
	{\end{outline}
}

\newcommand{\foot}[1]{\hyperlink{#1}{$_#1$}}

\begin{document}

\title{Africa}
\author{Avery Karlin}
\date{Fall 2017}
\newcommand{\textbook}{}
\newcommand{\teacher}{}

\maketitle
\newpage
\hypertarget{content}{\tableofcontents}
\vspace{11pt}
\noindent
\underline{Primary Textbook}: \textbook\\
\underline{Teacher}: \teacher
\newpage

\begin{outline*}
\1 Africa played an active role in the Scramble for Africa, starting to move away from former traditions, accepting the global economy and European influences, even before military conquests
	\2 On the optimistic side, they were have a cultural and political revolution to acclimate to the Western world, up until the military takeover
\1 Abolition led to anti-slave squads to enforce it, legitimate trade in raw materials, and return of slaves to their home countries in short term
	\2 On the coastal cities, it also led to conversion of pro-Western literate elites (creole class), calling for increased European influence, partially due to religion, gratitude over the end of slavery, and attempts to save the continent
		\3 This created an alternative to direct imperialism, mainly ignored by the European powers
	\2 In long term, it led to inequality and subordination (both in Western and African nations, continuing in an unofficial or pseudo-legal way, such as Jim Crow), Western view of Africa, modern pan-Africanism, imperialism
		\3 There was African elite anger as well about the removal of slavery, due to causing the destruction of the slave-based economy
	\2 It did not take place immediately, but rather just slowed until the 1870s, when all major European powers became industrialized
	\2 Towns were made in Sierra Leone and Liberia for freed slaves such as Freetown, often given financial aid by European powers, as a pseudo-colony
\end{outline*}
\end{document}