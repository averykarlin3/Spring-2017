\documentclass[11 pt, twoside]{article}
\usepackage{textcomp}
\usepackage[margin=1in]{geometry}
\usepackage[utf8]{inputenc}
\usepackage{color}
\usepackage{indentfirst} %Comment out for no first paragraph indent
\usepackage[parfill]{parskip}
\usepackage{setspace}
\usepackage{tikz}
\usepackage{amsmath}
\usepackage{amsfonts}
\usepackage{amssymb}
\usepackage{enumitem}
\usepackage{outlines}
\usepackage{physymb}

\usepackage{fancyhdr}
\pagestyle{fancy}
\cfoot{\hyperlink{content}{\thepage}}
\lhead{}
\chead{}
\rfoot{}
\lfoot{}
\rhead{}
\renewcommand{\headrulewidth}{0pt}
\renewcommand{\footrulewidth}{0pt}


\usepackage{hyperref}
\hypersetup {
	colorlinks,
	citecolor=black,
	filecolor=black,
	linkcolor=black,
	urlcolor=black
}

\newcommand{\sepitem}{0pt} %Added room between items on the list, not including a list and its sublist
\newcommand{\seppar}{1pt} %Between items and lists overall

\setenumerate[1]{itemsep=\sepitem, parsep=\seppar}
\setenumerate[2]{itemsep=\sepitem, parsep=\seppar}
\setenumerate[3]{itemsep=\sepitem, parsep=\seppar}
\setenumerate[4]{itemsep=\sepitem, parsep=\seppar}

\newenvironment{outline*}
{
	\begin{outline}[enumerate]
	}
	{\end{outline}
}

\newcommand{\foot}[1]{\hyperlink{#1}{$_#1$}}

\begin{document}

\title{Africa}
\author{Avery Karlin}
\date{Fall 2017}
\newcommand{\textbook}{}
\newcommand{\teacher}{}

\maketitle
\newpage
\hypertarget{content}{\tableofcontents}
\vspace{11pt}
\noindent
\underline{Primary Textbook}: \textbook\\
\underline{Teacher}: \teacher
\newpage

\begin{outline*}
\1 Africa played an active role in the Scramble for Africa, starting to move away from former traditions, accepting the global economy and European influences, even before military conquests
	\2 On the optimistic side, they were have a cultural and political revolution to acclimate to the Western world, up until the military takeover
\1 Abolition led to anti-slave squads to enforce it, legitimate trade in raw materials, and return of slaves to their home countries in short term
	\2 On the coastal cities, it also led to conversion of pro-Western literate elites (creole class), calling for increased European influence, partially due to religion, gratitude over the end of slavery, and attempts to save the continent
		\3 This created an alternative to direct imperialism, mainly ignored by the European powers
	\2 In long term, it led to inequality and subordination (both in Western and African nations, continuing in an unofficial or pseudo-legal way, such as Jim Crow), Western view of Africa, modern pan-Africanism, imperialism
		\3 There was African elite anger as well about the removal of slavery, due to causing the destruction of the slave-based economy
	\2 It did not take place immediately, but rather just slowed until the 1870s, when all major European powers became industrialized
	\2 Towns were made in Sierra Leone and Liberia for freed slaves such as Freetown, often given financial aid by European powers, as a pseudo-colony
\end{outline*}
\section{10/16}
\begin{outline*}
\1 Elite were assimilated into Frenchman in particular, since peasants were unable to be assimilated, allowed to rise in the ranks of colonial governance and later in the French government (unlike indirect rule, which more preserved preexisting structures?), giving them a bigger voice than their numbers
\1 The colonial era struck down the belief system, both in legislation and in comparison to the European technological superiority (which created both the downfall of African culture in Africa and the European idea of superiority)
	\2 The idea of superiority led to the European idea of being unable to release the colonies for a long amount of time and led to the civilizzing missions sent to Africa, who often did not understand the culture correctly
	\2 Africans responded to this by the assertion of value and legitimacy of African history, law, and custom (such as Kenyatta's thesis), African Independent church movements, and Negritude literary movement (both cultural nationalism as political activism prior to the right to do so)
		\3 These are 20th century movements, following the 19th century Ethiopianism and African Independent Churches such as the ancient-origin Ethiopian church, Edward Wilmot Blyden's African personality (?), and pan-Africanism (a network of educated Africans, believing to all have been subject to the same forces of Europeans)
		\3 Christianity was used as a tool of colonialism as an early frontier, with the elites and the converts as advocates for further colonization, and yet many continued to join, especially the educated, especially in evangelical movements which were hostile to African culture
			\4 Missionaries originally brought in outcasts and provided literacy and education to the converts (often the most vocally opposed to Europeans), since missionaries ran a majority of the schools in the colonies, not changed until independence
			\4 Christianity provided a response to community lost by the destruction of the age-grade system and the mass migration, and had a similar value to that of the age-grade system of community, as Kenyatta discussed, as well as the similar idea of prophets through dreams and visions to traditional African religion, and the idea of healing (both governments, divisions soul, body) through religion overlapped
			\4 Missionaries often defended the Africans against land takeovers or coercive labor laws, believing them to be uncivil
		\3 African Independent Churches were divided into Ethopian type churches (based in self-improvement and political rights) and Zionist type churchs (based in healing, prophecy, and spirit possession), differentiating from the values of European churches and missions which they broke off from (such as pro-polygamy)
			\4 In (Protestant?) regions in which the Bible was translated into African languages, such that the literate society members could read it, thousands of independent churches broke off
			\4 Chilembwe formed an independent church, critisizing atrocities of the cotton industry, akin to slavery, and the forced migration, starting to subtly request change, but progressively becoming militant, fighting against the decendent, Bruce Livingstone, who believed educated Africans had no place, such that Chilembwe eventually brought a military revolt, after which he was executed, but brought attention to the illegal practices of the plantation, as an example of a militant church
			\4 The Kimbanguist Church began to preach to copper belt workers, eventually arrested for dissent, but now an official Congo religion, although preaching a non-militant message, preaching a Zionist message of God being on their side and cultural strength of the African people as well as the Ethiopian message of the land belonging to Africans
		\3 The Negritude movement was formed by the educated/assimilated (Cesaire, Damas, Senghor) African elite in French Africa, formed from the secular schools created by the French ideal of separation of church and state, forming a literary society
\end{outline*}
\end{document}