\documentclass[11 pt, twoside]{article}
\usepackage{textcomp}
\usepackage[margin=1in]{geometry}
\usepackage[utf8]{inputenc}
\usepackage{color}
\usepackage{indentfirst} %Comment out for no first paragraph indent
\usepackage[parfill]{parskip}
\usepackage{setspace}
\usepackage{tikz}
\usepackage{amsmath}
\usepackage{amsfonts}
\usepackage{amssymb}
\usepackage{enumitem}
\usepackage{outlines}
\usepackage{physymb}

\usepackage{fancyhdr}
\pagestyle{fancy}
\cfoot{\hyperlink{content}{\thepage}}
\lhead{}
\chead{}
\rfoot{}
\lfoot{}
\rhead{}
\renewcommand{\headrulewidth}{0pt}
\renewcommand{\footrulewidth}{0pt}


\usepackage{hyperref}
\hypersetup {
	colorlinks,
	citecolor=black,
	filecolor=black,
	linkcolor=black,
	urlcolor=black
}

\newcommand{\sepitem}{0pt} %Added room between items on the list, not including a list and its sublist
\newcommand{\seppar}{1pt} %Between items and lists overall

\setenumerate[1]{itemsep=\sepitem, parsep=\seppar}
\setenumerate[2]{itemsep=\sepitem, parsep=\seppar}
\setenumerate[3]{itemsep=\sepitem, parsep=\seppar}
\setenumerate[4]{itemsep=\sepitem, parsep=\seppar}

\newenvironment{outline*}
{
	\begin{outline}[enumerate]
	}
	{\end{outline}
}

\newcommand{\foot}[1]{\hyperlink{#1}{$_#1$}}

\begin{document}

\title{The Social Contract}
\author{Avery Karlin}
\date{Fall 2017}
\newcommand{\textbook}{}
\newcommand{\teacher}{}

\maketitle
\newpage
\hypertarget{content}{\tableofcontents}
\vspace{11pt}
\noindent
\underline{Primary Textbook}: \textbook\\
\underline{Teacher}: \teacher
\newpage

\begin{outline*}
\1 Social contract provides a systematic view of political philosophy, highly influential to modern political order, with Hobbes as the main starting point during the modern period
	\2 Locke, while not mentioning Hobbes by name, wrote in direct opposition to his readings of Hobbes
\1 The main questions are those of legitimacy and justice/just political order, and what makes those states
	\2 The former can also be phrased as ``How does the state, if it can at all, come to have the right of coercion over individuals?''
	\2 The latter is phrased as ``What are the appropriate standards, or principles of justice, for regulating and ordering our public institutions and laws?''
\1 The question of legitimacy relies on consent within the social contract, but there is the question of conditions under which consent is valid

\1 The state of nature and human nature are fundamental concepts, understood differently by the different authors
	\2 The state of nature is the thought experiment or historical situation of a pre-political state, the human condition without government involved
		\3 The contract is made under this condition naturally, due to being mutally beneficial to do so
		\3 Interpretation of the state of nature's qualities create the type of governments that are considered acceptable, based on what is an improvement

\1 Hobbes wrote his works in light of the English Civil War, watching England get destroyed by questions of monarch power and civil duty
	\2 Powerful government is necessary to prevent civil war, and monarchy is the most efficient and effective form of powerful government, to protect against the violent human nature
		\3 Unlike previous justification (inheritance, divine right), he used political theory for justification of the monarchy
		\3 He didn't believe in any divinely given/inherant or natural morality, viewing humans materialistically as machines, tending deterministically towards violence
	\2 He was secular, such that the optimal solution is the appealing/utilitarian option, such that psychology defines morality
	\2 State of nature according to chapter 13 has life as nasty, brutish, and short, with too much chaos to have a guarantee of work and development yielding, such that it is not done
		\3 There is not constant war, but there is a constant threat and fear of violence and war, without property rights
		\3 Self-preservation is the only right in a state of nature, the only obligation, up to and including using other people
			\4 Rights impose an obligation on other people in the modern usage, but Hobbes in this case meant it as no moral/legal constraints on the goal
	\2 Psychological egoism is the modern term for the empirical thesis that people are motivated deep down by what they believe to be in their own self-interest
		\3 While Hobbes is often grouped under this, he did not believe it was completely true in all cases, but was in a state of nature
\end{outline*}
\end{document}