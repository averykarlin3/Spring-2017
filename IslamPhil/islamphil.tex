\documentclass[11 pt, twoside]{article}
\usepackage{textcomp}
\usepackage[margin=1in]{geometry}
\usepackage[utf8]{inputenc}
\usepackage{color}
\usepackage{indentfirst} %Comment out for no first paragraph indent
\usepackage[parfill]{parskip}
\usepackage{setspace}
\usepackage{tikz}
\usepackage{amsmath}
\usepackage{amsfonts}
\usepackage{amssymb}
\usepackage{enumitem}
\usepackage{outlines}

\usepackage{fancyhdr}
\pagestyle{fancy}
\cfoot{\hyperlink{content}{\thepage}}
\lhead{}
\chead{}
\rfoot{}
\lfoot{}
\rhead{}
\renewcommand{\headrulewidth}{0pt}
\renewcommand{\footrulewidth}{0pt}


\usepackage{hyperref}
\hypersetup {
	colorlinks,
	citecolor=black,
	filecolor=black,
	linkcolor=black,
	urlcolor=black
}

\newcommand{\sepitem}{0pt} %Added room between items on the list, not including a list and its sublist
\newcommand{\seppar}{1pt} %Between items and lists overall

\setenumerate[1]{itemsep=\sepitem, parsep=\seppar}
\setenumerate[2]{itemsep=\sepitem, parsep=\seppar}
\setenumerate[3]{itemsep=\sepitem, parsep=\seppar}
\setenumerate[4]{itemsep=\sepitem, parsep=\seppar}

\newenvironment{outline*}
{
	\begin{outline}[enumerate]
	}
	{\end{outline}
}

\newcommand{\foot}[1]{\hyperlink{#1}{$_#1$}}

\begin{document}

\title{Islamic Philosophy}
\author{Avery Karlin}
\date{Spring 2017}
%\newcommand{\textbook}{Modern Physics by Tipler and Llewellyn}
\newcommand{\teacher}{}

\maketitle
\newpage
\hypertarget{content}{\tableofcontents}
\vspace{11pt}
\noindent
%\underline{Primary Textbook}: \textbook\\
\underline{Teacher}: \teacher
\newpage

Phalsopha - history, philosophy, natural science, etc
Kalam - the act of speaking, defending faith against scholars of other faiths, theology
	  - viewed in West as contrasting phalosopha, opposing
	  - predated, sharing the motivations of phalsopha, differing in methods and the belief in the power of logic
- Islam originally in near east, eventually expanding to Spain, being pushed back, and growing into the full middle east and africa instead and diaspora
- Classical period to 900s, independent Islamic states with large Arab minorities, and other religious minorities
	- Religion did not spread as quickly as Arabic language, such that other religions also speak Arabic in regions
	- Begun from 570-632 CE with the prophet Mohammed, in Arabia in Mecca born, persecuted and migrated to Medina, at which point the Islamic calendar starts
	- Converted most in Medina, later returns to Mecca, converts them, and eventually unites the Arabic tribes under Islam
	- Succeeded by the Rightly Guided Caliphs, the 3rd had the Quran written down, not many documented sources of most early history, mainly based on Islamic legend, without opposing POV so accepted generally
	- Many questioned if it transcended tribal boundaries as a universal faith or if they were just monotheistic Arabs
	- With migration to the Fertile Crescent, became integrated with and adopted aspects of other early monotheistic religions, Christianity, Judaism
- Islam became fully formed after the Early/Classical period in the 10th century, after which there was less contrast between historical and theological debate
- 2nd Century CE onward was denoted the age of anxiety, with Rabbinical Judaism forming, and Christianity forming its own identity, fighting against Paganism
	- Sassanit/Persian state is founded, adopting Zoastroarianism, fighting against the Byzantine empire for centuries
	- Lot of religious competition and formation in the time leading up to Islam, more individual choice in religious conversion
	- Christianity weak and powerless, developing slowly, until Constantine in 312 converted and allowed it as the Roman Empire religion, and the Edict of Thessalonica in 80 CE set it as the official religion
	- Coplic church in Egypt develops in this point in time against 451 Council of Calcadon which declared that Jesus was perfect God and man with two natures as an official mystery of the church (faith), taken up by Byzantine
		- They stated a single nature, union of God and man, weakening the power of the Roman church in the Near East and the Byzantine Empire influence, showing relationship between theological and political issues
		- Coptic church is monofusite, meaning one nature
	- Arabia is less settled, urban, or agriultural, broken into tribal groups, such that Islam is the nomadic semetic spirit of the region
		- Some Jewish and Christian tribes, especially in the more urban Medina, Mecca
		- Well-established trade groups into Egypt, Africa, South Asia, Greece, etc, such that there is information inflow from abroad
- Quran has 144 Chapters/Suras, written in a poetic form, as the specific word of God with Mohammed as the vehicle, depicted as a living/animated world/nature, beginning often with nature odes
	- Living Gin beings are present, all familiar concepts to Arabs, as general spiritual beings
	- Mohammed is just the next prophet in a line of prophets, present in every period of history, starting with Adam, with Mohammed as the last prophet, viewed as a reflection of the unstable political nature
	- Similar Jewish poet at that time in Mecca, who claimed to be prophet, fairly common at that point, such that success of Mohammed implies political advantage of Islamic philosophy in Arabia at that time
	- Judaism and Christianity are often vaguely referenced, assuming familiarity with the religious texts at that time

Philosophy Hanabalite:
- Creed is a statement of the communal beliefs of a sect
- Hannibalite use of the name of sects meaning standard general terminology, except for Abu Hanifa, showing heretical, and distance?, specifically separating creatednessism
- why in obscure locations found for all of the creeds?
- Hadith = word of Mohammed
- How can someone with unrepented sin both be in Hell and have the option of being pardoned? Does dying in Islam mean belief and mainly action?
- Hypocrisy is critism of the founders of Islam, such that it is a sin, because founders and all commit sins? Why does that make him an innovator?
- Purpose of weighing works of a person?
- People of the Qibla? Basin? Munkar? Nakir?
- Intercession clause repetitive (11)?

Philosophy Hanafite:
- No mention of lack of doubt of Mohammed's allies, or are those also messengers?
- Adam not like God in appearance
- Quran is the eternal speech of God, but the reports (in it?) of his talking are just created descriptions of eternity?
- God abandoned to make them unbelive causing God to abandon them - people only born in religious state?
- Simultaneously their own choice and God's, prophet sins are mistakes, not sins?
- Reuniting of body with spirit?

Philosophy Al-Ashari:
- Raising up tombs mentioned again? Again conception only by God's grace (knowledge here)

\end{document}