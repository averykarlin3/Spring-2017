\documentclass[11 pt, twoside]{article}
\usepackage{textcomp}
\usepackage[margin=1in]{geometry}
\usepackage[utf8]{inputenc}
\usepackage{color}
\usepackage{indentfirst} %Comment out for no first paragraph indent
\usepackage[parfill]{parskip}
\usepackage{setspace}
\usepackage{tikz}
\usepackage{amsmath}
\usepackage{amsfonts}
\usepackage{amssymb}
\usepackage{enumitem}
\usepackage{outlines}

\usepackage{fancyhdr}
\pagestyle{fancy}
\cfoot{\hyperlink{content}{\thepage}}
\lhead{}
\chead{}
\rfoot{}
\lfoot{}
\rhead{}
\renewcommand{\headrulewidth}{0pt}
\renewcommand{\footrulewidth}{0pt}


\usepackage{hyperref}
\hypersetup {
	colorlinks,
	citecolor=black,
	filecolor=black,
	linkcolor=black,
	urlcolor=black
}

\newcommand{\sepitem}{0pt} %Added room between items on the list, not including a list and its sublist
\newcommand{\seppar}{1pt} %Between items and lists overall

\setenumerate[1]{itemsep=\sepitem, parsep=\seppar}
\setenumerate[2]{itemsep=\sepitem, parsep=\seppar}
\setenumerate[3]{itemsep=\sepitem, parsep=\seppar}
\setenumerate[4]{itemsep=\sepitem, parsep=\seppar}

\newenvironment{outline*}
{
	\begin{outline}[enumerate]
	}
	{\end{outline}
}

\newcommand{\foot}[1]{\hyperlink{#1}{$_#1$}}

\begin{document}

\title{Islamic Philosophy}
\author{Avery Karlin}
\date{Spring 2017}
%\newcommand{\textbook}{Modern Physics by Tipler and Llewellyn}
\newcommand{\teacher}{Dr. Nabeel Hamid}

\maketitle
\newpage
\hypertarget{content}{\tableofcontents}
\vspace{11pt}
\noindent
%\underline{Primary Textbook}: \textbook\\
\underline{Teacher}: \teacher
\newpage

\section{Kalam}

Phalsopha - history, philosophy, natural science, etc
Kalam - the act of speaking, defending faith against scholars of other faiths, theology
	  - viewed in West as contrasting phalosopha, opposing
	  - predated, sharing the motivations of phalsopha, differing in methods and the belief in the power of logic
- Islam originally in near east, eventually expanding to Spain, being pushed back, and growing into the full middle east and africa instead and diaspora
- Classical period to 900s, independent Islamic states with large Arab minorities, and other religious minorities
	- Religion did not spread as quickly as Arabic language, such that other religions also speak Arabic in regions
	- Begun from 570-632 CE with the prophet Mohammed, in Arabia in Mecca born, persecuted and migrated to Medina, at which point the Islamic calendar starts
	- Converted most in Medina, later returns to Mecca, converts them, and eventually unites the Arabic tribes under Islam
	- Succeeded by the Rightly Guided Caliphs, the 3rd had the Quran written down, not many documented sources of most early history, mainly based on Islamic legend, without opposing POV so accepted generally
	- Many questioned if it transcended tribal boundaries as a universal faith or if they were just monotheistic Arabs
	- With migration to the Fertile Crescent, became integrated with and adopted aspects of other early monotheistic religions, Christianity, Judaism
- Islam became fully formed after the Early/Classical period in the 10th century, after which there was less contrast between historical and theological debate
- 2nd Century CE onward was denoted the age of anxiety, with Rabbinical Judaism forming, and Christianity forming its own identity, fighting against Paganism
	- Sassanit/Persian state is founded, adopting Zoastroarianism, fighting against the Byzantine empire for centuries
	- Lot of religious competition and formation in the time leading up to Islam, more individual choice in religious conversion
	- Christianity weak and powerless, developing slowly, until Constantine in 312 converted and allowed it as the Roman Empire religion, and the Edict of Thessalonica in 80 CE set it as the official religion
	- Coplic church in Egypt develops in this point in time against 451 Council of Calcadon which declared that Jesus was perfect God and man with two natures as an official mystery of the church (faith), taken up by Byzantine
		- They stated a single nature, union of God and man, weakening the power of the Roman church in the Near East and the Byzantine Empire influence, showing relationship between theological and political issues
		- Coptic church is monofusite, meaning one nature
	- Arabia is less settled, urban, or agriultural, broken into tribal groups, such that Islam is the nomadic semetic spirit of the region
		- Some Jewish and Christian tribes, especially in the more urban Medina, Mecca
		- Well-established trade groups into Egypt, Africa, South Asia, Greece, etc, such that there is information inflow from abroad
- Quran has 144 Chapters/Suras, written in a poetic form, as the specific word of God with Mohammed as the vehicle, depicted as a living/animated world/nature, beginning often with nature odes
	- Living Gin beings are present, all familiar concepts to Arabs, as general spiritual beings
	- Mohammed is just the next prophet in a line of prophets, present in every period of history, starting with Adam, with Mohammed as the last prophet, viewed as a reflection of the unstable political nature
	- Similar Jewish poet at that time in Mecca, who claimed to be prophet, fairly common at that point, such that success of Mohammed implies political advantage of Islamic philosophy in Arabia at that time
	- Judaism and Christianity are often vaguely referenced, assuming familiarity with the religious texts at that time

Philosophy Hanabalite:
- Creed is a statement of the communal beliefs of a sect
- Hannibalite use of the name of sects meaning standard general terminology, except for Abu Hanifa, showing heretical, and distance?, specifically separating creatednessism
- why in obscure locations found for all of the creeds?
- Hadith = word of Mohammed
- How can someone with unrepented sin both be in Hell and have the option of being pardoned? Does dying in Islam mean belief and mainly action?
- Hypocrisy is critism of the founders of Islam, such that it is a sin, because founders and all commit sins? Why does that make him an innovator?
- Purpose of weighing works of a person?
- People of the Qibla? Basin? Munkar? Nakir?
- Intercession clause repetitive (11)?

Philosophy Hanafite:
- No mention of lack of doubt of Mohammed's allies, or are those also messengers?
- Adam not like God in appearance
- Quran is the eternal speech of God, but the reports (in it?) of his talking are just created descriptions of eternity?
- God abandoned to make them unbelive causing God to abandon them - people only born in religious state?
- Simultaneously their own choice and God's, prophet sins are mistakes, not sins?
- Reuniting of body with spirit?

Philosophy Al-Ashari:
- Raising up tombs mentioned again? Again conception only by God's grace (knowledge here)
- God chose to abandon the unbelievers
- Bring a group of messangers from hell as the Messanger said? Who? (31)
- Against debate and metaphysics (32)
- Not adding what not permitted, but following the consensus of Muslims as backup? (36)
- \textbf{Jahmiyya?}
- Impossible to mark the upright by God
- Fire for children of polytheists

Blankinship Creed:
- Faith level/afterlife debate, presence of God debate
- Sunnis branched off of the non-judging Murji'a combining with the original proto-Sunni traditionalists, faith unable to decrease, predestinarian, uncreated Quran
	- Hanbali traditionalists considered them illegitimate because of their emphasis on rationalism and kalam, tortured by Mutazils for not agreeing createdness, using literalism instead
		- Al-Ashari became Hanbali purist, increasing and decreasing faith based in action and belief, seeing God amodally, uncreated, limited punishment for believers, essential characteristics coexisting with God
			- Becamse the primary form of Sunniism, though using rationalism, while claiming to reject it, merged with Maturidi
	- Hanafis stated actions both followed and were simultaneous with God allowing
	- Believed caliph must be of Quraysh tribe of Mohammed
	- Aisha sect became Zubayrids, largely acting as intellectual memory keepers, located in Medinna, fighting against metaphysics, were predestinarian, severed because of debate on Uthman
		- This later became mainstream Sunnism, taking lead over Muawiya's proto-form
- Kharijite rebels rejected compromise, duty of resistance to any caliphate, sins negated faith fully, against human rule, elected leaders, free will doctrine
- Qadaris were based on pietists, stating humans had responsibility for sin, leading to doctrine of free will, dividing into pietist Sunni and Mutazilists
	- Mutazilists upheld the free will doctrine in a fully rationalistic defense, God is unique and thus cannot be anthropomorphic, even indivisably so
- Shia divided into Zaydis, electing inferior imams, and Twelvers, uncertain about predestination in early times
- Mutazilites believed attributes and form of God were metaphorical, with the internal characteristics meaning God acting through external, but identical
	- Muslim sinners between belief and unbelief, free will, Quran created
	- Gained power over Abbassid caliphate in the 800s, leading to persecution of Hanbali
	- God had an obligation to not be merciful, in order to justice, not as a personal deity, but for justice, humans empowered to act slightly before, atomizing time, equal guidence from God to all
	- Highly philisophical sect generally, eventually adopted partially by merging with Shiism, and making God more personal to fit with Sunni

Response 1:
	Early Islam seems to be filled with different sects, nuances, and parties than it appears to be made up of in modern day, showing either a consolidation over time, possibly in response to a more globalized world, persecution, or fighting. It is clear as well that a large portion of Islamic debate originate politically, almost acting like political parties, such as the debate over Uthman and Ali between the Shia and Sunni, or the debate over the anthropomorphic and thus literalist interpretation of the Quran, based on the political nature of separation from Christianity. This also extends to the debate over belief and unbelief, creating the political issue of treatment both of leaders, subjects, and those of other religions, and the debate over predetermination, determining if leaders are responsible for their own actions. On the other hand, the debate over the attributes of God or the createdness of the Quran, while related to the literalist debate, do not seem to have political reprocussions, and while possibly helping to divide sects who had conflict in the past, seem primarily theological in nature, raising the question of what allowed those debates to be unrelated to the political aspects in early Islam.
	Another point of interest is the debate over determinism, especially al-Ashari's insistance on the predeterminism of God, has aspects of the anti-metaphysical stance of the Hanbalites, for example, the idea of God having declared unfaith, such that a specific unfaith declaration of God can be negative, but the overall declaration of God cannot be said to be, brushing over the actual issue of God's predestined unfaith by simply avoiding statement of it. 
	Finally, a recurring point in each of the works is the order of leaders as the most faithful, stating Uthman third, then Ali, implying a form of the non-judgemental view, in that the order is simply all that counts, while the details of the circumstances of the first civil war are irrelevent. On the other hand, it also seems to imply the superiority of Uthman in terms of purity, possibly showing a tinge of Sunni bias in each of the original works read.

Lesson 2 Notes:
- Islamic philosophy since 1800s has been the POV that philosophy should be timeless, called the falsafa movement, isolated from other intellectual activity, largely done by European scholars, attempting to trace back to Greece and Rome
- Sharia movement of legal codes (from Quran and hadith/tradition), tradition-collection movement of traditions and annotations of Mohammed (discerning which are authentic)
	- Based on the idea of authenticity of the original community, after large amount of war, trying to regain truth by chain of transmission from Mohammed
	- Sharia is a case-based legal system as system of interpretation, made up of 2 source principles (Quran and hadith) and 2 method principles (ijma/consensus, ajyas/analogy)
		- Attempts to construct sets of laws from the 4 principles, like common law
- Kalam of debate and rational arguments to refute opponents and debate theology from the Quran, articulating distinction of Islam, falsapha as the 4th movement, distinct from behavior, laws, argument, theology, or authenticity of traditions
	- Instead falsapha is attempting to continue Greco-Roman ideas in Islam, non-ideal due to Arabs most likely using it for analysis of Islam theologically, but must be viewed in relation to other aspects, with the same scholars doing all four
	- Astrology and medicine from Greek texts led to the revival by Islamic leaders
- Rightly-guided calphs goes until the end of the 1st civil war and the assassination of the 4th, Ali
- Dome of the Rock built afterward, asserting Islamic identity as the legacy of Abrahamic faiths, Islam correcting the strayed Abrahamic faiths, releasing proper-Ummayad coins, rather than using Byzantine
	- Showed themselves as taking up the legacy of former imperial civilizations, while creating their own differentiating characteristics
- Ummayad caliphate failed partially due to the view of it being an Arab empire, reverting to Arab tribal rivalries, and calling non-Arab Muslims a lower status/client, the Abassid trying to correct that
	- Set Baghdad as an imperial capital city from Ummayad's Damascus, in concentric circles with the mosque at the center, setting it as a new cosmopolitan civilization
		- Not at Mecca or Medina due to keeping their capitals similar, not entangling in their religious identity too much, moving from region of conflict, or possibly better geographical land strategically for control and trade and in terms of resources
		- Mecca and Medina viewed as place of authenticity of true Islam, hadith
	- Start of 800s Harun al-Rashid reign viewed as the start of proper philosophy, theology, high learning, viewed as a golden age, acting as patrons to many foreign scholars
- Free will debate originated in status of the grave believing sinner, believer, unbeliever, in-between, beginning with Kharijjite debate over killing of Uthman, if God creates evil or if humans create evil
	- Predeterminism, possible actions predetermined, or complete free will
	- Hanafi school viewed faith as speech and belief, predeterminism, Hanbali school viewed as belief, speech, and works, free will, increasing and decreasing faith
- al-Ashari book is the qadar (determination/ism), the night of the Qadar is when all actions are determined by God, the role of human action in a deterministic system

Ch 5 al-Ashari:
- The wood is made by God, and the action, but not the meaning attributed to it? Or is the carving? (83) Fooling men was determined by God as well
	- Why would the literality of all phrase be equal, such that their deception is invalid? (84) Proof seems faulty, but the overall sentiment seems legitimate
- Since the hard road of faith must be harder, then it must have someone to create it as hard, else someone would have made it differently (85), aquisitional act
- Does God acquire the action or not? (88) Man does not acquire the action, they are a part of it, but does not acquire (90)
- No secondary causation of God, all directly, such that minor actions are as well, since needed actions are (91), concurrent with God's choice, versus a non-body action, purely by God
- God made all unneeded movements, because otherwise, somebody else would be able to control movements and all would be unknown (94)
	- Gain power only as an acquisition, while other motion is not an acquisition? (95)
- Impossibility of an unfinished set of actions (127)? Capacity requires the act to occur (133)? People must possess either the power to do something or opposite, not both (135)
	- All people have the bodily ability for a long journey (144)? No ability too difficult to resist contradicts it (149)?
- Mutazilla only able to use the direct antecedent in interpretation? (143)

Lecture 3:
- Mutazilla hated due to the idea of the impersonal God, going against Sunni orthodoxy
	- Five principles, Gods unity, justice (people of unity and justice), promise of threat, intermediate state, do good and forbid evil 
		- Justice requires free will, making decisions for themselves
	- Radically simple God, not composed of many parts, such that God = life/justice/knowledge, rather than having the attributes, amodally possessing (unknowable the manner by which God has the attributes)
		- Justice is essence of God, such that it is unchanging, mercy is the dispensing of justice
		- Amodal solution similar to the concept of solving God's physical appearance/meeting of God after judgement day
	- Believed Quran is not God's speech, thus created, due to not being amodal, due to that would require a change in God (when he started speaking the Quran)
	- Also embodiment of the state vs religious scholars, due to the Mutazilla losing the fight for the createdness
- Taklif = duty/burden, the age of being subjected to laws and justice, responsibility to understand and make proper decisions
- al-Ashari - God created the action of human beings, but humans acquired them, such that actions are predetermined, but humans take the actions on their own, assuming they possess taklif
\end{document}