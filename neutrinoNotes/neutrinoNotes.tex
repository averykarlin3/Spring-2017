\documentclass[11 pt, twoside]{article}
\usepackage{textcomp}
\usepackage[margin=1in]{geometry}
\usepackage[utf8]{inputenc}
\usepackage{color}
\usepackage{indentfirst} %Comment out for no first paragraph indent
\usepackage[parfill]{parskip}
\usepackage{setspace}
\usepackage{tikz}
\usepackage{amsmath}
\usepackage{amsfonts}
\usepackage{amssymb}
\usepackage{enumitem}
\usepackage{outlines}

\usepackage{fancyhdr}
\pagestyle{fancy}
\cfoot{\hyperlink{content}{\thepage}}
\lhead{}
\chead{}
\rfoot{}
\lfoot{}
\rhead{}
\renewcommand{\headrulewidth}{0pt}
\renewcommand{\footrulewidth}{0pt}


\usepackage{hyperref}
\hypersetup {
	colorlinks,
	citecolor=black,
	filecolor=black,
	linkcolor=black,
	urlcolor=black
}

\newcommand{\sepitem}{0pt} %Added room between items on the list, not including a list and its sublist
\newcommand{\seppar}{1pt} %Between items and lists overall

\setenumerate[1]{itemsep=\sepitem, parsep=\seppar}
\setenumerate[2]{itemsep=\sepitem, parsep=\seppar}
\setenumerate[3]{itemsep=\sepitem, parsep=\seppar}
\setenumerate[4]{itemsep=\sepitem, parsep=\seppar}

\newenvironment{outline*}
{
	\begin{outline}[enumerate]
	}
	{\end{outline}
}

\newcommand{\foot}[1]{\hyperlink{#1}{$_#1$}}

\begin{document}

\title{Neutrino Research Notes}
\author{Avery Karlin}
\date{Spring 2017}
%\newcommand{\textbook}{Modern Physics by Tipler and Llewellyn}
\newcommand{\teacher}{Dr. Chris Mauger}

\maketitle
\newpage
\hypertarget{content}{\tableofcontents}
\vspace{11pt}
\noindent
%\underline{Primary Textbook}: \textbook\\
\underline{Teacher}: \teacher
\newpage

Finished Reads: 
	-Griffiths 1-4.2 (Except Certain Strong Force Portions)
	-Holliday Modern Section on Particles
	-Rex Ch 5, 12, Neutrino Detection, 14 (Except Section 8)
	-Artin 1-2, 5.1, 6.1-6.4, 8.1-8.3, 10.1
	-p1-7 of Article 1 Done
	-p1-5 of Article 2 Done

Possible Reads:
	-Griffiths 4.3
	-Artin 8.4, 10.2-10.3

Neutrino source above muon production threshold - Background
Hadronic system - Background = baryon/meson produced
Why can't the electrons be captured from electron neutrinos

Does oxygen only capture one type of muon? - only negative muons in a specific fraction

Gadolinium captures the neutrons which are released by the oxygen
What does the purity of the sample mean/why is high purity required
Eliminates the neutrino energy factor, only distance/flavor

Neutrino fluxes?

How do we know not massless
What does massive neutrinos imply with standard model
Meaning of mass inhibiting structure

Mixing angle determined by nature?

Capture frequency postulated to be independent of neutrino energy, gain percentage of neutrons of neutrino and anti-neutrino within a sample

Purity meaning and purpose?

1/2. What software exists currently
3. Within what programming paradigm
5. Meaning of calibrating the experiment

Rex:
- Bosonic difference from fermions (exclusionary principle)?
- Do leptons and hadrons interact $->$ automatically weak?, can hadrons interacts with weak?
- Neutrinos only produce their cooresponding particles in interactions, Conservation of Lepton Number for Each Family (+- 1 per) - how with oscillation?, Conservation of Baryon Number, Conservataion of Strangeness, Topness, Bottomness, Charm (not weak)
- Why do particles have minimum kinetic energy? Does that just mean minimum possible maximum KE? (Smallest top of the range) - Hows does uncertainty give it?
- Wavelength perpendicular to energy? Why must it be continuous probability wavefunction?
- Why does decay into two pions violate CP? Does lack of parity imply the frame of reference matters?
- Pauli Exclusion == No Two Fermions in the Same State
- Quantum gravity is the graviton?

Griffiths:
- How does neutral interaction hurt parity, is it only for scattering?
- Least energy ideal?
- Why does crossing symmetry not require the photons by annihilation?
- How did Raines experiment know the inverse beta wasn't just the same as the inverse original without neutrino
- Wouldn't the spacing needed be twice the Compton wavelength of an electron
- Why is the Kobayash-Maskawa matrix give the sqrt probability for each component?
- Gluon coupling and why the OZI rule is valid, why less in short ranges?
- Meaning of coupling constant converging - interaction of Higgs field with others?

- No understanding of Tensor portions in Ch 3 or Einstein matrix notation
- No understanding of proof of always CM for massive particles

- How is SO(3) a rotation and how is it equal to SU(2)?
- How can every group be represented by a group of matrices (by homomorphic map)?
- Why can we only measure some values of spin by Heisenberg, and must it be restricted
- What is the decomposition of states/what does each states equation denote specifically?
- Derivation of matrix for spin calculation and subsequent calculations?

- Meaning of a left-handed doublet of the particles? Right-handed singlet? Why does the lack of right-handed neutrinos make them massless and conservation of lepton generation?
- Invisible width of Z boson and relationship to neutrino active flavors?
- Meaning of an adjoint field? Spinors?
- How are the Lagrangians axiomatic?

\section{Artin's Algebra}
\begin{outline*}
\1 Ch 2 - Known that d is the GCD versus another divider - Since it must be within the set, only if GCD
	 \2 Known m must be LCM why?
	 \2 How is the generated group found for some subset
	 \2 How are cosets equal if homomorphic result is equal
	 \2 injective = at most one from the domain, surjective = at least one, bijective = both
\1 Ch 1 - How is proving the multiplicative property for determinants assuming it valid?
	 \2 Proof of uniqueness of the determinant, $det(A) = det(A^t), det(A^-1) = 1/det(A)$?
\1 Ch 5 - Confused on the entire 2D rotation proof? What has determinate 1, all rotation?
	 \2 Orthoganal matrices are all rotation, or just SO? 
	 \2 Determinate -1 combines a reflection about the axis
	 \2 Trace independent of basis? Is trace of a matrix and its conjugate equal?
	 \2 Continuity used to figure definitive discrete value
	 \2 Proof of Euler's theorem? How can we know M' (matrix with respect to the orthoganal basis) is orthoganal and determinant 1?
\1 Ch 8 - How does a change of basis change a vector space coordinate system? Orthoganal change of basis preserves dot product?
	 \2 $PP^t$ always symmetric?
	 \2 Does a positive definite Hermitian matrix imply a positive definite Hermitian form?
	 \2 Orthonormal basis = basis of orthogonal unit vectors
\1 Ch 6 - Uncertain how reflection and rotation == reflection and such forth?
	 \2 Entire section of translation vectors?
\1 Ch 10 - Properties of the trace in general?
	  \2 Why are the rows and columns of the character of the representations orthogonal?
\end{outline*}

\section{Article 1}
\begin{outline*}
\1 p4 - singlet vs triplet states for a $p\mu$ atom?
\1 p4 - Where did the remaining 18\% (or 10\% because light is because of lack of emission) go?
\1 p4 - Is neutron production the emission or the conversion of protons, such that bound states produce one, but not emit
\1 p5 - virtual pion meaning? Also what is the parenthetical notation
\1 p5 - How are the proposed reactions similar?
\1 p6 - what does partial lifetime meaning?
\1 p7 - weak interaction == muon decay? lepton decay?
\1 p7 - mesonic states distinct from electron states or replacing them
\1 p7 - capture rate dependent on decay time, such that replies on CPT (symmetry)
\1 p7 - How can you say muons don't decay into electrons if the principle weak interaction formula (p7) stated the decay into an electron - no other mass loss, only energy meaning?
\1 p7 - heavy electron == excited state?
\end{outline*}

\section{Article 2}
\begin{outline*}
\1 p2 - Left handed partner has any proper meaning?
\1 p2 - CP violation fix by third generation, GUTs strong interactions
\1 p2 - Does mixing mean oscillation or something else
\1 p2 - Double beta decay, end point beta-decay anomalies
\1 p3 - What does it mean, why they are the masses they are, why these, why do they exist

\1 p3 - Entire mass see-saw mechanism section, Majorana particles, etc....????? - how does the see-saw prove they are Majorana or likewise for lack of conservation of total lepton
\2 Also definite flavors of quarks, weak charged current intezractions, mixed states
\2 Weak doublet, weak eigenstate meaning - eigenstate/doublet in weak interaction
\1 p4 - Mass not associated with particular flavor?
\1 p5 - What does it mean for the doublets to be $SU(2)_L$, and left and right handed components? How does this make it massless and conserved by family?
\1 p5 - Meaning of loop corrections?
\1 p5 - Are sterile neutrinos other flavors, or how are they differentiated from active? Do they act more when not at equilibrium?
\1 p5 - Big Bang Nucleosynthesis?
\1 p5 - Mass terms and see-saw mechanism?
\end{outline*}

\end{document}